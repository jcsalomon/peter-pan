% !TEX program = pdflatex
% !TEX encoding = UTF-8
% !TEX spellcheck = en_GB
% !TEX root = peter-pan.tex

\chapter{Wendy’s Story}

\endinput


Chapter 11 WENDY'S STORY


"Listen, then," said Wendy, settling down to her story, with Michael at
her feet and seven boys in the bed. "There was once a gentleman—"


"I had rather he had been a lady," Curly said.


"I wish he had been a white rat," said Nibs.


"Quiet," their mother admonished them. "There was a lady also,
and—"


"Oh, mummy," cried the first twin, "you mean that there is a lady also,
don't you? She is not dead, is she?"


"Oh, no."


"I am awfully glad she isn't dead," said Tootles. "Are you glad, John?"


"Of course I am."


"Are you glad, Nibs?"


"Rather."


"Are you glad, Twins?"


"We are glad."


"Oh dear," sighed Wendy.


"Little less noise there," Peter called out, determined that she should
have fair play, however beastly a story it might be in his opinion.


"The gentleman's name," Wendy continued, "was Mr. Darling, and her name
was Mrs. Darling."


"I knew them," John said, to annoy the others.


"I think I knew them," said Michael rather doubtfully.


"They were married, you know," explained Wendy, "and what do you think
they had?"


"White rats," cried Nibs, inspired.


"No."


"It's awfully puzzling," said Tootles, who knew the story by heart.


"Quiet, Tootles. They had three descendants."


"What is descendants?"


"Well, you are one, Twin."


"Did you hear that, John? I am a descendant."


"Descendants are only children," said John.


"Oh dear, oh dear," sighed Wendy. "Now these three children had a faithful
nurse called Nana; but Mr. Darling was angry with her and chained her up
in the yard, and so all the children flew away."


"It's an awfully good story," said Nibs.


"They flew away," Wendy continued, "to the Neverland, where the lost
children are."


"I just thought they did," Curly broke in excitedly. "I don't know how it
is, but I just thought they did!"


"O Wendy," cried Tootles, "was one of the lost children called Tootles?"


"Yes, he was."


"I am in a story. Hurrah, I am in a story, Nibs."


"Hush. Now I want you to consider the feelings of the unhappy parents with
all their children flown away."


"Oo!" they all moaned, though they were not really considering the
feelings of the unhappy parents one jot.


"Think of the empty beds!"


"Oo!"


"It's awfully sad," the first twin said cheerfully.


"I don't see how it can have a happy ending," said the second twin. "Do
you, Nibs?"


"I'm frightfully anxious."


"If you knew how great is a mother's love," Wendy told them triumphantly,
"you would have no fear." She had now come to the part that Peter hated.


"I do like a mother's love," said Tootles, hitting Nibs with a pillow. "Do
you like a mother's love, Nibs?"


"I do just," said Nibs, hitting back.


"You see," Wendy said complacently, "our heroine knew that the mother
would always leave the window open for her children to fly back by; so
they stayed away for years and had a lovely time."


"Did they ever go back?"


"Let us now," said Wendy, bracing herself up for her finest effort, "take
a peep into the future;" and they all gave themselves the twist that makes
peeps into the future easier. "Years have rolled by, and who is this
elegant lady of uncertain age alighting at London Station?"


"O Wendy, who is she?" cried Nibs, every bit as excited as if he didn't
know.


"Can it be—yes—no—it is—the fair Wendy!"


"Oh!"


"And who are the two noble portly figures accompanying her, now grown to
man's estate? Can they be John and Michael? They are!"


"Oh!"


"'See, dear brothers,' says Wendy pointing upwards, 'there is the window
still standing open. Ah, now we are rewarded for our sublime faith in a
mother's love.' So up they flew to their mummy and daddy, and pen cannot
describe the happy scene, over which we draw a veil."


That was the story, and they were as pleased with it as the fair narrator
herself. Everything just as it should be, you see. Off we skip like the
most heartless things in the world, which is what children are, but so
attractive; and we have an entirely selfish time, and then when we have
need of special attention we nobly return for it, confident that we shall
be rewarded instead of smacked.


So great indeed was their faith in a mother's love that they felt they
could afford to be callous for a bit longer.


But there was one there who knew better, and when Wendy finished he
uttered a hollow groan.


"What is it, Peter?" she cried, running to him, thinking he was ill. She
felt him solicitously, lower down than his chest. "Where is it, Peter?"


"It isn't that kind of pain," Peter replied darkly.


"Then what kind is it?"


"Wendy, you are wrong about mothers."


They all gathered round him in affright, so alarming was his agitation;
and with a fine candour he told them what he had hitherto concealed.


"Long ago," he said, "I thought like you that my mother would always keep
the window open for me, so I stayed away for moons and moons and moons,
and then flew back; but the window was barred, for mother had forgotten
all about me, and there was another little boy sleeping in my bed."


I am not sure that this was true, but Peter thought it was true; and it
scared them.


"Are you sure mothers are like that?"


"Yes."


So this was the truth about mothers. The toads!


Still it is best to be careful; and no one knows so quickly as a child
when he should give in. "Wendy, let us go home," cried John and
Michael together.


"Yes," she said, clutching them.


"Not to-night?" asked the lost boys bewildered. They knew in what they
called their hearts that one can get on quite well without a mother, and
that it is only the mothers who think you can't.


"At once," Wendy replied resolutely, for the horrible thought had come to
her: "Perhaps mother is in half mourning by this time."


This dread made her forgetful of what must be Peter's feelings, and she
said to him rather sharply, "Peter, will you make the necessary
arrangements?"


"If you wish it," he replied, as coolly as if she had asked him to pass
the nuts.


Not so much as a sorry-to-lose-you between them! If she did not mind the
parting, he was going to show her, was Peter, that neither did he.


But of course he cared very much; and he was so full of wrath against
grown-ups, who, as usual, were spoiling everything, that as soon as he got
inside his tree he breathed intentionally quick short breaths at the rate
of about five to a second. He did this because there is a saying in the
Neverland that, every time you breathe, a grown-up dies; and Peter was
killing them off vindictively as fast as possible.


Then having given the necessary instructions to the redskins he returned
to the home, where an unworthy scene had been enacted in his absence.
Panic-stricken at the thought of losing Wendy the lost boys had advanced
upon her threateningly.


"It will be worse than before she came," they cried.


"We shan't let her go."


"Let's keep her prisoner."


"Ay, chain her up."


In her extremity an instinct told her to which of them to turn.


"Tootles," she cried, "I appeal to you."


Was it not strange? She appealed to Tootles, quite the silliest one.


Grandly, however, did Tootles respond. For that one moment he dropped his
silliness and spoke with dignity.


"I am just Tootles," he said, "and nobody minds me. But the first who does
not behave to Wendy like an English gentleman I will blood him severely."


He drew back his hanger; and for that instant his sun was at noon. The
others held back uneasily. Then Peter returned, and they saw at once that
they would get no support from him. He would keep no girl in the Neverland
against her will.


"Wendy," he said, striding up and down, "I have asked the redskins to
guide you through the wood, as flying tires you so."


"Thank you, Peter."


"Then," he continued, in the short sharp voice of one accustomed to be
obeyed, "Tinker Bell will take you across the sea. Wake her, Nibs."


Nibs had to knock twice before he got an answer, though Tink had really
been sitting up in bed listening for some time.


"Who are you? How dare you? Go away," she cried.


"You are to get up, Tink," Nibs called, "and take Wendy on a journey."


Of course Tink had been delighted to hear that Wendy was going; but she
was jolly well determined not to be her courier, and she said so in still
more offensive language. Then she pretended to be asleep again.


"She says she won't!" Nibs exclaimed, aghast at such insubordination,
whereupon Peter went sternly toward the young lady's chamber.


"Tink," he rapped out, "if you don't get up and dress at once I will open
the curtains, and then we shall all see you in your \emph{négligée}."


This made her leap to the floor. "Who said I wasn't getting up?" she
cried.


In the meantime the boys were gazing very forlornly at Wendy, now equipped
with John and Michael for the journey. By this time they were dejected,
not merely because they were about to lose her, but also because they felt
that she was going off to something nice to which they had not been
invited. Novelty was beckoning to them as usual.


Crediting them with a nobler feeling Wendy melted.


"Dear ones," she said, "if you will all come with me I feel almost sure I
can get my father and mother to adopt you."


The invitation was meant specially for Peter, but each of the boys was
thinking exclusively of himself, and at once they jumped with joy.


"But won't they think us rather a handful?" Nibs asked in the middle of
his jump.


"Oh no," said Wendy, rapidly thinking it out, "it will only mean having a
few beds in the drawing-room; they can be hidden behind the screens on
first Thursdays."


"Peter, can we go?" they all cried imploringly. They took it for granted
that if they went he would go also, but really they scarcely cared. Thus
children are ever ready, when novelty knocks, to desert their dearest
ones.


"All right," Peter replied with a bitter smile, and immediately they
rushed to get their things.


"And now, Peter," Wendy said, thinking she had put everything right, "I am
going to give you your medicine before you go." She loved to give them
medicine, and undoubtedly gave them too much. Of course it was only water,
but it was out of a bottle, and she always shook the bottle and counted
the drops, which gave it a certain medicinal quality. On this occasion,
however, she did not give Peter his draught, for just as she had
prepared it, she saw a look on his face that made her heart sink.


"Get your things, Peter," she cried, shaking.


"No," he answered, pretending indifference, "I am not going with you,
Wendy."


"Yes, Peter."


"No."


To show that her departure would leave him unmoved, he skipped up and down
the room, playing gaily on his heartless pipes. She had to run about after
him, though it was rather undignified.


"To find your mother," she coaxed.


Now, if Peter had ever quite had a mother, he no longer missed her. He
could do very well without one. He had thought them out, and remembered
only their bad points.


"No, no," he told Wendy decisively; "perhaps she would say I was old, and
I just want always to be a little boy and to have fun."


"But, Peter—"


"No."


And so the others had to be told.


"Peter isn't coming."


Peter not coming! They gazed blankly at him, their sticks over their
backs, and on each stick a bundle. Their first thought was that if Peter
was not going he had probably changed his mind about letting them go.


But he was far too proud for that. "If you find your mothers," he said
darkly, "I hope you will like them."


The awful cynicism of this made an uncomfortable impression, and most of
them began to look rather doubtful. After all, their faces said, were they
not noodles to want to go?


"Now then," cried Peter, "no fuss, no blubbering; good-bye, Wendy;" and he
held out his hand cheerily, quite as if they must really go now, for he
had something important to do.


She had to take his hand, and there was no indication that he would prefer
a thimble.


"You will remember about changing your flannels, Peter?" she said,
lingering over him. She was always so particular about their flannels.


"Yes."


"And you will take your medicine?"


"Yes."


That seemed to be everything, and an awkward pause followed. Peter,
however, was not the kind that breaks down before other people. "Are you
ready, Tinker Bell?" he called out.


"Ay, ay."


"Then lead the way."


Tink darted up the nearest tree; but no one followed her, for it was at
this moment that the pirates made their dreadful attack upon the redskins.
Above, where all had been so still, the air was rent with shrieks and the
clash of steel. Below, there was dead silence. Mouths opened and remained
open. Wendy fell on her knees, but her arms were extended toward Peter.
All arms were extended to him, as if suddenly blown in his direction; they
were beseeching him mutely not to desert them. As for Peter, he seized his
sword, the same he thought he had slain Barbecue with, and the lust of
battle was in his eye.

