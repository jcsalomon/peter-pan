% !TEX program = pdflatex
% !TEX encoding = UTF-8
% !TEX spellcheck = en_GB
% !TEX root = peter-pan.tex

\chapter{The Never Bird}

The last sound Peter heard before he was quite alone
were the mermaids retiring one by one to their bedchambers under the sea.
He was too far away to hear their doors shut;
but every door in the coral caves where they live rings a tiny bell when it opens or closes
(as in all the nicest houses on the mainland),
and he heard the bells.

Steadily the waters rose till they were nibbling at his feet;
and to pass the time until they made their final gulp,
he watched the only thing on the lagoon.
He thought it was a piece of floating paper, perhaps part of the kite,
and wondered idly how long it would take to drift ashore.

Presently he noticed as an odd thing that it was undoubtedly out upon the lagoon with some definite purpose,
for it was fighting the tide, and sometimes winning;
and when it won, Peter, always sympathetic to the weaker side, could not help clapping;
it was such a gallant piece of paper.

It was not really a piece of paper;
it was the Never bird, making desperate efforts to reach Peter on the nest.
By working her wings, in a way she had learned since the nest fell into the water,
she was able to some extent to guide her strange craft,
but by the time Peter recognised her she was very exhausted.
She had come to save him, to give him her nest, though there were eggs in it.
I rather wonder at the bird, for though he had been nice to her, he had also sometimes tormented her.
I can suppose only that, like Mrs.\@ Darling and the rest of them, she was melted because he had all his first teeth.

She called out to him what she had come for,
and he called out to her what she was doing there;
but of course neither of them understood the other’s language.
In fanciful stories people can talk to the birds freely,
and I wish for the moment I could pretend that this were such a story,
and say that Peter replied intelligently to the Never bird;
but truth is best, and I want to tell you only what really happened.
Well, not only could they not understand each other, but they forgot their manners.

“I·—·want·—·you·—·to·—·get·—·into·—·the·—·nest,” the bird called,
speaking as slowly and distinctly as possible,
“and·—·then·—·you·—·can·—·drift·—·ashore,
but·—·I·—·am·—·too·—·tired·—·to·—·bring·—·it·—·any·—·nearer·—·so·—·you·—·must·—·try to·—·swim·—·to·—·it.”

“What are you quacking about?\@” Peter answered.
“Why don’t you let the nest drift as usual?”

“I·—·want·—·you·—” the bird said, and repeated it all over.

Then Peter tried slow and distinct.

“What·—·are·—·you·—·quacking·—·about?\@” and so on.

The Never bird became irritated;
they have very short tempers.

“You dunderheaded little jay,” she screamed,
“Why don’t you do as I tell you?”

Peter felt that she was calling him names, and at a venture he retorted hotly:

“So are you!”

Then rather curiously they both snapped out the same remark:

“Shut up!”

“Shut up!”

Nevertheless the bird was determined to save him if she could,
and by one last mighty effort she propelled the nest against the rock.
Then up she flew;
deserting her eggs, so as to make her meaning clear.

Then at last he understood, and clutched the nest and waved his thanks to the bird as she fluttered overhead.
It was not to receive his thanks, however, that she hung there in the sky;
it was not even to watch him get into the nest;
it was to see what he did with her eggs.

There were two large white eggs, and Peter lifted them up and reflected.
The bird covered her face with her wings, so as not to see the last of them;
but she could not help peeping between the feathers.

I forget whether I have told you that there was a stave on the rock,
driven into it by some buccaneers of long ago to mark the site of buried treasure.
The children had discovered the glittering hoard,
and when in a mischievous mood used to fling showers of moidores, diamonds, pearls and pieces of eight to the gulls,
who pounced upon them for food, and then flew away, raging at the scurvy trick that had been played upon them.
The stave was still there,
and on it Starkey had hung his hat, a deep tarpaulin, watertight, with a broad brim.
Peter put the eggs into this hat and set it on the lagoon.
It floated beautifully.

The Never bird saw at once what he was up to, and screamed her admiration of him;
and, alas, Peter crowed his agreement with her.
Then he got into the nest, reared the stave in it as a mast, and hung up his shirt for a sail.
At the same moment the bird fluttered down upon the hat and once more sat snugly on her eggs.
She drifted in one direction, and he was borne off in another, both cheering.

Of course when Peter landed he beached his barque in a place where the bird would easily find it;
but the hat was such a great success that she abandoned the nest.
It drifted about till it went to pieces,
and often Starkey came to the shore of the lagoon,
and with many bitter feelings watched the bird sitting on his hat.
As we shall not see her again,
it may be worth mentioning here that all Never birds now build in that shape of nest,
with a broad brim on which the youngsters take an airing.

Great were the rejoicings when Peter reached the home under the ground almost as soon as Wendy,
who had been carried hither and thither by the kite.
Every boy had adventures to tell;
but perhaps the biggest adventure of all was that they were several hours late for bed.
This so inflated them that they did various dodgy things to get staying up still longer,
such as demanding bandages;
but Wendy, though glorying in having them all home again safe and sound,
was scandalised by the lateness of the hour,
and cried, “To bed, to bed,” in a voice that had to be obeyed.
Next day, however, she was awfully tender,
and gave out bandages to every one,
and they played till bed‐time at limping about and carrying their arms in slings.

\endinput
