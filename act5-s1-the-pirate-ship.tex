% !TEX program = pdflatex
% !TEX encoding = UTF-8
% !TEX spellcheck = en_GB
% !TEX root = peter-pan.tex

\act

\Scene{The Pirate Ship}

\begin{Settings}
The stage directions for the opening of this scene are as follows:—%
1 Circuit Amber checked to 80.
Battens, all Amber checked,
3 ship’s lanterns alight,
Arcs: prompt perch 1.
Open dark Amber flooding back,
O.P. perch open dark Amber flooding upper deck.
Arc on tall steps at back of cabin to flood back cloth.
Open dark Amber.
Warning for slide.
Plank ready.
Call Hook.

In the strange light thus described we see what is happening on the deck of the \emph{Jolly Roger},
which is flying the skull and crossbones and lies low in the water.
There is no need to call Hook, for he is here already,
and indeed there is not a pirate aboard who would dare to call him.
Most of them are at present carousing in the bowels of the vessel,
but on the poop Mullins is visible, in the only great‐coat on the ship,
raking with his glass the monstrous rocks within which the lagoon is cooped.
Such a look‐out is supererogatory, for the pirate craft floats immune in the horror of her name.

From Hook’s cabin at the back Starkey appears and leans over the bulwark,
silently surveying the sullen waters.
He is bare‐headed and is perhaps thinking with bitterness of his hat,
which he sometimes sees still drifting past him with the Never bird sitting on it.
The black pirate is asleep on deck,
yet even in his dreams rolling mechanically out of the way when Hook draws near.
The only sound to be heard is made by Smee at his sewing‐machine,
which lends a touch of domesticity to the night.

Hook is now leaning against the mast, now prowling the deck, the double cigar in his mouth.
With Peter surely at last removed from his path we, who know how vain a tabernacle is man,
would not be surprised to find him bellied out by the winds of his success, but it is not so;
he is still uneasy, looking long and meaninglessly at familiar objects, such as the ship’s bell or the Long Tom,
like one who may shortly be a stranger to them.
It is as if Pan’s terrible oath ‘Hook or me this time!’ had already boarded the ship.
\end{Settings}

\begin{drama}

\hookspeaks[\delivery{communing with his ego}]
How still the night is; nothing sounds alive.
Now is the hour when children in their homes are a‐bed;
their lips bright‐browned with the good‐night chocolate,
and their tongues drowsily searching for belated crumbs housed insecurely on their shining cheeks.
Compare with them the children on this boat about to walk the plank.
Split my infinitives, but ’tis my hour of triumph!
\delivery{Clinging to this fair prospect he dances a few jubilant steps, but they fall below his usual form.}
And yet some disky spirit compels me now to make my dying speech, lest when dying there may be no time for it.
All mortals envy me, yet better perhaps for Hook to have had less ambition!
O fame, fame, thou glittering bauble, what if the very——¤
\delivery{\smee, engrossed in his labours at the sewing‐machine,
tears a piece of calico with a rending sound
which makes the Solitary think for a moment that the untoward has happened to his garments.}
No little children love me.
I am told they play at Peter Pan, and that the strongest always chooses to be Peter.
They would rather be a Twin than Hook; they force the baby to be Hook.
The baby!\@ that is where the canker gnaws.
\delivery{He contemplates his industrious boatswain.}
’Tis said they find Smee lovable.
But an hour agone I found him letting the youngest of them try on his spectacles.
Pathetic Smee, the Nonconformist pirate, a happy smile upon his face because he thinks they fear him!
How can I break it to him that they think him lovable?
No, bi‐carbonate of Soda, no, not even——¤
\delivery{Another rending of the calico disturbs him,
and he has a private consultation with \starkey,
who turns him round and evidently assures him that all is well.
The peroration of his speech is nevertheless for ever lost,
as eight bells strikes and his crew pour forth in bacchanalian orgy.
From, the poop he watches their dance till it frets him beyond bearing.}
Quiet, you dogs, or I’ll cast anchor in you!
\delivery{He descends to a barrel on which there are playing‐cards,
and his crew stand waiting, as ever, like whipped curs.}
Are all the prisoners chained, so that they can’t fly away?

\jukesspeaks
Ay, ay, Captain.

\hookspeaks
Then hoist them up.

\starkeyspeaks[\delivery{raising the door of the hold}]
Tumble up, you ungentlemanly lubbers.

\direction{The terrified boys are prodded up and tossed about the deck.
\hook seems to have forgotten them; he is sitting by the barrel with his cards.}

\hookspeaks[\delivery{suddenly}]
So!
Now then, you bullies, six of you walk the plank to‐night, but I have room for two cabin‐boys.
Which of you is it to be?
\delivery{He returns to his cards.}

\tootlesspeaks[\delivery{hoping to soothe him by putting the blame on the only person, vaguely remembered,
who is always willing to act as a buffer}]
You see, sir, I don’t think my mother would like me to be a pirate.
Would your mother like you to be a pirate, Slightly?

\slightlyspeaks[\delivery{implying that otherwise it would be a pleasure to him to oblige}]
I don’t think so.
Twin, would your mother like——

\hookspeaks
Stow this gab.
\delivery{To \john}¤
You boy, you look as if you had a little pluck in you.
Didst never want to be a pirate, my hearty?

\johnspeaks[\delivery{dazzled by being singled out}]
When I was at school I——what do you think, Michael?

\michaelspeaks[\delivery{stepping into prominence}]
What would you call me if I joined?

\hookspeaks
Blackbeard Joe.

\michaelspeaks
John, what do you think?

\johnspeaks
Stop, should we still be respectful subjects of King George?

\hookspeaks
You would have to swear ‘Down with King George.’

\johnspeaks[\delivery{grandly}]
Then I refuse!

\michaelspeaks
And I refuse.

\hookspeaks
That seals your doom.
Bring up their mother.

\direction{\wendy is driven up from the hold and thrown to him.
She sees at the first glance that the deck has not been scrubbed for years.}

So, my beauty, you are to see your children walk the plank.

\wendyspeaks[\delivery{with noble calmness}]
Are they to die?

\hookspeaks
They are.
Silence all, for a mother’s last words to her children.

\wendyspeaks
These are my last words.
Dear boys, I feel that I have a message to you from your real mothers, and it is this,
‘We hope our sons will die like English gentlemen.’

\direction{The boys go on fire.}

\tootlesspeaks
I am going to do what my mother hopes.
What are you to do, Twin?

\firsttwinspeaks
What my mother hopes.
John, what are——

\hookspeaks
Tie her up!
Get the plank ready.

\direction{\wendy is roped to the mast;
but no one regards her, for all eyes are fixed upon the plank now protruding from the poop over the ship’s side.
A great change, however, occurs in the time \hook takes to raise his claw and point to this deadly engine.
No one is now looking at the plank: for the tick, tick of the crocodile is heard.
Yet it is not to bear on the crocodile that all eyes slew round, it is that they may bear on \hook.
Otherwise prisoners and captors are equally inert,
like actors in some play who have found themselves ‘on’ in a scene in which they are not personally concerned.
Even the iron claw hangs inactive, as if aware that the crocodile is not coming for it.
Affection for their captain, now cowering from view, is not what has given \hook his dominance over the crew,
but as the menacing sound draws nearer they close their eyes respectfully.

There is no crocodile.
It is \peter, who has been circling the pirate ship, ticking as he flies far more superbly than any clock.
He drops into the water and climbs aboard, warning the captives with upraised finger \parenth{but still ticking}
not for the moment to give audible expression to their natural admiration.
Only one pirate sees him, \character{Whibbles} of the eye patch, who comes up from below.
\john claps a hand on \character{Whibbles}’s mouth to stifle the groan;
four boys hold him to prevent the thud;
\peter delivers the blow, and the carrion is thrown overboard.
‘One!\@’ says \slightly, beginning to count.

\starkey is the first pirate to open his eyes.
The ship seems to him to be precisely as when he closed them.
He cannot interpret the sparkle that has come into the faces of the captives,
who are cleverly pretending to be as afraid as ever.
He little knows that the door of the dark cabin has just closed on one more boy.
Indeed it is for \hook alone he looks, and he is a little surprised to see him.}

\starkeyspeaks[\delivery{hoarsely}]
It is gone, Captain!
There is not a sound.

\direction{The tenement that is \hook heaves tumultuously and he is himself again.}

\hookspeaks[\delivery{now convinced that some fair spirit watches over him}]
Then here is to Johnny Plank——

%\speakercontinues
\begin{verse}
	Avast, belay, the English brig\\
	We took and quickly sank,\\
	And for a warning to the crew\\
	We made them walk the plank!
\end{verse}

\direction{As he sings he capers detestably along an imaginary plank
and his copy‐cats do likewise, joining in the chorus.}

%\speakercontinues
\begin{verse}
	Yo ho, yo ho, the frisky cat,\\
	You walks along it so,\\
	Till it goes down and you goes down\\
	To tooral looral lo!
\end{verse}

\direction{The brave children try to stem this monstrous torrent by breaking into the National Anthem.}

\starkeyspeaks[\delivery{paling}]
I don’t like it, messmates!

\hookspeaks
Stow that, Starkey.
Do you boys want a touch of the cat before you walk the plank?
\delivery{He is more pitiless than ever now that he believes he has a charmed life.}
Fetch the cat, Jukes; it is in the cabin.

\jukesspeaks
Ay, ay, sir.
\delivery{It is one of his commonest remarks, and is only recorded now because he never makes another.
The stage direction ‘Exit \jukes’ has in this case a special significance.
But only the children know that some one is awaiting this unfortunate in the cabin,
and \hook tramples them down as he resumes his ditty:}

%\speakercontinues
\begin{verse}
	Yo ho, yo ho, the scratching cat\\
	Its tails are nine you know,\\
	And when they’re writ upon your back,\\
	You’re fit to——
\end{verse}

\direction{The last words will ever remain a matter of conjecture,
for from the dark cabin comes a curdling screech which wails through the ship and dies away.
It is followed by a sound, almost more eerie in the circumstances,
that can only be likened to the crowing of a cock.}

\hookspeaks
What was that?

\slightlyspeaks[\delivery{solemnly}]
Two!

\direction{\cecco swings into the cabin, and in a moment returns, livid.}

\hookspeaks[\delivery{with an effort}]
What is the matter with Bill Jukes, you dog?

\ceccospeaks
The matter with him is he is dead——stabbed.

\speaker{Pirates}
Bill Jukes dead!

\ceccospeaks
The cabin is as black as a pit, but there is something terrible in there:
the thing you heard a‐crowing.

\hookspeaks[\delivery{slowly}]
Cecco, go back and fetch me out that doodle‐doo.

\ceccospeaks[\delivery{unstrung}]
No, Captain, no.
\delivery{He supplicates on his knees, but his master advances on him implacably.}

\hookspeaks[\delivery{in his most syrupy voice}]
Did you say you would go, Cecco?

\direction{\cecco goes.
All listen.
There is one screech, one crow.}

\slightlyspeaks[\delivery{as if he were a bell tolling}]
Three!

\hookspeaks
’Sdeath and oddsfish, who is to bring me out that doodle‐doo?

\direction{No one steps forward.}

\starkeyspeaks[\delivery{injudiciously}]
Wait till Cecco comes out.

\direction{The black looks of some others encourage him.}

\hookspeaks
I think I heard you volunteer, Starkey.

\starkeyspeaks[\delivery{emphatically}]
No, by thunder!

\hookspeaks[\delivery{in that syrupy voice which might be more engaging when accompanied by his flute}]
My hook thinks you did.
I wonder if it would not be advisable, Starkey, to humour the hook?

\starkeyspeaks
I’ll swing before I go in there.

\hookspeaks[\delivery{gleaming}]
Is it mutiny?
Starkey is ringleader.
Shake hands, Starkey.

\direction{\starkey recoils from the claw.
It follows him till he leaps overboard.}

%\speakercontinues
Did any other gentleman say mutiny?

\direction{They indicate that they did not even know the late \starkey.}

\slightlyspeaks
Four!

\hookspeaks
I will bring out that doodle‐doo myself.

\direction{He raises a blunderbuss but casts it from him with a menacing gesture
which means that he has more faith in the claw.
With a lighted lantern in his hand he enters the cabin.
Not a sound is to be heard now on the ship, unless it be \slightly wetting his lips to say ‘Five.’
\hook staggers out.}

\hookspeaks[\delivery{unsteadily}]
Something blew out the light.

\mullinsspeaks[\delivery{with dark meaning}]
Some—thing?

\noodlerspeaks
What of Cecco?

\hookspeaks
He is as dead as Jukes.

\direction{They are superstitious like all sailors,
and \mullins has planted a dire conception in their minds.}

\cooksonspeaks.
They do say as the surest sign a ship’s accurst is when there is one aboard more than can be accounted for.

\noodlerspeaks
I’ve heard he allus boards the pirate craft at last.
\delivery{With dreadful significance}¤
Has he a tail, Captain?

\mullinsspeaks.
They say that when he comes it is in the likeness of the wickedest man aboard.

\cooksonspeaks[\delivery{clinching it}]
Has he a hook, Captain?

\direction{Knives and pistols come to hand, and there is a general cry ‘The ship is doomed!’
But it is not his dogs that can frighten \character{Jas Hook}
Hearing something like a cheer from the boys he wheels round, and his face brings them to their knees.}

\hookspeaks
So you like it, do you!
By Caius and Balbus, bullies, here is a notion: open the cabin door and drive them in.
Let them fight the doodle‐doo for their lives.
If they kill him we are so much the better; if he kills them we are none the worse.

\direction{This masterly stroke restores their confidence;
and the boys, affecting fear, are driven into the cabin.
Desperadoes though the pirates are, some of them have been boys themselves,
and all turn their backs to the cabin and listen,
with arms outstretched to it as if to ward off the horrors that are being enacted there.

Relieved by Peter of their manacles, and armed with such weapons as they can lay their hands on,
the boys steal out softly as snowflakes, and under their captain’s hushed order find hiding‐places on the poop.
He releases \wendy; and now it would be easy for them all to fly away,
but it is to be \hook or him this time.
He signs to her to join the others,
and with awful grimness folding her cloak around him, the hood over his head,
he takes her place by the mast, and crows.}

\mullinsspeaks.
The doodle‐doo has killed them all!

\speaker{Several}
The ship’s bewitched.

\direction{They are snapping at \hook again.}

\hookspeaks
I’ve thought it out, lads; there is a Jonah aboard.

\speaker{Several}[\delivery{advancing upon him}]
Ay, a man with a hook.

\direction{If he were to withdraw one step their knives would be in him, but he does not flinch.}

\hookspeaks[\delivery{temporising}]
No, lads, no, it is the girl.
Never was luck on a pirate ship wi’ a woman aboard.
We’ll right the ship when she has gone.

\mullinsspeaks[\delivery{lowering his cutlass}]
It’s worth trying.

\hookspeaks
Throw the girl overboard.

\mullinsspeaks[\delivery{jeering}]
There is none can save you now, missy.

\peterspeaks
There is one.

\mullinsspeaks.
Who is that?

\peterspeaks[\delivery{casting off the cloak}]
Peter Pan, the avenger!

\direction{He continues standing there to let the effect sink in.}

\hookspeaks[\delivery{throwing out a suggestion}]
Cleave him to the brisket.

\direction{But he has a sinking that this boy has no brisket.}

\noodlerspeaks
The ship’s accurst!

\peterspeaks
Down, boys, and at them!

\direction{The boys leap from their concealment and the clash of arms resounds through the vessel.
Man to man the pirates are the stronger, but they are unnerved by the suddenness of the onslaught and they scatter,
thus enabling their opponents to hunt in couples and choose their quarry.
Some are hurled into the lagoon; others are dragged from dark recesses.
There is no boy whose weapon is not reeking save \slightly, who runs about with a lantern, counting, ever counting.}

\wendyspeaks[\delivery{meeting \michael in a moment’s lull}]
Oh, Michael, stay with me, protect me!

\michaelspeaks[\delivery{reeling}]
Wendy, I’ve killed a pirate!

\wendyspeaks
It’s awful, awful.

\michaelspeaks
No, it isn’t, I like it, I like it.

\direction{He casts himself into the group of boys who are encircling \hook.
Again and again they close upon him and again and again he hews a clear space.}

\hookspeaks
Back, back, you mice.
It’s Hook; do you like him?
\delivery{He lifts up \michael with his claw and uses him as a buckler.
A terrible voice breaks in.}

\peterspeaks
Put up your swords, boys.
This man is mine.

\direction{\hook shakes \michael off his claw as if he were a drop of water,
and these two antagonists face each other for their final bout.
They measure swords at arms’ length, make a sweeping motion with them,
and bringing the points to the deck rest their hands upon the hilts.}

\hookspeaks[\delivery{with curling lip}]
So, Pan, this is all your doing!

\peterspeaks
Ay, Jas Hook, it is all my doing.

\hookspeaks
Proud and insolent youth, prepare to meet thy doom.

\peterspeaks
Dark and sinister man, have at thee.

\direction{Some say that he had to ask \tootles whether the word was sinister or canister.

\hook or \peter this time!
They fall to without another word.
\peter is a rare swordsman, and parries with dazzling rapidity,
sometimes before the other can make his stroke.
\hook, if not quite so nimble in wrist play, has the advantage of a yard or two in reach,
but though they close he cannot give the quietus with his claw, which seems to find nothing to tear at.
He does not, especially in the most heated moments, quite see \peter,
who to his eyes, now blurred or opened clearly for the first time,
is less like a boy than a mote of dust dancing in the sun.
By some impalpable stroke \hook’s sword is whipped from his grasp,
and when he stoops to raise it a little foot is on its blade.
There is no deep gash on \hook, but he is suffering torment as from innumerable jags.}

\speaker{Boys}[\delivery{exulting}]
Now, Peter, now!

\direction{\peter raises the sword by its blade,
and with an inclination of the head that is perhaps slightly overdone, presents the hilt to his enemy.}

\hookspeaks
’Tis some fiend fighting me!
Pan, who and what art thou?

\direction{The children listen eagerly for the answer, none quite so eagerly as \wendy.}

\peterspeaks[\delivery{at a venture}]
I’m youth, I’m joy, I’m a little bird that has broken out of the egg.

\hookspeaks
To ’t again!

\direction{He has now a damp feeling that this boy is the weapon which is to strike him from the lists of man;
but the grandeur of his mind still holds and, true to the traditions of his flag, he fights on like a human flail.
\peter flutters round and through and over these gyrations as if the wind of them blew him out of the danger zone,
and again and again he darts in and jags.}

\hookspeaks[\delivery{stung to madness}]
I’ll fire the powder magazine.
\delivery{He disappears they know not where.}

\speaker{Children}
Peter, save us!

\direction{\peter, alas, goes the wrong way and \hook returns.}

\hookspeaks[\delivery{sitting on the hold with gloomy satisfaction}]
In two minutes the ship will be blown to pieces.

\direction{They cast themselves before him in entreaty.}

\speaker{Children}
Mercy, mercy!

\hookspeaks
Back, you pewling spawn.
I’ll show you now the road to dusty death.
A holocaust of children, there is something grand in the idea!

{\peter appears with the smoking bomb in his hand, and tosses it overboard.
\hook has not really had much hope,
and he rushes at his other persecutors with his head down like some exasperated bull in the ring;
but with bantering cries they easily elude him by flying among the rigging.

Where is \peter?
The incredible boy has apparently forgotten the recent doings,
and is sitting on a barrel playing upon his pipes.
This may surprise others but does not surprise \hook.
Lifting a blunderbuss he strikes forlornly
not at the boy but at the barrel, which is hurled across the deck.
\peter remains sitting in the air still playing upon his pipes.
At this sight the great heart of \hook breaks.
That not wholly unheroic figure climbs the bulwarks murmuring ‘\emph{Floreat Etona},’
and prostrates himself into the water, where the crocodile is waiting for him open‐mouthed.
\hook knows the purpose of this yawning cavity,
but after what he has gone through he enters it like one greeting a friend.

The curtain rises to show \peter a very Napoleon on his ship.
It must not rise again lest we see him on the poop in \hook’s hat and cigars,
and with a small iron claw.}

\end{drama}

\endinput
