% !TEX program = pdflatex
% !TEX encoding = UTF-8
% !TEX spellcheck = en_GB
% !TEX root = peter-pan.tex

\chapter{When Wendy Grew Up}

As you look at Wendy,
you may see her hair becoming white,
and her figure little again,
for all this happened long ago.
Jane is now a common grown-up,
with a daughter called Margaret;
and every spring cleaning time,
except when he forgets,
Peter comes for Margaret and takes her to the Neverland,
where she tells him stories about himself,
to which he listens eagerly.
When Margaret grows up she will have a daughter,
who is to be Peter’s mother in turn;
and thus it will go on,
so long as children are gay and innocent and heartless.

\endinput


Chapter 17 WHEN WENDY GREW UP


I hope you want to know what became of the other boys. They were waiting
below to give Wendy time to explain about them; and when they had counted
five hundred they went up. They went up by the stair, because they thought
this would make a better impression. They stood in a row in front of Mrs.
Darling, with their hats off, and wishing they were not wearing their
pirate clothes. They said nothing, but their eyes asked her to have them.
They ought to have looked at Mr. Darling also, but they forgot about him.


Of course Mrs. Darling said at once that she would have them; but Mr.
Darling was curiously depressed, and they saw that he considered six a
rather large number.


"I must say," he said to Wendy, "that you don't do things by halves," a
grudging remark which the twins thought was pointed at them.


The first twin was the proud one, and he asked, flushing, "Do you think we
should be too much of a handful, sir? Because, if so, we can go away."


"Father!" Wendy cried, shocked; but still the cloud was on him. He knew he
was behaving unworthily, but he could not help it.


"We could lie doubled up," said Nibs.


"I always cut their hair myself," said Wendy.


"George!" Mrs. Darling exclaimed, pained to see her dear one showing
himself in such an unfavourable light.


Then he burst into tears, and the truth came out. He was as glad to have
them as she was, he said, but he thought they should have asked his
consent as well as hers, instead of treating him as a cypher [zero] in his
own house.


"I don't think he is a cypher," Tootles cried instantly. "Do you think he
is a cypher, Curly?"


"No, I don't. Do you think he is a cypher, Slightly?"


"Rather not. Twin, what do you think?"


It turned out that not one of them thought him a cypher; and he was
absurdly gratified, and said he would find space for them all in the
drawing-room if they fitted in.


"We'll fit in, sir," they assured him.


"Then follow the leader," he cried gaily. "Mind you, I am not sure that we
have a drawing-room, but we pretend we have, and it's all the same. Hoop
la!"


He went off dancing through the house, and they all cried "Hoop la!" and
danced after him, searching for the drawing-room; and I forget whether
they found it, but at any rate they found corners, and they all fitted in.


As for Peter, he saw Wendy once again before he flew away. He did not
exactly come to the window, but he brushed against it in passing so that
she could open it if she liked and call to him. That is what she did.


"Hullo, Wendy, good-bye," he said.


"Oh dear, are you going away?"


"Yes."


"You don't feel, Peter," she said falteringly, "that you would like to say
anything to my parents about a very sweet subject?"


"No."


"About me, Peter?"


"No."


Mrs. Darling came to the window, for at present she was keeping a sharp
eye on Wendy. She told Peter that she had adopted all the other boys, and
would like to adopt him also.


"Would you send me to school?" he inquired craftily.


"Yes."


"And then to an office?"


"I suppose so."


"Soon I would be a man?"


"Very soon."


"I don't want to go to school and learn solemn things," he told her
passionately. "I don't want to be a man. O Wendy's mother, if I was to
wake up and feel there was a beard!"


"Peter," said Wendy the comforter, "I should love you in a beard;" and
Mrs. Darling stretched out her arms to him, but he repulsed her.


"Keep back, lady, no one is going to catch me and make me a man."


"But where are you going to live?"


"With Tink in the house we built for Wendy. The fairies are to put it high
up among the tree tops where they sleep at nights."


"How lovely," cried Wendy so longingly that Mrs. Darling tightened her
grip.


"I thought all the fairies were dead," Mrs. Darling said.


"There are always a lot of young ones," explained Wendy, who was now quite
an authority, "because you see when a new baby laughs for the first time a
new fairy is born, and as there are always new babies there are always new
fairies. They live in nests on the tops of trees; and the mauve ones are
boys and the white ones are girls, and the blue ones are just little
sillies who are not sure what they are."


"I shall have such fun," said Peter, with eye on Wendy.


"It will be rather lonely in the evening," she said, "sitting by the
fire."


"I shall have Tink."


"Tink can't go a twentieth part of the way round," she reminded him a
little tartly.


"Sneaky tell-tale!" Tink called out from somewhere round the corner.


"It doesn't matter," Peter said.


"O Peter, you know it matters."


"Well, then, come with me to the little house."


"May I, mummy?"


"Certainly not. I have got you home again, and I mean to keep you."


"But he does so need a mother."


"So do you, my love."


"Oh, all right," Peter said, as if he had asked her from politeness
merely; but Mrs. Darling saw his mouth twitch, and she made this handsome
offer: to let Wendy go to him for a week every year to do his spring
cleaning. Wendy would have preferred a more permanent arrangement; and it
seemed to her that spring would be long in coming; but this promise sent
Peter away quite gay again. He had no sense of time, and was so full of
adventures that all I have told you about him is only a halfpenny-worth of
them. I suppose it was because Wendy knew this that her last words to him
were these rather plaintive ones:


"You won't forget me, Peter, will you, before spring cleaning time comes?"


Of course Peter promised; and then he flew away. He took Mrs. Darling's
kiss with him. The kiss that had been for no one else, Peter took quite
easily. Funny. But she seemed satisfied.


Of course all the boys went to school; and most of them got into Class
III, but Slightly was put first into Class IV and then into Class V. Class
I is the top class. Before they had attended school a week they saw what
goats they had been not to remain on the island; but it was too late now,
and soon they settled down to being as ordinary as you or me or Jenkins
minor [the younger Jenkins]. It is sad to have to say that the power to
fly gradually left them. At first Nana tied their feet to the bed-posts so
that they should not fly away in the night; and one of their diversions by
day was to pretend to fall off buses [the English double-deckers]; but by
and by they ceased to tug at their bonds in bed, and found that they hurt
themselves when they let go of the bus. In time they could not even fly
after their hats. Want of practice, they called it; but what it really
meant was that they no longer believed.


Michael believed longer than the other boys, though they jeered at him; so
he was with Wendy when Peter came for her at the end of the first year.
She flew away with Peter in the frock she had woven from leaves and
berries in the Neverland, and her one fear was that he might notice how
short it had become; but he never noticed, he had so much to say about
himself.


She had looked forward to thrilling talks with him about old times, but
new adventures had crowded the old ones from his mind.


"Who is Captain Hook?" he asked with interest when she spoke of the arch
enemy.


"Don't you remember," she asked, amazed, "how you killed him and saved all
our lives?"


"I forget them after I kill them," he replied carelessly.


When she expressed a doubtful hope that Tinker Bell would be glad to see
her he said, "Who is Tinker Bell?"


"O Peter," she said, shocked; but even when she explained he could not
remember.


"There are such a lot of them," he said. "I expect she is no more."


I expect he was right, for fairies don't live long, but they are so little
that a short time seems a good while to them.


Wendy was pained too to find that the past year was but as yesterday to
Peter; it had seemed such a long year of waiting to her. But he was
exactly as fascinating as ever, and they had a lovely spring cleaning in
the little house on the tree tops.


Next year he did not come for her. She waited in a new frock because the
old one simply would not meet; but he never came.


"Perhaps he is ill," Michael said.


"You know he is never ill."


Michael came close to her and whispered, with a shiver, "Perhaps there is
no such person, Wendy!" and then Wendy would have cried if Michael had not
been crying.


Peter came next spring cleaning; and the strange thing was that he never
knew he had missed a year.


That was the last time the girl Wendy ever saw him. For a little longer
she tried for his sake not to have growing pains; and she felt she was
untrue to him when she got a prize for general knowledge. But the years
came and went without bringing the careless boy; and when they met again
Wendy was a married woman, and Peter was no more to her than a little dust
in the box in which she had kept her toys. Wendy was grown up. You need
not be sorry for her. She was one of the kind that likes to grow up. In
the end she grew up of her own free will a day quicker than other girls.


All the boys were grown up and done for by this time; so it is scarcely
worth while saying anything more about them. You may see the twins and
Nibs and Curly any day going to an office, each carrying a little bag and
an umbrella. Michael is an engine-driver [train engineer]. Slightly
married a lady of title, and so he became a lord. You see that judge in a
wig coming out at the iron door? That used to be Tootles. The bearded man
who doesn't know any story to tell his children was once John.


Wendy was married in white with a pink sash. It is strange to think that
Peter did not alight in the church and forbid the banns [formal
announcement of a marriage].


Years rolled on again, and Wendy had a daughter. This ought not to be
written in ink but in a golden splash.


She was called Jane, and always had an odd inquiring look, as if from the
moment she arrived on the mainland she wanted to ask questions. When she
was old enough to ask them they were mostly about Peter Pan. She loved to
hear of Peter, and Wendy told her all she could remember in the very
nursery from which the famous flight had taken place. It was Jane's
nursery now, for her father had bought it at the three per cents [mortgage
rate] from Wendy's father, who was no longer fond of stairs. Mrs. Darling
was now dead and forgotten.


There were only two beds in the nursery now, Jane's and her nurse's; and
there was no kennel, for Nana also had passed away. She died of old age,
and at the end she had been rather difficult to get on with; being very
firmly convinced that no one knew how to look after children except
herself.


Once a week Jane's nurse had her evening off; and then it was Wendy's part
to put Jane to bed. That was the time for stories. It was Jane's invention
to raise the sheet over her mother's head and her own, thus making a tent,
and in the awful darkness to whisper:


"What do we see now?"


"I don't think I see anything to-night," says Wendy, with a feeling that
if Nana were here she would object to further conversation.


"Yes, you do," says Jane, "you see when you were a little girl."


"That is a long time ago, sweetheart," says Wendy. "Ah me, how time
flies!"


"Does it fly," asks the artful child, "the way you flew when you were a
little girl?"


"The way I flew? Do you know, Jane, I sometimes wonder whether I ever did
really fly."


"Yes, you did."


"The dear old days when I could fly!"


"Why can't you fly now, mother?"


"Because I am grown up, dearest. When people grow up they forget the way."


"Why do they forget the way?"


"Because they are no longer gay and innocent and heartless. It is only the
gay and innocent and heartless who can fly."


"What is gay and innocent and heartless? I do wish I were gay and innocent
and heartless."


Or perhaps Wendy admits she does see something.


"I do believe," she says, "that it is this nursery."


"I do believe it is," says Jane. "Go on."


They are now embarked on the great adventure of the night when Peter flew
in looking for his shadow.


"The foolish fellow," says Wendy, "tried to stick it on with soap, and
when he could not he cried, and that woke me, and I sewed it on for him."


"You have missed a bit," interrupts Jane, who now knows the story better
than her mother. "When you saw him sitting on the floor crying, what did
you say?"


"I sat up in bed and I said, 'Boy, why are you crying?'"


"Yes, that was it," says Jane, with a big breath.


"And then he flew us all away to the Neverland and the fairies and the
pirates and the redskins and the mermaid's lagoon, and the home under the
ground, and the little house."


"Yes! which did you like best of all?"


"I think I liked the home under the ground best of all."


"Yes, so do I. What was the last thing Peter ever said to you?"


"The last thing he ever said to me was, 'Just always be waiting for me,
and then some night you will hear me crowing.'"


"Yes."


"But, alas, he forgot all about me," Wendy said it with a smile. She was
as grown up as that.


"What did his crow sound like?" Jane asked one evening.


"It was like this," Wendy said, trying to imitate Peter's crow.


"No, it wasn't," Jane said gravely, "it was like this;" and she did it
ever so much better than her mother.


Wendy was a little startled. "My darling, how can you know?"


"I often hear it when I am sleeping," Jane said.


"Ah yes, many girls hear it when they are sleeping, but I was the only one
who heard it awake."


"Lucky you," said Jane.


And then one night came the tragedy. It was the spring of the year, and
the story had been told for the night, and Jane was now asleep in her bed.
Wendy was sitting on the floor, very close to the fire, so as to see to
darn, for there was no other light in the nursery; and while she sat
darning she heard a crow. Then the window blew open as of old, and Peter
dropped in on the floor.


He was exactly the same as ever, and Wendy saw at once that he still had
all his first teeth.


He was a little boy, and she was grown up. She huddled by the fire not
daring to move, helpless and guilty, a big woman.


"Hullo, Wendy," he said, not noticing any difference, for he was thinking
chiefly of himself; and in the dim light her white dress might have been
the nightgown in which he had seen her first.


"Hullo, Peter," she replied faintly, squeezing herself as small as
possible. Something inside her was crying "Woman, Woman, let go of me."


"Hullo, where is John?" he asked, suddenly missing the third bed.


"John is not here now," she gasped.


"Is Michael asleep?" he asked, with a careless glance at Jane.


"Yes," she answered; and now she felt that she was untrue to Jane as well
as to Peter.


"That is not Michael," she said quickly, lest a judgment should fall on
her.


Peter looked. "Hullo, is it a new one?"


"Yes."


"Boy or girl?"


"Girl."


Now surely he would understand; but not a bit of it.


"Peter," she said, faltering, "are you expecting me to fly away with you?"


"Of course; that is why I have come." He added a little sternly, "Have you
forgotten that this is spring cleaning time?"


She knew it was useless to say that he had let many spring cleaning times
pass.


"I can't come," she said apologetically, "I have forgotten how to fly."


"I'll soon teach you again."


"O Peter, don't waste the fairy dust on me."


She had risen; and now at last a fear assailed him. "What is it?" he
cried, shrinking.


"I will turn up the light," she said, "and then you can see for yourself."


For almost the only time in his life that I know of, Peter was afraid.
"Don't turn up the light," he cried.


She let her hands play in the hair of the tragic boy. She was not a little
girl heart-broken about him; she was a grown woman smiling at it all, but
they were wet eyed smiles.


Then she turned up the light, and Peter saw. He gave a cry of pain; and
when the tall beautiful creature stooped to lift him in her arms he drew
back sharply.


"What is it?" he cried again.


She had to tell him.


"I am old, Peter. I am ever so much more than twenty. I grew up long ago."


"You promised not to!"


"I couldn't help it. I am a married woman, Peter."


"No, you're not."


"Yes, and the little girl in the bed is my baby."


"No, she's not."


But he supposed she was; and he took a step towards the sleeping child
with his dagger upraised. Of course he did not strike. He sat down on the
floor instead and sobbed; and Wendy did not know how to comfort him,
though she could have done it so easily once. She was only a woman now,
and she ran out of the room to try to think.


Peter continued to cry, and soon his sobs woke Jane. She sat up in bed,
and was interested at once.


"Boy," she said, "why are you crying?"


Peter rose and bowed to her, and she bowed to him from the bed.


"Hullo," he said.


"Hullo," said Jane.


"My name is Peter Pan," he told her.


"Yes, I know."


"I came back for my mother," he explained, "to take her to the Neverland."


"Yes, I know," Jane said, "I have been waiting for you."


When Wendy returned diffidently she found Peter sitting on the bed-post
crowing gloriously, while Jane in her nighty was flying round the room in
solemn ecstasy.


"She is my mother," Peter explained; and Jane descended and stood by his
side, with the look in her face that he liked to see on ladies when they
gazed at him.


"He does so need a mother," Jane said.


"Yes, I know." Wendy admitted rather forlornly; "no one knows it so well
as I."


"Good-bye," said Peter to Wendy; and he rose in the air, and the shameless
Jane rose with him; it was already her easiest way of moving about.


Wendy rushed to the window.


"No, no," she cried.


"It is just for spring cleaning time," Jane said, "he wants me always to
do his spring cleaning."


"If only I could go with you," Wendy sighed.


"You see you can't fly," said Jane.


Of course in the end Wendy let them fly away together. Our last glimpse of
her shows her at the window, watching them receding into the sky until
they were as small as stars.


As you look at Wendy, you may see her hair becoming white, and her figure
little again, for all this happened long ago. Jane is now a common
grown-up, with a daughter called Margaret; and every spring cleaning time,
except when he forgets, Peter comes for Margaret and takes her to the
Neverland, where she tells him stories about himself, to which he listens
eagerly. When Margaret grows up she will have a daughter, who is to be
Peter's mother in turn; and thus it will go on, so long as children are
gay and innocent and heartless.

