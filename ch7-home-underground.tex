% !TEX program = pdflatex
% !TEX encoding = UTF-8
% !TEX spellcheck = en_GB
% !TEX root = peter-pan.tex

\chapter{The Home under the Ground}

One of the first things Peter did next day was to measure Wendy and John and Michael for hollow trees.
Hook, you remember, had sneered at the boys for thinking they needed a tree apiece,
but this was ignorance, for unless your tree fitted you it was difficult to go up and down,
and no two of the boys were quite the same size.
Once you fitted, you drew in your breath at the top, and down you went at exactly the right speed,
while to ascend you drew in and let out alternately, and so wriggled up.
Of course, when you have mastered the action you are able to do these things without thinking of them,
and nothing can be more graceful.

But you simply must fit, and Peter measures you for your tree as carefully as for a suit of clothes:
the only difference being that the clothes are made to fit you, while you have to be made to fit the tree.
Usually it is done quite easily, as by your wearing too many garments or too few,
but if you are bumpy in awkward places or the only available tree is an odd shape,
Peter does some things to you, and after that you fit.
Once you fit, great care must be taken to go on fitting,
and this, as Wendy was to discover to her delight, keeps a whole family in perfect condition.

Wendy and Michael fitted their trees at the first try, but John had to be altered a little.

After a few days’ practice they could go up and down as gaily as buckets in a well.
And how ardently they grew to love their home under the ground;
especially Wendy.
It consisted of one large room, as all houses should do,
with a floor in which you could dig if you wanted to go fishing,
and in this floor grew stout mushrooms of a charming colour, which were used as stools.
A Never tree tried hard to grow in the centre of the room,
but every morning they sawed the trunk through, level with the floor.
By tea‐time it was always about two feet high, and then they put a door on top of it,
the whole thus becoming a table;
as soon as they cleared away, they sawed off the trunk again, and thus there was more room to play.
There was an enormous fireplace which was in almost any part of the room where you cared to light it,
and across this Wendy stretched strings, made of fibre, from which she suspended her washing.
The bed was tilted against the wall by day, and let down at 6:30, when it filled nearly half the room;
and all the boys slept in it, except Michael, lying like sardines in a tin.
There was a strict rule against turning round until one gave the signal, when all turned at once.
Michael should have used it also, but Wendy would have a baby, and he was the littlest,
and you know what women are, and the short and long of it is that he was hung up in a basket.

It was rough and simple,
and not unlike what baby bears would have made of an underground house in the same circumstances.
But there was one recess in the wall, no larger than a bird‐cage,
which was the private apartment of Tinker Bell.
It could be shut off from the rest of the house by a tiny curtain,
which Tink, who was most fastidious, always kept drawn when dressing or undressing.
No woman, however large, could have had a more exquisite \emph{boudoir} and bed‐chamber combined.
The couch, as she always called it, was a genuine Queen Mab, with club legs;
and she varied the bedspreads according to what fruit‐blossom was in season.
Her mirror was a Puss‐in‐Boots, of which there are now only three, unchipped, known to fairy dealers;
the washstand was Pie‐crust and reversible,
the chest of drawers an authentic Charming the Sixth,
and the carpet and rugs the best (the early) period of Margery and Robin.
There was a chandelier from Tiddlywinks for the look of the thing,
but of course she lit the residence herself.
Tink was very contemptuous of the rest of the house, as indeed was perhaps inevitable,
and her chamber, though beautiful, looked rather conceited,
having the appearance of a nose permanently turned up.

I suppose it was all especially entrancing to Wendy,
because those rampagious boys of hers gave her so much to do.
Really there were whole weeks when, except perhaps with a stocking in the evening, she was never above ground.
The cooking, I can tell you, kept her nose to the pot,
and even if there was nothing in it, even if there was no pot,
she had to keep watching that it came aboil just the same.
You never exactly knew whether there would be a real meal or just a make‐believe,
it all depended upon Peter’s whim:
he could eat, really eat, if it was part of a game,
but he could not stodge just to feel stodgy, which is what most children like better than anything else;
the next best thing being to talk about it.
Make‐believe was so real to him that during a meal of it you could see him getting rounder.
Of course it was trying, but you simply had to follow his lead,
and if you could prove to him that you were getting loose for your tree he let you stodge.

Wendy’s favourite time for sewing and darning was after they had all gone to bed.
Then, as she expressed it, she had a breathing time for herself;
and she occupied it in making new things for them, and putting double pieces on the knees,
for they were all most frightfully hard on their knees.

When she sat down to a basketful of their stockings, every heel with a hole in it,
she would fling up her arms and exclaim, “Oh dear, I am sure I sometimes think spinsters are to be envied!”

Her face beamed when she exclaimed this.

You remember about her pet wolf.
Well, it very soon discovered that she had come to the island and it found her out,
and they just ran into each other’s arms.
After that it followed her about everywhere.

As time wore on did she think much about the beloved parents she had left behind her?
This is a difficult question,
because it is quite impossible to say how time does wear on in the Neverland,
where it is calculated by moons and suns, and there are ever so many more of them than on the mainland.
But I am afraid that Wendy did not really worry about her father and mother;
she was absolutely confident that they would always keep the window open for her to fly back by,
and this gave her complete ease of mind.
What did disturb her at times was that John remembered his parents vaguely only,
as people he had once known, while Michael was quite willing to believe that she was really his mother.
These things scared her a little,
and nobly anxious to do her duty, she tried to fix the old life in their minds
by setting them examination papers on it, as like as possible to the ones she used to do at school.
The other boys thought this awfully interesting, and insisted on joining,
and they made slates for themselves, and sat round the table,
writing and thinking hard about the questions she had written on another slate and passed round.
They were the most ordinary questions—%
“What was the colour of Mother’s eyes?
Which was taller, Father or Mother?
Was Mother blonde or brunette?
Answer all three questions if possible.”
“(A) Write an essay of not less than 40 words on How I spent my last Holidays,
or The Characters of Father and Mother compared.
Only one of these to be attempted.”
Or “(1) Describe Mother’s laugh;
(2) Describe Father’s laugh;
(3) Describe Mother’s Party Dress;
(4) Describe the Kennel and its Inmate.”

They were just everyday questions like these,
and when you could not answer them you were told to make a cross;
and it was really dreadful what a number of crosses even John made.
Of course the only boy who replied to every question was Slightly,
and no one could have been more hopeful of coming out first,
but his answers were perfectly ridiculous, and he really came out last:
a melancholy thing.

Peter did not compete.
For one thing he despised all mothers except Wendy,
and for another he was the only boy on the island who could neither write nor spell;
not the smallest word.
He was above all that sort of thing.

By the way, the questions were all written in the past tense.
What was the colour of Mother’s eyes, and so on.
Wendy, you see, had been forgetting, too.

Adventures, of course, as we shall see, were of daily occurrence;
but about this time Peter invented, with Wendy’s help, a new game that fascinated him enormously,
until he suddenly had no more interest in it,
which, as you have been told, was what always happened with his games.
It consisted in pretending not to have adventures,
in doing the sort of thing John and Michael had been doing all their lives,
sitting on stools flinging balls in the air, pushing each other,
going out for walks and coming back without having killed so much as a grizzly.
To see Peter doing nothing on a stool was a great sight;
he could not help looking solemn at such times, to sit still seemed to him such a comic thing to do.
He boasted that he had gone walking for the good of his health.
For several suns these were the most novel of all adventures to him;
and John and Michael had to pretend to be delighted also;
otherwise he would have treated them severely.

He often went out alone,
and when he came back you were never absolutely certain whether he had had an adventure or not.
He might have forgotten it so completely that he said nothing about it;
and then when you went out you found the body;
and, on the other hand, he might say a great deal about it, and yet you could not find the body.
Sometimes he came home with his head bandaged,
and then Wendy cooed over him and bathed it in lukewarm water, while he told a dazzling tale.
But she was never quite sure, you know.
There were, however, many adventures which she knew to be true because she was in them herself,
and there were still more that were at least partly true,
for the other boys were in them and said they were wholly true.
To describe them all would require a book as large as an English‐Latin, Latin‐English Dictionary,
and the most we can do is to give one as a specimen of an average hour on the island.
The difficulty is which one to choose.
Should we take the brush with the redskins at Slightly Gulch?
It was a sanguinary affair,
and especially interesting as showing one of Peter’s peculiarities,
which was that in the middle of a fight he would suddenly change sides.
At the Gulch, when victory was still in the balance,
sometimes leaning this way and sometimes that, he called out, “I’m redskin to‐day;
what are you, Tootles?”
And Tootles answered, “Redskin;
what are you, Nibs?\@” and Nibs said, “Redskin;
what are you Twin?\@” and so on;
and they were all redskins;
and of course this would have ended the fight
had not the real redskins fascinated by Peter’s methods,
agreed to be lost boys for that once, and so at it they all went again, more fiercely than ever.

The extraordinary upshot of this adventure was—%
but we have not decided yet that this is the adventure we are to narrate.
Perhaps a better one would be the night attack by the redskins on the house under the ground,
when several of them stuck in the hollow trees and had to be pulled out like corks.
Or we might tell how Peter saved Tiger Lily’s life in the Mermaids’ Lagoon, and so made her his ally.

Or we could tell of that cake the pirates cooked so that the boys might eat it and perish;
and how they placed it in one cunning spot after another;
but always Wendy snatched it from the hands of her children,
so that in time it lost its succulence, and became as hard as a stone, and was used as a missile,
and Hook fell over it in the dark.

Or suppose we tell of the birds that were Peter’s friends,
particularly of the Never bird that built in a tree overhanging the lagoon,
and how the nest fell into the water, and still the bird sat on her eggs,
and Peter gave orders that she was not to be disturbed.
That is a pretty story, and the end shows how grateful a bird can be;
but if we tell it we must also tell the whole adventure of the lagoon,
which would of course be telling two adventures rather than just one.
A shorter adventure, and quite as exciting, was Tinker Bell’s attempt, with the help of some street fairies,
to have the sleeping Wendy conveyed on a great floating leaf to the mainland.
Fortunately the leaf gave way and Wendy woke, thinking it was bath‐time, and swam back.
Or again, we might choose Peter’s defiance of the lions,
when he drew a circle round him on the ground with an arrow and dared them to cross it;
and though he waited for hours, with the other boys and Wendy looking on breathlessly from trees,
not one of them dared to accept his challenge.

Which of these adventures shall we choose?
The best way will be to toss for it.

I have tossed, and the lagoon has won.
This almost makes one wish that the gulch or the cake or Tink’s leaf had won.
Of course I could do it again, and make it best out of three;
however, perhaps fairest to stick to the lagoon.

\endinput
