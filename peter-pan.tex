% !TEX program = pdflatex
% !TEX encoding = UTF-8
% !TEX spellcheck = en_GB

\documentclass[draft, 10pt]{memoir}

\usepackage[utf8]{inputenc}
\usepackage[T1]{fontenc}
\usepackage[light,onlyrm,oldstylenums,largesmallcaps,noamsmath,notextcomp]{kpfonts}

\usepackage[british]{babel}

\usepackage{dramatist}

\usepackage{newunicodechar}

% See ‹http://tex.stackexchange.com/questions/59390/spacefactor-unicode-curly-quotes/2966›
\AtBeginDocument{
  \sfcode\csname\encodingdefault\string\textquotedblright\endcsname=0
}

% Dramatist uses xspace, and we’re using Unicode apostrophes
\xspaceaddexceptions{’}

% Force end-of sentence space; converse of \@
\newunicodechar{¤}{\spacefactor3000{}}

% Use U+2020 HYPHEN to allow hyphenation within compound words
\usepackage{hyphenat}
\newunicodechar{‐}{\hyp}

% Allow a bit of stretch or line breaking
% TODO: This ought to become part of the expansion for em dashes,
%       if only I can figure out how to handle consecutive dashes.
%       My own xpeek may be useful here; details left till later.
\newunicodechar{·}{\hspace{0pt plus 2pt}}

% Page Layout
\usepackage{calc}

\setstocksize{9in}{6in}
\settrimmedsize{\stockheight}{\stockwidth}{*}

\setlrmarginsandblock{1in}{1in}{*}
\setulmarginsandblock{1in + \onelineskip}{*}{1}

\setheadfoot{\onelineskip}{2\onelineskip}
\setheaderspaces{*}{\onelineskip}{*}

\checkandfixthelayout[lines]

% Book metadata
\title{Peter Pan}
\author{J. M. Barrie}
\newcommand{\thebooktitle}{Peter and Wendy}
\newcommand{\theplaytitle}{The Boy Who Wouldn’t Grow Up}
% TODO: Put this into the PDF metadata as well; see createspace.sty for tips

% Chapter headings
% TODO: Find a way to make act and scene headings look similar
\makechapterstyle{pan}{%
	\chapterstyle{thatcher}
	\renewcommand*{\chapnumfont}{\normalfont\otherscshape\MakeTextLowercase}
	\renewcommand*{\chaptitlefont}{\normalfont}
}
\chapterstyle{pan}

% At the books’ end
\newcommand{\theend}{\fancybreak{\scshape{The End}}}

% Use footnote symbols
\renewcommand*{\thefootnote}{\fnsymbol{footnote}}

% For use on the non-copyright page
\usepackage{ccicons}

\begin{document}
\frontmatter

% !TEX program = pdflatex
% !TEX encoding = UTF-8
% !TEX spellcheck = en_GB
% !TEX root = peter-pan.tex

\pagestyle{empty}

% half-title page

\vspace*{\fill}
\begin{center}
\HUGE \thetitle
\end{center}
\vspace*{\fill}
\cleardoublepage

% title page

\vspace*{\fill}
\begin{center}
\Large\theauthor’s\par
\HUGE\thetitle
\end{center}
\begin{center}\huge\thebooktitle\end{center}
\begin{center}\huge\emph{\Large and}\end{center}
\begin{center}\huge\theplaytitle\end{center}
\vspace*{\fill}
\clearpage

% copyright page

\begingroup
\footnotesize
\setlength{\parindent}{0pt}
\setlength{\parskip}{\baselineskip}

\ccPublicDomain\quad
The play \emph{\thetitle; or \theplaytitle}, written in 1904 and first published in 1928,
and the novel \emph{\thebooktitle}, published in 1911,
have in many jurisdictions entered the public domain;
in other jurisdictions the copyright is still in effect.

See ‹http://en.wikipedia.org/wiki/Peter\_and\_Wendy\#Copyright\_status›
for a discussion of the matter.

In 1929 Sir James Barrie assigned the copyright as a gift
to the Hospital for Sick Children, Great Ormond Street, London
(now called the Great Ormond Street Hospital).
In the U.K.,
due to the Copyright Designs and Patents Act (CDPA) of 1988,
the hospital has right to royalty in perpetuity
for public performance or commercial publication
even though the copyrights themselves have lapsed.

See ‹http://gosh.org/peterpan› regarding performance rights
of the play
in the U.K. or where the copyright is still valid.

\ccZero\quad
The \LaTeX\ code implementing the layout of this book,
or the HTML+CSS code implementing the layout of the EPUB version,
have been dedicated to the public domain.
No copyright is asserted.

You may find the code at ‹https://github.com/jcsalomon/peter-pan›.

\endgroup
\cleardoublepage

\tableofcontents*

\endinput


\mainmatter

\book{\thebooktitle}

% !TEX program = pdflatex
% !TEX encoding = UTF-8
% !TEX spellcheck = en_GB
% !TEX root = peter-pan.tex

\chapter{Peter Breaks Through}

All children, except one, grow up.
They soon know that they will grow up,
and the way Wendy knew was this.
One day when she was two years old she was playing in a garden,
and she plucked another flower and ran with it to her mother.
I suppose she must have looked rather delightful,
for Mrs.\@ Darling put her hand to her heart and cried,
“Oh, why can't you remain like this for ever!”
This was all that passed between them on the subject,
but henceforth Wendy knew that she must grow up.
You always know after you are two.
Two is the beginning of the end.

\endinput


Chapter 1 PETER BREAKS THROUGH


All children, except one, grow up. They soon know that they will grow up,
and the way Wendy knew was this. One day when she was two years old she
was playing in a garden, and she plucked another flower and ran with it to
her mother. I suppose she must have looked rather delightful, for Mrs.
Darling put her hand to her heart and cried, "Oh, why can't you remain
like this for ever!" This was all that passed between them on the subject,
but henceforth Wendy knew that she must grow up. You always know after you
are two. Two is the beginning of the end.


Of course they lived at 14 [their house number on their street], and until
Wendy came her mother was the chief one. She was a lovely lady, with a
romantic mind and such a sweet mocking mouth. Her romantic mind was like
the tiny boxes, one within the other, that come from the puzzling East,
however many you discover there is always one more; and her sweet mocking
mouth had one kiss on it that Wendy could never get, though there it was,
perfectly conspicuous in the right-hand corner.


The way Mr. Darling won her was this: the many gentlemen who had been boys
when she was a girl discovered simultaneously that they loved her, and
they all ran to her house to propose to her except Mr. Darling, who took a
cab and nipped in first, and so he got her. He got all of her, except the
innermost box and the kiss. He never knew about the box, and in time he
gave up trying for the kiss. Wendy thought Napoleon could have got it, but
I can picture him trying, and then going off in a passion, slamming the
door.


Mr. Darling used to boast to Wendy that her mother not only loved him but
respected him. He was one of those deep ones who know about stocks and
shares. Of course no one really knows, but he quite seemed to know, and he
often said stocks were up and shares were down in a way that would have
made any woman respect him.


Mrs. Darling was married in white, and at first she kept the books
perfectly, almost gleefully, as if it were a game, not so much as a
Brussels sprout was missing; but by and by whole cauliflowers dropped out,
and instead of them there were pictures of babies without faces. She drew
them when she should have been totting up. They were Mrs. Darling's
guesses.


Wendy came first, then John, then Michael.


For a week or two after Wendy came it was doubtful whether they would be
able to keep her, as she was another mouth to feed. Mr. Darling was
frightfully proud of her, but he was very honourable, and he sat on the
edge of Mrs. Darling's bed, holding her hand and calculating expenses,
while she looked at him imploringly. She wanted to risk it, come what
might, but that was not his way; his way was with a pencil and a piece of
paper, and if she confused him with suggestions he had to begin at the
beginning again.


"Now don't interrupt," he would beg of her.


"I have one pound seventeen here, and two and six at the office; I can cut
off my coffee at the office, say ten shillings, making two nine and six,
with your eighteen and three makes three nine seven, with five naught
naught in my cheque-book makes eight nine seven—who is that moving?—eight
nine seven, dot and carry seven—don't speak, my own—and the
pound you lent to that man who came to the door—quiet, child—dot
and carry child—there, you've done it!—did I say nine nine
seven? yes, I said nine nine seven; the question is, can we try it for a
year on nine nine seven?"


"Of course we can, George," she cried. But she was prejudiced in Wendy's
favour, and he was really the grander character of the two.


"Remember mumps," he warned her almost threateningly, and off he went
again. "Mumps one pound, that is what I have put down, but I daresay it
will be more like thirty shillings—don't speak—measles one
five, German measles half a guinea, makes two fifteen six—don't
waggle your finger—whooping-cough, say fifteen shillings"—and
so on it went, and it added up differently each time; but at last Wendy
just got through, with mumps reduced to twelve six, and the two kinds of
measles treated as one.


There was the same excitement over John, and Michael had even a narrower
squeak; but both were kept, and soon, you might have seen the three of
them going in a row to Miss Fulsom's Kindergarten school, accompanied by
their nurse.


Mrs. Darling loved to have everything just so, and Mr. Darling had a
passion for being exactly like his neighbours; so, of course, they had a
nurse. As they were poor, owing to the amount of milk the children drank,
this nurse was a prim Newfoundland dog, called Nana, who had belonged to
no one in particular until the Darlings engaged her. She had always
thought children important, however, and the Darlings had become
acquainted with her in Kensington Gardens, where she spent most of her
spare time peeping into perambulators, and was much hated by careless
nursemaids, whom she followed to their homes and complained of to their
mistresses. She proved to be quite a treasure of a nurse. How thorough she
was at bath-time, and up at any moment of the night if one of her charges
made the slightest cry. Of course her kennel was in the nursery. She had a
genius for knowing when a cough is a thing to have no patience with and
when it needs stocking around your throat. She believed to her last day in
old-fashioned remedies like rhubarb leaf, and made sounds of contempt over
all this new-fangled talk about germs, and so on. It was a lesson in
propriety to see her escorting the children to school, walking sedately by
their side when they were well behaved, and butting them back into line if
they strayed. On John's footer [in England soccer was called football,
"footer" for short] days she never once forgot his sweater, and she
usually carried an umbrella in her mouth in case of rain. There is a room
in the basement of Miss Fulsom's school where the nurses wait. They sat on
forms, while Nana lay on the floor, but that was the only difference. They
affected to ignore her as of an inferior social status to themselves, and
she despised their light talk. She resented visits to the nursery from
Mrs. Darling's friends, but if they did come she first whipped off
Michael's pinafore and put him into the one with blue braiding, and
smoothed out Wendy and made a dash at John's hair.


No nursery could possibly have been conducted more correctly, and Mr.
Darling knew it, yet he sometimes wondered uneasily whether the neighbours
talked.


He had his position in the city to consider.


Nana also troubled him in another way. He had sometimes a feeling that she
did not admire him. "I know she admires you tremendously, George," Mrs.
Darling would assure him, and then she would sign to the children to be
specially nice to father. Lovely dances followed, in which the only other
servant, Liza, was sometimes allowed to join. Such a midget she looked in
her long skirt and maid's cap, though she had sworn, when engaged, that
she would never see ten again. The gaiety of those romps! And gayest of
all was Mrs. Darling, who would pirouette so wildly that all you could see
of her was the kiss, and then if you had dashed at her you might have got
it. There never was a simpler happier family until the coming of Peter
Pan.


Mrs. Darling first heard of Peter when she was tidying up her children's
minds. It is the nightly custom of every good mother after her children
are asleep to rummage in their minds and put things straight for next
morning, repacking into their proper places the many articles that have
wandered during the day. If you could keep awake (but of course you can't)
you would see your own mother doing this, and you would find it very
interesting to watch her. It is quite like tidying up drawers. You would
see her on her knees, I expect, lingering humorously over some of your
contents, wondering where on earth you had picked this thing up, making
discoveries sweet and not so sweet, pressing this to her cheek as if it
were as nice as a kitten, and hurriedly stowing that out of sight. When
you wake in the morning, the naughtiness and evil passions with which you
went to bed have been folded up small and placed at the bottom of your
mind and on the top, beautifully aired, are spread out your prettier
thoughts, ready for you to put on.


I don't know whether you have ever seen a map of a person's mind. Doctors
sometimes draw maps of other parts of you, and your own map can become
intensely interesting, but catch them trying to draw a map of a child's
mind, which is not only confused, but keeps going round all the time.
There are zigzag lines on it, just like your temperature on a card, and
these are probably roads in the island, for the Neverland is always more
or less an island, with astonishing splashes of colour here and there, and
coral reefs and rakish-looking craft in the offing, and savages and lonely
lairs, and gnomes who are mostly tailors, and caves through which a river
runs, and princes with six elder brothers, and a hut fast going to decay,
and one very small old lady with a hooked nose. It would be an easy map if
that were all, but there is also first day at school, religion, fathers,
the round pond, needle-work, murders, hangings, verbs that take the
dative, chocolate pudding day, getting into braces, say ninety-nine,
three-pence for pulling out your tooth yourself, and so on, and either
these are part of the island or they are another map showing through, and
it is all rather confusing, especially as nothing will stand still.


Of course the Neverlands vary a good deal. John's, for instance, had a
lagoon with flamingoes flying over it at which John was shooting, while
Michael, who was very small, had a flamingo with lagoons flying over it.
John lived in a boat turned upside down on the sands, Michael in a wigwam,
Wendy in a house of leaves deftly sewn together. John had no friends,
Michael had friends at night, Wendy had a pet wolf forsaken by its
parents, but on the whole the Neverlands have a family resemblance, and if
they stood still in a row you could say of them that they have each
other's nose, and so forth. On these magic shores children at play are for
ever beaching their coracles [simple boat]. We too have been there; we can
still hear the sound of the surf, though we shall land no more.


Of all delectable islands the Neverland is the snuggest and most compact,
not large and sprawly, you know, with tedious distances between one
adventure and another, but nicely crammed. When you play at it by day with
the chairs and table-cloth, it is not in the least alarming, but in the
two minutes before you go to sleep it becomes very real. That is why there
are night-lights.


Occasionally in her travels through her children's minds Mrs. Darling
found things she could not understand, and of these quite the most
perplexing was the word Peter. She knew of no Peter, and yet he was here
and there in John and Michael's minds, while Wendy's began to be scrawled
all over with him. The name stood out in bolder letters than any of the
other words, and as Mrs. Darling gazed she felt that it had an oddly cocky
appearance.


"Yes, he is rather cocky," Wendy admitted with regret. Her mother had been
questioning her.


"But who is he, my pet?"


"He is Peter Pan, you know, mother."


At first Mrs. Darling did not know, but after thinking back into her
childhood she just remembered a Peter Pan who was said to live with the
fairies. There were odd stories about him, as that when children died he
went part of the way with them, so that they should not be frightened. She
had believed in him at the time, but now that she was married and full of
sense she quite doubted whether there was any such person.


"Besides," she said to Wendy, "he would be grown up by this time."


"Oh no, he isn't grown up," Wendy assured her confidently, "and he is just
my size." She meant that he was her size in both mind and body; she didn't
know how she knew, she just knew it.


Mrs. Darling consulted Mr. Darling, but he smiled pooh-pooh. "Mark my
words," he said, "it is some nonsense Nana has been putting into their
heads; just the sort of idea a dog would have. Leave it alone, and it will
blow over."


But it would not blow over and soon the troublesome boy gave Mrs. Darling
quite a shock.


Children have the strangest adventures without being troubled by them. For
instance, they may remember to mention, a week after the event happened,
that when they were in the wood they had met their dead father and had a
game with him. It was in this casual way that Wendy one morning made a
disquieting revelation. Some leaves of a tree had been found on the
nursery floor, which certainly were not there when the children went to
bed, and Mrs. Darling was puzzling over them when Wendy said with a
tolerant smile:


"I do believe it is that Peter again!"


"Whatever do you mean, Wendy?"


"It is so naughty of him not to wipe his feet," Wendy said, sighing. She
was a tidy child.


She explained in quite a matter-of-fact way that she thought Peter
sometimes came to the nursery in the night and sat on the foot of her bed
and played on his pipes to her. Unfortunately she never woke, so she
didn't know how she knew, she just knew.


"What nonsense you talk, precious. No one can get into the house without
knocking."


"I think he comes in by the window," she said.


"My love, it is three floors up."


"Were not the leaves at the foot of the window, mother?"


It was quite true; the leaves had been found very near the window.


Mrs. Darling did not know what to think, for it all seemed so natural to
Wendy that you could not dismiss it by saying she had been dreaming.


"My child," the mother cried, "why did you not tell me of this before?"


"I forgot," said Wendy lightly. She was in a hurry to get her breakfast.


Oh, surely she must have been dreaming.


But, on the other hand, there were the leaves. Mrs. Darling examined them
very carefully; they were skeleton leaves, but she was sure they did not
come from any tree that grew in England. She crawled about the floor,
peering at it with a candle for marks of a strange foot. She rattled the
poker up the chimney and tapped the walls. She let down a tape from the
window to the pavement, and it was a sheer drop of thirty feet, without so
much as a spout to climb up by.


Certainly Wendy had been dreaming.


But Wendy had not been dreaming, as the very next night showed, the night
on which the extraordinary adventures of these children may be said to
have begun.


On the night we speak of all the children were once more in bed. It
happened to be Nana's evening off, and Mrs. Darling had bathed them and
sung to them till one by one they had let go her hand and slid away into
the land of sleep.


All were looking so safe and cosy that she smiled at her fears now and sat
down tranquilly by the fire to sew.


It was something for Michael, who on his birthday was getting into shirts.
The fire was warm, however, and the nursery dimly lit by three
night-lights, and presently the sewing lay on Mrs. Darling's lap. Then her
head nodded, oh, so gracefully. She was asleep. Look at the four of them,
Wendy and Michael over there, John here, and Mrs. Darling by the fire.
There should have been a fourth night-light.


While she slept she had a dream. She dreamt that the Neverland had come
too near and that a strange boy had broken through from it. He did not
alarm her, for she thought she had seen him before in the faces of many
women who have no children. Perhaps he is to be found in the faces of some
mothers also. But in her dream he had rent the film that obscures the
Neverland, and she saw Wendy and John and Michael peeping through the gap.


The dream by itself would have been a trifle, but while she was dreaming
the window of the nursery blew open, and a boy did drop on the floor. He
was accompanied by a strange light, no bigger than your fist, which darted
about the room like a living thing and I think it must have been this
light that wakened Mrs. Darling.


She started up with a cry, and saw the boy, and somehow she knew at once
that he was Peter Pan. If you or I or Wendy had been there we should have
seen that he was very like Mrs. Darling's kiss. He was a lovely boy, clad
in skeleton leaves and the juices that ooze out of trees but the most
entrancing thing about him was that he had all his first teeth. When he
saw she was a grown-up, he gnashed the little pearls at her.


% !TEX program = pdflatex
% !TEX encoding = UTF-8
% !TEX spellcheck = en_GB
% !TEX root = peter-pan.tex

\chapter{The Shadow}

\endinput


Chapter 2 THE SHADOW


Mrs. Darling screamed, and, as if in answer to a bell, the door opened,
and Nana entered, returned from her evening out. She growled and sprang at
the boy, who leapt lightly through the window. Again Mrs. Darling
screamed, this time in distress for him, for she thought he was killed,
and she ran down into the street to look for his little body, but it was
not there; and she looked up, and in the black night she could see nothing
but what she thought was a shooting star.


She returned to the nursery, and found Nana with something in her mouth,
which proved to be the boy's shadow. As he leapt at the window Nana had
closed it quickly, too late to catch him, but his shadow had not had time
to get out; slam went the window and snapped it off.


You may be sure Mrs. Darling examined the shadow carefully, but it was
quite the ordinary kind.


Nana had no doubt of what was the best thing to do with this shadow. She
hung it out at the window, meaning "He is sure to come back for it; let us
put it where he can get it easily without disturbing the children."


But unfortunately Mrs. Darling could not leave it hanging out at the
window, it looked so like the washing and lowered the whole tone of the
house. She thought of showing it to Mr. Darling, but he was totting up
winter great-coats for John and Michael, with a wet towel around his head
to keep his brain clear, and it seemed a shame to trouble him; besides,
she knew exactly what he would say: "It all comes of having a dog for a
nurse."


She decided to roll the shadow up and put it away carefully in a drawer,
until a fitting opportunity came for telling her husband. Ah me!


The opportunity came a week later, on that never-to-be-forgotten Friday.
Of course it was a Friday.


"I ought to have been specially careful on a Friday," she used to say
afterwards to her husband, while perhaps Nana was on the other side of
her, holding her hand.


"No, no," Mr. Darling always said, "I am responsible for it all. I, George
Darling, did it. MEA CULPA, MEA CULPA." He had had a classical education.


They sat thus night after night recalling that fatal Friday, till every
detail of it was stamped on their brains and came through on the other
side like the faces on a bad coinage.


"If only I had not accepted that invitation to dine at 27," Mrs. Darling
said.


"If only I had not poured my medicine into Nana's bowl," said Mr. Darling.


"If only I had pretended to like the medicine," was what Nana's wet eyes
said.


"My liking for parties, George."


"My fatal gift of humour, dearest."


"My touchiness about trifles, dear master and mistress."


Then one or more of them would break down altogether; Nana at the thought,
"It's true, it's true, they ought not to have had a dog for a nurse." Many
a time it was Mr. Darling who put the handkerchief to Nana's eyes.


"That fiend!" Mr. Darling would cry, and Nana's bark was the echo of it,
but Mrs. Darling never upbraided Peter; there was something in the
right-hand corner of her mouth that wanted her not to call Peter names.


They would sit there in the empty nursery, recalling fondly every smallest
detail of that dreadful evening. It had begun so uneventfully, so
precisely like a hundred other evenings, with Nana putting on the water
for Michael's bath and carrying him to it on her back.


"I won't go to bed," he had shouted, like one who still believed that he
had the last word on the subject, "I won't, I won't. Nana, it isn't six
o'clock yet. Oh dear, oh dear, I shan't love you any more, Nana. I tell
you I won't be bathed, I won't, I won't!"


Then Mrs. Darling had come in, wearing her white evening-gown. She had
dressed early because Wendy so loved to see her in her evening-gown, with
the necklace George had given her. She was wearing Wendy's bracelet on her
arm; she had asked for the loan of it. Wendy loved to lend her bracelet to
her mother.


She had found her two older children playing at being herself and father
on the occasion of Wendy's birth, and John was saying:


"I am happy to inform you, Mrs. Darling, that you are now a mother," in
just such a tone as Mr. Darling himself may have used on the real
occasion.


Wendy had danced with joy, just as the real Mrs. Darling must have done.


Then John was born, with the extra pomp that he conceived due to the birth
of a male, and Michael came from his bath to ask to be born also, but John
said brutally that they did not want any more.


Michael had nearly cried. "Nobody wants me," he said, and of course the
lady in the evening-dress could not stand that.


"I do," she said, "I so want a third child."


"Boy or girl?" asked Michael, not too hopefully.


"Boy."


Then he had leapt into her arms. Such a little thing for Mr. and Mrs.
Darling and Nana to recall now, but not so little if that was to be
Michael's last night in the nursery.


They go on with their recollections.


"It was then that I rushed in like a tornado, wasn't it?" Mr. Darling
would say, scorning himself; and indeed he had been like a tornado.


Perhaps there was some excuse for him. He, too, had been dressing for the
party, and all had gone well with him until he came to his tie. It is an
astounding thing to have to tell, but this man, though he knew about
stocks and shares, had no real mastery of his tie. Sometimes the thing
yielded to him without a contest, but there were occasions when it would
have been better for the house if he had swallowed his pride and used a
made-up tie.


This was such an occasion. He came rushing into the nursery with the
crumpled little brute of a tie in his hand.


"Why, what is the matter, father dear?"


"Matter!" he yelled; he really yelled. "This tie, it will not tie." He
became dangerously sarcastic. "Not round my neck! Round the bed-post! Oh
yes, twenty times have I made it up round the bed-post, but round my neck,
no! Oh dear no! begs to be excused!"


He thought Mrs. Darling was not sufficiently impressed, and he went on
sternly, "I warn you of this, mother, that unless this tie is round my
neck we don't go out to dinner to-night, and if I don't go out to dinner
to-night, I never go to the office again, and if I don't go to the office
again, you and I starve, and our children will be flung into the streets."


Even then Mrs. Darling was placid. "Let me try, dear," she said, and
indeed that was what he had come to ask her to do, and with her nice cool
hands she tied his tie for him, while the children stood around to see
their fate decided. Some men would have resented her being able to do it
so easily, but Mr. Darling had far too fine a nature for that; he thanked
her carelessly, at once forgot his rage, and in another moment was dancing
round the room with Michael on his back.


"How wildly we romped!" says Mrs. Darling now, recalling it.


"Our last romp!" Mr. Darling groaned.


"O George, do you remember Michael suddenly said to me, 'How did you get
to know me, mother?'"


"I remember!"


"They were rather sweet, don't you think, George?"


"And they were ours, ours! and now they are gone."


The romp had ended with the appearance of Nana, and most unluckily Mr.
Darling collided against her, covering his trousers with hairs. They were
not only new trousers, but they were the first he had ever had with braid
on them, and he had had to bite his lip to prevent the tears coming. Of
course Mrs. Darling brushed him, but he began to talk again about its
being a mistake to have a dog for a nurse.


"George, Nana is a treasure."


"No doubt, but I have an uneasy feeling at times that she looks upon the
children as puppies."


"Oh no, dear one, I feel sure she knows they have souls."


"I wonder," Mr. Darling said thoughtfully, "I wonder." It was an
opportunity, his wife felt, for telling him about the boy. At first he
pooh-poohed the story, but he became thoughtful when she showed him the
shadow.


"It is nobody I know," he said, examining it carefully, "but it does look
a scoundrel."


"We were still discussing it, you remember," says Mr. Darling, "when Nana
came in with Michael's medicine. You will never carry the bottle in your
mouth again, Nana, and it is all my fault."


Strong man though he was, there is no doubt that he had behaved rather
foolishly over the medicine. If he had a weakness, it was for thinking
that all his life he had taken medicine boldly, and so now, when Michael
dodged the spoon in Nana's mouth, he had said reprovingly, "Be a man,
Michael."


"Won't; won't!" Michael cried naughtily. Mrs. Darling left the room to get
a chocolate for him, and Mr. Darling thought this showed want of firmness.


"Mother, don't pamper him," he called after her. "Michael, when I was your
age I took medicine without a murmur. I said, 'Thank you, kind parents,
for giving me bottles to make me well.'"


He really thought this was true, and Wendy, who was now in her night-gown,
believed it also, and she said, to encourage Michael, "That medicine you
sometimes take, father, is much nastier, isn't it?"


"Ever so much nastier," Mr. Darling said bravely, "and I would take it now
as an example to you, Michael, if I hadn't lost the bottle."


He had not exactly lost it; he had climbed in the dead of night to the top
of the wardrobe and hidden it there. What he did not know was that the
faithful Liza had found it, and put it back on his wash-stand.


"I know where it is, father," Wendy cried, always glad to be of service.
"I'll bring it," and she was off before he could stop her. Immediately his
spirits sank in the strangest way.


"John," he said, shuddering, "it's most beastly stuff. It's that nasty,
sticky, sweet kind."


"It will soon be over, father," John said cheerily, and then in rushed
Wendy with the medicine in a glass.


"I have been as quick as I could," she panted.


"You have been wonderfully quick," her father retorted, with a vindictive
politeness that was quite thrown away upon her. "Michael first," he said
doggedly.


"Father first," said Michael, who was of a suspicious nature.


"I shall be sick, you know," Mr. Darling said threateningly.


"Come on, father," said John.


"Hold your tongue, John," his father rapped out.


Wendy was quite puzzled. "I thought you took it quite easily, father."


"That is not the point," he retorted. "The point is, that there is more in
my glass than in Michael's spoon." His proud heart was nearly bursting.
"And it isn't fair: I would say it though it were with my last breath; it
isn't fair."


"Father, I am waiting," said Michael coldly.


"It's all very well to say you are waiting; so am I waiting."


"Father's a cowardly custard."


"So are you a cowardly custard."


"I'm not frightened."


"Neither am I frightened."


"Well, then, take it."


"Well, then, you take it."


Wendy had a splendid idea. "Why not both take it at the same time?"


"Certainly," said Mr. Darling. "Are you ready, Michael?"


Wendy gave the words, one, two, three, and Michael took his medicine, but
Mr. Darling slipped his behind his back.


There was a yell of rage from Michael, and "O father!" Wendy exclaimed.


"What do you mean by 'O father'?" Mr. Darling demanded. "Stop that row,
Michael. I meant to take mine, but I—I missed it."


It was dreadful the way all the three were looking at him, just as if they
did not admire him. "Look here, all of you," he said entreatingly, as soon
as Nana had gone into the bathroom. "I have just thought of a splendid
joke. I shall pour my medicine into Nana's bowl, and she will drink it,
thinking it is milk!"


It was the colour of milk; but the children did not have their father's
sense of humour, and they looked at him reproachfully as he poured the
medicine into Nana's bowl. "What fun!" he said doubtfully, and they did
not dare expose him when Mrs. Darling and Nana returned.


"Nana, good dog," he said, patting her, "I have put a little milk into
your bowl, Nana."


Nana wagged her tail, ran to the medicine, and began lapping it. Then she
gave Mr. Darling such a look, not an angry look: she showed him the great
red tear that makes us so sorry for noble dogs, and crept into her kennel.


Mr. Darling was frightfully ashamed of himself, but he would not give in.
In a horrid silence Mrs. Darling smelt the bowl. "O George," she said,
"it's your medicine!"


"It was only a joke," he roared, while she comforted her boys, and Wendy
hugged Nana. "Much good," he said bitterly, "my wearing myself to the bone
trying to be funny in this house."


And still Wendy hugged Nana. "That's right," he shouted. "Coddle her!
Nobody coddles me. Oh dear no! I am only the breadwinner, why should I be
coddled—why, why, why!"


"George," Mrs. Darling entreated him, "not so loud; the servants will hear
you." Somehow they had got into the way of calling Liza the servants.


"Let them!" he answered recklessly. "Bring in the whole world. But I
refuse to allow that dog to lord it in my nursery for an hour longer."


The children wept, and Nana ran to him beseechingly, but he waved her
back. He felt he was a strong man again. "In vain, in vain," he cried;
"the proper place for you is the yard, and there you go to be tied up this
instant."


"George, George," Mrs. Darling whispered, "remember what I told you about
that boy."


Alas, he would not listen. He was determined to show who was master in
that house, and when commands would not draw Nana from the kennel, he
lured her out of it with honeyed words, and seizing her roughly, dragged
her from the nursery. He was ashamed of himself, and yet he did it. It was
all owing to his too affectionate nature, which craved for admiration.
When he had tied her up in the back-yard, the wretched father went and sat
in the passage, with his knuckles to his eyes.


In the meantime Mrs. Darling had put the children to bed in unwonted
silence and lit their night-lights. They could hear Nana barking, and John
whimpered, "It is because he is chaining her up in the yard," but Wendy
was wiser.


"That is not Nana's unhappy bark," she said, little guessing what was
about to happen; "that is her bark when she smells danger."


Danger!


"Are you sure, Wendy?"


"Oh, yes."


Mrs. Darling quivered and went to the window. It was securely fastened.
She looked out, and the night was peppered with stars. They were crowding
round the house, as if curious to see what was to take place there, but
she did not notice this, nor that one or two of the smaller ones winked at
her. Yet a nameless fear clutched at her heart and made her cry, "Oh, how
I wish that I wasn't going to a party to-night!"


Even Michael, already half asleep, knew that she was perturbed, and he
asked, "Can anything harm us, mother, after the night-lights are lit?"


"Nothing, precious," she said; "they are the eyes a mother leaves behind
her to guard her children."


She went from bed to bed singing enchantments over them, and little
Michael flung his arms round her. "Mother," he cried, "I'm glad of you."
They were the last words she was to hear from him for a long time.


No. 27 was only a few yards distant, but there had been a slight fall of
snow, and Father and Mother Darling picked their way over it deftly not to
soil their shoes. They were already the only persons in the street, and
all the stars were watching them. Stars are beautiful, but they may not
take an active part in anything, they must just look on for ever. It is a
punishment put on them for something they did so long ago that no star now
knows what it was. So the older ones have become glassy-eyed and seldom
speak (winking is the star language), but the little ones still wonder.
They are not really friendly to Peter, who had a mischievous way of
stealing up behind them and trying to blow them out; but they are so fond
of fun that they were on his side to-night, and anxious to get the
grown-ups out of the way. So as soon as the door of 27 closed on Mr. and
Mrs. Darling there was a commotion in the firmament, and the smallest of
all the stars in the Milky Way screamed out:


"Now, Peter!"


% !TEX program = pdflatex
% !TEX encoding = UTF-8
% !TEX spellcheck = en_GB
% !TEX root = peter-pan.tex

\chapter{Come Away, Come Away!}

For a moment after Mr.\@ and Mrs.\@ Darling left the house
the night-lights by the beds of the three children continued to burn clearly.
They were awfully nice little night-lights,
and one cannot help wishing that they could have kept awake to see Peter;
but Wendy’s light blinked and gave such a yawn that the other two yawned also,
and before they could close their mouths all the three went out.

There was another light in the room now, a thousand times brighter than the night-lights,
and in the time we have taken to say this, it had been in all the drawers in the nursery,
looking for Peter’s shadow, rummaged the wardrobe and turned every pocket inside out.
It was not really a light;
it made this light by flashing about so quickly,
but when it came to rest for a second you saw it was a fairy, no longer than your hand, but still growing.
It was a girl called Tinker Bell exquisitely gowned in a skeleton leaf, cut low and square,
through which her figure could be seen to the best advantage.
She was slightly inclined to \emph{embonpoint}.

A moment after the fairy’s entrance the window was blown open by the breathing of the little stars,
and Peter dropped in.
He had carried Tinker Bell part of the way, and his hand was still messy with the fairy dust.

“Tinker Bell,” he called softly, after making sure that the children were asleep,
“Tink, where are you?”
She was in a jug for the moment, and liking it extremely;
she had never been in a jug before.

“Oh, do come out of that jug, and tell me, do you know where they put my shadow?”

The loveliest tinkle as of golden bells answered him.
It is the fairy language.
You ordinary children can never hear it,
but if you were to hear it you would know that you had heard it once before.

Tink said that the shadow was in the big box.
She meant the chest of drawers, and Peter jumped at the drawers,
scattering their contents to the floor with both hands, as kings toss ha’pence to the crowd.
In a moment he had recovered his shadow,
and in his delight he forgot that he had shut Tinker Bell up in the drawer.

If he thought at all, but I don’t believe he ever thought,
it was that he and his shadow, when brought near each other, would join like drops of water,
and when they did not he was appalled.
He tried to stick it on with soap from the bathroom, but that also failed.
A shudder passed through Peter, and he sat on the floor and cried.

His sobs woke Wendy, and she sat up in bed.
She was not alarmed to see a stranger crying on the nursery floor;
she was only pleasantly interested.

“Boy,” she said courteously, “why are you crying?”

Peter could be exceeding polite also,
having learned the grand manner at fairy ceremonies,
and he rose and bowed to her beautifully.
She was much pleased, and bowed beautifully to him from the bed.

“What’s your name?\@” he asked.

“Wendy Moira Angela Darling,” she replied with some satisfaction.
“What is your name?”

“Peter Pan.”

She was already sure that he must be Peter,
but it did seem a comparatively short name.

“Is that all?”

“Yes,” he said rather sharply.
He felt for the first time that it was a shortish name.

“I’m so sorry,” said Wendy Moira Angela.

“It doesn’t matter,” Peter gulped.

She asked where he lived.

“Second to the right,” said Peter,
“and then straight on till morning.”

“What a funny address!”

Peter had a sinking.
For the first time he felt that perhaps it was a funny address.

“No, it isn’t,” he said.

“I mean,” Wendy said nicely, remembering that she was hostess,
“is that what they put on the letters?”

He wished she had not mentioned letters.

“Don’t get any letters,” he said contemptuously.

“But your mother gets letters?”

“Don’t have a mother,” he said.
Not only had he no mother, but he had not the slightest desire to have one.
He thought them very over-rated persons.
Wendy, however, felt at once that she was in the presence of a tragedy.

“O Peter, no wonder you were crying,” she said, and got out of bed and ran to him.

“I wasn’t crying about mothers,” he said rather indignantly.
“I was crying because I can’t get my shadow to stick on.
Besides, I wasn’t crying.”

“It has come off?”

“Yes.”

Then Wendy saw the shadow on the floor, looking so draggled,
and she was frightfully sorry for Peter.
“How awful!\@” she said,
but she could not help smiling when she saw that he had been trying to stick it on with soap.
How exactly like a boy!

Fortunately she knew at once what to do.
“It must be sewn on,” she said, just a little patronisingly.

“What’s sewn?\@” he asked.

“You’re dreadfully ignorant.”

“No, I’m not.”

But she was exulting in his ignorance.
“I shall sew it on for you, my little man,” she said,
though he was tall as herself,
and she got out her housewife, and sewed the shadow on to Peter’s foot.

“I daresay it will hurt a little,” she warned him.

“Oh, I shan’t cry,” said Peter,
who was already of the opinion that he had never cried in his life.
And he clenched his teeth and did not cry,
and soon his shadow was behaving properly, though still a little creased.

“Perhaps I should have ironed it,” Wendy said thoughtfully,
but Peter, boylike, was indifferent to appearances,
and he was now jumping about in the wildest glee.
Alas, he had already forgotten that he owed his bliss to Wendy.
He thought he had attached the shadow himself.
“How clever I am!\@” he crowed rapturously, “oh, the cleverness of me!”

It is humiliating to have to confess that this conceit of Peter was one of his most fascinating qualities.
To put it with brutal frankness, there never was a cockier boy.

But for the moment Wendy was shocked.
“You conceit,” she exclaimed, with frightful sarcasm;
“of course I did nothing!”

“You did a little,” Peter said carelessly, and continued to dance.

“A little!\@” she replied with \emph{hauteur};
“if I am no use I can at least withdraw,”
and she sprang in the most dignified way into bed and covered her face with the blankets.

To induce her to look up he pretended to be going away,
and when this failed he sat on the end of the bed and tapped her gently with his foot.
“Wendy,” he said, “don’t withdraw.
I can’t help crowing, Wendy, when I’m pleased with myself.”
Still she would not look up, though she was listening eagerly.
“Wendy,” he continued, in a voice that no woman has ever yet been able to resist,
“Wendy, one girl is more use than twenty boys.”

Now Wendy was every inch a woman,
though there were not very many inches,
and she peeped out of the bed-clothes.

“Do you really think so, Peter?”

“Yes, I do.”

“I think it’s perfectly sweet of you,” she declared, “and I’ll get up again,”
and she sat with him on the side of the bed.
She also said she would give him a kiss if he liked,
but Peter did not know what she meant, and he held out his hand expectantly.

“Surely you know what a kiss is?\@” she asked, aghast.

“I shall know when you give it to me,” he replied stiffly,
and not to hurt his feeling she gave him a thimble.

“Now,” said he, “shall I give you a kiss?\@”
and she replied with a slight primness, “If you please.”
She made herself rather cheap by inclining her face toward him,
but he merely dropped an acorn button into her hand,
so she slowly returned her face to where it had been before,
and said nicely that she would wear his kiss on the chain around her neck.
It was lucky that she did put it on that chain, for it was afterwards to save her life.

When people in our set are introduced, it is customary for them to ask each other’s age,
and so Wendy, who always liked to do the correct thing, asked Peter how old he was.
It was not really a happy question to ask him;
it was like an examination paper that asks grammar,
when what you want to be asked is Kings of England.

“I don’t know,” he replied uneasily, “but I am quite young.”
He really knew nothing about it, he had merely suspicions,
but he said at a venture, “Wendy, I ran away the day I was born.”

Wendy was quite surprised, but interested;
and she indicated in the charming drawing-room manner, by a touch on her night-gown,
that he could sit nearer her.

“It was because I heard father and mother,” he explained in a low voice,
“talking about what I was to be when I became a man.”
He was extraordinarily agitated now.
“I don’t want ever to be a man,” he said with passion.
“I want always to be a little boy and to have fun.
So I ran away to Kensington Gardens and lived a long long time among the fairies.”

She gave him a look of the most intense admiration,
and he thought it was because he had run away,
but it was really because he knew fairies.
Wendy had lived such a home life that to know fairies struck her as quite delightful.
She poured out questions about them, to his surprise,
for they were rather a nuisance to him, getting in his way and so on,
and indeed he sometimes had to give them a hiding.
Still, he liked them on the whole, and he told her about the beginning of fairies.

“You see, Wendy, when the first baby laughed for the first time, its laugh broke into a thousand pieces,
and they all went skipping about, and that was the beginning of fairies.”

Tedious talk this, but being a stay-at-home she liked it.

“And so,” he went on good-naturedly, “there ought to be one fairy for every boy and girl.”

“Ought to be?
Isn’t there?”

“No.
You see children know such a lot now, they soon don’t believe in fairies,
and every time a child says, ‘I don’t believe in fairies,’ there is a fairy somewhere that falls down dead.”

Really, he thought they had now talked enough about fairies,
and it struck him that Tinker Bell was keeping very quiet.
“I can’t think where she has gone to,” he said, rising, and he called Tink by name.
Wendy’s heart went flutter with a sudden thrill.

“Peter,” she cried, clutching him,
“you don’t mean to tell me that there is a fairy in this room!”

“She was here just now,” he said a little impatiently.
“You don’t hear her, do you?\@” and they both listened.

“The only sound I hear,” said Wendy, “is like a tinkle of bells.”

“Well, that’s Tink, that’s the fairy language.
I think I hear her too.”

The sound came from the chest of drawers, and Peter made a merry face.
No one could ever look quite so merry as Peter, and the loveliest of gurgles was his laugh.
He had his first laugh still.

“Wendy,” he whispered gleefully, “I do believe I shut her up in the drawer!”

He let poor Tink out of the drawer, and she flew about the nursery screaming with fury.
“You shouldn’t say such things,” Peter retorted.
“Of course I’m very sorry, but how could I know you were in the drawer?”

Wendy was not listening to him.
“O Peter,” she cried, “if she would only stand still and let me see her!”

“They hardly ever stand still,” he said,
but for one moment Wendy saw the romantic figure come to rest on the cuckoo clock.
“O the lovely!\@” she cried, though Tink’s face was still distorted with passion.

“Tink,” said Peter amiably, “this lady says she wishes you were her fairy.”

Tinker Bell answered insolently.

“What does she say, Peter?”

He had to translate.
“She is not very polite.
She says you are a great ugly girl, and that she is my fairy.”

He tried to argue with Tink.
“You know you can’t be my fairy, Tink, because I am an gentleman and you are a lady.”

To this Tink replied in these words, “You silly ass,” and disappeared into the bathroom.
“She is quite a common fairy,” Peter explained apologetically,
“she is called Tinker Bell because she mends the pots and kettles.”

They were together in the armchair by this time,
and Wendy plied him with more questions.

“If you don’t live in Kensington Gardens now—”

“Sometimes I do still.”

“But where do you live mostly now?”

“With the lost boys.”

“Who are they?”

“They are the children who fall out of their perambulators when the nurse is looking the other way.
If they are not claimed in seven days they are sent far away to the Neverland to defray expenses.
I’m captain.”

“What fun it must be!”

“Yes,” said cunning Peter, “but we are rather lonely.
You see we have no female companionship.”

“Are none of the others girls?”

“Oh, no;
girls, you know, are much too clever to fall out of their prams.”

This flattered Wendy immensely.
“I think,” she said, “it is perfectly lovely the way you talk about girls;
John there just despises us.”

For reply Peter rose and kicked John out of bed, blankets and all; one kick.
This seemed to Wendy rather forward for a first meeting,
and she told him with spirit that he was not captain in her house.
However, John continued to sleep so placidly on the floor that she allowed him to remain there.
“And I know you meant to be kind,” she said, relenting, “so you may give me a kiss.”

For the moment she had forgotten his ignorance about kisses.
“I thought you would want it back,” he said a little bitterly,
and offered to return her the thimble.

“Oh dear,” said the nice Wendy,
“I don’t mean a kiss, I mean a thimble.”

“What’s that?”

“It’s like this.”
She kissed him.

“Funny!\@” said Peter gravely.
“Now shall I give you a thimble?”

“If you wish to,” said Wendy, keeping her head erect this time.

Peter thimbled her, and almost immediately she screeched.
“What is it, Wendy?”

“It was exactly as if someone were pulling my hair.”

“That must have been Tink.
I never knew her so naughty before.”

And indeed Tink was darting about again, using offensive language.

“She says she will do that to you, Wendy, every time I give you a thimble.”

“But why?”

“Why, Tink?”

Again Tink replied, “You silly ass.”
Peter could not understand why, but Wendy understood,
and she was just slightly disappointed when he admitted that he came to the nursery window
not to see her but to listen to stories.

“You see, I don’t know any stories.
None of the lost boys knows any stories.”

“How perfectly awful,” Wendy said.

“Do you know,” Peter asked “why swallows build in the eaves of houses?
It is to listen to the stories.
O Wendy, your mother was telling you such a lovely story.”

“Which story was it?”

“About the prince who couldn’t find the lady who wore the glass slipper.”

“Peter,” said Wendy excitedly, “that was Cinderella,
and he found her, and they lived happily ever after.”

Peter was so glad that he rose from the floor, where they had been sitting, and hurried to the window.

“Where are you going?\@” she cried with misgiving.

“To tell the other boys.”

“Don’t go Peter,” she entreated,
“I know such lots of stories.”

Those were her precise words, so there can be no denying that it was she who first tempted him.

He came back, and there was a greedy look in his eyes now which ought to have alarmed her, but did not.

“Oh, the stories I could tell to the boys!\@” she cried,
and then Peter gripped her and began to draw her toward the window.

“Let me go!\@” she ordered him.

“Wendy, do come with me and tell the other boys.”

Of course she was very pleased to be asked,
but she said, “Oh dear, I can’t.
Think of mummy!
Besides, I can’t fly.”

“I’ll teach you.”

“Oh, how lovely to fly.”

“I’ll teach you how to jump on the wind’s back, and then away we go.”

“Oo!\@” she exclaimed rapturously.

“Wendy, Wendy, when you are sleeping in your silly bed
you might be flying about with me saying funny things to the stars.”

“Oo!”

“And, Wendy, there are mermaids.”

“Mermaids!
With tails?”

“Such long tails.”

“Oh,” cried Wendy, “to see a mermaid!”

He had become frightfully cunning.
“Wendy,” he said, “how we should all respect you.”

She was wriggling her body in distress.
It was quite as if she were trying to remain on the nursery floor.

But he had no pity for her.

“Wendy,” he said, the sly one, “you could tuck us in at night.”

“Oo!”

“None of us has ever been tucked in at night.”

“Oo,” and her arms went out to him.

“And you could darn our clothes, and make pockets for us.
None of us has any pockets.”

How could she resist.
“Of course it’s awfully fascinating!\@” she cried.
“Peter, would you teach John and Michael to fly too?”

“If you like,” he said indifferently, and she ran to John and Michael and shook them.
“Wake up,” she cried, “Peter Pan has come and he is to teach us to fly.”

John rubbed his eyes.
“Then I shall get up,” he said.
Of course he was on the floor already.
“Hallo,” he said, “I am up!”

Michael was up by this time also, looking as sharp as a knife with six blades and a saw,
but Peter suddenly signed silence.
Their faces assumed the awful craftiness of children listening for sounds from the grown-up world.
All was as still as salt.
Then everything was right.
No, stop!
Everything was wrong.
Nana, who had been barking distressfully all the evening, was quiet now.
It was her silence they had heard.

“Out with the light!
Hide!
Quick!\@” cried John, taking command for the only time throughout the whole adventure.
And thus when Liza entered, holding Nana, the nursery seemed quite its old self, very dark,
and you would have sworn you heard its three wicked inmates breathing angelically as they slept.
They were really doing it artfully from behind the window curtains.

Liza was in a bad temper, for she was mixing the Christmas puddings in the kitchen,
and had been drawn from them, with a raisin still on her cheek, by Nana’s absurd suspicions.
She thought the best way of getting a little quiet was to take Nana to the nursery for a moment,
but in custody of course.

“There, you suspicious brute,” she said, not sorry that Nana was in disgrace.
“They are perfectly safe, aren’t they?
Every one of the little angels sound asleep in bed.
Listen to their gentle breathing.”

Here Michael, encouraged by his success, breathed so loudly that they were nearly detected.
Nana knew that kind of breathing, and she tried to drag herself out of Liza’s clutches.

But Liza was dense.
“No more of it, Nana,” she said sternly, pulling her out of the room.
“I warn you if you bark again I shall go straight for master and missus and bring them home from the party,
and then, oh, won’t master whip you, just.”

She tied the unhappy dog up again, but do you think Nana ceased to bark?
Bring master and missus home from the party!
Why, that was just what she wanted.
Do you think she cared whether she was whipped so long as her charges were safe?
Unfortunately Liza returned to her puddings,
and Nana, seeing that no help would come from her, strained and strained at the chain until at last she broke it.
In another moment she had burst into the dining-room of 27 and flung up her paws to heaven,
her most expressive way of making a communication.
Mr.\@ and Mrs.\@ Darling knew at once that something terrible was happening in their nursery,
and without a good-bye to their hostess they rushed into the street.

But it was now ten minutes since three scoundrels had been breathing behind the curtains,
and Peter Pan can do a great deal in ten minutes.

We now return to the nursery.

“It’s all right,” John announced, emerging from his hiding-place.
“I say, Peter, can you really fly?”

Instead of troubling to answer him Peter flew around the room, taking the mantelpiece on the way.

“How topping!\@” said John and Michael.

“How sweet!\@” cried Wendy.

“Yes, I’m sweet, oh, I am sweet!\@” said Peter, forgetting his manners again.

It looked delightfully easy, and they tried it first from the floor and then from the beds,
but they always went down instead of up.

“I say, how do you do it?\@” asked John, rubbing his knee.
He was quite a practical boy.

“You just think lovely wonderful thoughts,” Peter explained, “and they lift you up in the air.”

He showed them again.

“You’re so nippy at it,” John said, “couldn’t you do it very slowly once?”

Peter did it both slowly and quickly.
“I’ve got it now, Wendy!\@” cried John, but soon he found he had not.
Not one of them could fly an inch,
though even Michael was in words of two syllables, and Peter did not know A from Z\@.

Of course Peter had been trifling with them,
for no one can fly unless the fairy dust has been blown on him.
Fortunately, as we have mentioned, one of his hands was messy with it,
and he blew some on each of them, with the most superb results.

“Now just wiggle your shoulders this way,” he said, “and let go.”

They were all on their beds, and gallant Michael let go first.
He did not quite mean to let go, but he did it,
and immediately he was borne across the room.

“I flewed!\@” he screamed while still in mid-air.

John let go and met Wendy near the bathroom.

“Oh, lovely!”

“Oh, ripping!”

“Look at me!”

“Look at me!”

“Look at me!”

They were not nearly so elegant as Peter, they could not help kicking a little,
but their heads were bobbing against the ceiling,
and there is almost nothing so delicious as that.
Peter gave Wendy a hand at first, but had to desist, Tink was so indignant.

Up and down they went, and round and round.
Heavenly was Wendy’s word.

“I say,” cried John, “why shouldn’t we all go out?”

Of course it was to this that Peter had been luring them.

Michael was ready:
he wanted to see how long it took him to do a billion miles.
But Wendy hesitated.

“Mermaids!\@” said Peter again.

“Oo!”

“And there are pirates.”

“Pirates,” cried John, seizing his Sunday hat, “let us go at once.”

It was just at this moment that Mr.\@ and Mrs.\@ Darling hurried with Nana out of 27.
They ran into the middle of the street to look up at the nursery window;
and, yes, it was still shut,
but the room was ablaze with light,
and most heart-gripping sight of all,
they could see in shadow on the curtain three little figures in night attire circling round and round,
not on the floor but in the air.

Not three figures, four!

In a tremble they opened the street door.
Mr.\@ Darling would have rushed upstairs, but Mrs.\@ Darling signed him to go softly.
She even tried to make her heart go softly.

Will they reach the nursery in time?
If so, how delightful for them, and we shall all breathe a sigh of relief, but there will be no story.
On the other hand, if they are not in time, I solemnly promise that it will all come right in the end.

They would have reached the nursery in time had it not been that the little stars were watching them.
Once again the stars blew the window open, and that smallest star of all called out:

“\emph{Cave}, Peter!”

Then Peter knew that there was not a moment to lose.
“Come,” he cried imperiously, and soared out at once into the night,
followed by John and Michael and Wendy.

Mr.\@ and Mrs.\@ Darling and Nana rushed into the nursery too late.
The birds were flown.

\endinput

% !TEX program = pdflatex
% !TEX encoding = UTF-8
% !TEX spellcheck = en_GB
% !TEX root = peter-pan.tex

\chapter{The Flight}

\begin{center}
“Second to the right, and straight on till morning.”
\end{center}
That, Peter had told Wendy, was the way to the Neverland;
but even birds, carrying maps and consulting them at windy corners,
could not have sighted it with these instructions.
Peter, you see, just said anything that came into his head.

\endinput

At first his companions trusted him implicitly, and so great were the
delights of flying that they wasted time circling round church spires or
any other tall objects on the way that took their fancy.


John and Michael raced, Michael getting a start.


They recalled with contempt that not so long ago they had thought
themselves fine fellows for being able to fly round a room.


Not long ago. But how long ago? They were flying over the sea before this
thought began to disturb Wendy seriously. John thought it was their second
sea and their third night.


Sometimes it was dark and sometimes light, and now they were very cold and
again too warm. Did they really feel hungry at times, or were they merely
pretending, because Peter had such a jolly new way of feeding them? His
way was to pursue birds who had food in their mouths suitable for humans
and snatch it from them; then the birds would follow and snatch it back;
and they would all go chasing each other gaily for miles, parting at last
with mutual expressions of good-will. But Wendy noticed with gentle
concern that Peter did not seem to know that this was rather an odd way of
getting your bread and butter, nor even that there are other ways.


Certainly they did not pretend to be sleepy, they were sleepy; and that
was a danger, for the moment they popped off, down they fell. The awful
thing was that Peter thought this funny.


"There he goes again!" he would cry gleefully, as Michael suddenly dropped
like a stone.


"Save him, save him!" cried Wendy, looking with horror at the cruel sea
far below. Eventually Peter would dive through the air, and catch Michael
just before he could strike the sea, and it was lovely the way he did it;
but he always waited till the last moment, and you felt it was his
cleverness that interested him and not the saving of human life. Also he
was fond of variety, and the sport that engrossed him one moment would
suddenly cease to engage him, so there was always the possibility that the
next time you fell he would let you go.


He could sleep in the air without falling, by merely lying on his back and
floating, but this was, partly at least, because he was so light that if
you got behind him and blew he went faster.


"Do be more polite to him," Wendy whispered to John, when they were
playing "Follow my Leader."


"Then tell him to stop showing off," said John.


When playing Follow my Leader, Peter would fly close to the water and
touch each shark's tail in passing, just as in the street you may run your
finger along an iron railing. They could not follow him in this with much
success, so perhaps it was rather like showing off, especially as he kept
looking behind to see how many tails they missed.


"You must be nice to him," Wendy impressed on her brothers. "What could we
do if he were to leave us!"


"We could go back," Michael said.


"How could we ever find our way back without him?"


"Well, then, we could go on," said John.


"That is the awful thing, John. We should have to go on, for we don't know
how to stop."


This was true, Peter had forgotten to show them how to stop.


John said that if the worst came to the worst, all they had to do was to
go straight on, for the world was round, and so in time they must come
back to their own window.


"And who is to get food for us, John?"


"I nipped a bit out of that eagle's mouth pretty neatly, Wendy."


"After the twentieth try," Wendy reminded him. "And even though we became
good at picking up food, see how we bump against clouds and things if he
is not near to give us a hand."


Indeed they were constantly bumping. They could now fly strongly, though
they still kicked far too much; but if they saw a cloud in front of them,
the more they tried to avoid it, the more certainly did they bump into it.
If Nana had been with them, she would have had a bandage round Michael's
forehead by this time.


Peter was not with them for the moment, and they felt rather lonely up
there by themselves. He could go so much faster than they that he would
suddenly shoot out of sight, to have some adventure in which they had no
share. He would come down laughing over something fearfully funny he had
been saying to a star, but he had already forgotten what it was, or he
would come up with mermaid scales still sticking to him, and yet not be
able to say for certain what had been happening. It was really rather
irritating to children who had never seen a mermaid.


"And if he forgets them so quickly," Wendy argued, "how can we expect that
he will go on remembering us?"


Indeed, sometimes when he returned he did not remember them, at least not
well. Wendy was sure of it. She saw recognition come into his eyes as he
was about to pass them the time of day and go on; once even she had to
call him by name.


"I'm Wendy," she said agitatedly.


He was very sorry. "I say, Wendy," he whispered to her, "always if you see
me forgetting you, just keep on saying 'I'm Wendy,' and then I'll
remember."


Of course this was rather unsatisfactory. However, to make amends he
showed them how to lie out flat on a strong wind that was going their way,
and this was such a pleasant change that they tried it several times and
found that they could sleep thus with security. Indeed they would have
slept longer, but Peter tired quickly of sleeping, and soon he would cry
in his captain voice, "We get off here." So with occasional tiffs, but on
the whole rollicking, they drew near the Neverland; for after many moons
they did reach it, and, what is more, they had been going pretty straight
all the time, not perhaps so much owing to the guidance of Peter or Tink
as because the island was looking for them. It is only thus that any one
may sight those magic shores.


"There it is," said Peter calmly.


"Where, where?"


"Where all the arrows are pointing."


Indeed a million golden arrows were pointing it out to the children, all
directed by their friend the sun, who wanted them to be sure of their way
before leaving them for the night.


Wendy and John and Michael stood on tip-toe in the air to get their first
sight of the island. Strange to say, they all recognized it at once, and
until fear fell upon them they hailed it, not as something long dreamt of
and seen at last, but as a familiar friend to whom they were returning
home for the holidays.


"John, there's the lagoon."


"Wendy, look at the turtles burying their eggs in the sand."


"I say, John, I see your flamingo with the broken leg!"


"Look, Michael, there's your cave!"


"John, what's that in the brushwood?"


"It's a wolf with her whelps. Wendy, I do believe that's your little
whelp!"


"There's my boat, John, with her sides stove in!"


"No, it isn't. Why, we burned your boat."


"That's her, at any rate. I say, John, I see the smoke of the redskin
camp!"


"Where? Show me, and I'll tell you by the way smoke curls whether they are
on the war-path."


"There, just across the Mysterious River."


"I see now. Yes, they are on the war-path right enough."


Peter was a little annoyed with them for knowing so much, but if he wanted
to lord it over them his triumph was at hand, for have I not told you that
anon fear fell upon them?


It came as the arrows went, leaving the island in gloom.


In the old days at home the Neverland had always begun to look a little
dark and threatening by bedtime. Then unexplored patches arose in it and
spread, black shadows moved about in them, the roar of the beasts of prey
was quite different now, and above all, you lost the certainty that you
would win. You were quite glad that the night-lights were on. You even
liked Nana to say that this was just the mantelpiece over here, and that
the Neverland was all make-believe.


Of course the Neverland had been make-believe in those days, but it was
real now, and there were no night-lights, and it was getting darker every
moment, and where was Nana?


They had been flying apart, but they huddled close to Peter now. His
careless manner had gone at last, his eyes were sparkling, and a tingle
went through them every time they touched his body. They were now over the
fearsome island, flying so low that sometimes a tree grazed their feet.
Nothing horrid was visible in the air, yet their progress had become slow
and laboured, exactly as if they were pushing their way through hostile
forces. Sometimes they hung in the air until Peter had beaten on it with
his fists.


"They don't want us to land," he explained.


"Who are they?" Wendy whispered, shuddering.


But he could not or would not say. Tinker Bell had been asleep on his
shoulder, but now he wakened her and sent her on in front.


Sometimes he poised himself in the air, listening intently, with his hand
to his ear, and again he would stare down with eyes so bright that they
seemed to bore two holes to earth. Having done these things, he went on
again.


His courage was almost appalling. "Would you like an adventure now," he
said casually to John, "or would you like to have your tea first?"


Wendy said "tea first" quickly, and Michael pressed her hand in gratitude,
but the braver John hesitated.


"What kind of adventure?" he asked cautiously.


"There's a pirate asleep in the pampas just beneath us," Peter told him.
"If you like, we'll go down and kill him."


"I don't see him," John said after a long pause.


"I do."


"Suppose," John said, a little huskily, "he were to wake up."


Peter spoke indignantly. "You don't think I would kill him while he was
sleeping! I would wake him first, and then kill him. That's the way I
always do."


"I say! Do you kill many?"


"Tons."


John said "How ripping," but decided to have tea first. He asked if there
were many pirates on the island just now, and Peter said he had never
known so many.


"Who is captain now?"


"Hook," answered Peter, and his face became very stern as he said that
hated word.


"Jas. Hook?"


"Ay."


Then indeed Michael began to cry, and even John could speak in gulps only,
for they knew Hook's reputation.


"He was Blackbeard's bo'sun," John whispered huskily. "He is the worst of
them all. He is the only man of whom Barbecue was afraid."


"That's him," said Peter.


"What is he like? Is he big?"


"He is not so big as he was."


"How do you mean?"


"I cut off a bit of him."


"You!"


"Yes, me," said Peter sharply.


"I wasn't meaning to be disrespectful."


"Oh, all right."


"But, I say, what bit?"


"His right hand."


"Then he can't fight now?"


"Oh, can't he just!"


"Left-hander?"


"He has an iron hook instead of a right hand, and he claws with it."


"Claws!"


"I say, John," said Peter.


"Yes."


"Say, 'Ay, ay, sir.'"


"Ay, ay, sir."


"There is one thing," Peter continued, "that every boy who serves under me
has to promise, and so must you."


John paled.


"It is this, if we meet Hook in open fight, you must leave him to me."


"I promise," John said loyally.


For the moment they were feeling less eerie, because Tink was flying with
them, and in her light they could distinguish each other. Unfortunately
she could not fly so slowly as they, and so she had to go round and round
them in a circle in which they moved as in a halo. Wendy quite liked it,
until Peter pointed out the drawbacks.


"She tells me," he said, "that the pirates sighted us before the darkness
came, and got Long Tom out."


"The big gun?"


"Yes. And of course they must see her light, and if they guess we are near
it they are sure to let fly."


"Wendy!"


"John!"


"Michael!"


"Tell her to go away at once, Peter," the three cried simultaneously, but
he refused.


"She thinks we have lost the way," he replied stiffly, "and she is rather
frightened. You don't think I would send her away all by herself when she
is frightened!"


For a moment the circle of light was broken, and something gave Peter a
loving little pinch.


"Then tell her," Wendy begged, "to put out her light."


"She can't put it out. That is about the only thing fairies can't do. It
just goes out of itself when she falls asleep, same as the stars."


"Then tell her to sleep at once," John almost ordered.


"She can't sleep except when she's sleepy. It is the only other thing
fairies can't do."


"Seems to me," growled John, "these are the only two things worth doing."


Here he got a pinch, but not a loving one.


"If only one of us had a pocket," Peter said, "we could carry her in it."
However, they had set off in such a hurry that there was not a pocket
between the four of them.


He had a happy idea. John's hat!


Tink agreed to travel by hat if it was carried in the hand. John carried
it, though she had hoped to be carried by Peter. Presently Wendy took the
hat, because John said it struck against his knee as he flew; and this, as
we shall see, led to mischief, for Tinker Bell hated to be under an
obligation to Wendy.


In the black topper the light was completely hidden, and they flew on in
silence. It was the stillest silence they had ever known, broken once by a
distant lapping, which Peter explained was the wild beasts drinking at the
ford, and again by a rasping sound that might have been the branches of
trees rubbing together, but he said it was the redskins sharpening their
knives.


Even these noises ceased. To Michael the loneliness was dreadful. "If only
something would make a sound!" he cried.


As if in answer to his request, the air was rent by the most tremendous
crash he had ever heard. The pirates had fired Long Tom at them.


The roar of it echoed through the mountains, and the echoes seemed to cry
savagely, "Where are they, where are they, where are they?"


Thus sharply did the terrified three learn the difference between an
island of make-believe and the same island come true.


When at last the heavens were steady again, John and Michael found
themselves alone in the darkness. John was treading the air mechanically,
and Michael without knowing how to float was floating.


"Are you shot?" John whispered tremulously.


"I haven't tried yet," Michael whispered back.


We know now that no one had been hit. Peter, however, had been carried by
the wind of the shot far out to sea, while Wendy was blown upwards with no
companion but Tinker Bell.


It would have been well for Wendy if at that moment she had dropped the
hat.


I don't know whether the idea came suddenly to Tink, or whether she had
planned it on the way, but she at once popped out of the hat and began to
lure Wendy to her destruction.


Tink was not all bad; or, rather, she was all bad just now, but, on the
other hand, sometimes she was all good. Fairies have to be one thing or
the other, because being so small they unfortunately have room for one
feeling only at a time. They are, however, allowed to change, only it must
be a complete change. At present she was full of jealousy of Wendy. What
she said in her lovely tinkle Wendy could not of course understand, and I
believe some of it was bad words, but it sounded kind, and she flew back
and forward, plainly meaning "Follow me, and all will be well."


What else could poor Wendy do? She called to Peter and John and Michael,
and got only mocking echoes in reply. She did not yet know that Tink hated
her with the fierce hatred of a very woman. And so, bewildered, and now
staggering in her flight, she followed Tink to her doom.


% !TEX program = pdflatex
% !TEX encoding = UTF-8
% !TEX spellcheck = en_GB
% !TEX root = peter-pan.tex

\chapter{The Island Come True}

% !TEX program = pdflatex
% !TEX encoding = UTF-8
% !TEX spellcheck = en_GB
% !TEX root = peter-pan.tex

\chapter{The Little House}

Foolish Tootles was standing like a conqueror over Wendy's body
when the other boys sprang, armed, from their trees.

"You are too late," he cried proudly, "I have shot the Wendy.
Peter will be so pleased with me."

Overhead Tinker Bell shouted "Silly ass!\@" and darted into hiding.
The others did not hear her.
They had crowded round Wendy, and as they looked a terrible silence fell upon the wood.
If Wendy's heart had been beating they would all have heard it.

Slightly was the first to speak.
"This is no bird," he said in a scared voice.
"I think this must be a lady."

"A lady?\@" said Tootles, and fell a-trembling.

"And we have killed her," Nibs said hoarsely.

They all whipped off their caps.

"Now I see," Curly said:
"Peter was bringing her to us."
He threw himself sorrowfully on the ground.

"A lady to take care of us at last," said one of the twins,
"and you have killed her!"

They were sorry for him, but sorrier for themselves,
and when he took a step nearer them they turned from him.

Tootles' face was very white, but there was a dignity about him now that had never been there before.

"I did it," he said, reflecting.
"When ladies used to come to me in dreams,
I said, 'Pretty mother, pretty mother.'
But when at last she really came, I shot her."

He moved slowly away.

"Don't go," they called in pity.

"I must," he answered, shaking;
"I am so afraid of Peter."

It was at this tragic moment that they heard a sound which made the heart of every one of them rise to his mouth.
They heard Peter crow.

"Peter!\@" they cried, for it was always thus that he signalled his return.

"Hide her," they whispered, and gathered hastily around Wendy.
But Tootles stood aloof.

Again came that ringing crow, and Peter dropped in front of them.
"Greetings, boys," he cried, and mechanically they saluted, and then again was silence.

He frowned.

"I am back," he said hotly, "why do you not cheer?"

They opened their mouths, but the cheers would not come.
He overlooked it in his haste to tell the glorious tidings.

"Great news, boys," he cried, "I have brought at last a mother for you all."

Still no sound, except a little thud from Tootles as he dropped on his knees.

"Have you not seen her?\@" asked Peter, becoming troubled.
"She flew this way."

"Ah me!\@" once voice said, and another said, "Oh, mournful day."

Tootles rose.
"Peter," he said quietly, "I will show her to you,"
and when the others would still have hidden her he said, "Back, twins, let Peter see."

So they all stood back, and let him see,
and after he had looked for a little time he did not know what to do next.

"She is dead," he said uncomfortably.
"Perhaps she is frightened at being dead."

He thought of hopping off in a comic sort of way till he was out of sight of her,
and then never going near the spot any more.
They would all have been glad to follow if he had done this.

But there was the arrow.
He took it from her heart and faced his band.

"Whose arrow?\@" he demanded sternly.

"Mine, Peter," said Tootles on his knees.

"Oh, dastard hand," Peter said, and he raised the arrow to use it as a dagger.

Tootles did not flinch.
He bared his breast.
"Strike, Peter," he said firmly, "strike true."

Twice did Peter raise the arrow, and twice did his hand fall.
"I cannot strike," he said with awe, "there is something stays my hand."

All looked at him in wonder, save Nibs, who fortunately looked at Wendy.

"It is she," he cried, "the Wendy lady, see, her arm!"

Wonderful to relate, Wendy had raised her arm.
Nibs bent over her and listened reverently.
"I think she said, 'Poor Tootles,'" he whispered.

"She lives," Peter said briefly.

Slightly cried instantly, "The Wendy lady lives."

Then Peter knelt beside her and found his button.
You remember she had put it on a chain that she wore round her neck.

"See," he said, "the arrow struck against this.
It is the kiss I gave her.
It has saved her life."

"I remember kisses," Slightly interposed quickly, "let me see it.
Ay, that's a kiss."

Peter did not hear him.
He was begging Wendy to get better quickly, so that he could show her the mermaids.
Of course she could not answer yet, being still in a frightful faint;
but from overhead came a wailing note.

"Listen to Tink," said Curly, "she is crying because the Wendy lives."

Then they had to tell Peter of Tink's crime, and almost never had they seen him look so stern.

"Listen, Tinker Bell," he cried, "I am your friend no more.
Begone from me for ever."

She flew on to his shoulder and pleaded, but he brushed her off.
Not until Wendy again raised her arm did he relent sufficiently to say,
"Well, not for ever, but for a whole week."

Do you think Tinker Bell was grateful to Wendy for raising her arm?
Oh dear no, never wanted to pinch her so much.
Fairies indeed are strange, and Peter, who understood them best, often cuffed them.

But what to do with Wendy in her present delicate state of health?

"Let us carry her down into the house," Curly suggested.

"Ay," said Slightly, "that is what one does with ladies."

"No, no," Peter said, "you must not touch her.
It would not be sufficiently respectful."

"That," said Slightly, "is what I was thinking."

"But if she lies there," Tootles said, "she will die."

"Ay, she will die," Slightly admitted, "but there is no way out."

"Yes, there is," cried Peter.
"Let us build a little house round her."

They were all delighted.
"Quick," he ordered them, "bring me each of you the best of what we have.
Gut our house.
Be sharp."

In a moment they were as busy as tailors the night before a wedding.
They scurried this way and that, down for bedding, up for firewood,
and while they were at it, who should appear but John and Michael.
As they dragged along the ground they fell asleep standing,
stopped, woke up, moved another step and slept again.

"John, John," Michael would cry, "wake up!
Where is Nana, John, and mother?"

And then John would rub his eyes and mutter, "It is true, we did fly."

You may be sure they were very relieved to find Peter.

"Hullo, Peter," they said.

"Hullo," replied Peter amicably, though he had quite forgotten them.
He was very busy at the moment measuring Wendy with his feet to see how large a house she would need.
Of course he meant to leave room for chairs and a table.
John and Michael watched him.

"Is Wendy asleep?\@" they asked.

"Yes."

"John," Michael proposed, "let us wake her and get her to make supper for us,"
but as he said it some of the other boys rushed on carrying branches for the building of the house.
"Look at them!\@" he cried.

"Curly," said Peter in his most captainy voice, "see that these boys help in the building of the house."

"Ay, ay, sir."

"Build a house?\@" exclaimed John.

"For the Wendy," said Curly.

"For Wendy?\@" John said, aghast.
"Why, she is only a girl!"

"That," explained Curly, "is why we are her servants."

"You?
Wendy's servants!"

"Yes," said Peter, "and you also.
Away with them."

The astounded brothers were dragged away to hack and hew and carry.
"Chairs and a fender first," Peter ordered.
"Then we shall build a house round them."

"Ay," said Slightly, "that is how a house is built;
it all comes back to me."

Peter thought of everything.
"Slightly," he cried, "fetch a doctor."

"Ay, ay," said Slightly at once, and disappeared, scratching his head.
But he knew Peter must be obeyed, and he returned in a moment, wearing John's hat and looking solemn.

"Please, sir," said Peter, going to him, "are you a doctor?"

The difference between him and the other boys at such a time was that they knew it was make-believe,
while to him make-believe and true were exactly the same thing.
This sometimes troubled them, as when they had to make-believe that they had had their dinners.

If they broke down in their make-believe he rapped them on the knuckles.

"Yes, my little man," Slightly anxiously replied, who had chapped knuckles.

"Please, sir," Peter explained, "a lady lies very ill."

She was lying at their feet, but Slightly had the sense not to see her.

"Tut, tut, tut," he said, "where does she lie?"

"In yonder glade."

"I will put a glass thing in her mouth," said Slightly,
and he made-believe to do it, while Peter waited.
It was an anxious moment when the glass thing was withdrawn.

"How is she?\@" inquired Peter.

"Tut, tut, tut," said Slightly, "this has cured her."

"I am glad!\@" Peter cried.

"I will call again in the evening," Slightly said;
"give her beef tea out of a cup with a spout to it;"
but after he had returned the hat to John he blew big breaths,
which was his habit on escaping from a difficulty.

In the meantime the wood had been alive with the sound of axes;
almost everything needed for a cosy dwelling already lay at Wendy's feet.

"If only we knew," said one, "the kind of house she likes best."

"Peter," shouted another, "she is moving in her sleep."

"Her mouth opens," cried a third, looking respectfully into it.
"Oh, lovely!"

"Perhaps she is going to sing in her sleep," said Peter.
"Wendy, sing the kind of house you would like to have."

Immediately, without opening her eyes, Wendy began to sing:

\begin{verse}
	"I wish I had a pretty house,\\
	The littlest ever seen,\\
	With funny little red walls\\
	And roof of mossy green."
\end{verse}

They gurgled with joy at this,
for by the greatest good luck the branches they had brought were sticky with red sap,
and all the ground was carpeted with moss.
As they rattled up the little house they broke into song themselves:

\begin{verse}
	"We've built the little walls and roof\\
	And made a lovely door,\\
	So tell us, mother Wendy,\\
	What are you wanting more?"
\end{verse}

To this she answered greedily:

\begin{verse}
	"Oh, really next I think I'll have\\
	Gay windows all about,\\
	With roses peeping in, you know,\\
	And babies peeping out."
\end{verse}

With a blow of their fists they made windows,
and large yellow leaves were the blinds.
But roses—?

"Roses," cried Peter sternly.

Quickly they made-believe to grow the loveliest roses up the walls.

Babies?

To prevent Peter ordering babies they hurried into song again:

\begin{verse}
	"We've made the roses peeping out,\\
	The babes are at the door,\\
	We cannot make ourselves, you know,\\
	'cos we've been made before."
\end{verse}

Peter, seeing this to be a good idea, at once pretended that it was his own.
The house was quite beautiful, and no doubt Wendy was very cosy within,
though, of course, they could no longer see her.
Peter strode up and down, ordering finishing touches.
Nothing escaped his eagle eyes.
Just when it seemed absolutely finished:

"There's no knocker on the door," he said.

They were very ashamed, but Tootles gave the sole of his shoe,
and it made an excellent knocker.

Absolutely finished now, they thought.

Not of bit of it.
"There's no chimney," Peter said;
"we must have a chimney."

"It certainly does need a chimney," said John importantly.
This gave Peter an idea.
He snatched the hat off John's head, knocked out the bottom, and put the hat on the roof.
The little house was so pleased to have such a capital chimney that,
as if to say thank you, smoke immediately began to come out of the hat.

Now really and truly it was finished.
Nothing remained to do but to knock.

"All look your best," Peter warned them;
"first impressions are awfully important."

He was glad no one asked him what first impressions are;
they were all too busy looking their best.

He knocked politely, and now the wood was as still as the children,
not a sound to be heard except from Tinker Bell,
who was watching from a branch and openly sneering.

What the boys were wondering was, would any one answer the knock?
If a lady, what would she be like?

The door opened and a lady came out.
It was Wendy.
They all whipped off their hats.

She looked properly surprised, and this was just how they had hoped she would look.

"Where am I?\@" she said.

Of course Slightly was the first to get his word in.
"Wendy lady," he said rapidly, "for you we built this house."

"Oh, say you're pleased," cried Nibs.

"Lovely, darling house," Wendy said, and they were the very words they had hoped she would say.

"And we are your children," cried the twins.

Then all went on their knees, and holding out their arms cried, "O Wendy lady, be our mother."

"Ought I?\@" Wendy said, all shining.
"Of course it's frightfully fascinating, but you see I am only a little girl.
I have no real experience."

"That doesn't matter," said Peter, as if he were the only person present who knew all about it,
though he was really the one who knew least.
"What we need is just a nice motherly person."

"Oh dear!\@" Wendy said, "you see, I feel that is exactly what I am."

"It is, it is," they all cried;
"we saw it at once."

"Very well," she said, "I will do my best.
Come inside at once, you naughty children;
I am sure your feet are damp.
And before I put you to bed I have just time to finish the story of Cinderella."

In they went;
I don't know how there was room for them, but you can squeeze very tight in the Neverland.
And that was the first of the many joyous evenings they had with Wendy.
By and by she tucked them up in the great bed in the home under the trees,
but she herself slept that night in the little house,
and Peter kept watch outside with drawn sword,
for the pirates could be heard carousing far away and the wolves were on the prowl.
The little house looked so cosy and safe in the darkness, with a bright light showing through its blinds,
and the chimney smoking beautifully, and Peter standing on guard.
After a time he fell asleep, and some unsteady fairies had to climb over him on their way home from an orgy.
Any of the other boys obstructing the fairy path at night they would have mischiefed,
but they just tweaked Peter's nose and passed on.

\endinput

% !TEX program = pdflatex
% !TEX encoding = UTF-8
% !TEX spellcheck = en_GB
% !TEX root = peter-pan.tex

\chapter{The Home under the Ground}

\endinput


Chapter 7 THE HOME UNDER THE GROUND


One of the first things Peter did next day was to measure Wendy and John
and Michael for hollow trees. Hook, you remember, had sneered at the boys
for thinking they needed a tree apiece, but this was ignorance, for unless
your tree fitted you it was difficult to go up and down, and no two of the
boys were quite the same size. Once you fitted, you drew in [let out] your
breath at the top, and down you went at exactly the right speed, while to
ascend you drew in and let out alternately, and so wriggled up. Of course,
when you have mastered the action you are able to do these things without
thinking of them, and nothing can be more graceful.


But you simply must fit, and Peter measures you for your tree as carefully
as for a suit of clothes: the only difference being that the clothes are
made to fit you, while you have to be made to fit the tree. Usually it is
done quite easily, as by your wearing too many garments or too few, but if
you are bumpy in awkward places or the only available tree is an odd
shape, Peter does some things to you, and after that you fit. Once you
fit, great care must be taken to go on fitting, and this, as Wendy was to
discover to her delight, keeps a whole family in perfect condition.


Wendy and Michael fitted their trees at the first try, but John had to be
altered a little.


After a few days' practice they could go up and down as gaily as buckets
in a well. And how ardently they grew to love their home under the ground;
especially Wendy. It consisted of one large room, as all houses should do,
with a floor in which you could dig [for worms] if you wanted to go
fishing, and in this floor grew stout mushrooms of a charming colour,
which were used as stools. A Never tree tried hard to grow in the centre
of the room, but every morning they sawed the trunk through, level with
the floor. By tea-time it was always about two feet high, and then they
put a door on top of it, the whole thus becoming a table; as soon as they
cleared away, they sawed off the trunk again, and thus there was more room
to play. There was an enormous fireplace which was in almost any part of
the room where you cared to light it, and across this Wendy stretched
strings, made of fibre, from which she suspended her washing. The bed was
tilted against the wall by day, and let down at 6:30, when it filled
nearly half the room; and all the boys slept in it, except Michael, lying
like sardines in a tin. There was a strict rule against turning round
until one gave the signal, when all turned at once. Michael should have
used it also, but Wendy would have [desired] a baby, and he was the
littlest, and you know what women are, and the short and long of it is
that he was hung up in a basket.


It was rough and simple, and not unlike what baby bears would have made of
an underground house in the same circumstances. But there was one recess
in the wall, no larger than a bird-cage, which was the private apartment
of Tinker Bell. It could be shut off from the rest of the house by a tiny
curtain, which Tink, who was most fastidious [particular], always kept
drawn when dressing or undressing. No woman, however large, could have had
a more exquisite boudoir [dressing room] and bed-chamber combined. The
couch, as she always called it, was a genuine Queen Mab, with club legs;
and she varied the bedspreads according to what fruit-blossom was in
season. Her mirror was a Puss-in-Boots, of which there are now only three,
unchipped, known to fairy dealers; the washstand was Pie-crust and
reversible, the chest of drawers an authentic Charming the Sixth, and the
carpet and rugs the best (the early) period of Margery and Robin. There
was a chandelier from Tiddlywinks for the look of the thing, but of course
she lit the residence herself. Tink was very contemptuous of the rest of
the house, as indeed was perhaps inevitable, and her chamber, though
beautiful, looked rather conceited, having the appearance of a nose
permanently turned up.


I suppose it was all especially entrancing to Wendy, because those
rampagious boys of hers gave her so much to do. Really there were whole
weeks when, except perhaps with a stocking in the evening, she was never
above ground. The cooking, I can tell you, kept her nose to the pot, and
even if there was nothing in it, even if there was no pot, she had to keep
watching that it came aboil just the same. You never exactly knew whether
there would be a real meal or just a make-believe, it all depended upon
Peter's whim: he could eat, really eat, if it was part of a game, but he
could not stodge [cram down the food] just to feel stodgy [stuffed with
food], which is what most children like better than anything else; the
next best thing being to talk about it. Make-believe was so real to him
that during a meal of it you could see him getting rounder. Of course it
was trying, but you simply had to follow his lead, and if you could prove
to him that you were getting loose for your tree he let you stodge.


Wendy's favourite time for sewing and darning was after they had all gone
to bed. Then, as she expressed it, she had a breathing time for herself;
and she occupied it in making new things for them, and putting double
pieces on the knees, for they were all most frightfully hard on their
knees.


When she sat down to a basketful of their stockings, every heel with a
hole in it, she would fling up her arms and exclaim, "Oh dear, I am sure I
sometimes think spinsters are to be envied!"


Her face beamed when she exclaimed this.


You remember about her pet wolf. Well, it very soon discovered that she
had come to the island and it found her out, and they just ran into each
other's arms. After that it followed her about everywhere.


As time wore on did she think much about the beloved parents she had left
behind her? This is a difficult question, because it is quite impossible
to say how time does wear on in the Neverland, where it is calculated by
moons and suns, and there are ever so many more of them than on the
mainland. But I am afraid that Wendy did not really worry about her father
and mother; she was absolutely confident that they would always keep the
window open for her to fly back by, and this gave her complete ease of
mind. What did disturb her at times was that John remembered his parents
vaguely only, as people he had once known, while Michael was quite willing
to believe that she was really his mother. These things scared her a
little, and nobly anxious to do her duty, she tried to fix the old life in
their minds by setting them examination papers on it, as like as possible
to the ones she used to do at school. The other boys thought this awfully
interesting, and insisted on joining, and they made slates for themselves,
and sat round the table, writing and thinking hard about the questions she
had written on another slate and passed round. They were the most ordinary
questions—"What was the colour of Mother's eyes? Which was taller,
Father or Mother? Was Mother blonde or brunette? Answer all three
questions if possible." "(A) Write an essay of not less than 40 words on
How I spent my last Holidays, or The Characters of Father and Mother
compared. Only one of these to be attempted." Or "(1) Describe Mother's
laugh; (2) Describe Father's laugh; (3) Describe Mother's Party Dress; (4)
Describe the Kennel and its Inmate."


They were just everyday questions like these, and when you could not
answer them you were told to make a cross; and it was really dreadful what
a number of crosses even John made. Of course the only boy who replied to
every question was Slightly, and no one could have been more hopeful of
coming out first, but his answers were perfectly ridiculous, and he really
came out last: a melancholy thing.


Peter did not compete. For one thing he despised all mothers except Wendy,
and for another he was the only boy on the island who could neither write
nor spell; not the smallest word. He was above all that sort of thing.


By the way, the questions were all written in the past tense. What was the
colour of Mother's eyes, and so on. Wendy, you see, had been forgetting,
too.


Adventures, of course, as we shall see, were of daily occurrence; but
about this time Peter invented, with Wendy's help, a new game that
fascinated him enormously, until he suddenly had no more interest in it,
which, as you have been told, was what always happened with his games. It
consisted in pretending not to have adventures, in doing the sort of thing
John and Michael had been doing all their lives, sitting on stools
flinging balls in the air, pushing each other, going out for walks and
coming back without having killed so much as a grizzly. To see Peter doing
nothing on a stool was a great sight; he could not help looking solemn at
such times, to sit still seemed to him such a comic thing to do. He
boasted that he had gone walking for the good of his health. For several
suns these were the most novel of all adventures to him; and John and
Michael had to pretend to be delighted also; otherwise he would have
treated them severely.


He often went out alone, and when he came back you were never absolutely
certain whether he had had an adventure or not. He might have forgotten it
so completely that he said nothing about it; and then when you went out
you found the body; and, on the other hand, he might say a great deal
about it, and yet you could not find the body. Sometimes he came home with
his head bandaged, and then Wendy cooed over him and bathed it in lukewarm
water, while he told a dazzling tale. But she was never quite sure, you
know. There were, however, many adventures which she knew to be true
because she was in them herself, and there were still more that were at
least partly true, for the other boys were in them and said they were
wholly true. To describe them all would require a book as large as an
English-Latin, Latin-English Dictionary, and the most we can do is to give
one as a specimen of an average hour on the island. The difficulty is
which one to choose. Should we take the brush with the redskins at
Slightly Gulch? It was a sanguinary affair, and especially
interesting as showing one of Peter's peculiarities, which was that in the
middle of a fight he would suddenly change sides. At the Gulch, when
victory was still in the balance, sometimes leaning this way and sometimes
that, he called out, "I'm redskin to-day; what are you, Tootles?" And
Tootles answered, "Redskin; what are you, Nibs?" and Nibs said, "Redskin;
what are you Twin?" and so on; and they were all redskins; and of course
this would have ended the fight had not the real redskins fascinated by
Peter's methods, agreed to be lost boys for that once, and so at it they
all went again, more fiercely than ever.


The extraordinary upshot of this adventure was—but we have not
decided yet that this is the adventure we are to narrate. Perhaps a better
one would be the night attack by the redskins on the house under the
ground, when several of them stuck in the hollow trees and had to be
pulled out like corks. Or we might tell how Peter saved Tiger Lily's life
in the Mermaids' Lagoon, and so made her his ally.


Or we could tell of that cake the pirates cooked so that the boys might
eat it and perish; and how they placed it in one cunning spot after
another; but always Wendy snatched it from the hands of her children, so
that in time it lost its succulence, and became as hard as a stone, and
was used as a missile, and Hook fell over it in the dark.


Or suppose we tell of the birds that were Peter's friends, particularly of
the Never bird that built in a tree overhanging the lagoon, and how the
nest fell into the water, and still the bird sat on her eggs, and Peter
gave orders that she was not to be disturbed. That is a pretty story, and
the end shows how grateful a bird can be; but if we tell it we must also
tell the whole adventure of the lagoon, which would of course be telling
two adventures rather than just one. A shorter adventure, and quite as
exciting, was Tinker Bell's attempt, with the help of some street fairies,
to have the sleeping Wendy conveyed on a great floating leaf to the
mainland. Fortunately the leaf gave way and Wendy woke, thinking it was
bath-time, and swam back. Or again, we might choose Peter's defiance of
the lions, when he drew a circle round him on the ground with an arrow and
dared them to cross it; and though he waited for hours, with the other
boys and Wendy looking on breathlessly from trees, not one of them dared
to accept his challenge.


Which of these adventures shall we choose? The best way will be to toss
for it.


I have tossed, and the lagoon has won. This almost makes one wish that the
gulch or the cake or Tink's leaf had won. Of course I could do it again,
and make it best out of three; however, perhaps fairest to stick to the
lagoon.


% !TEX program = pdflatex
% !TEX encoding = UTF-8
% !TEX spellcheck = en_GB
% !TEX root = peter-pan.tex

\chapter{The Mermaids’ Lagoon}

\endinput

If you shut your eyes and are a lucky one, you may see at times a
shapeless pool of lovely pale colours suspended in the darkness; then if
you squeeze your eyes tighter, the pool begins to take shape, and the
colours become so vivid that with another squeeze they must go on fire.
But just before they go on fire you see the lagoon. This is the nearest
you ever get to it on the mainland, just one heavenly moment; if there
could be two moments you might see the surf and hear the mermaids singing.


The children often spent long summer days on this lagoon, swimming or
floating most of the time, playing the mermaid games in the water, and so
forth. You must not think from this that the mermaids were on friendly
terms with them: on the contrary, it was among Wendy's lasting regrets
that all the time she was on the island she never had a civil word from
one of them. When she stole softly to the edge of the lagoon she might see
them by the score, especially on Marooners' Rock, where they loved to
bask, combing out their hair in a lazy way that quite irritated her; or
she might even swim, on tiptoe as it were, to within a yard of them, but
then they saw her and dived, probably splashing her with their tails, not
by accident, but intentionally.


They treated all the boys in the same way, except of course Peter, who
chatted with them on Marooners' Rock by the hour, and sat on their tails
when they got cheeky. He gave Wendy one of their combs.


The most haunting time at which to see them is at the turn of the moon,
when they utter strange wailing cries; but the lagoon is dangerous for
mortals then, and until the evening of which we have now to tell, Wendy
had never seen the lagoon by moonlight, less from fear, for of course
Peter would have accompanied her, than because she had strict rules about
every one being in bed by seven. She was often at the lagoon, however, on
sunny days after rain, when the mermaids come up in extraordinary numbers
to play with their bubbles. The bubbles of many colours made in rainbow
water they treat as balls, hitting them gaily from one to another with
their tails, and trying to keep them in the rainbow till they burst. The
goals are at each end of the rainbow, and the keepers only are allowed to
use their hands. Sometimes a dozen of these games will be going on in the
lagoon at a time, and it is quite a pretty sight.


But the moment the children tried to join in they had to play by
themselves, for the mermaids immediately disappeared. Nevertheless we have
proof that they secretly watched the interlopers, and were not above
taking an idea from them; for John introduced a new way of hitting the
bubble, with the head instead of the hand, and the mermaids adopted it.
This is the one mark that John has left on the Neverland.


It must also have been rather pretty to see the children resting on a rock
for half an hour after their mid-day meal. Wendy insisted on their doing
this, and it had to be a real rest even though the meal was make-believe.
So they lay there in the sun, and their bodies glistened in it, while she
sat beside them and looked important.


It was one such day, and they were all on Marooners' Rock. The rock was
not much larger than their great bed, but of course they all knew how not
to take up much room, and they were dozing, or at least lying with their
eyes shut, and pinching occasionally when they thought Wendy was not
looking. She was very busy, stitching.


While she stitched a change came to the lagoon. Little shivers ran over
it, and the sun went away and shadows stole across the water, turning it
cold. Wendy could no longer see to thread her needle, and when she looked
up, the lagoon that had always hitherto been such a laughing place seemed
formidable and unfriendly.


It was not, she knew, that night had come, but something as dark as night
had come. No, worse than that. It had not come, but it had sent that
shiver through the sea to say that it was coming. What was it?


There crowded upon her all the stories she had been told of Marooners'
Rock, so called because evil captains put sailors on it and leave them
there to drown. They drown when the tide rises, for then it is submerged.


Of course she should have roused the children at once; not merely because
of the unknown that was stalking toward them, but because it was no longer
good for them to sleep on a rock grown chilly. But she was a young mother
and she did not know this; she thought you simply must stick to your rule
about half an hour after the mid-day meal. So, though fear was upon her,
and she longed to hear male voices, she would not waken them. Even when
she heard the sound of muffled oars, though her heart was in her mouth,
she did not waken them. She stood over them to let them have their sleep
out. Was it not brave of Wendy?


It was well for those boys then that there was one among them who could
sniff danger even in his sleep. Peter sprang erect, as wide awake at once
as a dog, and with one warning cry he roused the others.


He stood motionless, one hand to his ear.


"Pirates!" he cried. The others came closer to him. A strange smile was
playing about his face, and Wendy saw it and shuddered. While that smile
was on his face no one dared address him; all they could do was to stand
ready to obey. The order came sharp and incisive.


"Dive!"


There was a gleam of legs, and instantly the lagoon seemed deserted.
Marooners' Rock stood alone in the forbidding waters as if it were itself
marooned.


The boat drew nearer. It was the pirate dinghy, with three figures in her,
Smee and Starkey, and the third a captive, no other than Tiger Lily. Her
hands and ankles were tied, and she knew what was to be her fate. She was
to be left on the rock to perish, an end to one of her race more terrible
than death by fire or torture, for is it not written in the book of the
tribe that there is no path through water to the happy hunting-ground? Yet
her face was impassive; she was the daughter of a chief, she must die as a
chief's daughter, it is enough.


They had caught her boarding the pirate ship with a knife in her mouth. No
watch was kept on the ship, it being Hook's boast that the wind of his
name guarded the ship for a mile around. Now her fate would help to guard
it also. One more wail would go the round in that wind by night.


In the gloom that they brought with them the two pirates did not see the
rock till they crashed into it.


"Luff, you lubber," cried an Irish voice that was Smee's; "here's the
rock. Now, then, what we have to do is to hoist the redskin on to it and
leave her here to drown."


It was the work of one brutal moment to land the beautiful girl on the
rock; she was too proud to offer a vain resistance.


Quite near the rock, but out of sight, two heads were bobbing up and down,
Peter's and Wendy's. Wendy was crying, for it was the first tragedy she
had seen. Peter had seen many tragedies, but he had forgotten them all. He
was less sorry than Wendy for Tiger Lily: it was two against one that
angered him, and he meant to save her. An easy way would have been to wait
until the pirates had gone, but he was never one to choose the easy way.


There was almost nothing he could not do, and he now imitated the voice of
Hook.


"Ahoy there, you lubbers!" he called. It was a marvellous imitation.


"The captain!" said the pirates, staring at each other in surprise.


"He must be swimming out to us," Starkey said, when they had looked for
him in vain.


"We are putting the redskin on the rock," Smee called out.


"Set her free," came the astonishing answer.


"Free!"


"Yes, cut her bonds and let her go."


"But, captain—"


"At once, d'ye hear," cried Peter, "or I'll plunge my hook in you."


"This is queer!" Smee gasped.


"Better do what the captain orders," said Starkey nervously.


"Ay, ay." Smee said, and he cut Tiger Lily's cords. At once like an eel
she slid between Starkey's legs into the water.


Of course Wendy was very elated over Peter's cleverness; but she knew that
he would be elated also and very likely crow and thus betray himself, so
at once her hand went out to cover his mouth. But it was stayed even in
the act, for "Boat ahoy!" rang over the lagoon in Hook's voice, and this
time it was not Peter who had spoken.


Peter may have been about to crow, but his face puckered in a whistle of
surprise instead.


"Boat ahoy!" again came the voice.


Now Wendy understood. The real Hook was also in the water.


He was swimming to the boat, and as his men showed a light to guide him he
had soon reached them. In the light of the lantern Wendy saw his hook grip
the boat's side; she saw his evil swarthy face as he rose dripping from
the water, and, quaking, she would have liked to swim away, but Peter
would not budge. He was tingling with life and also top-heavy with
conceit. "Am I not a wonder, oh, I am a wonder!" he whispered to her, and
though she thought so also, she was really glad for the sake of his
reputation that no one heard him except herself.


He signed to her to listen.


The two pirates were very curious to know what had brought their captain
to them, but he sat with his head on his hook in a position of profound
melancholy.


"Captain, is all well?" they asked timidly, but he answered with a hollow
moan.


"He sighs," said Smee.


"He sighs again," said Starkey.


"And yet a third time he sighs," said Smee.


Then at last he spoke passionately.


"The game's up," he cried, "those boys have found a mother."


Affrighted though she was, Wendy swelled with pride.


"O evil day!" cried Starkey.


"What's a mother?" asked the ignorant Smee.


Wendy was so shocked that she exclaimed. "He doesn't know!" and always
after this she felt that if you could have a pet pirate Smee would be her
one.


Peter pulled her beneath the water, for Hook had started up, crying, "What
was that?"


"I heard nothing," said Starkey, raising the lantern over the waters, and
as the pirates looked they saw a strange sight. It was the nest I have
told you of, floating on the lagoon, and the Never bird was sitting on it.


"See," said Hook in answer to Smee's question, "that is a mother. What a
lesson! The nest must have fallen into the water, but would the mother
desert her eggs? No."


There was a break in his voice, as if for a moment he recalled innocent
days when—but he brushed away this weakness with his hook.


Smee, much impressed, gazed at the bird as the nest was borne past, but
the more suspicious Starkey said, "If she is a mother, perhaps she is
hanging about here to help Peter."


Hook winced. "Ay," he said, "that is the fear that haunts me."


He was roused from this dejection by Smee's eager voice.


"Captain," said Smee, "could we not kidnap these boys' mother and make her
our mother?"


"It is a princely scheme," cried Hook, and at once it took practical shape
in his great brain. "We will seize the children and carry them to the
boat: the boys we will make walk the plank, and Wendy shall be our
mother."


Again Wendy forgot herself.


"Never!" she cried, and bobbed.


"What was that?"


But they could see nothing. They thought it must have been a leaf in the
wind. "Do you agree, my bullies?" asked Hook.


"There is my hand on it," they both said.


"And there is my hook. Swear."


They all swore. By this time they were on the rock, and suddenly Hook
remembered Tiger Lily.


"Where is the redskin?" he demanded abruptly.


He had a playful humour at moments, and they thought this was one of the
moments.


"That is all right, captain," Smee answered complacently; "we let her go."


"Let her go!" cried Hook.


"'Twas your own orders," the bo'sun faltered.


"You called over the water to us to let her go," said Starkey.


"Brimstone and gall," thundered Hook, "what cozening is going
on here!" His face had gone black with rage, but he saw that they believed
their words, and he was startled. "Lads," he said, shaking a little, "I
gave no such order."


"It is passing queer," Smee said, and they all fidgeted uncomfortably.
Hook raised his voice, but there was a quiver in it.


"Spirit that haunts this dark lagoon to-night," he cried, "dost hear me?"


Of course Peter should have kept quiet, but of course he did not. He
immediately answered in Hook's voice:


"Odds, bobs, hammer and tongs, I hear you."


In that supreme moment Hook did not blanch, even at the gills, but Smee
and Starkey clung to each other in terror.


"Who are you, stranger? Speak!" Hook demanded.


"I am James Hook," replied the voice, "captain of the \emph{Jolly Roger}."


"You are not; you are not," Hook cried hoarsely.


"Brimstone and gall," the voice retorted, "say that again, and I'll cast
anchor in you."


Hook tried a more ingratiating manner. "If you are Hook," he said almost
humbly, "come tell me, who am I?"


"A codfish," replied the voice, "only a codfish."


"A codfish!" Hook echoed blankly, and it was then, but not till then, that
his proud spirit broke. He saw his men draw back from him.


"Have we been captained all this time by a codfish!" they muttered. "It is
lowering to our pride."


They were his dogs snapping at him, but, tragic figure though he had
become, he scarcely heeded them. Against such fearful evidence it was not
their belief in him that he needed, it was his own. He felt his ego
slipping from him. "Don't desert me, bully," he whispered hoarsely to it.


In his dark nature there was a touch of the feminine, as in all the great
pirates, and it sometimes gave him intuitions. Suddenly he tried the
guessing game.


"Hook," he called, "have you another voice?"


Now Peter could never resist a game, and he answered blithely in his own
voice, "I have."


"And another name?"


"Ay, ay."


"Vegetable?" asked Hook.


"No."


"Mineral?"


"No."


"Animal?"


"Yes."


"Man?"


"No!" This answer rang out scornfully.


"Boy?"


"Yes."


"Ordinary boy?"


"No!"


"Wonderful boy?"


To Wendy's pain the answer that rang out this time was "Yes."


"Are you in England?"


"No."


"Are you here?"


"Yes."


Hook was completely puzzled. "You ask him some questions," he said to the
others, wiping his damp brow.


Smee reflected. "I can't think of a thing," he said regretfully.


"Can't guess, can't guess!" crowed Peter. "Do you give it up?"


Of course in his pride he was carrying the game too far, and the
miscreants saw their chance.


"Yes, yes," they answered eagerly.


"Well, then," he cried, "I am Peter Pan."


Pan!


In a moment Hook was himself again, and Smee and Starkey were his faithful
henchmen.


"Now we have him," Hook shouted. "Into the water, Smee. Starkey, mind the
boat. Take him dead or alive!"


He leaped as he spoke, and simultaneously came the gay voice of Peter.


"Are you ready, boys?"


"Ay, ay," from various parts of the lagoon.


"Then lam into the pirates."


The fight was short and sharp. First to draw blood was John, who gallantly
climbed into the boat and held Starkey. There was fierce struggle, in
which the cutlass was torn from the pirate's grasp. He wriggled overboard
and John leapt after him. The dinghy drifted away.


Here and there a head bobbed up in the water, and there was a flash of
steel followed by a cry or a whoop. In the confusion some struck at their
own side. The corkscrew of Smee got Tootles in the fourth rib, but he was
himself pinked in turn by Curly. Farther from the rock Starkey
was pressing Slightly and the twins hard.


Where all this time was Peter? He was seeking bigger game.


The others were all brave boys, and they must not be blamed for backing
from the pirate captain. His iron claw made a circle of dead water round
him, from which they fled like affrighted fishes.


But there was one who did not fear him: there was one prepared to enter
that circle.


Strangely, it was not in the water that they met. Hook rose to the rock to
breathe, and at the same moment Peter scaled it on the opposite side. The
rock was slippery as a ball, and they had to crawl rather than climb.
Neither knew that the other was coming. Each feeling for a grip met the
other's arm: in surprise they raised their heads; their faces were almost
touching; so they met.


Some of the greatest heroes have confessed that just before they fell to
they had a sinking. Had it been so
with Peter at that moment I would admit it. After all, he was the only man
that the Sea-Cook had feared. But Peter had no sinking, he had one feeling
only, gladness; and he gnashed his pretty teeth with joy. Quick as thought
he snatched a knife from Hook's belt and was about to drive it home, when
he saw that he was higher up the rock that his foe. It would not have been
fighting fair. He gave the pirate a hand to help him up.


It was then that Hook bit him.


Not the pain of this but its unfairness was what dazed Peter. It made him
quite helpless. He could only stare, horrified. Every child is affected
thus the first time he is treated unfairly. All he thinks he has a right
to when he comes to you to be yours is fairness. After you have been
unfair to him he will love you again, but will never afterwards be quite
the same boy. No one ever gets over the first unfairness; no one except
Peter. He often met it, but he always forgot it. I suppose that was the
real difference between him and all the rest.


So when he met it now it was like the first time; and he could just stare,
helpless. Twice the iron hand clawed him.


A few moments afterwards the other boys saw Hook in the water striking
wildly for the ship; no elation on the pestilent face now, only white
fear, for the crocodile was in dogged pursuit of him. On ordinary
occasions the boys would have swum alongside cheering; but now they were
uneasy, for they had lost both Peter and Wendy, and were scouring the
lagoon for them, calling them by name. They found the dinghy and went home
in it, shouting "Peter, Wendy" as they went, but no answer came save
mocking laughter from the mermaids. "They must be swimming back or
flying," the boys concluded. They were not very anxious, because they had
such faith in Peter. They chuckled, boylike, because they would be late
for bed; and it was all mother Wendy's fault!


When their voices died away there came cold silence over the lagoon, and
then a feeble cry.


"Help, help!"


Two small figures were beating against the rock; the girl had fainted and
lay on the boy's arm. With a last effort Peter pulled her up the rock and
then lay down beside her. Even as he also fainted he saw that the water
was rising. He knew that they would soon be drowned, but he could do no
more.


As they lay side by side a mermaid caught Wendy by the feet, and began
pulling her softly into the water. Peter, feeling her slip from him, woke
with a start, and was just in time to draw her back. But he had to tell
her the truth.


"We are on the rock, Wendy," he said, "but it is growing smaller. Soon the
water will be over it."


She did not understand even now.


"We must go," she said, almost brightly.


"Yes," he answered faintly.


"Shall we swim or fly, Peter?"


He had to tell her.


"Do you think you could swim or fly as far as the island, Wendy, without
my help?"


She had to admit that she was too tired.


He moaned.


"What is it?" she asked, anxious about him at once.


"I can't help you, Wendy. Hook wounded me. I can neither fly nor swim."


"Do you mean we shall both be drowned?"


"Look how the water is rising."


They put their hands over their eyes to shut out the sight. They thought
they would soon be no more. As they sat thus something brushed against
Peter as light as a kiss, and stayed there, as if saying timidly, "Can I
be of any use?"


It was the tail of a kite, which Michael had made some days before. It had
torn itself out of his hand and floated away.


"Michael's kite," Peter said without interest, but next moment he had
seized the tail, and was pulling the kite toward him.


"It lifted Michael off the ground," he cried; "why should it not carry
you?"


"Both of us!"


"It can't lift two; Michael and Curly tried."


"Let us draw lots," Wendy said bravely.


"And you a lady; never." Already he had tied the tail round her. She clung
to him; she refused to go without him; but with a "Good-bye, Wendy," he
pushed her from the rock; and in a few minutes she was borne out of his
sight. Peter was alone on the lagoon.


The rock was very small now; soon it would be submerged. Pale rays of
light tiptoed across the waters; and by and by there was to be heard a
sound at once the most musical and the most melancholy in the world: the
mermaids calling to the moon.


Peter was not quite like other boys; but he was afraid at last. A tremour
ran through him, like a shudder passing over the sea; but on the sea one
shudder follows another till there are hundreds of them, and Peter felt
just the one. Next moment he was standing erect on the rock again, with
that smile on his face and a drum beating within him. It was saying, "To
die will be an awfully big adventure."


% !TEX program = pdflatex
% !TEX encoding = UTF-8
% !TEX spellcheck = en_GB
% !TEX root = peter-pan.tex

\chapter{The Never Bird}

\endinput


Chapter 9 THE NEVER BIRD


The last sound Peter heard before he was quite alone were the mermaids
retiring one by one to their bedchambers under the sea. He was too far
away to hear their doors shut; but every door in the coral caves where
they live rings a tiny bell when it opens or closes (as in all the nicest
houses on the mainland), and he heard the bells.


Steadily the waters rose till they were nibbling at his feet; and to pass
the time until they made their final gulp, he watched the only thing on
the lagoon. He thought it was a piece of floating paper, perhaps part of
the kite, and wondered idly how long it would take to drift ashore.


Presently he noticed as an odd thing that it was undoubtedly out upon the
lagoon with some definite purpose, for it was fighting the tide, and
sometimes winning; and when it won, Peter, always sympathetic to the
weaker side, could not help clapping; it was such a gallant piece of
paper.


It was not really a piece of paper; it was the Never bird, making
desperate efforts to reach Peter on the nest. By working her wings, in a
way she had learned since the nest fell into the water, she was able to
some extent to guide her strange craft, but by the time Peter recognised
her she was very exhausted. She had come to save him, to give him her
nest, though there were eggs in it. I rather wonder at the bird, for
though he had been nice to her, he had also sometimes tormented her. I can
suppose only that, like Mrs. Darling and the rest of them, she was melted
because he had all his first teeth.


She called out to him what she had come for, and he called out to her what
she was doing there; but of course neither of them understood the other's
language. In fanciful stories people can talk to the birds freely, and I
wish for the moment I could pretend that this were such a story, and say
that Peter replied intelligently to the Never bird; but truth is best, and
I want to tell you only what really happened. Well, not only could they
not understand each other, but they forgot their manners.


"I—want—you—to—get—into—the—nest,"
the bird called, speaking as slowly and distinctly as possible, "and—then—you—can—drift—ashore,
but—I—am—too—tired—to—bring—it—any—nearer—so—you—must—try
to—swim—to—it."


"What are you quacking about?" Peter answered. "Why don't you let the nest
drift as usual?"


"I—want—you—" the bird said, and repeated it all over.


Then Peter tried slow and distinct.


"What—are—you—quacking—about?" and so on.


The Never bird became irritated; they have very short tempers.


"You dunderheaded little jay," she screamed, "Why don't you do as I tell
you?"


Peter felt that she was calling him names, and at a venture he retorted
hotly:


"So are you!"


Then rather curiously they both snapped out the same remark:


"Shut up!"


"Shut up!"


Nevertheless the bird was determined to save him if she could, and by one
last mighty effort she propelled the nest against the rock. Then up she
flew; deserting her eggs, so as to make her meaning clear.


Then at last he understood, and clutched the nest and waved his thanks to
the bird as she fluttered overhead. It was not to receive his thanks,
however, that she hung there in the sky; it was not even to watch him get
into the nest; it was to see what he did with her eggs.


There were two large white eggs, and Peter lifted them up and reflected.
The bird covered her face with her wings, so as not to see the last of
them; but she could not help peeping between the feathers.


I forget whether I have told you that there was a stave on the rock,
driven into it by some buccaneers of long ago to mark the site of buried
treasure. The children had discovered the glittering hoard, and when in a
mischievous mood used to fling showers of moidores, diamonds, pearls and
pieces of eight to the gulls, who pounced upon them for food, and then
flew away, raging at the scurvy trick that had been played upon them. The
stave was still there, and on it Starkey had hung his hat, a deep
tarpaulin, watertight, with a broad brim. Peter put the eggs into this hat
and set it on the lagoon. It floated beautifully.


The Never bird saw at once what he was up to, and screamed her admiration
of him; and, alas, Peter crowed his agreement with her. Then he got into
the nest, reared the stave in it as a mast, and hung up his shirt for a
sail. At the same moment the bird fluttered down upon the hat and once
more sat snugly on her eggs. She drifted in one direction, and he was
borne off in another, both cheering.


Of course when Peter landed he beached his barque [small ship, actually
the Never Bird's nest in this particular case in point] in a place where
the bird would easily find it; but the hat was such a great success that
she abandoned the nest. It drifted about till it went to pieces, and often
Starkey came to the shore of the lagoon, and with many bitter feelings
watched the bird sitting on his hat. As we shall not see her again, it may
be worth mentioning here that all Never birds now build in that shape of
nest, with a broad brim on which the youngsters take an airing.


Great were the rejoicings when Peter reached the home under the ground
almost as soon as Wendy, who had been carried hither and thither by the
kite. Every boy had adventures to tell; but perhaps the biggest adventure
of all was that they were several hours late for bed. This so inflated
them that they did various dodgy things to get staying up still longer,
such as demanding bandages; but Wendy, though glorying in having them all
home again safe and sound, was scandalised by the lateness of the hour,
and cried, "To bed, to bed," in a voice that had to be obeyed. Next day,
however, she was awfully tender, and gave out bandages to every one, and
they played till bed-time at limping about and carrying their arms in
slings.


% !TEX program = pdflatex
% !TEX encoding = UTF-8
% !TEX spellcheck = en_GB
% !TEX root = peter-pan.tex

\chapter{The Happy Home}

\endinput

    <h2>
      Chapter 10 THE HAPPY HOME
    </h2>
    <p>
      One important result of the brush [with the pirates] on the lagoon was
      that it made the redskins their friends. Peter had saved Tiger Lily from a
      dreadful fate, and now there was nothing she and her braves would not do
      for him. All night they sat above, keeping watch over the home under the
      ground and awaiting the big attack by the pirates which obviously could
      not be much longer delayed. Even by day they hung about, smoking the pipe
      of peace, and looking almost as if they wanted tit-bits to eat.
    </p>
    <p>
      They called Peter the Great White Father, prostrating themselves [lying
      down] before him; and he liked this tremendously, so that it was not
      really good for him.
    </p>
    <p>
      "The great white father," he would say to them in a very lordly manner, as
      they grovelled at his feet, "is glad to see the Piccaninny warriors
      protecting his wigwam from the pirates."
    </p>
    <p>
      "Me Tiger Lily," that lovely creature would reply. "Peter Pan save me, me
      his velly nice friend. Me no let pirates hurt him."
    </p>
    <p>
      She was far too pretty to cringe in this way, but Peter thought it his
      due, and he would answer condescendingly, "It is good. Peter Pan has
      spoken."
    </p>
    <p>
      Always when he said, "Peter Pan has spoken," it meant that they must now
      shut up, and they accepted it humbly in that spirit; but they were by no
      means so respectful to the other boys, whom they looked upon as just
      ordinary braves. They said "How-do?" to them, and things like that; and
      what annoyed the boys was that Peter seemed to think this all right.
    </p>
    <p>
      Secretly Wendy sympathised with them a little, but she was far too loyal a
      housewife to listen to any complaints against father. "Father knows best,"
      she always said, whatever her private opinion must be. Her private opinion
      was that the redskins should not call her a squaw.
    </p>
    <p>
      We have now reached the evening that was to be known among them as the
      Night of Nights, because of its adventures and their upshot. The day, as
      if quietly gathering its forces, had been almost uneventful, and now the
      redskins in their blankets were at their posts above, while, below, the
      children were having their evening meal; all except Peter, who had gone
      out to get the time. The way you got the time on the island was to find
      the crocodile, and then stay near him till the clock struck.
    </p>
    <p>
      The meal happened to be a make-believe tea, and they sat around the board,
      guzzling in their greed; and really, what with their chatter and
      recriminations, the noise, as Wendy said, was positively deafening. To be
      sure, she did not mind noise, but she simply would not have them grabbing
      things, and then excusing themselves by saying that Tootles had pushed
      their elbow. There was a fixed rule that they must never hit back at
      meals, but should refer the matter of dispute to Wendy by raising the
      right arm politely and saying, "I complain of so-and-so;" but what usually
      happened was that they forgot to do this or did it too much.
    </p>
    <p>
      "Silence," cried Wendy when for the twentieth time she had told them that
      they were not all to speak at once. "Is your mug empty, Slightly darling?"
    </p>
    <p>
      "Not quite empty, mummy," Slightly said, after looking into an imaginary
      mug.
    </p>
    <p>
      "He hasn't even begun to drink his milk," Nibs interposed.
    </p>
    <p>
      This was telling, and Slightly seized his chance.
    </p>
    <p>
      "I complain of Nibs," he cried promptly.
    </p>
    <p>
      John, however, had held up his hand first.
    </p>
    <p>
      "Well, John?"
    </p>
    <p>
      "May I sit in Peter's chair, as he is not here?"
    </p>
    <p>
      "Sit in father's chair, John!" Wendy was scandalised. "Certainly not."
    </p>
    <p>
      "He is not really our father," John answered. "He didn't even know how a
      father does till I showed him."
    </p>
    <p>
      This was grumbling. "We complain of John," cried the twins.
    </p>
    <p>
      Tootles held up his hand. He was so much the humblest of them, indeed he
      was the only humble one, that Wendy was specially gentle with him.
    </p>
    <p>
      "I don't suppose," Tootles said diffidently [bashfully or timidly], "that
      I could be father."
    </p>
    <p>
      "No, Tootles."
    </p>
    <p>
      Once Tootles began, which was not very often, he had a silly way of going
      on.
    </p>
    <p>
      "As I can't be father," he said heavily, "I don't suppose, Michael, you
      would let me be baby?"
    </p>
    <p>
      "No, I won't," Michael rapped out. He was already in his basket.
    </p>
    <p>
      "As I can't be baby," Tootles said, getting heavier and heavier and
      heavier, "do you think I could be a twin?"
    </p>
    <p>
      "No, indeed," replied the twins; "it's awfully difficult to be a twin."
    </p>
    <p>
      "As I can't be anything important," said Tootles, "would any of you like
      to see me do a trick?"
    </p>
    <p>
      "No," they all replied.
    </p>
    <p>
      Then at last he stopped. "I hadn't really any hope," he said.
    </p>
    <p>
      The hateful telling broke out again.
    </p>
    <p>
      "Slightly is coughing on the table."
    </p>
    <p>
      "The twins began with cheese-cakes."
    </p>
    <p>
      "Curly is taking both butter and honey."
    </p>
    <p>
      "Nibs is speaking with his mouth full."
    </p>
    <p>
      "I complain of the twins."
    </p>
    <p>
      "I complain of Curly."
    </p>
    <p>
      "I complain of Nibs."
    </p>
    <p>
      "Oh dear, oh dear," cried Wendy, "I'm sure I sometimes think that
      spinsters are to be envied."
    </p>
    <p>
      She told them to clear away, and sat down to her work-basket, a heavy load
      of stockings and every knee with a hole in it as usual.
    </p>
    <p>
      "Wendy," remonstrated [scolded] Michael, "I'm too big for a cradle."
    </p>
    <p>
      "I must have somebody in a cradle," she said almost tartly, "and you are
      the littlest. A cradle is such a nice homely thing to have about a house."
    </p>
    <p>
      While she sewed they played around her; such a group of happy faces and
      dancing limbs lit up by that romantic fire. It had become a very familiar
      scene, this, in the home under the ground, but we are looking on it for
      the last time.
    </p>
    <p>
      There was a step above, and Wendy, you may be sure, was the first to
      recognize it.
    </p>
    <p>
      "Children, I hear your father's step. He likes you to meet him at the
      door."
    </p>
    <p>
      Above, the redskins crouched before Peter.
    </p>
    <p>
      "Watch well, braves. I have spoken."
    </p>
    <p>
      And then, as so often before, the gay children dragged him from his tree.
      As so often before, but never again.
    </p>
    <p>
      He had brought nuts for the boys as well as the correct time for Wendy.
    </p>
    <p>
      "Peter, you just spoil them, you know," Wendy simpered [exaggerated a
      smile].
    </p>
    <p>
      "Ah, old lady," said Peter, hanging up his gun.
    </p>
    <p>
      "It was me told him mothers are called old lady," Michael whispered to
      Curly.
    </p>
    <p>
      "I complain of Michael," said Curly instantly.
    </p>
    <p>
      The first twin came to Peter. "Father, we want to dance."
    </p>
    <p>
      "Dance away, my little man," said Peter, who was in high good humour.
    </p>
    <p>
      "But we want you to dance."
    </p>
    <p>
      Peter was really the best dancer among them, but he pretended to be
      scandalised.
    </p>
    <p>
      "Me! My old bones would rattle!"
    </p>
    <p>
      "And mummy too."
    </p>
    <p>
      "What," cried Wendy, "the mother of such an armful, dance!"
    </p>
    <p>
      "But on a Saturday night," Slightly insinuated.
    </p>
    <p>
      It was not really Saturday night, at least it may have been, for they had
      long lost count of the days; but always if they wanted to do anything
      special they said this was Saturday night, and then they did it.
    </p>
    <p>
      "Of course it is Saturday night, Peter," Wendy said, relenting.
    </p>
    <p>
      "People of our figure, Wendy!"
    </p>
    <p>
      "But it is only among our own progeny [children]."
    </p>
    <p>
      "True, true."
    </p>
    <p>
      So they were told they could dance, but they must put on their nighties
      first.
    </p>
    <p>
      "Ah, old lady," Peter said aside to Wendy, warming himself by the fire and
      looking down at her as she sat turning a heel, "there is nothing more
      pleasant of an evening for you and me when the day's toil is over than to
      rest by the fire with the little ones near by."
    </p>
    <p>
      "It is sweet, Peter, isn't it?" Wendy said, frightfully gratified. "Peter,
      I think Curly has your nose."
    </p>
    <p>
      "Michael takes after you."
    </p>
    <p>
      She went to him and put her hand on his shoulder.
    </p>
    <p>
      "Dear Peter," she said, "with such a large family, of course, I have now
      passed my best, but you don't want to [ex]change me, do you?"
    </p>
    <p>
      "No, Wendy."
    </p>
    <p>
      Certainly he did not want a change, but he looked at her uncomfortably,
      blinking, you know, like one not sure whether he was awake or asleep.
    </p>
    <p>
      "Peter, what is it?"
    </p>
    <p>
      "I was just thinking," he said, a little scared. "It is only make-believe,
      isn't it, that I am their father?"
    </p>
    <p>
      "Oh yes," Wendy said primly [formally and properly].
    </p>
    <p>
      "You see," he continued apologetically, "it would make me seem so old to
      be their real father."
    </p>
    <p>
      "But they are ours, Peter, yours and mine."
    </p>
    <p>
      "But not really, Wendy?" he asked anxiously.
    </p>
    <p>
      "Not if you don't wish it," she replied; and she distinctly heard his sigh
      of relief. "Peter," she asked, trying to speak firmly, "what are your
      exact feelings to [about] me?"
    </p>
    <p>
      "Those of a devoted son, Wendy."
    </p>
    <p>
      "I thought so," she said, and went and sat by herself at the extreme end
      of the room.
    </p>
    <p>
      "You are so queer," he said, frankly puzzled, "and Tiger Lily is just the
      same. There is something she wants to be to me, but she says it is not my
      mother."
    </p>
    <p>
      "No, indeed, it is not," Wendy replied with frightful emphasis. Now we
      know why she was prejudiced against the redskins.
    </p>
    <p>
      "Then what is it?"
    </p>
    <p>
      "It isn't for a lady to tell."
    </p>
    <p>
      "Oh, very well," Peter said, a little nettled. "Perhaps Tinker Bell will
      tell me."
    </p>
    <p>
      "Oh yes, Tinker Bell will tell you," Wendy retorted scornfully. "She is an
      abandoned little creature."
    </p>
    <p>
      Here Tink, who was in her bedroom, eavesdropping, squeaked out something
      impudent.
    </p>
    <p>
      "She says she glories in being abandoned," Peter interpreted.
    </p>
    <p>
      He had a sudden idea. "Perhaps Tink wants to be my mother?"
    </p>
    <p>
      "You silly ass!" cried Tinker Bell in a passion.
    </p>
    <p>
      She had said it so often that Wendy needed no translation.
    </p>
    <p>
      "I almost agree with her," Wendy snapped. Fancy Wendy snapping! But she
      had been much tried, and she little knew what was to happen before the
      night was out. If she had known she would not have snapped.
    </p>
    <p>
      None of them knew. Perhaps it was best not to know. Their ignorance gave
      them one more glad hour; and as it was to be their last hour on the
      island, let us rejoice that there were sixty glad minutes in it. They sang
      and danced in their night-gowns. Such a deliciously creepy song it was, in
      which they pretended to be frightened at their own shadows, little witting
      that so soon shadows would close in upon them, from whom they would shrink
      in real fear. So uproariously gay was the dance, and how they buffeted
      each other on the bed and out of it! It was a pillow fight rather than a
      dance, and when it was finished, the pillows insisted on one bout more,
      like partners who know that they may never meet again. The stories they
      told, before it was time for Wendy's good-night story! Even Slightly tried
      to tell a story that night, but the beginning was so fearfully dull that
      it appalled not only the others but himself, and he said happily:
    </p>
    <p>
      "Yes, it is a dull beginning. I say, let us pretend that it is the end."
    </p>
    <p>
      And then at last they all got into bed for Wendy's story, the story they
      loved best, the story Peter hated. Usually when she began to tell this
      story he left the room or put his hands over his ears; and possibly if he
      had done either of those things this time they might all still be on the
      island. But to-night he remained on his stool; and we shall see what
      happened.
    </p>

% !TEX program = pdflatex
% !TEX encoding = UTF-8
% !TEX spellcheck = en_GB
% !TEX root = peter-pan.tex

\chapter{Wendy’s Story}

"Listen, then," said Wendy, settling down to her story,
with Michael at her feet and seven boys in the bed.
"There was once a gentleman—"

"I had rather he had been a lady," Curly said.

"I wish he had been a white rat," said Nibs.

"Quiet," their mother admonished them.
"There was a lady also, and—"

"Oh, mummy," cried the first twin,
"you mean that there is a lady also, don't you?
She is not dead, is she?"

"Oh, no."

"I am awfully glad she isn't dead," said Tootles.
"Are you glad, John?"

"Of course I am."

"Are you glad, Nibs?"

"Rather."

"Are you glad, Twins?"

"We are glad."

"Oh dear," sighed Wendy.

"Little less noise there," Peter called out,
determined that she should have fair play,
however beastly a story it might be in his opinion.

"The gentleman's name," Wendy continued, "was Mr.\@ Darling, and her name was Mrs.\@ Darling."

"I knew them," John said, to annoy the others.

"I think I knew them," said Michael rather doubtfully.

"They were married, you know," explained Wendy,
"and what do you think they had?"

"White rats," cried Nibs, inspired.

"No."

"It's awfully puzzling," said Tootles,
who knew the story by heart.

"Quiet, Tootles.
They had three descendants."

"What is descendants?"

"Well, you are one, Twin."

"Did you hear that, John?
I am a descendant."

"Descendants are only children," said John.

"Oh dear, oh dear," sighed Wendy.
"Now these three children had a faithful nurse called Nana;
but Mr.\@ Darling was angry with her and chained her up in the yard,
and so all the children flew away."

"It's an awfully good story," said Nibs.

"They flew away," Wendy continued, "to the Neverland,
where the lost children are."

"I just thought they did," Curly broke in excitedly.
"I don't know how it is, but I just thought they did!"

"O Wendy," cried Tootles, "was one of the lost children called Tootles?"

"Yes, he was."

"I am in a story.
Hurrah, I am in a story, Nibs."

"Hush.
Now I want you to consider the feelings of the unhappy parents with all their children flown away."

"Oo!\@" they all moaned,
though they were not really considering the feelings of the unhappy parents one jot.

"Think of the empty beds!"

"Oo!"

"It's awfully sad," the first twin said cheerfully.

"I don't see how it can have a happy ending," said the second twin.
"Do you, Nibs?"

"I'm frightfully anxious."

"If you knew how great is a mother's love," Wendy told them triumphantly,
"you would have no fear."
She had now come to the part that Peter hated.

"I do like a mother's love," said Tootles, hitting Nibs with a pillow.
"Do you like a mother's love, Nibs?"

"I do just," said Nibs, hitting back.

"You see," Wendy said complacently,
"our heroine knew that the mother would always leave the window open for her children to fly back by;
so they stayed away for years and had a lovely time."

"Did they ever go back?"

"Let us now," said Wendy, bracing herself up for her finest effort, "take a peep into the future;"
and they all gave themselves the twist that makes peeps into the future easier.
"Years have rolled by, and who is this elegant lady of uncertain age alighting at London Station?"

"O Wendy, who is she?\@" cried Nibs, every bit as excited as if he didn't know.

"Can it be—yes—no—it is—the fair Wendy!"

"Oh!"

"And who are the two noble portly figures accompanying her, now grown to man's estate?
Can they be John and Michael?
They are!"

"Oh!"

"'See, dear brothers,' says Wendy pointing upwards,
'there is the window still standing open.
Ah, now we are rewarded for our sublime faith in a mother's love.'
So up they flew to their mummy and daddy,
and pen cannot describe the happy scene, over which we draw a veil."

That was the story, and they were as pleased with it as the fair narrator herself.
Everything just as it should be, you see.
Off we skip like the most heartless things in the world,
which is what children are, but so attractive;
and we have an entirely selfish time, and then when we have need of special attention we nobly return for it,
confident that we shall be rewarded instead of smacked.

So great indeed was their faith in a mother's love that they felt they could afford to be callous for a bit longer.

But there was one there who knew better,
and when Wendy finished he uttered a hollow groan.

"What is it, Peter?\@" she cried, running to him, thinking he was ill.
She felt him solicitously, lower down than his chest.
"Where is it, Peter?"

"It isn't that kind of pain," Peter replied darkly.

"Then what kind is it?"

"Wendy, you are wrong about mothers."

They all gathered round him in affright, so alarming was his agitation;
and with a fine candour he told them what he had hitherto concealed.

"Long ago," he said, "I thought like you that my mother would always keep the window open for me,
so I stayed away for moons and moons and moons, and then flew back;
but the window was barred,
for mother had forgotten all about me,
and there was another little boy sleeping in my bed."

I am not sure that this was true, but Peter thought it was true;
and it scared them.

"Are you sure mothers are like that?"

"Yes."

So this was the truth about mothers.
The toads!

Still it is best to be careful;
and no one knows so quickly as a child when he should give in.
"Wendy, let us go home," cried John and Michael together.

"Yes," she said, clutching them.

"Not to-night?\@" asked the lost boys bewildered.
They knew in what they called their hearts that one can get on quite well without a mother,
and that it is only the mothers who think you can't.

"At once," Wendy replied resolutely,
for the horrible thought had come to her:
"Perhaps mother is in half mourning by this time."

This dread made her forgetful of what must be Peter's feelings,
and she said to him rather sharply, "Peter, will you make the necessary arrangements?"

"If you wish it," he replied, as coolly as if she had asked him to pass the nuts.

Not so much as a sorry-to-lose-you between them!
If she did not mind the parting, he was going to show her, was Peter, that neither did he.

But of course he cared very much;
and he was so full of wrath against grown-ups,
who, as usual, were spoiling everything,
that as soon as he got inside his tree he breathed intentionally quick short breaths
at the rate of about five to a second.
He did this because there is a saying in the Neverland that, every time you breathe, a grown-up dies;
and Peter was killing them off vindictively as fast as possible.

Then having given the necessary instructions to the redskins he returned to the home,
where an unworthy scene had been enacted in his absence.
Panic-stricken at the thought of losing Wendy the lost boys had advanced upon her threateningly.

"It will be worse than before she came," they cried.

"We shan't let her go."

"Let's keep her prisoner."

"Ay, chain her up."

In her extremity an instinct told her to which of them to turn.

"Tootles," she cried, "I appeal to you."

Was it not strange?
She appealed to Tootles, quite the silliest one.

Grandly, however, did Tootles respond.
For that one moment he dropped his silliness and spoke with dignity.

"I am just Tootles," he said, "and nobody minds me.
But the first who does not behave to Wendy like an English gentleman I will blood him severely."

He drew back his hanger;
and for that instant his sun was at noon.
The others held back uneasily.
Then Peter returned, and they saw at once that they would get no support from him.
He would keep no girl in the Neverland against her will.

"Wendy," he said, striding up and down,
"I have asked the redskins to guide you through the wood,
as flying tires you so."

"Thank you, Peter."

"Then," he continued, in the short sharp voice of one accustomed to be obeyed,
"Tinker Bell will take you across the sea.
Wake her, Nibs."

Nibs had to knock twice before he got an answer,
though Tink had really been sitting up in bed listening for some time.

"Who are you?
How dare you?
Go away," she cried.

"You are to get up, Tink," Nibs called, "and take Wendy on a journey."

Of course Tink had been delighted to hear that Wendy was going;
but she was jolly well determined not to be her courier,
and she said so in still more offensive language.
Then she pretended to be asleep again.

"She says she won't!\@" Nibs exclaimed, aghast at such insubordination,
whereupon Peter went sternly toward the young lady's chamber.

"Tink," he rapped out,
"if you don't get up and dress at once I will open the curtains,
and then we shall all see you in your \emph{négligée}."

This made her leap to the floor.
"Who said I wasn't getting up?\@" she cried.

In the meantime the boys were gazing very forlornly at Wendy,
now equipped with John and Michael for the journey.
By this time they were dejected,
not merely because they were about to lose her,
but also because they felt that she was going off to something nice to which they had not been invited.
Novelty was beckoning to them as usual.

Crediting them with a nobler feeling Wendy melted.

"Dear ones," she said, "if you will all come with me
I feel almost sure I can get my father and mother to adopt you."

The invitation was meant specially for Peter,
but each of the boys was thinking exclusively of himself, and at once they jumped with joy.

"But won't they think us rather a handful?\@" Nibs asked in the middle of his jump.

"Oh no," said Wendy, rapidly thinking it out,
"it will only mean having a few beds in the drawing-room;
they can be hidden behind the screens on first Thursdays."

"Peter, can we go?\@" they all cried imploringly.
They took it for granted that if they went he would go also,
but really they scarcely cared.
Thus children are ever ready, when novelty knocks, to desert their dearest ones.

"All right," Peter replied with a bitter smile,
and immediately they rushed to get their things.

"And now, Peter," Wendy said, thinking she had put everything right,
"I am going to give you your medicine before you go."
She loved to give them medicine, and undoubtedly gave them too much.
Of course it was only water, but it was out of a bottle,
and she always shook the bottle and counted the drops, which gave it a certain medicinal quality.
On this occasion, however, she did not give Peter his draught,
for just as she had prepared it, she saw a look on his face that made her heart sink.

"Get your things, Peter," she cried, shaking.

"No," he answered, pretending indifference, "I am not going with you, Wendy."

"Yes, Peter."

"No."

To show that her departure would leave him unmoved,
he skipped up and down the room, playing gaily on his heartless pipes.
She had to run about after him, though it was rather undignified.

"To find your mother," she coaxed.

Now, if Peter had ever quite had a mother, he no longer missed her.
He could do very well without one.
He had thought them out, and remembered only their bad points.

"No, no," he told Wendy decisively;
"perhaps she would say I was old, and I just want always to be a little boy and to have fun."

"But, Peter—"

"No."

And so the others had to be told.

"Peter isn't coming."

Peter not coming!
They gazed blankly at him, their sticks over their backs, and on each stick a bundle.
Their first thought was that if Peter was not going he had probably changed his mind about letting them go.

But he was far too proud for that.
"If you find your mothers," he said darkly, "I hope you will like them."

The awful cynicism of this made an uncomfortable impression,
and most of them began to look rather doubtful.
After all, their faces said, were they not noodles to want to go?

"Now then," cried Peter, "no fuss, no blubbering;
good-bye, Wendy;"
and he held out his hand cheerily,
quite as if they must really go now, for he had something important to do.

She had to take his hand, and there was no indication that he would prefer a thimble.

"You will remember about changing your flannels, Peter?\@" she said, lingering over him.
She was always so particular about their flannels.

"Yes."

"And you will take your medicine?"

"Yes."

That seemed to be everything, and an awkward pause followed.
Peter, however, was not the kind that breaks down before other people.
"Are you ready, Tinker Bell?\@" he called out.

"Ay, ay."

"Then lead the way."

Tink darted up the nearest tree;
but no one followed her, for it was at this moment that the pirates made their dreadful attack upon the redskins.
Above, where all had been so still, the air was rent with shrieks and the clash of steel.
Below, there was dead silence.
Mouths opened and remained open.
Wendy fell on her knees, but her arms were extended toward Peter.
All arms were extended to him, as if suddenly blown in his direction;
they were beseeching him mutely not to desert them.
As for Peter, he seized his sword, the same he thought he had slain Barbecue with,
and the lust of battle was in his eye.

\endinput

% !TEX program = pdflatex
% !TEX encoding = UTF-8
% !TEX spellcheck = en_GB
% !TEX root = peter-pan.tex

\chapter{The Children are Carried Off}

The pirate attack had been a complete surprise:
a sure proof that the unscrupulous Hook had conducted it improperly,
for to surprise redskins fairly is beyond the wit of the white man.

By all the unwritten laws of savage warfare it is always the redskin who attacks,
and with the wiliness of his race he does it just before the dawn,
at which time he knows the courage of the whites to be at its lowest ebb.
The white men have in the meantime made a rude stockade on the summit of yonder undulating ground,
at the foot of which a stream runs, for it is destruction to be too far from water.
There they await the onslaught,
the inexperienced ones clutching their revolvers and treading on twigs,
but the old hands sleeping tranquilly until just before the dawn.
Through the long black night the savage scouts wriggle, snake-like, among the grass without stirring a blade.
The brushwood closes behind them, as silently as sand into which a mole has dived.
Not a sound is to be heard, save when they give vent to a wonderful imitation of the lonely call of the coyote.
The cry is answered by other braves;
and some of them do it even better than the coyotes, who are not very good at it.
So the chill hours wear on,
and the long suspense is horribly trying to the paleface who has to live through it for the first time;
but to the trained hand those ghastly calls and still ghastlier silences
are but an intimation of how the night is marching.

That this was the usual procedure was so well known to Hook
that in disregarding it he cannot be excused on the plea of ignorance.

The Piccaninnies, on their part, trusted implicitly to his honour,
and their whole action of the night stands out in marked contrast to his.
They left nothing undone that was consistent with the reputation of their tribe.
With that alertness of the senses which is at once the marvel and despair of civilised peoples,
they knew that the pirates were on the island from the moment one of them trod on a dry stick;
and in an incredibly short space of time the coyote cries began.
Every foot of ground between the spot where Hook had landed his forces and the home under the trees
was stealthily examined by braves wearing their moccasins with the heels in front.
They found only one hillock with a stream at its base, so that Hook had no choice;
here he must establish himself and wait for just before the dawn.
Everything being thus mapped out with almost diabolical cunning,
the main body of the redskins folded their blankets around them,
and in the phlegmatic manner that is to them, the pearl of manhood squatted above the children's home,
awaiting the cold moment when they should deal pale death.

Here dreaming, though wide-awake, of the exquisite tortures to which they were to put him at break of day,
those confiding savages were found by the treacherous Hook.
From the accounts afterwards supplied by such of the scouts as escaped the carnage,
he does not seem even to have paused at the rising ground,
though it is certain that in that grey light he must have seen it:
no thought of waiting to be attacked appears from first to last to have visited his subtle mind;
he would not even hold off till the night was nearly spent;
on he pounded with no policy but to fall to.
What could the bewildered scouts do, masters as they were of every war-like artifice save this one,
but trot helplessly after him, exposing themselves fatally to view,
while they gave pathetic utterance to the coyote cry.

Around the brave Tiger Lily were a dozen of her stoutest warriors,
and they suddenly saw the perfidious pirates bearing down upon them.
Fell from their eyes then the film through which they had looked at victory.
No more would they torture at the stake.
For them the happy hunting-grounds was now.
They knew it;
but as their father's sons they acquitted themselves.
Even then they had time to gather in a phalanx that would have been hard to break had they risen quickly,
but this they were forbidden to do by the traditions of their race.
It is written that the noble savage must never express surprise in the presence of the white.
Thus terrible as the sudden appearance of the pirates must have been to them,
they remained stationary for a moment, not a muscle moving;
as if the foe had come by invitation.
Then, indeed, the tradition gallantly upheld, they seized their weapons,
and the air was torn with the war-cry;
but it was now too late.

It is no part of ours to describe what was a massacre rather than a fight.
Thus perished many of the flower of the Piccaninny tribe.
Not all unavenged did they die,
for with Lean Wolf fell Alf Mason, to disturb the Spanish Main no more,
and among others who bit the dust were Geo.\@ Scourie, Chas.\@ Turley, and the Alsatian Foggerty.
Turley fell to the tomahawk of the terrible Panther,
who ultimately cut a way through the pirates with Tiger Lily and a small remnant of the tribe.

To what extent Hook is to blame for his tactics on this occasion is for the historian to decide.
Had he waited on the rising ground till the proper hour he and his men would probably have been butchered;
and in judging him it is only fair to take this into account.
What he should perhaps have done was to acquaint his opponents that he proposed to follow a new method.
On the other hand, this, as destroying the element of surprise, would have made his strategy of no avail,
so that the whole question is beset with difficulties.
One cannot at least withhold a reluctant admiration for the wit that had conceived so bold a scheme,
and the fell genius with which it was carried out.

What were his own feelings about himself at that triumphant moment?
Fain would his dogs have known,
as breathing heavily and wiping their cutlasses, they gathered at a discreet distance from his hook,
and squinted through their ferret eyes at this extraordinary man.
Elation must have been in his heart, but his face did not reflect it:
ever a dark and solitary enigma, he stood aloof from his followers in spirit as in substance.

The night's work was not yet over, for it was not the redskins he had come out to destroy;
they were but the bees to be smoked, so that he should get at the honey.
It was Pan he wanted, Pan and Wendy and their band, but chiefly Pan.

Peter was such a small boy that one tends to wonder at the man's hatred of him.
True he had flung Hook's arm to the crocodile,
but even this and the increased insecurity of life to which it led, owing to the crocodile's pertinacity,
hardly account for a vindictiveness so relentless and malignant.
The truth is that there was a something about Peter which goaded the pirate captain to frenzy.
It was not his courage, it was not his engaging appearance, it was not—.
There is no beating about the bush, for we know quite well what it was, and have got to tell.
It was Peter's cockiness.

This had got on Hook's nerves;
it made his iron claw twitch, and at night it disturbed him like an insect.
While Peter lived, the tortured man felt that he was a lion in a cage into which a sparrow had come.

The question now was how to get down the trees, or how to get his dogs down?
He ran his greedy eyes over them, searching for the thinnest ones.
They wriggled uncomfortably, for they knew he would not scruple to ram them down with poles.

In the meantime, what of the boys?
We have seen them at the first clang of the weapons, turned as it were into stone figures, open-mouthed,
all appealing with outstretched arms to Peter;
and we return to them as their mouths close, and their arms fall to their sides.
The pandemonium above has ceased almost as suddenly as it arose, passed like a fierce gust of wind;
but they know that in the passing it has determined their fate.

Which side had won?

The pirates, listening avidly at the mouths of the trees, heard the question put by every boy,
and alas, they also heard Peter's answer.

"If the redskins have won," he said, "they will beat the tom-tom;
it is always their sign of victory."

Now Smee had found the tom-tom, and was at that moment sitting on it.
"You will never hear the tom-tom again," he muttered,
but inaudibly of course, for strict silence had been enjoined.
To his amazement Hook signed him to beat the tom-tom,
and slowly there came to Smee an understanding of the dreadful wickedness of the order.
Never, probably, had this simple man admired Hook so much.

Twice Smee beat upon the instrument, and then stopped to listen gleefully.

"The tom-tom," the miscreants heard Peter cry;
"an Indian victory!"

The doomed children answered with a cheer that was music to the black hearts above,
and almost immediately they repeated their good-byes to Peter.
This puzzled the pirates,
but all their other feelings were swallowed by a base delight that the enemy were about to come up the trees.
They smirked at each other and rubbed their hands.
Rapidly and silently Hook gave his orders:
one man to each tree, and the others to arrange themselves in a line two yards apart.

\endinput

% !TEX program = pdflatex
% !TEX encoding = UTF-8
% !TEX spellcheck = en_GB
% !TEX root = peter-pan.tex

\chapter{Do You Believe in Fairies?}

\endinput

The more quickly this horror is disposed of the better.
The first to emerge from his tree was Curly.
He rose out of it into the arms of Cecco, who flung him to Smee, who flung him to Starkey, who flung him to Bill Jukes, who flung him to Noodler, and so he was tossed from one to another till he fell at the feet of the black pirate.
All the boys were plucked from their trees in this ruthless manner;
and several of them were in the air at a time, like bales of goods flung from hand to hand.

A different treatment was accorded to Wendy, who came last.
With ironical politeness Hook raised his hat to her, and, offering her his arm, escorted her to the spot where the others were being gagged.
He did it with such an air, he was so frightfully \emph{distingué}, that she was too fascinated to cry out.
She was only a little girl.

Perhaps it is tell-tale to divulge that for a moment Hook entranced her, and we tell on her only because her slip led to strange results.
Had she haughtily unhanded him (and we should have loved to write it of her), she would have been hurled through the air like the others, and then Hook would probably not have been present at the tying of the children;
and had he not been at the tying he would not have discovered Slightly's secret, and without the secret he could not presently have made his foul attempt on Peter's life.

They were tied to prevent their flying away, doubled up with their knees close to their ears;
and for the trussing of them the black pirate had cut a rope into nine equal pieces.
All went well until Slightly's turn came, when he was found to be like those irritating parcels that use up all the string in going round and leave no tags with which to tie a knot.
The pirates kicked him in their rage, just as you kick the parcel (though in fairness you should kick the string);
and strange to say it was Hook who told them to belay their violence.
His lip was curled with malicious triumph.
While his dogs were merely sweating because every time they tried to pack the unhappy lad tight in one part he bulged out in another, Hook's master mind had gone far beneath Slightly's surface, probing not for effects but for causes;
and his exultation showed that he had found them.
Slightly, white to the gills, knew that Hook had surprised his secret, which was this, that no boy so blown out could use a tree wherein an average man need stick.
Poor Slightly, most wretched of all the children now, for he was in a panic about Peter, bitterly regretted what he had done.
Madly addicted to the drinking of water when he was hot, he had swelled in consequence to his present girth, and instead of reducing himself to fit his tree he had, unknown to the others, whittled his tree to make it fit him.

Sufficient of this Hook guessed to persuade him that Peter at last lay at his mercy, but no word of the dark design that now formed in the subterranean caverns of his mind crossed his lips;
he merely signed that the captives were to be conveyed to the ship, and that he would be alone.

How to convey them?
Hunched up in their ropes they might indeed be rolled down hill like barrels, but most of the way lay through a morass.
Again Hook's genius surmounted difficulties.
He indicated that the little house must be used as a conveyance.
The children were flung into it, four stout pirates raised it on their shoulders, the others fell in behind, and singing the hateful pirate chorus the strange procession set off through the wood.
I don't know whether any of the children were crying;
if so, the singing drowned the sound;
but as the little house disappeared in the forest, a brave though tiny jet of smoke issued from its chimney as if defying Hook.

Hook saw it, and it did Peter a bad service.
It dried up any trickle of pity for him that may have remained in the pirate's infuriated breast.

The first thing he did on finding himself alone in the fast falling night was to tiptoe to Slightly's tree, and make sure that it provided him with a passage.
Then for long he remained brooding;
his hat of ill omen on the sward, so that any gentle breeze which had arisen might play refreshingly through his hair.
Dark as were his thoughts his blue eyes were as soft as the periwinkle.
Intently he listened for any sound from the nether world, but all was as silent below as above;
the house under the ground seemed to be but one more empty tenement in the void.
Was that boy asleep, or did he stand waiting at the foot of Slightly's tree, with his dagger in his hand?

There was no way of knowing, save by going down.
Hook let his cloak slip softly to the ground, and then biting his lips till a lewd blood stood on them, he stepped into the tree.
He was a brave man, but for a moment he had to stop there and wipe his brow, which was dripping like a candle.
Then, silently, he let himself go into the unknown.

He arrived unmolested at the foot of the shaft, and stood still again, biting at his breath, which had almost left him.
As his eyes became accustomed to the dim light various objects in the home under the trees took shape;
but the only one on which his greedy gaze rested, long sought for and found at last, was the great bed.
On the bed lay Peter fast asleep.

Unaware of the tragedy being enacted above, Peter had continued, for a little time after the children left, to play gaily on his pipes:
no doubt rather a forlorn attempt to prove to himself that he did not care.
Then he decided not to take his medicine, so as to grieve Wendy.
Then he lay down on the bed outside the coverlet, to vex her still more;
for she had always tucked them inside it, because you never know that you may not grow chilly at the turn of the night.
Then he nearly cried;
but it struck him how indignant she would be if he laughed instead;
so he laughed a haughty laugh and fell asleep in the middle of it.

Sometimes, though not often, he had dreams, and they were more painful than the dreams of other boys.
For hours he could not be separated from these dreams, though he wailed piteously in them.
They had to do, I think, with the riddle of his existence.
At such times it had been Wendy's custom to take him out of bed and sit with him on her lap, soothing him in dear ways of her own invention, and when he grew calmer to put him back to bed before he quite woke up, so that he should not know of the indignity to which she had subjected him.
But on this occasion he had fallen at once into a dreamless sleep.
One arm dropped over the edge of the bed, one leg was arched, and the unfinished part of his laugh was stranded on his mouth, which was open, showing the little pearls.

Thus defenceless Hook found him.
He stood silent at the foot of the tree looking across the chamber at his enemy.
Did no feeling of compassion disturb his sombre breast?
The man was not wholly evil;
he loved flowers (I have been told) and sweet music (he was himself no mean performer on the harpsichord);
and, let it be frankly admitted, the idyllic nature of the scene stirred him profoundly.
Mastered by his better self he would have returned reluctantly up the tree, but for one thing.

What stayed him was Peter's impertinent appearance as he slept.
The open mouth, the drooping arm, the arched knee:
they were such a personification of cockiness as, taken together, will never again, one may hope, be presented to eyes so sensitive to their offensiveness.
They steeled Hook's heart.
If his rage had broken him into a hundred pieces every one of them would have disregarded the incident, and leapt at the sleeper.

Though a light from the one lamp shone dimly on the bed, Hook stood in darkness himself, and at the first stealthy step forward he discovered an obstacle, the door of Slightly's tree.
It did not entirely fill the aperture, and he had been looking over it.
Feeling for the catch, he found to his fury that it was low down, beyond his reach.
To his disordered brain it seemed then that the irritating quality in Peter's face and figure visibly increased, and he rattled the door and flung himself against it.
Was his enemy to escape him after all?

But what was that?
The red in his eye had caught sight of Peter's medicine standing on a ledge within easy reach.
He fathomed what it was straightaway, and immediately knew that the sleeper was in his power.

Lest he should be taken alive, Hook always carried about his person a dreadful drug, blended by himself of all the death-dealing rings that had come into his possession.
These he had boiled down into a yellow liquid quite unknown to science, which was probably the most virulent poison in existence.

Five drops of this he now added to Peter's cup.
His hand shook, but it was in exultation rather than in shame.
As he did it he avoided glancing at the sleeper, but not lest pity should unnerve him;
merely to avoid spilling.
Then one long gloating look he cast upon his victim, and turning, wormed his way with difficulty up the tree.
As he emerged at the top he looked the very spirit of evil breaking from its hole.
Donning his hat at its most rakish angle, he wound his cloak around him, holding one end in front as if to conceal his person from the night, of which it was the blackest part, and muttering strangely to himself, stole away through the trees.

Peter slept on.
The light guttered and went out, leaving the tenement in darkness;
but still he slept.
It must have been not less than ten o'clock by the crocodile, when he suddenly sat up in his bed, wakened by he knew not what.
It was a soft cautious tapping on the door of his tree.

Soft and cautious, but in that stillness it was sinister.
Peter felt for his dagger till his hand gripped it.
Then he spoke.

"Who is that?"

For long there was no answer:
then again the knock.

"Who are you?"

No answer.

He was thrilled, and he loved being thrilled.
In two strides he reached the door.
Unlike Slightly's door, it filled the aperture, so that he could not see beyond it, nor could the one knocking see him.

"I won't open unless you speak," Peter cried.

Then at last the visitor spoke, in a lovely bell-like voice.

"Let me in, Peter."

It was Tink, and quickly he unbarred to her.
She flew in excitedly, her face flushed and her dress stained with mud.

"What is it?"

"Oh, you could never guess!\@" she cried, and offered him three guesses.
"Out with it!\@" he shouted, and in one ungrammatical sentence, as long as the ribbons that conjurers pull from their mouths, she told of the capture of Wendy and the boys.

Peter's heart bobbed up and down as he listened.
Wendy bound, and on the pirate ship;
she who loved everything to be just so!

"I'll rescue her!\@" he cried, leaping at his weapons.
As he leapt he thought of something he could do to please her.
He could take his medicine.

His hand closed on the fatal draught.

"No!\@" shrieked Tinker Bell, who had heard Hook mutter about his deed as he sped through the forest.

"Why not?"

"It is poisoned."

"Poisoned?
Who could have poisoned it?"

"Hook."

"Don't be silly.
How could Hook have got down here?"

Alas, Tinker Bell could not explain this, for even she did not know the dark secret of Slightly's tree.
Nevertheless Hook's words had left no room for doubt.
The cup was poisoned.

"Besides," said Peter, quite believing himself "I never fell asleep."

He raised the cup.
No time for words now;
time for deeds;
and with one of her lightning movements Tink got between his lips and the draught, and drained it to the dregs.

"Why, Tink, how dare you drink my medicine?"

But she did not answer.
Already she was reeling in the air.

"What is the matter with you?\@" cried Peter, suddenly afraid.

"It was poisoned, Peter," she told him softly;
"and now I am going to be dead."

"O Tink, did you drink it to save me?"

"Yes."

"But why, Tink?"

Her wings would scarcely carry her now, but in reply she alighted on his shoulder and gave his nose a loving bite.
She whispered in his ear "You silly ass," and then, tottering to her chamber, lay down on the bed.

His head almost filled the fourth wall of her little room as he knelt near her in distress.
Every moment her light was growing fainter;
and he knew that if it went out she would be no more.
She liked his tears so much that she put out her beautiful finger and let them run over it.

Her voice was so low that at first he could not make out what she said.
Then he made it out.
She was saying that she thought she could get well again if children believed in fairies.

Peter flung out his arms.
There were no children there, and it was night time;
but he addressed all who might be dreaming of the Neverland, and who were therefore nearer to him than you think:
boys and girls in their nighties, and naked papooses in their baskets hung from trees.

"Do you believe?\@" he cried.

Tink sat up in bed almost briskly to listen to her fate.

She fancied she heard answers in the affirmative, and then again she wasn't sure.

"What do you think?\@" she asked Peter.

"If you believe," he shouted to them, "clap your hands;
don't let Tink die."

Many clapped.

Some didn't.

A few beasts hissed.

The clapping stopped suddenly;
as if countless mothers had rushed to their nurseries to see what on earth was happening;
but already Tink was saved.
First her voice grew strong, then she popped out of bed, then she was flashing through the room more merry and impudent than ever.
She never thought of thanking those who believed, but she would have like to get at the ones who had hissed.

"And now to rescue Wendy!"

The moon was riding in a cloudy heaven when Peter rose from his tree, begirt with weapons and wearing little else, to set out upon his perilous quest.
It was not such a night as he would have chosen.
He had hoped to fly, keeping not far from the ground so that nothing unwonted should escape his eyes;
but in that fitful light to have flown low would have meant trailing his shadow through the trees, thus disturbing birds and acquainting a watchful foe that he was astir.

He regretted now that he had given the birds of the island such strange names that they are very wild and difficult of approach.

There was no other course but to press forward in redskin fashion, at which happily he was an adept.
But in what direction, for he could not be sure that the children had been taken to the ship?
A light fall of snow had obliterated all footmarks;
and a deathly silence pervaded the island, as if for a space Nature stood still in horror of the recent carnage.
He had taught the children something of the forest lore that he had himself learned from Tiger Lily and Tinker Bell, and knew that in their dire hour they were not likely to forget it.
Slightly, if he had an opportunity, would blaze the trees, for instance, Curly would drop seeds, and Wendy would leave her handkerchief at some important place.
The morning was needed to search for such guidance, and he could not wait.
The upper world had called him, but would give no help.

The crocodile passed him, but not another living thing, not a sound, not a movement;
and yet he knew well that sudden death might be at the next tree, or stalking him from behind.

He swore this terrible oath:
"Hook or me this time."

Now he crawled forward like a snake, and again erect, he darted across a space on which the moonlight played, one finger on his lip and his dagger at the ready.
He was frightfully happy.

% !TEX program = pdflatex
% !TEX encoding = UTF-8
% !TEX spellcheck = en_GB
% !TEX root = peter-pan.tex

\chapter{The Pirate Ship}

\endinput


Chapter 14 THE PIRATE SHIP


One green light squinting over Kidd's Creek, which is near the mouth of
the pirate river, marked where the brig, the JOLLY ROGER, lay, low in the
water; a rakish-looking [speedy-looking] craft foul to the hull, every
beam in her detestable, like ground strewn with mangled feathers. She was
the cannibal of the seas, and scarce needed that watchful eye, for she
floated immune in the horror of her name.


She was wrapped in the blanket of night, through which no sound from her
could have reached the shore. There was little sound, and none agreeable
save the whir of the ship's sewing machine at which Smee sat, ever
industrious and obliging, the essence of the commonplace, pathetic Smee. I
know not why he was so infinitely pathetic, unless it were because he was
so pathetically unaware of it; but even strong men had to turn hastily
from looking at him, and more than once on summer evenings he had touched
the fount of Hook's tears and made it flow. Of this, as of almost
everything else, Smee was quite unconscious.


A few of the pirates leant over the bulwarks, drinking in the miasma
[putrid mist] of the night; others sprawled by barrels over games of dice
and cards; and the exhausted four who had carried the little house lay
prone on the deck, where even in their sleep they rolled skillfully to
this side or that out of Hook's reach, lest he should claw them
mechanically in passing.


Hook trod the deck in thought. O man unfathomable. It was his hour of
triumph. Peter had been removed for ever from his path, and all the other
boys were in the brig, about to walk the plank. It was his grimmest deed
since the days when he had brought Barbecue to heel; and knowing as we do
how vain a tabernacle is man, could we be surprised had he now paced the
deck unsteadily, bellied out by the winds of his success?


But there was no elation in his gait, which kept pace with the action of
his sombre mind. Hook was profoundly dejected.


He was often thus when communing with himself on board ship in the
quietude of the night. It was because he was so terribly alone. This
inscrutable man never felt more alone than when surrounded by his dogs.
They were socially inferior to him.


Hook was not his true name. To reveal who he really was would even at this
date set the country in a blaze; but as those who read between the lines
must already have guessed, he had been at a famous public school; and its
traditions still clung to him like garments, with which indeed they are
largely concerned. Thus it was offensive to him even now to board a ship
in the same dress in which he grappled [attacked] her, and he still
adhered in his walk to the school's distinguished slouch. But above all he
retained the passion for good form.


Good form! However much he may have degenerated, he still knew that this
is all that really matters.


From far within him he heard a creaking as of rusty portals, and through
them came a stern tap-tap-tap, like hammering in the night when one cannot
sleep. "Have you been good form to-day?" was their eternal question.


"Fame, fame, that glittering bauble, it is mine," he cried.


"Is it quite good form to be distinguished at anything?" the tap-tap from
his school replied.


"I am the only man whom Barbecue feared," he urged, "and Flint feared
Barbecue."


"Barbecue, Flint—what house?" came the cutting retort.


Most disquieting reflection of all, was it not bad form to think about
good form?


His vitals were tortured by this problem. It was a claw within him sharper
than the iron one; and as it tore him, the perspiration dripped down his
tallow [waxy] countenance and streaked his doublet. Ofttimes he drew his
sleeve across his face, but there was no damming that trickle.


Ah, envy not Hook.


There came to him a presentiment of his early dissolution [death]. It was
as if Peter's terrible oath had boarded the ship. Hook felt a gloomy
desire to make his dying speech, lest presently there should be no time
for it.


"Better for Hook," he cried, "if he had had less ambition!" It was in his
darkest hours only that he referred to himself in the third person.


"No little children to love me!"


Strange that he should think of this, which had never troubled him before;
perhaps the sewing machine brought it to his mind. For long he muttered to
himself, staring at Smee, who was hemming placidly, under the conviction
that all children feared him.


Feared him! Feared Smee! There was not a child on board the brig that
night who did not already love him. He had said horrid things to them and
hit them with the palm of his hand, because he could not hit with his
fist, but they had only clung to him the more. Michael had tried on his
spectacles.


To tell poor Smee that they thought him lovable! Hook itched to do it, but
it seemed too brutal. Instead, he revolved this mystery in his mind: why
do they find Smee lovable? He pursued the problem like the sleuth-hound
that he was. If Smee was lovable, what was it that made him so? A terrible
answer suddenly presented itself—"Good form?"


Had the bo'sun good form without knowing it, which is the best form of
all?


He remembered that you have to prove you don't know you have it before you
are eligible for Pop [an elite social club at Eton].


With a cry of rage he raised his iron hand over Smee's head; but he did
not tear. What arrested him was this reflection:


"To claw a man because he is good form, what would that be?"


"Bad form!"


The unhappy Hook was as impotent [powerless] as he was damp, and he fell
forward like a cut flower.


His dogs thinking him out of the way for a time, discipline instantly
relaxed; and they broke into a bacchanalian [drunken] dance, which brought
him to his feet at once, all traces of human weakness gone, as if a bucket
of water had passed over him.


"Quiet, you scugs," he cried, "or I'll cast anchor in you;" and at once
the din was hushed. "Are all the children chained, so that they cannot fly
away?"


"Ay, ay."


"Then hoist them up."


The wretched prisoners were dragged from the hold, all except Wendy, and
ranged in line in front of him. For a time he seemed unconscious of their
presence. He lolled at his ease, humming, not unmelodiously, snatches of a
rude song, and fingering a pack of cards. Ever and anon the light from his
cigar gave a touch of colour to his face.


"Now then, bullies," he said briskly, "six of you walk the plank to-night,
but I have room for two cabin boys. Which of you is it to be?"


"Don't irritate him unnecessarily," had been Wendy's instructions in the
hold; so Tootles stepped forward politely. Tootles hated the idea of
signing under such a man, but an instinct told him that it would be
prudent to lay the responsibility on an absent person; and though a
somewhat silly boy, he knew that mothers alone are always willing to be
the buffer. All children know this about mothers, and despise them for it,
but make constant use of it.


So Tootles explained prudently, "You see, sir, I don't think my mother
would like me to be a pirate. Would your mother like you to be a pirate,
Slightly?"


He winked at Slightly, who said mournfully, "I don't think so," as if he
wished things had been otherwise. "Would your mother like you to be a
pirate, Twin?"


"I don't think so," said the first twin, as clever as the others. "Nibs,
would—"


"Stow this gab," roared Hook, and the spokesmen were dragged back. "You,
boy," he said, addressing John, "you look as if you had a little pluck in
you. Didst never want to be a pirate, my hearty?"


Now John had sometimes experienced this hankering at maths. prep.; and he
was struck by Hook's picking him out.


"I once thought of calling myself Red-handed Jack," he said diffidently.


"And a good name too. We'll call you that here, bully, if you join."


"What do you think, Michael?" asked John.


"What would you call me if I join?" Michael demanded.


"Blackbeard Joe."


Michael was naturally impressed. "What do you think, John?" He wanted John
to decide, and John wanted him to decide.


"Shall we still be respectful subjects of the King?" John inquired.


Through Hook's teeth came the answer: "You would have to swear, 'Down with
the King.'"


Perhaps John had not behaved very well so far, but he shone out now.


"Then I refuse," he cried, banging the barrel in front of Hook.


"And I refuse," cried Michael.


"Rule Britannia!" squeaked Curly.


The infuriated pirates buffeted them in the mouth; and Hook roared out,
"That seals your doom. Bring up their mother. Get the plank ready."


They were only boys, and they went white as they saw Jukes and Cecco
preparing the fatal plank. But they tried to look brave when Wendy was
brought up.


No words of mine can tell you how Wendy despised those pirates. To the
boys there was at least some glamour in the pirate calling; but all that
she saw was that the ship had not been tidied for years. There was not a
porthole on the grimy glass of which you might not have written with your
finger "Dirty pig"; and she had already written it on several. But as the
boys gathered round her she had no thought, of course, save for them.


"So, my beauty," said Hook, as if he spoke in syrup, "you are to see your
children walk the plank."


Fine gentlemen though he was, the intensity of his communings had soiled
his ruff, and suddenly he knew that she was gazing at it. With a hasty
gesture he tried to hide it, but he was too late.


"Are they to die?" asked Wendy, with a look of such frightful contempt
that he nearly fainted.


"They are," he snarled. "Silence all," he called gloatingly, "for a
mother's last words to her children."


At this moment Wendy was grand. "These are my last words, dear boys," she
said firmly. "I feel that I have a message to you from your real mothers,
and it is this: 'We hope our sons will die like English gentlemen.'"


Even the pirates were awed, and Tootles cried out hysterically, "I am
going to do what my mother hopes. What are you to do, Nibs?"


"What my mother hopes. What are you to do, Twin?"


"What my mother hopes. John, what are—"


But Hook had found his voice again.


"Tie her up!" he shouted.


It was Smee who tied her to the mast. "See here, honey," he whispered,
"I'll save you if you promise to be my mother."


But not even for Smee would she make such a promise. "I would almost
rather have no children at all," she said disdainfully [scornfully].


It is sad to know that not a boy was looking at her as Smee tied her to
the mast; the eyes of all were on the plank: that last little walk they
were about to take. They were no longer able to hope that they would walk
it manfully, for the capacity to think had gone from them; they could
stare and shiver only.


Hook smiled on them with his teeth closed, and took a step toward Wendy.
His intention was to turn her face so that she should see the boys walking
the plank one by one. But he never reached her, he never heard the cry of
anguish he hoped to wring from her. He heard something else instead.


It was the terrible tick-tick of the crocodile.


They all heard it—pirates, boys, Wendy; and immediately every head
was blown in one direction; not to the water whence the sound proceeded,
but toward Hook. All knew that what was about to happen concerned him
alone, and that from being actors they were suddenly become spectators.


Very frightful was it to see the change that came over him. It was as if
he had been clipped at every joint. He fell in a little heap.


The sound came steadily nearer; and in advance of it came this ghastly
thought, "The crocodile is about to board the ship!"


Even the iron claw hung inactive; as if knowing that it was no intrinsic
part of what the attacking force wanted. Left so fearfully alone, any
other man would have lain with his eyes shut where he fell: but the
gigantic brain of Hook was still working, and under its guidance he
crawled on the knees along the deck as far from the sound as he could go.
The pirates respectfully cleared a passage for him, and it was only when
he brought up against the bulwarks that he spoke.


"Hide me!" he cried hoarsely.


They gathered round him, all eyes averted from the thing that was coming
aboard. They had no thought of fighting it. It was Fate.


Only when Hook was hidden from them did curiosity loosen the limbs of the
boys so that they could rush to the ship's side to see the crocodile
climbing it. Then they got the strangest surprise of the Night of Nights;
for it was no crocodile that was coming to their aid. It was Peter.


He signed to them not to give vent to any cry of admiration that might
rouse suspicion. Then he went on ticking.


% !TEX program = pdflatex
% !TEX encoding = UTF-8
% !TEX spellcheck = en_GB
% !TEX root = peter-pan.tex

\chapter{“Hook or Me this Time”}

\endinput


Chapter 15 "HOOK OR ME THIS TIME"


Odd things happen to all of us on our way through life without our
noticing for a time that they have happened. Thus, to take an instance, we
suddenly discover that we have been deaf in one ear for we don't know how
long, but, say, half an hour. Now such an experience had come that night
to Peter. When last we saw him he was stealing across the island with one
finger to his lips and his dagger at the ready. He had seen the crocodile
pass by without noticing anything peculiar about it, but by and by he
remembered that it had not been ticking. At first he thought this eerie,
but soon concluded rightly that the clock had run down.


Without giving a thought to what might be the feelings of a
fellow-creature thus abruptly deprived of its closest companion, Peter
began to consider how he could turn the catastrophe to his own use; and he
decided to tick, so that wild beasts should believe he was the crocodile
and let him pass unmolested. He ticked superbly, but with one unforeseen
result. The crocodile was among those who heard the sound, and it followed
him, though whether with the purpose of regaining what it had lost, or
merely as a friend under the belief that it was again ticking itself, will
never be certainly known, for, like slaves to a fixed idea, it was a
stupid beast.


Peter reached the shore without mishap, and went straight on, his legs
encountering the water as if quite unaware that they had entered a new
element. Thus many animals pass from land to water, but no other human of
whom I know. As he swam he had but one thought: "Hook or me this time." He
had ticked so long that he now went on ticking without knowing that he was
doing it. Had he known he would have stopped, for to board the brig by
help of the tick, though an ingenious idea, had not occurred to him.


On the contrary, he thought he had scaled her side as noiseless as a
mouse; and he was amazed to see the pirates cowering from him, with Hook
in their midst as abject as if he had heard the crocodile.


The crocodile! No sooner did Peter remember it than he heard the ticking.
At first he thought the sound did come from the crocodile, and he looked
behind him swiftly. Then he realised that he was doing it himself, and in
a flash he understood the situation. "How clever of me!" he thought at
once, and signed to the boys not to burst into applause.


It was at this moment that Ed Teynte the quartermaster emerged from the
forecastle and came along the deck. Now, reader, time what happened by
your watch. Peter struck true and deep. John clapped his hands on the
ill-fated pirate's mouth to stifle the dying groan. He fell forward. Four
boys caught him to prevent the thud. Peter gave the signal, and the
carrion was cast overboard. There was a splash, and then silence. How long
has it taken?


"One!" (Slightly had begun to count.)


None too soon, Peter, every inch of him on tiptoe, vanished into the
cabin; for more than one pirate was screwing up his courage to look round.
They could hear each other's distressed breathing now, which showed them
that the more terrible sound had passed.


"It's gone, captain," Smee said, wiping off his spectacles. "All's still
again."


Slowly Hook let his head emerge from his ruff, and listened so intently
that he could have caught the echo of the tick. There was not a sound, and
he drew himself up firmly to his full height.


"Then here's to Johnny Plank!" he cried brazenly, hating the boys more
than ever because they had seen him unbend. He broke into the villainous
ditty:

\begin{verse}
     "Yo ho, yo ho, the frisky plank,
     You walks along it so,
     Till it goes down and you goes down
     To Davy Jones below!"
\end{verse}

To terrorize the prisoners the more, though with a certain loss of
dignity, he danced along an imaginary plank, grimacing at them as he sang;
and when he finished he cried, "Do you want a touch of the cat [o' nine
tails] before you walk the plank?"


At that they fell on their knees. "No, no!" they cried so piteously that
every pirate smiled.


"Fetch the cat, Jukes," said Hook; "it's in the cabin."


The cabin! Peter was in the cabin! The children gazed at each other.


"Ay, ay," said Jukes blithely, and he strode into the cabin. They followed
him with their eyes; they scarce knew that Hook had resumed his song, his
dogs joining in with him:

\begin{verse}
"Yo ho, yo ho, the scratching cat,
Its tails are nine, you know,
And when they're writ upon your back—"
\end{verse}

What was the last line will never be known, for of a sudden the song was
stayed by a dreadful screech from the cabin. It wailed through the ship,
and died away. Then was heard a crowing sound which was well understood by
the boys, but to the pirates was almost more eerie than the screech.


"What was that?" cried Hook.


"Two," said Slightly solemnly.


The Italian Cecco hesitated for a moment and then swung into the cabin. He
tottered out, haggard.


"What's the matter with Bill Jukes, you dog?" hissed Hook, towering over
him.


"The matter wi' him is he's dead, stabbed," replied Cecco in a hollow
voice.


"Bill Jukes dead!" cried the startled pirates.


"The cabin's as black as a pit," Cecco said, almost gibbering, "but there
is something terrible in there: the thing you heard crowing."


The exultation of the boys, the lowering looks of the pirates, both were
seen by Hook.


"Cecco," he said in his most steely voice, "go back and fetch me out that
doodle-doo."


Cecco, bravest of the brave, cowered before his captain, crying "No, no";
but Hook was purring to his claw.


"Did you say you would go, Cecco?" he said musingly.


Cecco went, first flinging his arms despairingly. There was no more
singing, all listened now; and again came a death-screech and again a
crow.


No one spoke except Slightly. "Three," he said.


Hook rallied his dogs with a gesture. "'S'death and odds fish," he
thundered, "who is to bring me that doodle-doo?"


"Wait till Cecco comes out," growled Starkey, and the others took up the
cry.


"I think I heard you volunteer, Starkey," said Hook, purring again.


"No, by thunder!" Starkey cried.


"My hook thinks you did," said Hook, crossing to him. "I wonder if it
would not be advisable, Starkey, to humour the hook?"


"I'll swing before I go in there," replied Starkey doggedly, and again he
had the support of the crew.


"Is this mutiny?" asked Hook more pleasantly than ever. "Starkey's
ringleader!"


"Captain, mercy!" Starkey whimpered, all of a tremble now.


"Shake hands, Starkey," said Hook, proffering his claw.


Starkey looked round for help, but all deserted him. As he backed up Hook
advanced, and now the red spark was in his eye. With a despairing scream
the pirate leapt upon Long Tom and precipitated himself into the sea.


"Four," said Slightly.


"And now," Hook said courteously, "did any other gentlemen say mutiny?"
Seizing a lantern and raising his claw with a menacing gesture, "I'll
bring out that doodle-doo myself," he said, and sped into the cabin.


"Five." How Slightly longed to say it. He wetted his lips to be ready, but
Hook came staggering out, without his lantern.


"Something blew out the light," he said a little unsteadily.


"Something!" echoed Mullins.


"What of Cecco?" demanded Noodler.


"He's as dead as Jukes," said Hook shortly.


His reluctance to return to the cabin impressed them all unfavourably, and
the mutinous sounds again broke forth. All pirates are superstitious, and
Cookson cried, "They do say the surest sign a ship's accurst is when
there's one on board more than can be accounted for."


"I've heard," muttered Mullins, "he always boards the pirate craft last.
Had he a tail, captain?"


"They say," said another, looking viciously at Hook, "that when he comes
it's in the likeness of the wickedest man aboard."


"Had he a hook, captain?" asked Cookson insolently; and one after another
took up the cry, "The ship's doomed!" At this the children could not
resist raising a cheer. Hook had well-nigh forgotten his prisoners, but as
he swung round on them now his face lit up again.


"Lads," he cried to his crew, "now here's a notion. Open the cabin door
and drive them in. Let them fight the doodle-doo for their lives. If they
kill him, we're so much the better; if he kills them, we're none the
worse."


For the last time his dogs admired Hook, and devotedly they did his
bidding. The boys, pretending to struggle, were pushed into the cabin and
the door was closed on them.


"Now, listen!" cried Hook, and all listened. But not one dared to face the
door. Yes, one, Wendy, who all this time had been bound to the mast. It
was for neither a scream nor a crow that she was watching, it was for the
reappearance of Peter.


She had not long to wait. In the cabin he had found the thing for which he
had gone in search: the key that would free the children of their
manacles, and now they all stole forth, armed with such weapons as they
could find. First signing them to hide, Peter cut Wendy's bonds, and then
nothing could have been easier than for them all to fly off together; but
one thing barred the way, an oath, "Hook or me this time." So when he had
freed Wendy, he whispered for her to conceal herself with the others, and
himself took her place by the mast, her cloak around him so that he should
pass for her. Then he took a great breath and crowed.


To the pirates it was a voice crying that all the boys lay slain in the
cabin; and they were panic-stricken. Hook tried to hearten them; but like
the dogs he had made them they showed him their fangs, and he knew that if
he took his eyes off them now they would leap at him.


"Lads," he said, ready to cajole or strike as need be, but never quailing
for an instant, "I've thought it out. There's a Jonah aboard."


"Ay," they snarled, "a man wi' a hook."


"No, lads, no, it's the girl. Never was luck on a pirate ship wi' a woman
on board. We'll right the ship when she's gone."


Some of them remembered that this had been a saying of Flint's. "It's
worth trying," they said doubtfully.


"Fling the girl overboard," cried Hook; and they made a rush at the figure
in the cloak.


"There's none can save you now, missy," Mullins hissed jeeringly.


"There's one," replied the figure.


"Who's that?"


"Peter Pan the avenger!" came the terrible answer; and as he spoke Peter
flung off his cloak. Then they all knew who 'twas that had been undoing
them in the cabin, and twice Hook essayed to speak and twice he failed. In
that frightful moment I think his fierce heart broke.


At last he cried, "Cleave him to the brisket!" but without conviction.


"Down, boys, and at them!" Peter's voice rang out; and in another moment
the clash of arms was resounding through the ship. Had the pirates kept
together it is certain that they would have won; but the onset came when
they were still unstrung, and they ran hither and thither, striking
wildly, each thinking himself the last survivor of the crew. Man to man
they were the stronger; but they fought on the defensive only, which
enabled the boys to hunt in pairs and choose their quarry. Some of the
miscreants leapt into the sea; others hid in dark recesses, where they
were found by Slightly, who did not fight, but ran about with a lantern
which he flashed in their faces, so that they were half blinded and fell
as an easy prey to the reeking swords of the other boys. There was little
sound to be heard but the clang of weapons, an occasional screech or
splash, and Slightly monotonously counting—five—six—seven
eight—nine—ten—eleven.


I think all were gone when a group of savage boys surrounded Hook, who
seemed to have a charmed life, as he kept them at bay in that circle of
fire. They had done for his dogs, but this man alone seemed to be a match
for them all. Again and again they closed upon him, and again and again he
hewed a clear space. He had lifted up one boy with his hook, and was using
him as a buckler [shield], when another, who had just passed his sword
through Mullins, sprang into the fray.


"Put up your swords, boys," cried the newcomer, "this man is mine."


Thus suddenly Hook found himself face to face with Peter. The others drew
back and formed a ring around them.


For long the two enemies looked at one another, Hook shuddering slightly,
and Peter with the strange smile upon his face.


"So, Pan," said Hook at last, "this is all your doing."


"Ay, James Hook," came the stern answer, "it is all my doing."


"Proud and insolent youth," said Hook, "prepare to meet thy doom."


"Dark and sinister man," Peter answered, "have at thee."


Without more words they fell to, and for a space there was no advantage to
either blade. Peter was a superb swordsman, and parried with dazzling
rapidity; ever and anon he followed up a feint with a lunge that got past
his foe's defence, but his shorter reach stood him in ill stead, and he
could not drive the steel home. Hook, scarcely his inferior in brilliancy,
but not quite so nimble in wrist play, forced him back by the weight of
his onset, hoping suddenly to end all with a favourite thrust, taught him
long ago by Barbecue at Rio; but to his astonishment he found this thrust
turned aside again and again. Then he sought to close and give the quietus
with his iron hook, which all this time had been pawing the air; but Peter
doubled under it and, lunging fiercely, pierced him in the ribs. At the
sight of his own blood, whose peculiar colour, you remember, was offensive
to him, the sword fell from Hook's hand, and he was at Peter's mercy.


"Now!" cried all the boys, but with a magnificent gesture Peter invited
his opponent to pick up his sword. Hook did so instantly, but with a
tragic feeling that Peter was showing good form.


Hitherto he had thought it was some fiend fighting him, but darker
suspicions assailed him now.


"Pan, who and what art thou?" he cried huskily.


"I'm youth, I'm joy," Peter answered at a venture, "I'm a little bird that
has broken out of the egg."


This, of course, was nonsense; but it was proof to the unhappy Hook that
Peter did not know in the least who or what he was, which is the very
pinnacle of good form.


"To't again," he cried despairingly.


He fought now like a human flail, and every sweep of that terrible sword
would have severed in twain any man or boy who obstructed it; but Peter
fluttered round him as if the very wind it made blew him out of the danger
zone. And again and again he darted in and pricked.


Hook was fighting now without hope. That passionate breast no longer asked
for life; but for one boon it craved: to see Peter show bad form before it
was cold forever.


Abandoning the fight he rushed into the powder magazine and fired it.


"In two minutes," he cried, "the ship will be blown to pieces."


Now, now, he thought, true form will show.


But Peter issued from the powder magazine with the shell in his hands, and
calmly flung it overboard.


What sort of form was Hook himself showing? Misguided man though he was,
we may be glad, without sympathising with him, that in the end he was true
to the traditions of his race. The other boys were flying around him now,
flouting, scornful; and he staggered about the deck striking up at them
impotently, his mind was no longer with them; it was slouching in the
playing fields of long ago, or being sent up [to the headmaster] for good,
or watching the wall-game from a famous wall. And his shoes were right,
and his waistcoat was right, and his tie was right, and his socks were
right.


James Hook, thou not wholly unheroic figure, farewell.


For we have come to his last moment.


Seeing Peter slowly advancing upon him through the air with dagger poised,
he sprang upon the bulwarks to cast himself into the sea. He did not know
that the crocodile was waiting for him; for we purposely stopped the clock
that this knowledge might be spared him: a little mark of respect from us
at the end.


He had one last triumph, which I think we need not grudge him. As he stood
on the bulwark looking over his shoulder at Peter gliding through the air,
he invited him with a gesture to use his foot. It made Peter kick instead
of stab.


At last Hook had got the boon for which he craved.


"Bad form," he cried jeeringly, and went content to the crocodile.


Thus perished James Hook.


"Seventeen," Slightly sang out; but he was not quite correct in his
figures. Fifteen paid the penalty for their crimes that night; but two
reached the shore: Starkey to be captured by the redskins, who made him
nurse for all their papooses, a melancholy come-down for a pirate; and
Smee, who henceforth wandered about the world in his spectacles, making a
precarious living by saying he was the only man that Jas. Hook had feared.


Wendy, of course, had stood by taking no part in the fight, though
watching Peter with glistening eyes; but now that all was over she became
prominent again. She praised them equally, and shuddered delightfully when
Michael showed her the place where he had killed one; and then she took
them into Hook's cabin and pointed to his watch which was hanging on a
nail. It said "half-past one!"


The lateness of the hour was almost the biggest thing of all. She got them
to bed in the pirates' bunks pretty quickly, you may be sure; all but
Peter, who strutted up and down on the deck, until at last he fell asleep
by the side of Long Tom. He had one of his dreams that night, and cried in
his sleep for a long time, and Wendy held him tightly.


% !TEX program = pdflatex
% !TEX encoding = UTF-8
% !TEX spellcheck = en_GB
% !TEX root = peter-pan.tex

\chapter{The Return Home}

\endinput


Chapter 16 THE RETURN HOME


By three bells that morning they were all stirring their stumps [legs];
for there was a big sea running; and Tootles, the bo'sun, was among them,
with a rope's end in his hand and chewing tobacco. They all donned pirate
clothes cut off at the knee, shaved smartly, and tumbled up, with the true
nautical roll and hitching their trousers.


It need not be said who was the captain. Nibs and John were first and
second mate. There was a woman aboard. The rest were tars [sailors] before
the mast, and lived in the fo'c'sle. Peter had already lashed himself to
the wheel; but he piped all hands and delivered a short address to them;
said he hoped they would do their duty like gallant hearties, but that he
knew they were the scum of Rio and the Gold Coast, and if they snapped at
him he would tear them. The bluff strident words struck the note sailors
understood, and they cheered him lustily. Then a few sharp orders were
given, and they turned the ship round, and nosed her for the mainland.


Captain Pan calculated, after consulting the ship's chart, that if this
weather lasted they should strike the Azores about the 21st of June, after
which it would save time to fly.


Some of them wanted it to be an honest ship and others were in favour of
keeping it a pirate; but the captain treated them as dogs, and they dared
not express their wishes to him even in a round robin [one person after
another, as they had to Cpt. Hook]. Instant obedience was the only safe
thing. Slightly got a dozen for looking perplexed when told to take
soundings. The general feeling was that Peter was honest just now to lull
Wendy's suspicions, but that there might be a change when the new suit was
ready, which, against her will, she was making for him out of some of
Hook's wickedest garments. It was afterwards whispered among them that on
the first night he wore this suit he sat long in the cabin with Hook's
cigar-holder in his mouth and one hand clenched, all but for the
forefinger, which he bent and held threateningly aloft like a hook.


Instead of watching the ship, however, we must now return to that desolate
home from which three of our characters had taken heartless flight so long
ago. It seems a shame to have neglected No. 14 all this time; and yet we
may be sure that Mrs. Darling does not blame us. If we had returned sooner
to look with sorrowful sympathy at her, she would probably have cried,
"Don't be silly; what do I matter? Do go back and keep an eye on the
children." So long as mothers are like this their children will take
advantage of them; and they may lay to [bet on] that.


Even now we venture into that familiar nursery only because its lawful
occupants are on their way home; we are merely hurrying on in advance of
them to see that their beds are properly aired and that Mr. and Mrs.
Darling do not go out for the evening. We are no more than servants. Why
on earth should their beds be properly aired, seeing that they left them
in such a thankless hurry? Would it not serve them jolly well right if
they came back and found that their parents were spending the week-end in
the country? It would be the moral lesson they have been in need of ever
since we met them; but if we contrived things in this way Mrs. Darling
would never forgive us.


One thing I should like to do immensely, and that is to tell her, in the
way authors have, that the children are coming back, that indeed they will
be here on Thursday week. This would spoil so completely the surprise to
which Wendy and John and Michael are looking forward. They have been
planning it out on the ship: mother's rapture, father's shout of joy,
Nana's leap through the air to embrace them first, when what they ought to
be prepared for is a good hiding. How delicious to spoil it all by
breaking the news in advance; so that when they enter grandly Mrs. Darling
may not even offer Wendy her mouth, and Mr. Darling may exclaim pettishly,
"Dash it all, here are those boys again." However, we should get no thanks
even for this. We are beginning to know Mrs. Darling by this time, and may
be sure that she would upbraid us for depriving the children of their
little pleasure.


"But, my dear madam, it is ten days till Thursday week; so that by telling
you what's what, we can save you ten days of unhappiness."


"Yes, but at what a cost! By depriving the children of ten minutes of
delight."


"Oh, if you look at it in that way!"


"What other way is there in which to look at it?"


You see, the woman had no proper spirit. I had meant to say
extraordinarily nice things about her; but I despise her, and not one of
them will I say now. She does not really need to be told to have things
ready, for they are ready. All the beds are aired, and she never leaves
the house, and observe, the window is open. For all the use we are to her,
we might well go back to the ship. However, as we are here we may as well
stay and look on. That is all we are, lookers-on. Nobody really wants us.
So let us watch and say jaggy things, in the hope that some of them will
hurt.


The only change to be seen in the night-nursery is that between nine and
six the kennel is no longer there. When the children flew away, Mr.
Darling felt in his bones that all the blame was his for having chained
Nana up, and that from first to last she had been wiser than he. Of
course, as we have seen, he was quite a simple man; indeed he might have
passed for a boy again if he had been able to take his baldness off; but
he had also a noble sense of justice and a lion's courage to do what
seemed right to him; and having thought the matter out with anxious care
after the flight of the children, he went down on all fours and crawled
into the kennel. To all Mrs. Darling's dear invitations to him to come out
he replied sadly but firmly:


"No, my own one, this is the place for me."


In the bitterness of his remorse he swore that he would never leave the
kennel until his children came back. Of course this was a pity; but
whatever Mr. Darling did he had to do in excess, otherwise he soon gave up
doing it. And there never was a more humble man than the once proud George
Darling, as he sat in the kennel of an evening talking with his wife of
their children and all their pretty ways.


Very touching was his deference to Nana. He would not let her come into
the kennel, but on all other matters he followed her wishes implicitly.


Every morning the kennel was carried with Mr. Darling in it to a cab,
which conveyed him to his office, and he returned home in the same way at
six. Something of the strength of character of the man will be seen if we
remember how sensitive he was to the opinion of neighbours: this man whose
every movement now attracted surprised attention. Inwardly he must have
suffered torture; but he preserved a calm exterior even when the young
criticised his little home, and he always lifted his hat courteously to
any lady who looked inside.


It may have been Quixotic, but it was magnificent. Soon the inward meaning
of it leaked out, and the great heart of the public was touched. Crowds
followed the cab, cheering it lustily; charming girls scaled it to get his
autograph; interviews appeared in the better class of papers, and society
invited him to dinner and added, "Do come in the kennel."


On that eventful Thursday week, Mrs. Darling was in the night-nursery
awaiting George's return home; a very sad-eyed woman. Now that we look at
her closely and remember the gaiety of her in the old days, all gone now
just because she has lost her babes, I find I won't be able to say nasty
things about her after all. If she was too fond of her rubbishy children,
she couldn't help it. Look at her in her chair, where she has fallen
asleep. The corner of her mouth, where one looks first, is almost withered
up. Her hand moves restlessly on her breast as if she had a pain there.
Some like Peter best, and some like Wendy best, but I like her best.
Suppose, to make her happy, we whisper to her in her sleep that the brats
are coming back. They are really within two miles of the window now, and
flying strong, but all we need whisper is that they are on the way. Let's.


It is a pity we did it, for she has started up, calling their names; and
there is no one in the room but Nana.


"O Nana, I dreamt my dear ones had come back."


Nana had filmy eyes, but all she could do was put her paw gently on her
mistress's lap; and they were sitting together thus when the kennel was
brought back. As Mr. Darling puts his head out to kiss his wife, we see
that his face is more worn than of yore, but has a softer expression.


He gave his hat to Liza, who took it scornfully; for she had no
imagination, and was quite incapable of understanding the motives of such
a man. Outside, the crowd who had accompanied the cab home were still
cheering, and he was naturally not unmoved.


"Listen to them," he said; "it is very gratifying."


"Lots of little boys," sneered Liza.


"There were several adults to-day," he assured her with a faint flush; but
when she tossed her head he had not a word of reproof for her. Social
success had not spoilt him; it had made him sweeter. For some time he sat
with his head out of the kennel, talking with Mrs. Darling of this
success, and pressing her hand reassuringly when she said she hoped his
head would not be turned by it.


"But if I had been a weak man," he said. "Good heavens, if I had been a
weak man!"


"And, George," she said timidly, "you are as full of remorse as ever,
aren't you?"


"Full of remorse as ever, dearest! See my punishment: living in a kennel."


"But it is punishment, isn't it, George? You are sure you are not enjoying
it?"


"My love!"


You may be sure she begged his pardon; and then, feeling drowsy, he curled
round in the kennel.


"Won't you play me to sleep," he asked, "on the nursery piano?" and as she
was crossing to the day-nursery he added thoughtlessly, "And shut that
window. I feel a draught."


"O George, never ask me to do that. The window must always be left open
for them, always, always."


Now it was his turn to beg her pardon; and she went into the day-nursery
and played, and soon he was asleep; and while he slept, Wendy and John and
Michael flew into the room.


Oh no. We have written it so, because that was the charming arrangement
planned by them before we left the ship; but something must have happened
since then, for it is not they who have flown in, it is Peter and Tinker
Bell.


Peter's first words tell all.


"Quick Tink," he whispered, "close the window; bar it! That's right. Now
you and I must get away by the door; and when Wendy comes she will think
her mother has barred her out; and she will have to go back with me."


Now I understand what had hitherto puzzled me, why when Peter had
exterminated the pirates he did not return to the island and leave Tink to
escort the children to the mainland. This trick had been in his head all
the time.


Instead of feeling that he was behaving badly he danced with glee; then he
peeped into the day-nursery to see who was playing. He whispered to Tink,
"It's Wendy's mother! She is a pretty lady, but not so pretty as my
mother. Her mouth is full of thimbles, but not so full as my mother's
was."


Of course he knew nothing whatever about his mother; but he sometimes
bragged about her.


He did not know the tune, which was "Home, Sweet Home," but he knew it was
saying, "Come back, Wendy, Wendy, Wendy"; and he cried exultantly, "You
will never see Wendy again, lady, for the window is barred!"


He peeped in again to see why the music had stopped, and now he saw that
Mrs. Darling had laid her head on the box, and that two tears were sitting
on her eyes.


"She wants me to unbar the window," thought Peter, "but I won't, not I!"


He peeped again, and the tears were still there, or another two had taken
their place.


"She's awfully fond of Wendy," he said to himself. He was angry with her
now for not seeing why she could not have Wendy.


The reason was so simple: "I'm fond of her too. We can't both have her,
lady."


But the lady would not make the best of it, and he was unhappy. He ceased
to look at her, but even then she would not let go of him. He skipped
about and made funny faces, but when he stopped it was just as if she were
inside him, knocking.


"Oh, all right," he said at last, and gulped. Then he unbarred the window.
"Come on, Tink," he cried, with a frightful sneer at the laws of nature;
"we don't want any silly mothers;" and he flew away.


Thus Wendy and John and Michael found the window open for them after all,
which of course was more than they deserved. They alighted on the floor,
quite unashamed of themselves, and the youngest one had already forgotten
his home.


"John," he said, looking around him doubtfully, "I think I have been here
before."


"Of course you have, you silly. There is your old bed."


"So it is," Michael said, but not with much conviction.


"I say," cried John, "the kennel!" and he dashed across to look into it.


"Perhaps Nana is inside it," Wendy said.


But John whistled. "Hullo," he said, "there's a man inside it."


"It's father!" exclaimed Wendy.


"Let me see father," Michael begged eagerly, and he took a good look. "He
is not so big as the pirate I killed," he said with such frank
disappointment that I am glad Mr. Darling was asleep; it would have been
sad if those had been the first words he heard his little Michael say.


Wendy and John had been taken aback somewhat at finding their father in
the kennel.


"Surely," said John, like one who had lost faith in his memory, "he used
not to sleep in the kennel?"


"John," Wendy said falteringly, "perhaps we don't remember the old life as
well as we thought we did."


A chill fell upon them; and serve them right.


"It is very careless of mother," said that young scoundrel John, "not to
be here when we come back."


It was then that Mrs. Darling began playing again.


"It's mother!" cried Wendy, peeping.


"So it is!" said John.


"Then are you not really our mother, Wendy?" asked Michael, who was surely
sleepy.


"Oh dear!" exclaimed Wendy, with her first real twinge of remorse [for
having gone], "it was quite time we came back."


"Let us creep in," John suggested, "and put our hands over her eyes."


But Wendy, who saw that they must break the joyous news more gently, had a
better plan.


"Let us all slip into our beds, and be there when she comes in, just as if
we had never been away."


And so when Mrs. Darling went back to the night-nursery to see if her
husband was asleep, all the beds were occupied. The children waited for
her cry of joy, but it did not come. She saw them, but she did not believe
they were there. You see, she saw them in their beds so often in her
dreams that she thought this was just the dream hanging around her still.


She sat down in the chair by the fire, where in the old days she had
nursed them.


They could not understand this, and a cold fear fell upon all the three of
them.


"Mother!" Wendy cried.


"That's Wendy," she said, but still she was sure it was the dream.


"Mother!"


"That's John," she said.


"Mother!" cried Michael. He knew her now.


"That's Michael," she said, and she stretched out her arms for the three
little selfish children they would never envelop again. Yes, they did,
they went round Wendy and John and Michael, who had slipped out of bed and
run to her.


"George, George!" she cried when she could speak; and Mr. Darling woke to
share her bliss, and Nana came rushing in. There could not have been a
lovelier sight; but there was none to see it except a little boy who was
staring in at the window. He had had ecstasies innumerable that other
children can never know; but he was looking through the window at the one
joy from which he must be for ever barred.


% !TEX program = pdflatex
% !TEX encoding = UTF-8
% !TEX spellcheck = en_GB
% !TEX root = peter-pan.tex

\chapter{When Wendy Grew Up}

As you look at Wendy,
you may see her hair becoming white,
and her figure little again,
for all this happened long ago.
Jane is now a common grown-up,
with a daughter called Margaret;
and every spring cleaning time,
except when he forgets,
Peter comes for Margaret and takes her to the Neverland,
where she tells him stories about himself,
to which he listens eagerly.
When Margaret grows up she will have a daughter,
who is to be Peter’s mother in turn;
and thus it will go on,
so long as children are gay and innocent and heartless.

\endinput

    <h2>
      Chapter 17 WHEN WENDY GREW UP
    </h2>
    <p>
      I hope you want to know what became of the other boys. They were waiting
      below to give Wendy time to explain about them; and when they had counted
      five hundred they went up. They went up by the stair, because they thought
      this would make a better impression. They stood in a row in front of Mrs.
      Darling, with their hats off, and wishing they were not wearing their
      pirate clothes. They said nothing, but their eyes asked her to have them.
      They ought to have looked at Mr. Darling also, but they forgot about him.
    </p>
    <p>
      Of course Mrs. Darling said at once that she would have them; but Mr.
      Darling was curiously depressed, and they saw that he considered six a
      rather large number.
    </p>
    <p>
      "I must say," he said to Wendy, "that you don't do things by halves," a
      grudging remark which the twins thought was pointed at them.
    </p>
    <p>
      The first twin was the proud one, and he asked, flushing, "Do you think we
      should be too much of a handful, sir? Because, if so, we can go away."
    </p>
    <p>
      "Father!" Wendy cried, shocked; but still the cloud was on him. He knew he
      was behaving unworthily, but he could not help it.
    </p>
    <p>
      "We could lie doubled up," said Nibs.
    </p>
    <p>
      "I always cut their hair myself," said Wendy.
    </p>
    <p>
      "George!" Mrs. Darling exclaimed, pained to see her dear one showing
      himself in such an unfavourable light.
    </p>
    <p>
      Then he burst into tears, and the truth came out. He was as glad to have
      them as she was, he said, but he thought they should have asked his
      consent as well as hers, instead of treating him as a cypher [zero] in his
      own house.
    </p>
    <p>
      "I don't think he is a cypher," Tootles cried instantly. "Do you think he
      is a cypher, Curly?"
    </p>
    <p>
      "No, I don't. Do you think he is a cypher, Slightly?"
    </p>
    <p>
      "Rather not. Twin, what do you think?"
    </p>
    <p>
      It turned out that not one of them thought him a cypher; and he was
      absurdly gratified, and said he would find space for them all in the
      drawing-room if they fitted in.
    </p>
    <p>
      "We'll fit in, sir," they assured him.
    </p>
    <p>
      "Then follow the leader," he cried gaily. "Mind you, I am not sure that we
      have a drawing-room, but we pretend we have, and it's all the same. Hoop
      la!"
    </p>
    <p>
      He went off dancing through the house, and they all cried "Hoop la!" and
      danced after him, searching for the drawing-room; and I forget whether
      they found it, but at any rate they found corners, and they all fitted in.
    </p>
    <p>
      As for Peter, he saw Wendy once again before he flew away. He did not
      exactly come to the window, but he brushed against it in passing so that
      she could open it if she liked and call to him. That is what she did.
    </p>
    <p>
      "Hullo, Wendy, good-bye," he said.
    </p>
    <p>
      "Oh dear, are you going away?"
    </p>
    <p>
      "Yes."
    </p>
    <p>
      "You don't feel, Peter," she said falteringly, "that you would like to say
      anything to my parents about a very sweet subject?"
    </p>
    <p>
      "No."
    </p>
    <p>
      "About me, Peter?"
    </p>
    <p>
      "No."
    </p>
    <p>
      Mrs. Darling came to the window, for at present she was keeping a sharp
      eye on Wendy. She told Peter that she had adopted all the other boys, and
      would like to adopt him also.
    </p>
    <p>
      "Would you send me to school?" he inquired craftily.
    </p>
    <p>
      "Yes."
    </p>
    <p>
      "And then to an office?"
    </p>
    <p>
      "I suppose so."
    </p>
    <p>
      "Soon I would be a man?"
    </p>
    <p>
      "Very soon."
    </p>
    <p>
      "I don't want to go to school and learn solemn things," he told her
      passionately. "I don't want to be a man. O Wendy's mother, if I was to
      wake up and feel there was a beard!"
    </p>
    <p>
      "Peter," said Wendy the comforter, "I should love you in a beard;" and
      Mrs. Darling stretched out her arms to him, but he repulsed her.
    </p>
    <p>
      "Keep back, lady, no one is going to catch me and make me a man."
    </p>
    <p>
      "But where are you going to live?"
    </p>
    <p>
      "With Tink in the house we built for Wendy. The fairies are to put it high
      up among the tree tops where they sleep at nights."
    </p>
    <p>
      "How lovely," cried Wendy so longingly that Mrs. Darling tightened her
      grip.
    </p>
    <p>
      "I thought all the fairies were dead," Mrs. Darling said.
    </p>
    <p>
      "There are always a lot of young ones," explained Wendy, who was now quite
      an authority, "because you see when a new baby laughs for the first time a
      new fairy is born, and as there are always new babies there are always new
      fairies. They live in nests on the tops of trees; and the mauve ones are
      boys and the white ones are girls, and the blue ones are just little
      sillies who are not sure what they are."
    </p>
    <p>
      "I shall have such fun," said Peter, with eye on Wendy.
    </p>
    <p>
      "It will be rather lonely in the evening," she said, "sitting by the
      fire."
    </p>
    <p>
      "I shall have Tink."
    </p>
    <p>
      "Tink can't go a twentieth part of the way round," she reminded him a
      little tartly.
    </p>
    <p>
      "Sneaky tell-tale!" Tink called out from somewhere round the corner.
    </p>
    <p>
      "It doesn't matter," Peter said.
    </p>
    <p>
      "O Peter, you know it matters."
    </p>
    <p>
      "Well, then, come with me to the little house."
    </p>
    <p>
      "May I, mummy?"
    </p>
    <p>
      "Certainly not. I have got you home again, and I mean to keep you."
    </p>
    <p>
      "But he does so need a mother."
    </p>
    <p>
      "So do you, my love."
    </p>
    <p>
      "Oh, all right," Peter said, as if he had asked her from politeness
      merely; but Mrs. Darling saw his mouth twitch, and she made this handsome
      offer: to let Wendy go to him for a week every year to do his spring
      cleaning. Wendy would have preferred a more permanent arrangement; and it
      seemed to her that spring would be long in coming; but this promise sent
      Peter away quite gay again. He had no sense of time, and was so full of
      adventures that all I have told you about him is only a halfpenny-worth of
      them. I suppose it was because Wendy knew this that her last words to him
      were these rather plaintive ones:
    </p>
    <p>
      "You won't forget me, Peter, will you, before spring cleaning time comes?"
    </p>
    <p>
      Of course Peter promised; and then he flew away. He took Mrs. Darling's
      kiss with him. The kiss that had been for no one else, Peter took quite
      easily. Funny. But she seemed satisfied.
    </p>
    <p>
      Of course all the boys went to school; and most of them got into Class
      III, but Slightly was put first into Class IV and then into Class V. Class
      I is the top class. Before they had attended school a week they saw what
      goats they had been not to remain on the island; but it was too late now,
      and soon they settled down to being as ordinary as you or me or Jenkins
      minor [the younger Jenkins]. It is sad to have to say that the power to
      fly gradually left them. At first Nana tied their feet to the bed-posts so
      that they should not fly away in the night; and one of their diversions by
      day was to pretend to fall off buses [the English double-deckers]; but by
      and by they ceased to tug at their bonds in bed, and found that they hurt
      themselves when they let go of the bus. In time they could not even fly
      after their hats. Want of practice, they called it; but what it really
      meant was that they no longer believed.
    </p>
    <p>
      Michael believed longer than the other boys, though they jeered at him; so
      he was with Wendy when Peter came for her at the end of the first year.
      She flew away with Peter in the frock she had woven from leaves and
      berries in the Neverland, and her one fear was that he might notice how
      short it had become; but he never noticed, he had so much to say about
      himself.
    </p>
    <p>
      She had looked forward to thrilling talks with him about old times, but
      new adventures had crowded the old ones from his mind.
    </p>
    <p>
      "Who is Captain Hook?" he asked with interest when she spoke of the arch
      enemy.
    </p>
    <p>
      "Don't you remember," she asked, amazed, "how you killed him and saved all
      our lives?"
    </p>
    <p>
      "I forget them after I kill them," he replied carelessly.
    </p>
    <p>
      When she expressed a doubtful hope that Tinker Bell would be glad to see
      her he said, "Who is Tinker Bell?"
    </p>
    <p>
      "O Peter," she said, shocked; but even when she explained he could not
      remember.
    </p>
    <p>
      "There are such a lot of them," he said. "I expect she is no more."
    </p>
    <p>
      I expect he was right, for fairies don't live long, but they are so little
      that a short time seems a good while to them.
    </p>
    <p>
      Wendy was pained too to find that the past year was but as yesterday to
      Peter; it had seemed such a long year of waiting to her. But he was
      exactly as fascinating as ever, and they had a lovely spring cleaning in
      the little house on the tree tops.
    </p>
    <p>
      Next year he did not come for her. She waited in a new frock because the
      old one simply would not meet; but he never came.
    </p>
    <p>
      "Perhaps he is ill," Michael said.
    </p>
    <p>
      "You know he is never ill."
    </p>
    <p>
      Michael came close to her and whispered, with a shiver, "Perhaps there is
      no such person, Wendy!" and then Wendy would have cried if Michael had not
      been crying.
    </p>
    <p>
      Peter came next spring cleaning; and the strange thing was that he never
      knew he had missed a year.
    </p>
    <p>
      That was the last time the girl Wendy ever saw him. For a little longer
      she tried for his sake not to have growing pains; and she felt she was
      untrue to him when she got a prize for general knowledge. But the years
      came and went without bringing the careless boy; and when they met again
      Wendy was a married woman, and Peter was no more to her than a little dust
      in the box in which she had kept her toys. Wendy was grown up. You need
      not be sorry for her. She was one of the kind that likes to grow up. In
      the end she grew up of her own free will a day quicker than other girls.
    </p>
    <p>
      All the boys were grown up and done for by this time; so it is scarcely
      worth while saying anything more about them. You may see the twins and
      Nibs and Curly any day going to an office, each carrying a little bag and
      an umbrella. Michael is an engine-driver [train engineer]. Slightly
      married a lady of title, and so he became a lord. You see that judge in a
      wig coming out at the iron door? That used to be Tootles. The bearded man
      who doesn't know any story to tell his children was once John.
    </p>
    <p>
      Wendy was married in white with a pink sash. It is strange to think that
      Peter did not alight in the church and forbid the banns [formal
      announcement of a marriage].
    </p>
    <p>
      Years rolled on again, and Wendy had a daughter. This ought not to be
      written in ink but in a golden splash.
    </p>
    <p>
      She was called Jane, and always had an odd inquiring look, as if from the
      moment she arrived on the mainland she wanted to ask questions. When she
      was old enough to ask them they were mostly about Peter Pan. She loved to
      hear of Peter, and Wendy told her all she could remember in the very
      nursery from which the famous flight had taken place. It was Jane's
      nursery now, for her father had bought it at the three per cents [mortgage
      rate] from Wendy's father, who was no longer fond of stairs. Mrs. Darling
      was now dead and forgotten.
    </p>
    <p>
      There were only two beds in the nursery now, Jane's and her nurse's; and
      there was no kennel, for Nana also had passed away. She died of old age,
      and at the end she had been rather difficult to get on with; being very
      firmly convinced that no one knew how to look after children except
      herself.
    </p>
    <p>
      Once a week Jane's nurse had her evening off; and then it was Wendy's part
      to put Jane to bed. That was the time for stories. It was Jane's invention
      to raise the sheet over her mother's head and her own, thus making a tent,
      and in the awful darkness to whisper:
    </p>
    <p>
      "What do we see now?"
    </p>
    <p>
      "I don't think I see anything to-night," says Wendy, with a feeling that
      if Nana were here she would object to further conversation.
    </p>
    <p>
      "Yes, you do," says Jane, "you see when you were a little girl."
    </p>
    <p>
      "That is a long time ago, sweetheart," says Wendy. "Ah me, how time
      flies!"
    </p>
    <p>
      "Does it fly," asks the artful child, "the way you flew when you were a
      little girl?"
    </p>
    <p>
      "The way I flew? Do you know, Jane, I sometimes wonder whether I ever did
      really fly."
    </p>
    <p>
      "Yes, you did."
    </p>
    <p>
      "The dear old days when I could fly!"
    </p>
    <p>
      "Why can't you fly now, mother?"
    </p>
    <p>
      "Because I am grown up, dearest. When people grow up they forget the way."
    </p>
    <p>
      "Why do they forget the way?"
    </p>
    <p>
      "Because they are no longer gay and innocent and heartless. It is only the
      gay and innocent and heartless who can fly."
    </p>
    <p>
      "What is gay and innocent and heartless? I do wish I were gay and innocent
      and heartless."
    </p>
    <p>
      Or perhaps Wendy admits she does see something.
    </p>
    <p>
      "I do believe," she says, "that it is this nursery."
    </p>
    <p>
      "I do believe it is," says Jane. "Go on."
    </p>
    <p>
      They are now embarked on the great adventure of the night when Peter flew
      in looking for his shadow.
    </p>
    <p>
      "The foolish fellow," says Wendy, "tried to stick it on with soap, and
      when he could not he cried, and that woke me, and I sewed it on for him."
    </p>
    <p>
      "You have missed a bit," interrupts Jane, who now knows the story better
      than her mother. "When you saw him sitting on the floor crying, what did
      you say?"
    </p>
    <p>
      "I sat up in bed and I said, 'Boy, why are you crying?'"
    </p>
    <p>
      "Yes, that was it," says Jane, with a big breath.
    </p>
    <p>
      "And then he flew us all away to the Neverland and the fairies and the
      pirates and the redskins and the mermaid's lagoon, and the home under the
      ground, and the little house."
    </p>
    <p>
      "Yes! which did you like best of all?"
    </p>
    <p>
      "I think I liked the home under the ground best of all."
    </p>
    <p>
      "Yes, so do I. What was the last thing Peter ever said to you?"
    </p>
    <p>
      "The last thing he ever said to me was, 'Just always be waiting for me,
      and then some night you will hear me crowing.'"
    </p>
    <p>
      "Yes."
    </p>
    <p>
      "But, alas, he forgot all about me," Wendy said it with a smile. She was
      as grown up as that.
    </p>
    <p>
      "What did his crow sound like?" Jane asked one evening.
    </p>
    <p>
      "It was like this," Wendy said, trying to imitate Peter's crow.
    </p>
    <p>
      "No, it wasn't," Jane said gravely, "it was like this;" and she did it
      ever so much better than her mother.
    </p>
    <p>
      Wendy was a little startled. "My darling, how can you know?"
    </p>
    <p>
      "I often hear it when I am sleeping," Jane said.
    </p>
    <p>
      "Ah yes, many girls hear it when they are sleeping, but I was the only one
      who heard it awake."
    </p>
    <p>
      "Lucky you," said Jane.
    </p>
    <p>
      And then one night came the tragedy. It was the spring of the year, and
      the story had been told for the night, and Jane was now asleep in her bed.
      Wendy was sitting on the floor, very close to the fire, so as to see to
      darn, for there was no other light in the nursery; and while she sat
      darning she heard a crow. Then the window blew open as of old, and Peter
      dropped in on the floor.
    </p>
    <p>
      He was exactly the same as ever, and Wendy saw at once that he still had
      all his first teeth.
    </p>
    <p>
      He was a little boy, and she was grown up. She huddled by the fire not
      daring to move, helpless and guilty, a big woman.
    </p>
    <p>
      "Hullo, Wendy," he said, not noticing any difference, for he was thinking
      chiefly of himself; and in the dim light her white dress might have been
      the nightgown in which he had seen her first.
    </p>
    <p>
      "Hullo, Peter," she replied faintly, squeezing herself as small as
      possible. Something inside her was crying "Woman, Woman, let go of me."
    </p>
    <p>
      "Hullo, where is John?" he asked, suddenly missing the third bed.
    </p>
    <p>
      "John is not here now," she gasped.
    </p>
    <p>
      "Is Michael asleep?" he asked, with a careless glance at Jane.
    </p>
    <p>
      "Yes," she answered; and now she felt that she was untrue to Jane as well
      as to Peter.
    </p>
    <p>
      "That is not Michael," she said quickly, lest a judgment should fall on
      her.
    </p>
    <p>
      Peter looked. "Hullo, is it a new one?"
    </p>
    <p>
      "Yes."
    </p>
    <p>
      "Boy or girl?"
    </p>
    <p>
      "Girl."
    </p>
    <p>
      Now surely he would understand; but not a bit of it.
    </p>
    <p>
      "Peter," she said, faltering, "are you expecting me to fly away with you?"
    </p>
    <p>
      "Of course; that is why I have come." He added a little sternly, "Have you
      forgotten that this is spring cleaning time?"
    </p>
    <p>
      She knew it was useless to say that he had let many spring cleaning times
      pass.
    </p>
    <p>
      "I can't come," she said apologetically, "I have forgotten how to fly."
    </p>
    <p>
      "I'll soon teach you again."
    </p>
    <p>
      "O Peter, don't waste the fairy dust on me."
    </p>
    <p>
      She had risen; and now at last a fear assailed him. "What is it?" he
      cried, shrinking.
    </p>
    <p>
      "I will turn up the light," she said, "and then you can see for yourself."
    </p>
    <p>
      For almost the only time in his life that I know of, Peter was afraid.
      "Don't turn up the light," he cried.
    </p>
    <p>
      She let her hands play in the hair of the tragic boy. She was not a little
      girl heart-broken about him; she was a grown woman smiling at it all, but
      they were wet eyed smiles.
    </p>
    <p>
      Then she turned up the light, and Peter saw. He gave a cry of pain; and
      when the tall beautiful creature stooped to lift him in her arms he drew
      back sharply.
    </p>
    <p>
      "What is it?" he cried again.
    </p>
    <p>
      She had to tell him.
    </p>
    <p>
      "I am old, Peter. I am ever so much more than twenty. I grew up long ago."
    </p>
    <p>
      "You promised not to!"
    </p>
    <p>
      "I couldn't help it. I am a married woman, Peter."
    </p>
    <p>
      "No, you're not."
    </p>
    <p>
      "Yes, and the little girl in the bed is my baby."
    </p>
    <p>
      "No, she's not."
    </p>
    <p>
      But he supposed she was; and he took a step towards the sleeping child
      with his dagger upraised. Of course he did not strike. He sat down on the
      floor instead and sobbed; and Wendy did not know how to comfort him,
      though she could have done it so easily once. She was only a woman now,
      and she ran out of the room to try to think.
    </p>
    <p>
      Peter continued to cry, and soon his sobs woke Jane. She sat up in bed,
      and was interested at once.
    </p>
    <p>
      "Boy," she said, "why are you crying?"
    </p>
    <p>
      Peter rose and bowed to her, and she bowed to him from the bed.
    </p>
    <p>
      "Hullo," he said.
    </p>
    <p>
      "Hullo," said Jane.
    </p>
    <p>
      "My name is Peter Pan," he told her.
    </p>
    <p>
      "Yes, I know."
    </p>
    <p>
      "I came back for my mother," he explained, "to take her to the Neverland."
    </p>
    <p>
      "Yes, I know," Jane said, "I have been waiting for you."
    </p>
    <p>
      When Wendy returned diffidently she found Peter sitting on the bed-post
      crowing gloriously, while Jane in her nighty was flying round the room in
      solemn ecstasy.
    </p>
    <p>
      "She is my mother," Peter explained; and Jane descended and stood by his
      side, with the look in her face that he liked to see on ladies when they
      gazed at him.
    </p>
    <p>
      "He does so need a mother," Jane said.
    </p>
    <p>
      "Yes, I know." Wendy admitted rather forlornly; "no one knows it so well
      as I."
    </p>
    <p>
      "Good-bye," said Peter to Wendy; and he rose in the air, and the shameless
      Jane rose with him; it was already her easiest way of moving about.
    </p>
    <p>
      Wendy rushed to the window.
    </p>
    <p>
      "No, no," she cried.
    </p>
    <p>
      "It is just for spring cleaning time," Jane said, "he wants me always to
      do his spring cleaning."
    </p>
    <p>
      "If only I could go with you," Wendy sighed.
    </p>
    <p>
      "You see you can't fly," said Jane.
    </p>
    <p>
      Of course in the end Wendy let them fly away together. Our last glimpse of
      her shows her at the window, watching them receding into the sky until
      they were as small as stars.
    </p>
    <p>
      As you look at Wendy, you may see her hair becoming white, and her figure
      little again, for all this happened long ago. Jane is now a common
      grown-up, with a daughter called Margaret; and every spring cleaning time,
      except when he forgets, Peter comes for Margaret and takes her to the
      Neverland, where she tells him stories about himself, to which he listens
      eagerly. When Margaret grows up she will have a daughter, who is to be
      Peter's mother in turn; and thus it will go on, so long as children are
      gay and innocent and heartless.
    </p>


\book[\theplaytitle]{\thetitle\\\emph{or}\\\theplaytitle}

% !TEX program = pdflatex
% !TEX encoding = UTF-8
% !TEX spellcheck = en_GB
% !TEX root = peter-pan.tex

\emph{Peter Pan} was first produced at the Duke of York’s Theatre, London, on December 27, 1904.

The play ran for 145 performances.

\endinput

% TODO: Format this information nicely

Original cast:

Peter Pan - Miss Nina Boucicault
Mr.\@ Darling - Mr.\@ Gerald du Maurier
Mrs\@. Darling - Miss Dorothea Baird
Wendy Moira Angela Darling - Miss Hilda Trevelyan
John Napoleon Darling - Master George Hersee
Michael Nicholas Darling - Miss Winifred Geoghgan
Nana - Mr.\@ Arthur Lupino
Tinker Bell - Miss Jane Wren

Members of Peter's Band
Tootles - Miss Joan Burnett
Nibs - Miss Christine Silver
Slightly - Mr.\@ A. W. Baskcomb
Curly - Miss Alice DuBarry
1st Twin - Miss Pauline Chase
2nd Twin - Miss Phyllis Beadon

Jas.\@ Hook - Mr.\@ Gerald du Maurier

Pirates
Smee - Mr.\@ George Shelton
Gentleman Starkey - Mr.\@ Sidney Harcourt
Cookson - Mr.\@ Charles Trevor
Cecco - Mr.\@ Frederick Annerley
Mullins - Mr.\@ Hubert Willis
Jukes - Mr.\@ James English
Noodler - Mr.\@ John Kelt

Redskins
Great Big Little Panther - Mr.\@ Philip Darwin
Tiger Lily - Miss Miriam Nesbitt

Liza - Ela Q. May

Redskins, Pirates, Crocodile, Eagle, Ostrich, Pack of Wolves,
by Misses 


First Pirate /Gerald Malvern Second Pirate / J. Grahame Black Pirate/ S. Spencer Crocodile / A. Ganker & C. Lawton Ostrich / G. Henson
Director - Mr. Dion Boucicault Jr

% !TEX program = pdflatex
% !TEX encoding = UTF-8
% !TEX spellcheck = en_GB
% !TEX root = peter-pan.tex

\centerline{\chapnumfont{To the Five}}\afterchapternum
{\centering \chaptitlefont\MakeTextUppercase{A Dedication}\afterchaptertitle}
\addcontentsline{toc}{chapter}{To the Five: A Dedication}

Some disquieting confessions must be made in printing at last the play of \emph{Peter Pan};
among them this, that I have no recollection of having written it.
Of that, however, anon.
What I want to do first is to give Peter to the Five without whom he never would have existed.
I hope, my dear sirs,
that in memory of what we have been to each other
you will accept this dedication with your friend’s love.
The play of Peter is streaky with you still,
though none may see this save ourselves.
A score of Acts had to be left out, and you were in them all.
We first brought Peter down, didn’t we, with a blunt‐headed arrow in Kensington Gardens?
I seem to remember that we believed we had killed him,
though he was only winded,
and that after a spasm of exultation in our prowess
the more soft hearted among us wept and all of us thought of the police.
There was not one of you who would not have sworn as an eye‐witness to this occurrence;
no doubt I was abetting,
but you used to provide corroboration that was never given to you by me.
As for myself,
I suppose I always knew that I made Peter by rubbing the five of you violently together,
as savages with two sticks produce a flame.
That is all he is, the spark I got from you.

We had good sport of him before we clipped him small to make him fit the boards.
Some of you were not born when the story began
and yet were hefty figures before we saw that the game was up.
Do you remember a garden at Burpham and the initiation there of No.~4 when he was six weeks old,
and three of you grudged letting him in so young?
Have you, No.~3, forgotten the white violets at the Cistercian abbey
in which we cassocked our first fairies
(all little friends of St.\@ Benedict),
or your cry to the Gods, ‘Do I just kill one pirate all the time?’
Do you remember Marooners’ Hut in the haunted groves of Waverley,
and the St.\@ Bernard dog in a tiger’s mask who so frequently attacked you,
and the literary record of that summer, \emph{The Boy Castaways},
which is so much the best and the rarest of this author’s works?
What was it that made us eventually give to the public in the thin form of a play
that which had been woven for ourselves alone?
Alas, I know what it was, I was losing my grip.
One by one as you swung monkey‐wise from branch to branch in the wood of make‐believe
you reached the tree of knowledge.
Sometimes you swung back into the wood,
as the unthinking may at a cross‐road take a familiar path that no longer leads to home;
or you perched ostentatiously on its boughs to please me,
pretending that you still belonged;
soon you knew it only as the vanished wood,
for it vanishes if one needs to look for it.
A time came when I saw that No.~1, the most gallant of you all,
ceased to believe that he was ploughing woods incarnadine,
and with an apologetic eye for me derided the lingering faith of No.~2;
when even No.~3 questioned gloomily whether he did not really spend his nights in bed.
There were still two who knew no better, but their day was dawning.
In these circumstances, I suppose, was begun the writing of the play of Peter.
That was a quarter of a century ago,
and I clutch my brows in vain to remember
whether it was a last desperate throw to retain the five of you for a little longer,
or merely a cold decision to turn you into bread and butter.

This brings us back to my uncomfortable admission
that I have no recollection of writing the play of \emph{Peter Pan},
now being published for the first time so long after he made his bow upon the stage.
You had played it until you tired of it,
and tossed it in the air and gored it and left it derelict in the mud
and went on your way singing other songs;
and then I stole back and sewed some of the gory fragments together with a pen‐nib.
That is what must have happened, but I cannot remember doing it.
I remember writing the story of \emph{Peter and Wendy} many years after the production of the play,
but I might have cribbed that from some typed copy.
I can haul back to mind the writing of almost every other assay of mine,
however forgotten by the pretty public;
but this play of Peter, no.
Even my beginning as an amateur playwright,
that noble mouthful, \emph{Bandelero the Bandit},
I remember every detail of its composition in my school days at Dumfries.
Not less vivid is my first little piece, produced by Mr.\@ Toole.
It was called \emph{Ibsen’s Ghost},
and was a parody of the mightiest craftsman that ever wrote for our kind friends in front.
To save the management the cost of typing I wrote out the ‘parts,’
after being told what parts were,
and I can still recall my first words, spoken so plaintively by a now famous actress,—%
‘To run away from my second husband just as I ran away from my first,
it feels quite like old times.’
On the first night a man in the pit found \emph{Ibsen’s Ghost} so diverting
that he had to be removed in hysterics.
After that no one seems to have thought of it at all.
But what a man to carry about with one!
How odd, too, that these trifles should adhere to the mind
that cannot remember the long job of writing Peter.
It does seem almost suspicious,
especially as I have not the original MS. of \emph{Peter Pan} (except a few stray pages)
with which to support my claim.
I have indeed another MS., lately made, but that ‘proves nothing.’
I know not whether I lost that original MS. or destroyed it or happily gave it away.
I talk of dedicating the play to you, but how can I prove it is mine?
How ought I to act if some other hand, who could also have made a copy,
thinks it worthwhile to contest the cold rights?
Cold they are to me now as that laughter of yours in which Peter came into being
long before he was caught and written down.
There is Peter still, but to me he lies sunk in the gay Black Lake.

Any one of you five brothers has a better claim to the authorship than most,
and I would not fight you for it,
but you should have launched your case long ago in the days when you most admired me,
which were in the first year of the play,
owing to a rumour’s reaching you that my spoils were one‐and‐sixpence a night.
This was untrue, but it did give me a standing among you.
You watched for my next play with peeled eyes, not for entertainment
but lest it contained some chance witticism of yours that could be challenged as collaboration;
indeed I believe there still exists a legal document,
full of the Aforesaid and Henceforward to be called Part‐Author,
in which for some such snatching I was tied down
to pay No.~2 one halfpenny daily throughout the run of the piece.

During the rehearsals of Peter
(and it is evidence in my favour that I was admitted to them)
a depressed man in overalls, carrying a mug of tea or a paint‐pot,
used often to appear by my side in the shadowy stalls and say to me,
‘The gallery boys won’t stand it.’
He then mysteriously faded away as if he were the theatre ghost.
This hopelessness of his is what all dramatists are said to feel at such times,
so perhaps he was the author.
Again, a large number of children whom I have seen playing Peter in their homes
with careless mastership, constantly putting in better words,
could have thrown it off with ease.
It was for such as they that after the first production
I had to add something to the play at the request of parents
(who thus showed that they thought me the responsible person)
about no one being able to fly until the fairy dust had been blown on him;
so many children having gone home and tried it from their beds and needed surgical attention.

Notwithstanding other possibilities, I think I wrote Peter,
and if so it must have been in the usual inky way.
Some of it, I like to think, was done in that native place which is the dearest spot on earth to me,
though my last heart‐beats shall be with my beloved solitary London that was so hard to reach.
I must have sat at a table with that great dog waiting for me to stop,
not complaining, for he knew it was thus we made our living,
but giving me a look when he found he was to be in the play, with his sex changed.
In after years when the actor who was Nana had to go to the wars
he first taught his wife how to take his place as the dog till he came back,
and I am glad that I see nothing funny in this;
it seems to me to belong to the play.
I offer this obtuseness on my part as my first proof that I am the author.

Some say that we are different people at different periods of our lives,
changing not through effort of will, which is a brave affair,
but in the easy course of nature every ten years or so.
I suppose this theory might explain my present trouble, but I don’t hold with it;
I think one remains the same person throughout,
merely passing, as it were, in these lapses of time from one room to another,
but all in the same house.
If we unlock the rooms of the far past we can peer in and see ourselves,
busily occupied in beginning to become you and me.
Thus, if I am the author in question
the way he is to go should already be showing in the occupant of my first compartment,
at whom I now take the liberty to peep.
Here he is at the age of seven or so with his fellow‐conspirator Robb, both in glengarry bonnets.
They are giving an entertainment in a tiny old washing‐house that still stands.
The charge for admission is preens, a bool, or a peerie
(I taught you a good deal of Scotch, so possibly you can follow that),
and apparently the culminating Act consists in our trying to put each other into the boiler,
though some say that I also addressed the spell‐bound audience.
This washing‐house is not only the theatre of my first play,
but has a still closer connection with Peter.
It is the original of the little house the Lost Boys built in the NeverLand for Wendy,
the chief difference being that it never wore John’s tall hat as a chimney.
If Robb had owned a lumhat I have no doubt that it would have been placed on the washing‐house.

Here is that boy again some four years older,
and the reading he is munching feverishly is about desert islands;
he calls them wrecked islands.
He buys his sanguinary tales surreptitiously in penny numbers.
I see a change coming over him;
he is blanching as he reads in the high‐class magazine, \emph{Chatterbox},
a fulmination against such literature,
and sees that unless his greed for islands is quenched he is for ever lost.
With gloaming he steals out of the house, his library bulging beneath his palpitating waistcoat.
I follow like his shadow, as indeed I am,
and watch him dig a hole in a field at Pathhead farm and bury his islands in it;
it was ages ago, but I could walk straight to that hole in the field now and delve for the remains.
I peep into the next compartment.
There he is again, ten years older,
an undergraduate now and craving to be a real explorer,
one of those who do things instead of prating of them,
but otherwise unaltered;
he might be painted at twenty on top of a mast,
in his hand a spy‐glass through which he rakes the horizon for an elusive strand.
I go from room to room,
and he is now a man, real exploration abandoned
(though only because no one would have him).
Soon he is even concocting other plays,
and quaking a little lest some low person counts how many islands there are in them.
I note that with the years the islands grow more sinister,
but it is only because he has now to write with the left hand, the right having given out;
evidently one thinks more darkly down the left arm.
Go to the keyhole of the compartment where he and I join up,
and you may see us wondering whether they would stand one more island.
This journey through the house may not convince any one that I wrote Peter,
but it does suggest me as a likely person.
I pause to ask myself whether I read \emph{Chatterbox} again, suffered the old agony,
and buried that MS. of the play in a hole in a field.

Of course this is over‐charged.
Perhaps we do change;
except a little something in us which is no larger than a mote in the eye,
and that, like it, dances in front of us beguiling us all our days.
I cannot cut the hair by which it hangs.

The strongest evidence that I am the author is to be found, I think,
in a now melancholy volume, the aforementioned \emph{The Boy Castaways};
so you must excuse me for parading that work here.
Officer of the Court, call \emph{The Boy Castaways}.
The witness steps forward and proves to be a book you remember well
though you have not glanced at it these many years.
I pulled it out of a bookcase just now not without difficulty,
for its recent occupation has been to support the shelf above.
I suppose, though I am uncertain,
that it was I and not you who hammered it into that place of utility.
It is a little battered and bent after the manner of those who shoulder burdens,
and ought (to our shame) to remind us of the witnesses
who sometimes get an hour off from the cells to give evidence before his Lordship.
I have said that it is the rarest of my printed works, as it must be,
for the only edition was limited to two copies, of which one
(there was always some devilry in any matter connected with Peter)
instantly lost itself in a railway carriage.
This is the survivor.
The idlers in court may have assumed that it is a handwritten screed, and are impressed by its bulk.
It is printed by Constable’s
(how handsomely you did us, dear *Blaikie*),
it contains thirty‐five illustrations
and is bound in cloth with a picture stamped on the cover
of the three eldest of you ‘setting out to be wrecked.’
This record is supposed to be edited by the youngest of the three,
and I must have granted him that honour
to make up for his being so often lifted bodily out of our adventures by his nurse,
who kept breaking into them for the fell purpose of giving him a midday rest.
No.~4 rested so much at this period that he was merely an honorary member of the band,
waving his foot to you for luck when you set off with bow and arrow to shoot his dinner for him;
and one may rummage the book in vain for any trace of No.~5.
Here is the title page, except that you are numbered instead of named—

\begin{center}
\begin{samepage}
	\uppercase{The Boy\\Castaways\\of Black Lake Island}

	Being a record of the Terrible\\
	Adventures of Three Brothers\\
	in the summer of 1901\\
	faithfully set forth\\by No.~3.

	LONDON\\[\baselineskip]

	Published by J. M. Barrie\\
	in the Gloucester Road\\
	1901
\end{samepage}
\end{center}

There is a long preface by No.~3 in which we gather your ages at this first flight.
‘No.~1 was eight and a month,
No.~2 was approaching his seventh lustrum,
and I was a good bit past four.’
Of his two elders, while commending their fearless dispositions,
the editor complains that they wanted to do all the shooting
and carried the whole equipment of arrows inside their shirts.
He is attractively modest about himself,
‘Of No.~3 I prefer to say nothing,
hoping that the tale as it is unwound will show that he was a boy of deeds rather than of words,’
a quality which he hints did not unduly protrude upon the brows of Nos.\@ 1 and 2.
His preface ends on a high note,
‘I should say that the work was in the first instance compiled as a record
simply at which we could whet our memories,
and that it is now published for No.~4’s benefit.
If it teaches him by example lessons in fortitude and manly endurance
we shall consider that we were not wrecked in vain.’

Published to whet your memories.
Does it whet them?
Do you hear once more, like some long‐forgotten whistle beneath your window
(Robb at dawn calling me to the fishing!\@)
the not quite mortal blows that still echo in some of, the chapter headings?—%
‘Chapter II, No.~1 teaches Wilkinson (his master) a Stern Lesson—We Run away to Sea.
Chapter III, A Fearful Hurricane—Wreck of the “Anna Pink”—%
We go crazy from Want of Food—Proposal to eat No.~3—Land Ahoy.’
Such are two chapters out of sixteen.
Are these again your javelins cutting tunes in the blue haze of the pines;
do you sweat as you scale the dreadful Valley of Rolling Stones,
and cleanse your hands of pirate blood by scouring them carelessly in Mother Earth?
Can you still make a fire
(you could do it once,
Mr.\@ Seton‐Thompson taught us in, surely an odd place, the Reform Club)
by rubbing those sticks together?
Was it the travail of hut‐building that subsequently advised Peter to find a ‘home under the ground’?
The bottle and mugs in that lurid picture, ‘Last night on the Island,’
seem to suggest that you had changed from Lost Boys into pirates,
which was probably also a tendency of Peter’s.
Listen again to our stolen saw‐mill, man’s proudest invention;
when he made the saw‐mill he beat the birds for music in a wood.

The illustrations (full‐paged) in \emph{The Boy Castaways} are all photographs taken by myself;
some of them indeed of phenomena that had to be invented afterwards,
for you were always off doing the wrong things when I pressed the button.
I see that we combined instruction with amusement;
perhaps we had given our kingly word to that effect.
How otherwise account for such wording to the pictures as these:
‘It is undoubtedly,’ says No.~1 in a fir tree that is bearing unwonted fruit, recently tied to it,
‘the \emph{Cocos nucifera},
for observe the slender columns supporting the crown of leaves which fall
with a grace that no art can imitate.’
‘Truly,’ continues No.~1 under the same tree in another forest as he leans upon his trusty gun,
‘though the perils of these happenings are great,
yet would I rejoice to endure still greater privations
to be thus rewarded by such wondrous studies of Nature.’
He is soon back to the practical, however,
‘recognising the Mango (\emph{Magnifera indica})
by its lancet‐shaped leaves and the cucumber‐shaped fruit.’
No.~1 was certainly the right sort of voyager to be wrecked with,
though if my memory fails me not, No.~2, to whom these strutting observations were addressed,
sometimes protested because none of them was given to him.
No.~3 being the author is in surprisingly few of the pictures,
but this, you may remember, was because the lady already darkly referred to
used to pluck him from our midst for his siesta at 12 o’clock,
which was the hour that best suited the camera.
With a skill on which he has never been complimented
the photographer sometimes got No.~3 nominally included in a wild‐life picture
when he was really in a humdrum house kicking on the sofa.
Thus in a scene representing Nos.\@ 1 and 2 sitting scowling outside the hut
it is untruly written that they scowled because
‘their brother was within singing and playing on a barbaric instrument.
The music,’ the unseen No.~3 is represented as saying (obviously forestalling No.~1),
‘is rude and to a cultured ear discordant,
but the songs like those of the Arabs are full of poetic imagery.’
He was perhaps allowed to say this sulkily on the sofa.

Though \emph{The Boy Castaways} has sixteen chapter‐headings,
there is no other letterpress;
an absence which possible purchasers might complain of,
though there are surely worse ways of writing a book than this.
These headings anticipate much of the play of \emph{Peter Pan},
but there were many incidents of our Kensington Gardens days that never got into the book,
such as our Antarctic exploits
when we reached the Pole in advance of our friend Captain Scott
and cut our initials on it for him to find,
a strange foreshadowing of what was really to happen.
In \emph{The Boy Castaways} Captain Hook has arrived but is called Captain Swarthy,
and he seems from the pictures to have been a black man.
This character, as you do not need to be told,
is held by those in the know to be autobiographical.
You had many tussles with him
(though you never, I think, got his right arm)
before you reached the terrible chapter
(which might be taken from the play)
entitled ‘We Board the Pirate Ship at Dawn—%
A Rakish Craft—No.~1 Hew‐them‐Down and No.~2 of the Red Hatchet—%
A Holocaust of Pirates—Rescue of Peter.’
(Hullo, Peter rescued instead of rescuing others?
I know what that means and so do you, but we are not going to give away all our secrets.)
The scene of the Holocaust is the Black Lake
(afterwards, when we let women in, the Mermaids’ Lagoon).
The pirate captain’s end was not in the mouth of a crocodile though we had crocodiles on the spot
(‘while No.~2 was removing the crocodiles from the stream
No.~1 shot a few parrots, \emph{Psittacidae}, for our evening meal’).
I think our captain had divers deaths owing to unseemly competition among you,
each wanting to slay him single‐handed.
On a special occasion, such as when No.~3 pulled out the tooth himself,
you gave the deed to him, but took it from him while he rested.
The only pictorial representation in the book of Swarthy’s fate is in two parts.
In one, called briefly ‘We string him up,’
Nos.\@ 1 and 2, stern as Athos, are hauling him up a tree by a rope,
his face snarling as if it were a grinning mask (which indeed it was),
and his garments very like some of my own stuffed with bracken.
The other, the same scene next day, is called ‘The Vultures had Picked him Clean,’
and tells its own tale.

The dog in \emph{The Boy Castaways} seems never to have been called Nana
but was evidently in training for that post.
He originally belonged to Swarthy (or to Captain Marryat?),
and the first picture of him, lean, skulking, and hunched (how did I get that effect?),
‘patrolling the island’ in the monster’s interests,
gives little indication of the domestic paragon he was to become.
We lured him away to the better life,
and there is, later, a touching picture, a clear forecast of the Darling nursery,
entitled ‘We trained the dog to watch over us while we slept.’
In this he also is sleeping, in a position that is a careful copy of his charges;
indeed any trouble we had with him was because, once he knew he was in a story,
he thought his safest course was to imitate you in everything you did.
How anxious he was to show that he understood the game,
and more generous than you, he never pretended that he was the one who killed Captain Swarthy.
I must not imply that he was entirely without initiative,
for it was his own idea to bark warningly a minute or two before twelve o’clock
as a signal to No.~3 that his keeper was probably on her way for him (Disappearance of No.~3);
and he became so used to living in the world of Pretend
that when we reached the hut of a morning he was often there waiting for us,
looking, it is true, rather idiotic,
but with a new bark he had invented which puzzled us
until we decided that he was demanding the password.
He was always willing to do any extra jobs, such as becoming the tiger in mask,
and when after a fierce engagement you carried home that mask in triumph,
he joined in the procession proudly and never let on that the trophy had ever been part of him.
Long afterwards he saw the play from a box in the theatre,
and as familiar scenes were unrolled before his eyes I have never seen a dog so bothered.
At one matinee we even let him for a moment take the place of the actor who played Nana,
and I don’t know that any members of the audience ever noticed the change,
though he introduced some ‘business’ that was new to them but old to you and me.
Heigh‐ho, I suspect that in this reminiscence I am mixing him up with his successor,
for such a one there had to be,
the loyal Newfoundland who, perhaps in the following year, applied, so to say, for the part
by bringing hedgehogs to the hut in his mouth as offerings for our evening repasts.
The head and coat of him were copied for the Nana of the play.

They do seem to be emerging out of our island, don’t they,
the little people of the play,
all except that sly one, the chief figure,
who draws farther and farther into the wood as we advance upon him?
He so dislikes being tracked, as if there were something odd about him,
that when he dies he means to get up and blow away the particle that will be his ashes.

Wendy has not yet appeared,
but she has been trying to come
ever since that loyal nurse cast the humorous shadow of woman upon the scene
and made us feel that it might be fun to let in a disturbing element.
Perhaps she would have bored her way in at last whether we wanted her or not.
It may be that even Peter did not really bring her to the Never Land of his free will,
but merely pretended to do so because she would not stay away.
Even Tinker Bell had reached our island before we left it.
It was one evening when we climbed the wood carrying No.~4
to show him what the trail was like by twilight.
As our lanterns twinkled among the leaves
No.~4 saw a twinkle stand still for a moment and he waved his foot gaily to it,
thus creating Tink.
It must not be thought, however,
that there were any other sentimental passages between No.~4 and Tink;
indeed, as he got to know her better
he suspected her of frequenting the hut to see what we had been having for supper,
and to partake of the same,
and he pursued her with malignancy.

A safe but sometimes chilly way of recalling the past is to force open a crammed drawer.
If you are searching for anything in particular you don’t find it,
but something falls out at the back that is often more interesting.
It is in this way that I get my desultory reading,
which includes the few stray leaves of the original MS. of Peter that I have said I do possess,
though even they, when returned to the drawer, are gone again,
as if that touch of devilry lurked in them still.
They show that in early days I hacked at and added to the play.
In the drawer I find some scraps of Mr.\@ Crook’s delightful music,
and other incomplete matter relating to Peter.
Here is the reply of a boy whom I favoured with a seat in my box
and injudiciously asked at the end what he had liked best.
‘What I think I liked best,’ he said,
‘was tearing up the programme and dropping the bits on people’s heads.’
Thus am I often laid low.
A copy of my favourite programme of the play is still in the drawer.
In the first or second year of Peter No.~4 could not attend through illness,
so we took the play to his nursery, far away in the country,
an array of vehicles almost as glorious as a travelling circus;
the leading parts were played by the youngest children in the London company,
and No.~4, aged five, looked on solemnly at the performance from his bed and never smiled once.
That was my first and only appearance on the real stage,
and this copy of the programme shows I was thought so meanly of as an actor
that they printed my name in smaller letters than the others.

I have said little here of Nos.\@ 4 and 5, and it is high time I had finished.
They had a long summer day, and I turn round twice and now they are off to school.
On Monday, as it seems,
I was escorting No.~5 to a children’s party and brushing his hair in the ante‐room;
and by Thursday he is placing me against the wall of an underground station and saying,
‘Now I am going to get the tickets;
don’t move till I come back for you or you’ll lose yourself.’
No.~4 jumps from being astride my shoulders fishing, I knee‐deep in the stream,
to becoming, while still a schoolboy, the sternest of my literary critics.
Anything he shook his head over I abandoned,
and conceivably the world has thus been deprived of masterpieces.
There was for instance an unfortunate little tragedy which I liked
until I foolishly told No.~4 its subject,
when he frowned and said he had better have a look at it.
He read it, and then, patting me on the back, as only he and No.~1 could touch me, said,
‘You know you can’t do this sort of thing.’
End of a tragedian.
Sometimes, however, No.~4 liked my efforts,
and I walked in the azure that day when he returned \emph{Dear Brutus} to me
with the comment ‘Not so bad.’
In earlier days, when he was ten, I offered him the MS. of my book \emph{Margaret Ogilvy}.
‘Oh, thanks,’ he said almost immediately, and added, ‘Of course my desk is awfully full.’
I reminded him that he could take out some of its more ridiculous contents.
He said, ‘I have read it already in the book.’
This I had not known, and I was secretly elated,
but I said that people sometimes liked to preserve this kind of thing as a curiosity.
He said ‘Oh’ again.
I said tartly that he was not compelled to take it if he didn’t want it.
He said, ‘Of course I want it, but my desk———’¤
Then he wriggled out of the room and came back in a few minutes
dragging in No.~5 and announcing triumphantly, ‘No.~5 will have it.’

The rebuffs I have got from all of you!
They were especially crushing in those early days when one by one
you came out of your belief in fairies and lowered on me as the deceiver.
My grandest triumph, the best thing in the play of \emph{Peter Pan}
(though it is not in it),
is that long after No.~4 had ceased to believe,
I brought him back to the faith for at least two minutes.
We were on our way in a boat to fish the Outer Hebrides
(where we caught \emph{Mary Rose}),
and though it was a journey of days he wore his fishing basket on his back all the time,
so as to be able to begin at once.
His one pain was the absence of Johnny Mackay,
for Johnny was the loved gillie of the previous summer
who had taught him everything that is worth knowing
(which is a matter of flies)
but could not be with us this time as he would have had to cross and re‐cross Scotland to reach us.
As the boat drew near the Kyle of Lochalsh pier
I told Nos.\@ 4 and 5 it was such a famous wishing pier
that they had now but to wish and they should have.
No.~5 believed at once and expressed a wish to meet himself
(I afterwards found him on the pier searching faces confidently),
but No.~4 thought it more of my untimely nonsense and doggedly declined to humour me.
‘Whom do you want to see most, No.~4?’
‘Of course I would like most to see Johnny Mackay.’
‘Well, then, wish for him.’
‘Oh, rot.’
‘It can’t do any harm to wish.’
Contemptuously he wished,
and as the ropes were thrown on the pier he saw Johnny waiting for him,
loaded with angling paraphernalia.
I know no one less like a fairy than Johnny Mackay,
but for two minutes No.~4 was quivering in another world than ours.
When he came to he gave me a smile which meant that we understood each other,
and thereafter neglected me for a month, being always with Johnny.
As I have said, this episode is not in the play;
so though I dedicate \emph{Peter Pan} to you I keep the smile,
with the few other broken fragments of immortality that have come my way.

\endinput

% !TEX program = pdflatex
% !TEX encoding = UTF-8
% !TEX spellcheck = en_GB
% !TEX root = peter-pan.tex

\Character[Peter Pan]{PETER}{peter}
\Character[Tinker Bell]{TINK}{tink}

\Character[Mr.\@ George Darling]{MR.\@ DARLING}{mrdarling}
\Character[Mrs\@. (Mary) Darling]{MRS.\@ DARLING}{mrsdarling}
\Character[Wendy Moira Angela Darling]{WENDY}{wendy}
\Character[John Napoleon Darling]{JOHN}{john}
\Character[Michael Nicholas Darling]{MICHAEL}{michael}
\Character[Nana]{NANA}{nana}
\Character[Liza]{LIZA}{liza}

\begin{CharacterGroup}{Lost Boys}
\GCharacter{Tootles}{TOOTLES}{tootles}
\GCharacter{Nibs}{NIBS}{nibs}
\GCharacter{Slightly}{SLIGHTLY}{slightly}
\GCharacter{Curly}{CURLY}{curly}
\GCharacter{The Twins}{TWINS}{twins}
\end{CharacterGroup}

\Character[Captain Jas.\@ Hook]{HOOK}{hook}

\begin{CharacterGroup}{Pirates}
\GCharacter{Smee}{SMEE}{smee}
\GCharacter{Gentleman Starkey}{STARKEY}{starkey}
\GCharacter{Cookson}{COOKSON}{cookson}
\GCharacter{Cecco}{CECCO}{cecco}
\GCharacter{Mullins}{MULLINS}{mullins}
\GCharacter{Bill Jukes}{JUKES}{jukes}
\GCharacter{Noodler}{NOODLER}{noodler}
\end{CharacterGroup}

\begin{CharacterGroup}{Redskins}
\GCharacter{Great Big Little Panther}{PANTHER}{panther}
\GCharacter{Tiger Lily}{TIGER LILY}{tigerlily}
\end{CharacterGroup}

\DramPer

% Combined in dramatis personæ
\Character{FIRST TWIN}{firsttwin}
\Character{SECOND TWIN}{secondtwin}

\endinput


% !TEX program = pdflatex
% !TEX encoding = UTF-8
% !TEX spellcheck = en_GB
% !TEX root = peter-pan.tex

\Act{The Nursery}

\begin{stagedir}
The night nursery of the Darling family,
which is the scene of our opening Act,
is at the top of a rather depressed street in Bloomsbury.
We have a right to place it where we will,
and the reason Bloomsbury is chosen is that Mr.\@ Roget once lived there.
So did we in days when his \emph{Thesaurus} was our only companion in London;
and we whom he has helped to wend our way through life
have always wanted to pay him a little compliment.
The Darlings therefore lived in Bloomsbury.

It is a corner house whose top window, the important one,
looks upon a leafy square from which Peter used to fly up to it,
to the delight of three children and no doubt the irritation of passers-by.
The street is still there, though the steaming sausage shop has gone;
and apparently the same cards perch now as then over the doors,
inviting homeless ones to come and stay with the hospitable inhabitants.
Since the days of the Darlings, however, a lick of paint has been applied;
and our corner house in particular, which has swallowed its neighbour,
blooms with awful freshness as if the colours had been discharged upon it through a hose.
Its card now says 'No children'
meaning maybe that the goings-on of Wendy and her brothers have given the house a bad name.
As for ourselves, we have not been in it since we went back to reclaim our old \emph{Thesaurus}.

That is what we call the Darling house,
but you may dump it down anywhere you like, and if you think it was your house you are very probably right.
It wanders about London looking for anybody in need of it, like the little house in the Never Land.

The blind
(which is what Peter would have called the theatre curtain if he had ever seen one)
rises on that top room,
a shabby little room if Mrs.\@ Darling had not made it the hub of creation by her certainty that such it was,
and adorned it to match with a loving heart and all the scrapings of her purse.
The door on the right leads into the day nursery,
which she has no right to have, but she made it herself with nails in her mouth and a paste-pot in her hand.
This is the door the children will come in by.
There are three beds and (rather oddly) a large dog-kennel;
two of these beds, with the kennel, being on the left and the other on the right.
The coverlets of the beds (if visitors are expected) are made out of Mrs.\@ Darling's wedding-gown,
which was such a grand affair that it still keeps them pinched.
Over each bed is a china house, the size of a linnet's nest, containing a night-light.
The fire, which is on our right, is burning as discreetly as if it were in custody,
which in a sense it is, for supporting the mantelshelf are two wooden soldiers, home-made,
begun by Mr.\@ Darling, finished by Mrs.\@ Darling, repainted (unfortunately) by John Darling.
On the fire-guard hang incomplete parts of children's night attire.
The door the parents will come in by is on the left.
At the back is the bathroom door, with a cuckoo clock over it;
and in the centre is the window, which is at present ever so staid and respectable,
but half an hour hence (namely at 6.30 p.m.) will be able to tell a very strange tale to the police.

The only occupant of the room at present is Nana the nurse,
reclining, not as you might expect on the one soft chair, but on the floor.
She is a Newfoundland dog, and though this may shock the grandiose,
the not exactly affluent will make allowances.
The Darlings could not afford to have a nurse,
they could not afford indeed to have children;
and now you are beginning to understand how they did it.
Of course Nana has been trained by Mrs.\@ Darling,
but like all treasures she was born to it.
In this play we shall see her chiefly inside the house,
but she was just as exemplary outside,
escorting the two elders to school with an umbrella in her mouth, for instance,
and butting them back into line if they strayed.

The cuckoo clock strikes six, and Nana springs into life.
This first moment in the play is tremendously important,
for if the actor playing Nana does not spring properly we are undone.
She will probably be played by a boy, if one clever enough can be found,
and must never be on two legs except on those rare occasions when an ordinary nurse would be on four.
This Nana must go about all her duties in a most ordinary manner,
so that you know in your bones that she performs them just so every evening at six;
naturalness must be her passion;
indeed, it should be the aim of every one in the play, for which she is now setting the pace.
All the characters, whether grown-ups or babes,
must wear a child's outlook on life as their only important adornment.
If they cannot help being funny they are begged to go away.
A good motto for all would be 'The little less, and how much it is.'

Nana, making much use of her mouth, 'turns down' the beds,
and carries the various articles on the fire-guard across to them.
Then pushing the bathroom door open,
she is seen at work on the taps preparing Michael's bath;
after which she enters from the day nursery with the youngest of the family on her back.
\end{stagedir}

\begin{drama}

\michaelspeaks \direct{obstreperous}¤
I won't go to bed, I won't, I won't.
Nana, it isn't six o'clock yet.
Two minutes more, please, one minute more?
Nana, I won't be bathed, I tell you I will not be bathed.

\end{drama}

\endinput


<p><i>(Here the bathroom door closes on them, and</i> MRS. DARLING,
<i>who has perhaps heard his cry, enters the nursery. She is the
loveliest lady in Bloomsbury, with a sweet mocking mouth, and as she
is going out to dinner tonight she is already wearing her evening
gown because she knows her children like to see her in it. It is a
delicious confection made by herself out of nothing and other
people's mistakes. She does not often go out to dinner, preferring
when the children are in bed to sit beside them tidying up their
minds, just as if they were drawers. If</i> WENDY <i>and the boys
could keep awake they might see her repacking into their proper
places the many articles of the mind that have strayed during the
day, lingering humorously over some of their contents, wondering
where on earth they picked this thing up, making discoveries sweet
and not so sweet, pressing this to her cheek and hurriedly stowing
that out of sight. When they wake in the morning the naughtinesses
with which they went to bed are not, alas, blown away, but they are
placed at the bottom of the drawer; and on the top, beautifully
aired, are their prettier thoughts ready forthe new day.</i></p>

<p><i>As she enters the room she is startled to see a strange little
face outside the window and a hand groping as if it wanted to come
in.)</i></p>

<p>MRS. DARLlNG. Who are you? <i>(The unknown disappears; she hurries
to the window.)</i> No one there. And yet I feel sure I saw a face.
My children! <i>(She throws open the bathroom door and</i> MICHAEL'S
<i>head appears gaily over the bath. He splashes; she throws kisses
to him and closes the door.'Wendy, John,' she cries, and gets
reassuring answers from the day nursery. She sits down, relieved,
on</i> WENDY'S <i>bed; and</i> WENDY <i>and</i> JOHN <i>come in,
looking their smallest size, as children tend to do to a mother
suddenly in fear for them.)</i></p>

<p>JOHN <i>(histrionically).</i> We are doing an act; we are playing
at being you and father. <i>(He imitates the only father who has come
under his special notice.)</i> A little less noise there.</p>

<p>WENDY. Now let us pretend we have a baby.</p>

<p>JOHN <i>(good-naturedly).</i> I am happy to inform you,
Mrs.Darling, that you are now a mother. (WENDY <i>gives way to
ecstasy.)</i> You have missed the chief thing; you haven't asked,
'boy or girl?'</p>

<p>WENDY. I am so glad to have one at all, I don't care which it
is.</p>

<p>JOHN <i>(crushingly).</i> That is just the difference between
gentlemen and ladies. Now you tell me.</p>

<p>WENDY. I am happy to acquaint you, Mr. Darling, you are now a
father.</p>

<p>JOHN. Boy or girl?</p>

<p>WENDY <i>(presenting herself).</i> Girl.</p>

<p>JOHN. Tuts.</p>

<p>WENDY. You horrid.</p>

<p>JOHN. Go on.</p>

<p>WENDY. I am happy to acquaint you, Mr. Darling, you are again a
father.</p>

<p>JOHN. Boy or girl?</p>

<p>WENDY. Boy. (JOHN <i>beams.)</i> Mummy, it's hateful of him.</p>

<blockquote>(MICHAEL <i>emerges from the bathroom in</i> JOHN'S
<i>old pyjamas and giving his face a last wipe with the
towel.)</i></blockquote>

<p>MICHAEL <i>(expanding).</i> Now, John, have me.</p>

<p>JOHN. We don't want any more.</p>

<p>MICHAEL <i>(contracting).</i> Am I not to be born at all?</p>

<p>JOHN. Two is enough.</p>

<p>MICHAEL <i>(wheedling).</i> Come, John; boy, John.
<i>(Appalled)</i> Nobody wants me!</p>

<p>MRS. DARLING. I do.</p>

<p>MICHAEL <i>(with a glimmer of hope).</i> Boy or girl?</p>

<p>MRS. DARLING <i>(with one of those happy thoughts of hers).</i>
Boy.</p>

<blockquote>(<i>Triumph of</i> MICHAEL; <i>discomfiture of</i> JOHN.
MR.DARLING <i>arrives, in no mood unfortunately to gloat over this
domestic scene. He is really a good man as breadwinners go, and it is
hard luck for him to be propelled into the room now, when if we had
brought him in a few minutes earlier or later he might have made a
fairer impression. In the city where he sits on a stool all day, as
fixed as a postage stamp, he is so like all the others on stools that
you recognise him not by his face but by his stool, but at home the
way to gratify him is to say that he has a distinct personality. He
is very conscientious, and in the days when</i> MRS. DARLING <i>gave
up keeping the house books correctly and drew pictures instead (which
he called her guesses), he did all the totting up for her, holding
her hand while he calculated whether they could have Wendy or not,
and coming down on the right side. It is with regret, therefore, that
we introduce him as a tornado, rushing into the nursery in evening
dress, but without his coat, and brandishing in his hand a
recalcitrant white tie.)</i></blockquote>

<p>MR. DARLING <i>(implying that he has searched for her everywhere
and that the nursery is a strange place in which to find her).</i>
Oh, here you are, Mary.</p>

<p>MRS. DARLING <i>(knowing at once what is the matter).</i> What is
the matter, George dear?</p>

<p>MR. DARLING <i>(as if the word were monstrous).</i> Matter! This
tie, it will not tie. <i>(He waxes sarcastic.)</i> Not round my neck.
Round the bed-post, oh yes; twenty times have I made it up round the
bed-post, but round my neck, oh dear no; begs to be excused.</p>

<p>MICHAEL <i>(in a joyous transport).</i> Say it again, father, say
it again!</p>

<p>MR. DARLING (<i>witheringly</i>). Thank you. <i>(Goaded by
asuspiciously crooked smile on</i> MRS. DARLING'S <i>face)</i> I warn
you, Mary, that unless this tie is round my neck we don't goout to
dinner to-night, and if I don't go out to dinner to-night I never go
to the office again, and if I don't go to the office again you and I
starve, and our children will be thrown into the streets.</p>

<blockquote><i>(The children blanch as they grasp the gravity of the
situation.)</i></blockquote>

<p>MRS. DARLING. Let me try, dear.</p>

<blockquote><i>(In a terrible silence their progeny cluster round
them. Will she succeed? Their fate depends on it. She
fails</i>&mdash;<i>no, she succeeds. In another moment they are
wildly gay, romping round the room on each other's shoulders. Father
is even a better horse than mother.</i> MICHAEL <i>is dropped upon
his bed,</i> WENDY <i>retires to prepare for hers,</i> JOHN <i>runs
from</i> NANA, <i>who has reappeared with the bath
towel.)</i></blockquote>

<p>JOHN <i>(rebellious).</i> I won't be bathed. You needn't think
it.</p>

<p>MR. DARLING <i>(in the grand manner).</i> Go and be bathed at
once, sir.</p>

<blockquote><i>(With bent head</i> JOHN <i>follows</i> NANA <i>into
the bathroom.</i> MR. DARLING <i>swells.)</i></blockquote>

<p>MICHAEL <i>(as he is put between the sheets).</i> Mother, how did
you get to know me?</p>

<p>MR. DARLING. A little less noise there.</p>

<p>MICHAEL <i>(growing solemn).</i> At what time was I born,
mother?</p>

<p>MRS. DARLING. At two o'clock in the night-time, dearest.</p>

<p>MICHAEL. Oh, mother, I hope I didn't wake you.</p>

<p>MRS. DARLING. They are rather sweet, don't you think,George?</p>

<p>MR. DARLING <i>(doting).</i> There is not their equal on earth,
and they are ours, ours!</p>

<blockquote><i>(Unfortunately</i> NANA <i>has come from the bathroom
for a sponge and she collides with his trousers<sub>)</sub> the first
pair</i> <i>he has ever had with braid on them.)</i></blockquote>

<p>MR. DARLING. Mary, it is too bad; just look at this; covered with
hairs. Clumsy, clumsy!</p>

<blockquote>(NANA <i>goes, a drooping figure.)</i></blockquote>

<p>MRS. DARLING. Let me brush you, dear.</p>

<blockquote><i>(Once more she is successful. They are now by the
fire, and</i> MICHAEL <i>is in bed doing idiotic things with a teddy
bear.)</i></blockquote>

<p>MR. DARLING <i>(depressed).</i> I sometimes think, Mary, that it
is a mistake to have a dog for a nurse.</p>

<p>MRS. DARLING. George, Nana is a treasure.</p>

<p>MR. DARLING. No doubt; but I have an uneasy feeling at times that
she looks upon the children as puppies.</p>

<p>MRS. DARLING <i>(rather faintly).</i> Oh no, dear one, I am sure
she knows they have souls.</p>

<p>MR. DARLING <i>(profoundly).</i> I wonder, I wonder.</p>

<blockquote><i>(The opportunity has come for her to tell him of
something that is on her mind.)</i></blockquote>

<p>MRS. DARLING. George, we must keep Nana. I will tell you why.
<i>(Her seriousness impresses him.)</i> My dear, when I came into
this room to-night I saw a face at the window.</p>

<p>MR. DARLING <i>(incredulous). A</i> face at the window, three
floors up? Pooh!</p>

<p>MRS. DARLING. It was the face of a little boy; he was trying to
get in. George, this is not the first time I have seen that boy.</p>

<p>MR. DARLING <i>(beginning to think that this may be a man's
job).</i> Oho!</p>

<p>MRS. DARLING <i>(making sure that</i> MICHAEL <i>does not
hear).</i> The first time was a week ago. It was Nana's night out,
and I had been drowsing here by the fire when suddenly I felt a
draught, as if the window were open. I looked round and I saw that
boy&mdash;in the room.</p>

<p>MR. DARLING. In the room?</p>

<p>MRS. DARLING. I screamed. Just then Nana came back and she at once
sprang at him. The boy leapt for the window. She pulled down the sash
quickly, but was too late to catch him.</p>

<p>MR. DARLING <i>(who knows he would not have been too late).</i> I
thought so!</p>

<p>MRS. DARLING. Wait. The boy escaped, but his shadow had not time
to get out; down came the window and cut it clean off.</p>

<p>MR. DARLING <i>(heavily).</i> Mary, Mary, why didn't you keep that
shadow?</p>

<p>MRS. DARLING <i>(scoring).</i> I did. I rolled it up, George; and
here it is.</p>

<blockquote><i>(She produces it from a drawer. They unroll and
examine the flimsy thing, which is not more material than a puff of
smoke, and if let go would probably float into the ceiling without
discolouring it. Yet it has human shape. As they nod their heads over
it they present the most satisfying picture on earth, two happy
parents conspiring cosily by the fire for the good of their
children.)</i></blockquote>

<p>MR. DARLING. It is nobody I know, but he does look ascoundrel.</p>

<p>MRS. DARLING. I think he comes back to get his shadow,George.</p>

<p>MR. DARLING <i>(meaning that the miscreant has now a father to
deal with).</i> I dare say. <i>(He sees himself telling the story to
the other stools at the office.)</i> There is money in this, my love.
I shall take it to the British Museum to-morrow and have it
priced.</p>

<blockquote><i>(The shadow is rolled up and replaced in the
drawer.)</i></blockquote>

<p>MRS. DARLING <i>(like a guilty person).</i> George, I have not
told you all; I am afraid to.</p>

<p>MR. DARLING <i>(who knows exactly the right moment to treat a
woman as a beloved child).</i> Cowardy, cowardy custard.</p>

<p>MRS. DARLING <i>(pouting).</i> No, I 'm not.</p>

<p>MR. DARLING. Oh yes, you are.</p>

<p>MRS. DARLING. George, I 'm not.</p>

<p>MR. DARLING. Then why not tell? <i>(Thus cleverly soothed she goes
on.)</i></p>

<p>MRS. DARLING. The boy was not alone that first time. He was
accompanied by&mdash;I don't know how to describe it; by a ball of
light, not as big as my fist, but it darted about the room like a
living thing.</p>

<p>MR. DARLING <i>(though open-minded).</i> That is very unusual. It
escaped with the boy?</p>

<p>MRS. DARLING. Yes. <i>(Sliding her hand into his.)</i> George,
what can all this mean?</p>

<p>MR. DARLING <i>(ever ready).</i> What indeed!</p>

<blockquote><i>(This intimate scene is broken by the return of</i>
NANA <i>with a bottle in her mouth.)</i></blockquote>

<p>MRS. DARLING <i>(at once dissembling).</i> What is that, Nana? Ah,
of course; Michael, it is your medicine.</p>

<p>MICHAEL <i>(promptly).</i> Won't take it.</p>

<p>MR. DARLING <i>(recalling his youth).</i> Be a man, Michael.</p>

<p>MICHAEL. Won't.</p>

<p>MRS. DARLING <i>(weakly).</i> I'll get you a lovely chocky to take
after it. <i>(She leaves the room, though her husband calls after
her.)</i></p>

<p>MR. DARLING. Mary, don't pamper him. When I was your age, Michael,
I took medicine without a murmur. I said 'Thank you, kind parents,
for giving me bottles to make me well.'</p>

<blockquote>(WENDY, <i>who has appeared in her nightgown, hears this
and believes.)</i></blockquote>

<p>WENDY. That medicine you sometimes take is much nastier, isn't it,
father?</p>

<p>MR. DARLING <i>(valuing her support).</i> Ever so much nastier.And
as an example to you, Michael, I would take it now
<i>(thankfully)</i> if I hadn't lost the bottle.</p>

<p>WENDY <i>(always glad to be of service).</i> I know where itis,
father. I'll fetch it.</p>

<blockquote><i>(She is gone before he can stop her. He turns for help
to</i> JOHN, <i>who has come from the bathroom attired for
bed.)</i></blockquote>

<p>MR. DARLING. John, it is the most beastly stuff. It is that sticky
sweet kind.</p>

<p>JOHN <i>(who is perhaps still playing at parents).</i> Never mind,
father, it will soon be over.</p>

<blockquote><i>(A spasm of ill-will to</i> JOHN <i>cuts through</i>
MR. DARLING, <i>and is gone.</i> WENDY <i>returns
panting.)</i></blockquote>

<p>WENDY. Here it is, father; I have been as quick as I could.</p>

<p>MR. DARLING <i>(with a sarcasm that is completely thrown away on
her).</i> You have been wonderfully quick, precious quick!</p>

<blockquote><i>(He is now at the foot of</i> MICHAEL'S <i>bed,</i>
NANA <i>is by its side, holding the medicine spoon insinuatingly in
her mouth.)</i></blockquote>

<p>WENDY <i>(proudly, as she pours out</i> MR. DARLING'S
<i>medicine).</i> Michael, now you will see how father takes it.</p>

<p>MR. DARLING <i>(hedging).</i> Michael first.</p>

<p>MICHAEL <i>(full of unworthy suspicions).</i> Father first.</p>

<p>MR. DARLING. It will make me sick, you know.</p>

<p>JOHN <i>(lightly).</i> Come on, father.</p>

<p>MR. DARLING. Hold your tongue, sir.</p>

<p>WENDY <i>(disturbed).</i> I thought you took it quite easily,
father, saying 'Thank you, kind parents,
for&mdash;&mdash;&mdash;'</p>

<p>MR. DARLING. That is not the point; the point is that there is
more in my glass than in Michael's spoon. It isn't fair, I swear
though it were with my last breath, it is not fair.</p>

<p>MICHAEL <i>(coldly).</i> Father, I'm waiting.</p>

<p>MR. DARLING. It's all very well to say you are waiting; soam I
waiting.</p>

<p>MICHAEL. Father 's a cowardy custard.</p>

<p>MR. DARLING. So are you a cowardy custard.</p>

<blockquote><i>(They are now glaring at each other.)</i></blockquote>

<p>MICHAEL. I am not frightened.</p>

<p>MR. DARLING. Neither am I frightened.</p>

<p>MICHAEL. Well, then, take it.</p>

<p>MR. DARLING. Well, then, you take it.</p>

<p>WENDY <i>(butting in again).</i> Why not take it at the same
time?</p>

<p>MR. DARLING <i>(haughtily).</i> Certainly. Are you ready,
Michael?</p>

<p>WENDY <i>(as nothing has happened).</i>
One&mdash;two&mdash;three.</p>

<blockquote>(MICHAEL <i>partakes, but</i> MR. DARLING <i>resorts to
hanky-panky.)</i></blockquote>

<p>JOHN. Father hasn't taken his!</p>

<blockquote>(MICHAEL <i>howls.)</i></blockquote>

<p>WENDY <i>(inexpressibly pained).</i> Oh father!</p>

<p>MR. DARLING <i>(who has been hiding the glass behind him).</i>What
do you mean by 'oh father'? Stop that row, Michael. I meant to take
mine but I&mdash;missed it. (NANA <i>shakes her head sadly over him,
and goes into the bathroom. They are all looking as if they did not
admire him, and nothing so dashes a temperamental man.)</i> I say, I
have just thought of a splendid joke. <i>(They brighten.)</i> I shall
pour my medicine into Nana's bowl, and she will drink it thinking it
is milk! <i>The pleasantry does not appeal, but he prepares the joke,
listening for appreciation.)</i></p>

<p>WENDY. Poor darling Nana!</p>

<p>MR. DARLING. You silly little things; to your beds everyone of
you; I am ashamed of you.</p>

<blockquote><i>(They steal to their beds as</i> MRS. DARLING
<i>returns with the chocolate.)</i></blockquote>

<p>MRS. DARLING. Well, is it all over?</p>

<p>MICHAEL. Father didn't&mdash;&mdash;<i>(Father glares.)</i></p>

<p>MR. DARLING. All over, dear, quite satisfactorily. (NANA <i>comes
back.)</i> Nana, good dog, good girl; I have put a little milk into
your bowl. <i>(The bowl is by the kennel, and</i> NANA <i>begins to
lap, only begins. She retreats into the kennel.)</i></p>

<p>MRS. DARLING. What is the matter, Nana?</p>

<p>MR. DARLING <i>(uneasily).</i> Nothing, nothing.</p>

<p>MRS. DARLING <i>(smelling the bowl).</i> George, it is your
medicine!</p>

<blockquote><i>(The children break into lamentation. He gives his
wife an imploring look; he is begging for one smile, but does not get
it. In consequence he goes from bad to worse.)</i></blockquote>

<p>MR. DARLING. It was only a joke. Much good my wearing myself to
the bone trying to be funny in this house.</p>

<p>WENDY <i>(on her knees by the kennel).</i> Father, Nana is
crying.</p>

<p>MR. DARLING. Coddle her; nobody coddles me. Oh dear no. I am only
the bread-winner, why should I be coddled? Why, why, why?</p>

<p>MRS. DARLING. George, not so loud; the servants will hearyou.</p>

<blockquote><i>(There is only one maid, absurdly small too, but they
have got into the way of calling her the servants.)</i></blockquote>

<p>MR. DARLING <i>(defiant).</i> Let them hear me; bring in the whole
world. ( <i>The desperate man, who has not been in fresh air for
days, has now lost all self-control.)</i> I refuse to allow that dog
to lord it in my nursery for one hour longer. (NANA <i>supplicates
him.)</i> In vain, in vain, the proper place for you is the yard, and
there you go to be tied up this instant.</p>

<blockquote>(NANA <i>again retreats into the kennel, and the children
add their prayers to hers.)</i></blockquote>

<p>MRS. DARLING <i>(who knows how contrite he will be for this
presently).</i> George, George, remember what I told you about that
boy.</p>

<p>MR. DARLING. Am I master in this house or is she? <i>(To</i> NANA
<i>fiercely)</i> Come along. <i>(He thunders at her, but she
indicates that she has reasons not worth troubling him with for
remaining where she is. He resorts to a false bonhomie.)</i> There,
there, did she think he was angry with her, poor Nana? <i>(She
wriggles a response in the affirmative.)</i> Good Nana, pretty Nana.
<i>(She has seldom been called pretty, and it has the old effect. She
plays rub-a-dub with her paws, which is how a dog blushes.)</i> She
will come to her kind master, won't she? won't she? <i>(She advances,
retreats, waggles her head, her tail, and eventually goes to him. He
seizes her collar in an iron grip and amid the cries of his progeny
drags her from the room. They listen, for her remonstrances are not
inaudible.)</i></p>

<p>MRS. DARLING. Be brave, my dears.</p>

<p>WENDY. He is chaining Nana up!</p>

<blockquote><i>(This unfortunately is what he is doing, though we
cannot see him. Let us hope that he then retires to his study, looks
up the word 'temper' in his</i> Thesaurus, <i>and under the influence
of those benign pages becomes a better man. In the meantime the
children have been put to bed in unwonted silence, and</i> MRS.
DARLING <i>lights the night-lights over the beds.)</i></blockquote>

<p>JOHN <i>(as the barking below goes on).</i> She is awfully
unhappy.</p>

<p>WENDY. That is not Nana's unhappy bark. That is her bark when she
smells danger.</p>

<p>MRS. DARLING <i>(remembering that boy).</i> Danger! Are you sure,
Wendy?</p>

<p>WENDY <i>(the one of the family, for there is one in every family,
who can be trusted to know or not to know).</i> Oh yes.</p>

<blockquote><i>(Her mother looks this way and that from the
window.)</i></blockquote>

<p>JOHN. Is anything there?</p>

<p>MRS. DARLING. All quite quiet and still. Oh, how I wish I was not
going out to dinner to-night.</p>

<p>MICHAEL. Can anything harm us, mother, after the night-lights are
lit?</p>

<p>MRS. DARLING. Nothing precious. They are the eyes amother leaves
behind her to guard her children.</p>

<blockquote><i>(Nevertheless we may be sure she means to tell</i>
LIZA, <i>the little maid, to look in on them frequently till she
comes home. She goes from bed to bed, after her custom, tucking them
in and crooning a lullaby.)</i></blockquote>

<p>MICHAEL <i>(drowsily).</i> Mother, I 'm glad of you.</p>

<p>MRS. DARLING <i>(with a last look round, her hand on the
switch).</i> Dear night-lights that protect my sleeping babes, burn
clear and steadfast to-night.</p>

<blockquote><i>(The nursery darkens and she is gone, intentionally
leaving the door ajar. Something uncanny is going to happen, we
expect, for a quiver has passed through the room, just sufficient to
touch the night-lights. They blink three times one after the other
and go out, precisely as children (whom familiarity has made them
resemble) fall asleep. There is another light in the room now, no
larger than</i> MRS. DARLING'S <i>fist, and in the time we have taken
to say this it has been into the drawers and wardrobe and searched
pockets, as it darts about looking for a certain shadow. Then the
window is blown open, probably by the smallest and therefore most
mischievous star, and</i> PETER PAN <i>flies into the room. In so far
as he is dressed at all it is in autumn leaves and
cobwebs.)</i></blockquote>

<p>PETER (in <i>a whisper).</i> Tinker Bell, Tink, are you there?
<i>(A jug lights up.)</i> Oh, do come out of that jug.
(TINK<i>flashes hither and thither?)</i> Do you know where they put
it? <i>(The answer comes as of a tinkle of bells; it is the fairy
language.</i> PETER <i>can speak it, but it bores him.)</i> Which big
box? This one? But which drawer? Yes, do show me. (TINK <i>pops into
the drawer where the shadow is, but before</i>PETER <i>can reach
it,</i> WENDY <i>moves in her sleep. He flies onto the mantelshelf as
a hiding-place. Then, as she has not waked, he flutters over the beds
as an easy way to observe the occupants, closes the window softly,
wafts himself to the drawer and scatters its contents to the floor,
as kings on their wedding day toss ha'pence to the crowd. In his joy
at finding his shadow he forgets that he has shut up</i> TINK <i>in
the drawer. He sits on the floor with the shadow, confident that he
and it will join like drops of water. Then he tries to stick it on
with soap from the bathroom, and this failing also, he subsides
dejectedly on the floor. This wakens</i> WENDY, <i>who sits up, and
is pleasantly interested to see a stranger.)</i></p>

<p><i>WENDY (courteously).</i> Boy, why are you crying?</p>

<blockquote><i>(He jump up, and crossing to the foot of the bed bows
to her in the fairy way.</i> WENDY, <i>impressed, bows to him from
the bed.)</i></blockquote>

<p>PETER. What is your name?</p>

<p>WENDY <i>(well satisfied).</i> Wendy Moira Angela Darling.What is
yours?</p>

<p>PETER <i>(finding it lamentably brief).</i> Peter Pan.</p>

<p>WENDY. Is that all?</p>

<p>PETER <i>(biting his lip).</i> Yes.</p>

<p>WENDY <i>(politely).</i> I am so sorry.</p>

<p>PETER. It doesn't matter.</p>

<p>WENDY. Where do you live?</p>

<p>PETER. Second to the right and then straight on till morning.</p>

<p>WENDY. What a funny address!</p>

<p>PETER. No, it isn't.</p>

<p>WENDY. I mean, is that what they put on the letters?</p>

<p>PETER. Don't get any letters.</p>

<p>WENDY. But your mother gets letters?</p>

<p>PETER. Don't have a mother.</p>

<p>WENDY. Peter!</p>

<blockquote><i>(She leaps out of bed to put her arms round him, but
he draws back; he does not know why, but he knows he must draw
back.)</i></blockquote>

<p>PETER. You mustn't touch me.</p>

<p>WENDY. Why?</p>

<p>PETER. No one must ever touch me.</p>

<p>WENDY. Why?</p>

<p>PETER. I don't know.</p>

<blockquote><i>(He is never touched by any one in the
play.)</i></blockquote>

<p>WENDY. No wonder you were crying.</p>

<p>PETER. I wasn't crying. But I can't get my shadow to stick on.</p>

<p>WENDY. It has come off! How awful. <i>(Looking at the spot where
he had lain.)</i> Peter, you have been trying to stick it on with
soap!</p>

<p>PETER <i>(snappily).</i> Well then?</p>

<p>WENDY. It must be sewn on.</p>

<p>PETER. What is 'sewn'?</p>

<p>WENDY. You are dreadfully ignorant.</p>

<p>PETER. No, I 'm not.</p>

<p>WENDY. I will sew it on for you, my little man. But we must have
more light. <i>(She touches something, and to his astonishment the
room is illuminated.)</i> Sit here. I dare say it will hurt a
little.</p>

<p>PETER <i>(a recent remark of hers rankling).</i> I never cry.
<i>(She seems to attach the shadow. He tests the combination.)</i> It
isn't quite itself yet.</p>

<p>WENDY. Perhaps I should have ironed it. <i>(It awakes and is as
glad to be back with him as he to have it. He and his shadow dance
together. He is showing off now. He crows</i> <i>like a cock. He
would fly in order to impress</i> WENDY <i>further</i> <i>if he knew
that there is anything unusual in that.)</i></p>

<p>PETER. Wendy, look, look; oh the cleverness of me!</p>

<p>WENDY. You conceit, of course I did nothing!</p>

<p>PETER. You did a little.</p>

<p>WENDY <i>(wounded).</i> A little! If I am no use I can at least
withdraw.</p>

<blockquote><i>(With one haughty leap she is again in bed with the
sheet over her face. Popping on to the end of the bed the artful one
appeals.)</i></blockquote>

<p>PETER. Wendy,. don't withdraw. I can't help crowing, Wendy, when
I'm pleased with myself. Wendy, one girl is worth more than twenty
boys.</p>

<p>WENDY <i>(peeping over the sheet).</i> You really think so,
Peter?</p>

<p>PETER. Yes, I do.</p>

<p>WENDY. I think it's perfectly sweet of you, and I shall get up
again. <i>(They sit together on the side of the bed.)</i> I shall
give you a kiss if you like.</p>

<p>PETER. Thank you. <i>(He holds out his hand.)</i></p>

<p>WENDY <i>(aghast).</i> Don't you know what a kiss is?</p>

<p>PETER. I shall know when you give it me. <i>(Not to hurt his
feelings she gives him her thimble.)</i> Now shall I give youa
kiss?</p>

<p>WENDY <i>(primly).</i> If you please. <i>(He pulls an acorn button
off his person and bestows it on her. She is shocked but
considerate.)</i> I will wear it on this chain round my neck. Peter,
how old are you?</p>

<p>PETER <i>(blithely).</i> I don't know, but quite young, Wendy. I
ran away the day I was born.</p>

<p>WENDY. Ran away, why?</p>

<p>PETER. Because I heard father and mother talking of what I was to
be when I became a man. I want always to be a little boy and to have
fun; so I ran away to Kensington Gardens and lived a long time among
the fairies.</p>

<p>WENDY <i>(with great eyes).</i> You know fairies, Peter!</p>

<p>PETER <i>(surprised that this should be a recommendation).</i>
Yes, but they are nearly all dead now. <i>(Baldly)</i> You see,
Wendy, when the first baby laughed for the first time, the laugh
broke into a thousand pieces and they all went skipping about, and
that was the beginning of fairies. And now when every new baby is
born its first laugh becomes a fairy. So there ought to be one fairy
for every boy or girl,</p>

<p>WENDY <i>(breathlessly).</i> Ought to be? Isn't there?</p>

<p>PETER. Oh no. Children know such a lot now. Soon they don't
believe in fairies, and every time a child says 'I don't believe in
fairies' there is a fairy somewhere that falls down dead. <i>(He
skips about heartlessly.)</i></p>

<p>WENDY. Poor things!</p>

<p>PETER. (<i>to whom this statement recalls a forgotten friend)</i>.
I can't think where she has gone. Tinker Bell, Tink, where are
you?</p>

<p>WENDY <i>(thrilling).</i> Peter, you don't mean to tell me that
there is a fairy in this room!</p>

<p>PETER (<i>flitting about in search</i>). She came with me. You
don't hear anything, do you?</p>

<p>WENDY. I hear&mdash;the only sound I hear is like a tinkle of
bells.</p>

<p>PETER. That is the fairy language. I hear it too.</p>

<p>WENDY. It seems to come from over there.</p>

<p>PETER. (<i>with shameless glee.</i>) Wendy, I believe I shut her
up in that drawer!</p>

<blockquote><i>(He releases</i> TINK, <i>who darts about in a fury
using language it is perhaps as well we don't
understand.)</i></blockquote>

You needn't say that; I 'm very sorry, but how could I know you were
in the drawer? 

<p>WENDY <i>(her eyes dancing in pursuit of the delicious
creature).</i> Oh, Peter, if only she would stand still and let me
see her!</p>

<p>PETER <i>(indifferently).</i> They hardly ever stand still.</p>

<blockquote><i>(To show that she can do even this</i> TINK <i>pauses
between two ticks of the cuckoo clock.)</i></blockquote>

<p>WENDY. I see her, the lovely! where is she now?</p>

<p>PETER. She is behind the clock. Tink, this lady wishes you were
her fairy. <i>(The answer comes immediately.)</i></p>

<p>WENDY. What does she say?</p>

<p>PETER. She is not very polite. She says you are a great ugly girl,
and that she is my fairy. You know, Tink, you can't be my fairy
because I am a gentleman and you are a lady.</p>

<blockquote>
<p>(TINK <i>replies.)</i></p>
</blockquote>

<p>WENDY. What did she say?</p>

<p>PETER. She said 'You silly ass.' She is quite a common girl, you
know. She is called Tinker Bell because she mends the fairy pots and
kettles.</p>

<blockquote>(<i>They have reached a chair,</i> WENDY <i>in the
ordinary way and</i> PETER <i>through a hole in the
back.)</i></blockquote>

<p>WENDY. Where do you live now?</p>

<p>PETER. With the lost boys.</p>

<p>WENDY. Who are they?</p>

<p>PETER. They are the children who fall out of their prams when the
nurse is looking the other way. If they are not claimed in seven days
they are sent far away to the Never-Land. I 'm captain.</p>

<p>WENDY. What fun it must be.</p>

<p>PETER <i>(craftily).</i> Yes, but we are rather lonely. You see,
Wendy, we have no female companionship.</p>

<p>WENDY. Are none of the other children girls?</p>

<p>PETER. Oh no; girls, you know, are much too clever to fall out of
their prams.</p>

<p>WENDY. Peter, it is perfectly lovely the way you talk about girls.
John there just depises us.<br>
(PETER, <i>for the first time, has a good look at</i> JOHN. <i>He
then neatly tumbles him out of bed.)</i></p>

<p>You wicked! you are not captain here. <i>(She bends over her
brother who is prone on the floor.)</i> After all he hasn't wakened,
and you meant to be kind. <i>(Having now done her duty she
forgets</i> JOHN, <i>who blissfully sleep on.)</i> Peter, you may
give me a kiss.</p>

<p>PETER <i>(cynically).</i> I thought you would want it back. <i>(He
offers her the thimble.)</i></p>

<p>WENDY <i>(artfully).</i> Oh dear, I didn't mean a kiss, Peter. I
meant a thimble.</p>

<p>PETER <i>(only half placated).</i>What is that?</p>

<p>WENDY. It is like this. <i>(She leans forward to give a
demonstration, but something prevents the meeting of their
faces.)</i></p>

<p>PETER <i>(satisfied).</i> Now shall I give you a thimble?</p>

<p>WENDY. If you please. <i>(Before he can even draw near she
screams.)</i></p>

<p>PETER. What is it?</p>

<p>WENDY. It was exactly as if some one were pulling my hairl</p>

<p>PETER. That must have been Tink. I never knew her so naughty
before.</p>

<blockquote>(TINK <i>speaks. She is in the jug
again.)</i></blockquote>

<p>WENDY. What does she say?</p>

<p>PETER. She says she will do that every time I give you a
thimble.</p>

<p>WENDY. But why?</p>

<p>PETER <i>(equally nonplussed).</i> Why, Tink? <i>(He has to
translate the answer.)</i> She said 'You silly ass' again.</p>

<p>WENDY. She is very impertinent. <i>(They are sitting on the floor
now.)</i> Peter, why did you come to our nursery window?</p>

<p>PETER. To try to hear stories None of us knows any stories.</p>

<p>WENDY. How perfectly awful!</p>

<p>PETER. Do you know why swallows build in the eaves of houses? It
is to listen to the stories. Wendy, your mother was telling you such
a lovely story.</p>

<p>WENDY. Which story was it?</p>

<p>PETER. About the prince, and he couldn't find the lady who wore
the glass slipper.</p>

<p>WENDY. That was Cinderella. Peter, he found her and they were
happy ever after.</p>

<p>PETER. I am glad. <i>(They have worked their way along the floor
close to each other, but he now jumps up.)</i></p>

<p>WENDY. Where are you going?</p>

<p>PETER <i>(already on his way to the window).</i> To tell the other
boys.</p>

<p>WENDY. Don't go, Peter. I know lots of stories. The stories I
could tell to the boys!</p>

<p>PETER <i>(gleaming).</i> Come on! We'll fly.</p>

<p>WENDY. Fly? You can fly!</p>

<blockquote><i>(How he would like to rip those stories out of her; he
is dangerous now.)</i></blockquote>

<p>PETER. Wendy, come with me.</p>

<p>WENDY. Oh dear, I mustn't. Think of mother. Besides, I can't
fly.</p>

<p>PETER. I'll teach you.</p>

<p>WENDY. How lovely to fly!</p>

<p>PETER. I'll teach you how to jump on the wind's back and then away
we go. Wendy, when you are sleeping in your silly bed you might be
flying about with me, saying funny things tothe stars. There are
mermaids, Wendy, with long tails. <i>(She just succeeds in remaining
on the nursery floor.)</i> Wendy, how we should all respect you.</p>

<blockquote><i>(At this she strikes her colours.)</i></blockquote>

<p>WENDY. Of course it's awfully fascinating! Would you teach John
and Michael to fly too?</p>

<p>PETER <i>(indifferently).</i> If you like.</p>

<p>WENDY <i>(playing rum-turn on</i> JOHN). John, wake up; there is a
boy here who is to teach us to fly.</p>

<p>JOHN. Is there? Then I shall get up. <i>(He raises his headfrom
the floor.)</i> Hullo, I am up!</p>

<p>WENDY. Michael, open your eyes. This boy is to teach us to
fly.</p>

<blockquote><i>(The sleepers are at once as awake as their father's
razor;but before a question can be asked</i> NANA'S <i>bark is
heard.)</i></blockquote>

<p>JOHN. Out with the light, quick, hide!</p>

<blockquote><i>(When the maid</i> LIZA, <i>who is so small that when
she says she will never see ten again one can scarcely believe her,
enters with a firm hand on the troubled</i> NANA'S <i>chain the room
is in comparative darkness.)</i></blockquote>

<p>LIZA. There, you suspicious brute, they are perfectly safe, aren't
they? Every one of the little angels sound asleep in bed. Listen to
their gentle breathing. (NANA'S <i>sense of smell here helps to her
undoing instead of hindering it. She knows that they are in the
room.</i> MICHAEL, <i>who is behind the window curtain, is so
encouraged by</i> LIZA'S <i>last remark that he breathes too
loudly.</i> NANA <i>knows that kind of breathing and tries to break
from her keeper's control.)</i> No more of it, Nana. <i>(Wagging a
finger at her)</i> I warn you if you bark again I shall go straight
for master and missus and bring them home from the party, and then
won't master whip you just! Come along, you naughty dog.</p>

<blockquote>
<p><i>(The unhappy</i> NANA is <i>led away. The children emerge
exulting from their various hiding-places. In their brief absence
from the scene strange things have been done to them; but it is not
for us to reveal a mysterious secret of the stage. They look just the
same.)</i></p>
</blockquote>

<p>JOHN. I say, can you really fly.</p>

<p>PETER. Look! <i>(He is now over their heads.)</i></p>

<p>WENDY. Oh, how sweet!</p>

<p>PETER. I 'm sweet, oh, I am sweet!</p>

<blockquote><i>(It looks so easy that they try it first from the
floor <b>and</b>then from their beds, without encouraging
results.)</i></blockquote>

<p>JOHN <i>(rubbing his knees).</i> How do you do it?</p>

<p>PETER <i>(descending).</i> You just think lovely wonderful
thoughts and they lift you up in the air. <i>(He is off
again.)</i></p>

<p>JOHN. You are so nippy at it; couldn't you do it very slowly once?
(PETER <i>does it slowly.)</i> I 've got it now, Wendy. <i>(He tries;
no, he has not got it, poor stay-at-home, though he knows the names
of all the counties in England and</i> PETER <i>does not know
one.)</i></p>

<p>PETER. I must blow the fairy dust on you first. <i>(Fortunately
his garments are smeared with it and he blows some dust on each.)</i>
Now, try; try from the bed. Just wiggle your shoulders this way, and
then let go.</p>

<blockquote><i>(The gallant</i> MICHAEL <i>is the first to let go,
and is borne across the room.)</i></blockquote>

<p>MICHAEL <i>(with a yell that should have disturbed</i> LIZA). I
flewed!</p>

<blockquote>(JOHN <i>lets go, and meets</i> WENDY <i>near the
bathroom door though they had both aimed in an opposite
direction.)</i></blockquote>

<p>WENDY. Oh, lovely!</p>

<p>JOHN <i>(tending to be upside down).</i> How ripping!</p>

<p>MICHAEL <i>(playing whack on a chair).</i> I do like it!</p>

<p>THE THREE. Look at me, look at me, look at me!</p>

<blockquote><i>(They are not nearly so elegant in the air as</i>
PETER, <i>but their heads have bumped the ceiling, and there is
nothing more delicious than that.)</i></blockquote>

<p>JOHN <i>(who can even go backwards).</i> I say, why shouldn't we
go out?</p>

<p>PETER. There are pirates.</p>

<p>JOHN. Pirates! <i>(He grabs his tall Sunday hat.)</i> Let us go at
once!</p>

<blockquote>(TINK <i>does not like it. She darts at their hair. From
down below in the street the lighted window must present an unwonted
spectacle: the shadows of children revolving in the room like a
merry-go-round. This is perhaps what</i> MR. <i>and</i> MRS. DARLING
<i>see as they come hurrying home from the party, brought by</i> NANA
<i>who, you may be sure, has broken her chain.</i> PETER'S
<i>accomplice, the little star, has seen them coming, and again the
window blows open.)</i></blockquote>

<p>PETER <i>(as if he had heard the star whisper 'Cave').</i> Now
come!</p>

<blockquote>
<p><i>(Breaking the circle he flies out of the window over the trees
of the square and over the house-tops, and the others follow like a
flight of birds. The broken-hearted father and mother arrive just in
time to get a nip from TINK as she too sets out for the Never
Land.)</i></p>
</blockquote>

% !TEX program = pdflatex
% !TEX encoding = UTF-8
% !TEX spellcheck = en_GB
% !TEX root = peter-pan.tex

\Act{The Never Land}

\begin{Settings}
When the blind goes up all is so dark that you scarcely know it has gone up.
This is because if you were to see the island bang
\parenth{as Peter would say}
the wonders of it might hurt your eyes.
If you all came in spectacles perhaps you could see it bang,
but to make a rule of that kind would be a pity.
The first thing seen is merely some whitish dots trudging along the sward,
and you can guess from their tinkling that they are probably fairies of the commoner sort
going home afoot from some party and having a cheery tiff by the way.
Then Peter’s star wakes up,
and in the blink of it, which is much stronger than in our stars, you can make out masses of trees,
and you think you see wild beasts stealing past to drink,
though what you see is not the beasts themselves but only the shadows of them.
They are really out pictorially to greet Peter in the way they think he would like them to greet him;
and for the same reason the mermaids basking in the lagoon beyond the trees are carefully combing their hair;
and for the same reason the pirates are landing invisibly from the longboat,
invisibly to you but not to the redskins, whom none can see or hear because they are on the war‐path.
The whole island, in short, which has been having a slack time in Peter’s absence,
is now in a ferment because the tidings has leaked out that he is on his way back;
and everybody and everything know that they will catch it from him if they don’t give satisfaction.
While you have been told this the sun
\parenth{another of his servants}
has been bestirring himself.
Those of you who may have thought it wiser after all to begin this Act in spectacles may now take them off.

What you see is the Never Land.
You have often half seen it before, or even three‐quarters, after the night‐lights were lit,
and you might then have beached your coracle on it if you had not always at the great moment fallen asleep.
I dare say you have chucked things on to it, the things you can’t find in the morning.
In the daytime you think the Never Land is only make‐believe, and so it is to the likes of you,
but this is the Never Land come true.
It is an open‐air scene, a forest,
with a beautiful lagoon beyond but not really far away, for the Never Land is very compact,
not large and sprawly with tedious distances between one adventure and another, but nicely crammed.
It is summer time on the trees and on the lagoon but winter on the river,
which is not remarkable on Peter’s island where all the four seasons may pass while you are filling a jug at the well.
Peter’s home is at this very spot,
but you could not point out the way into it even if you were told which is the entrance,
not even if you were told that there are seven of them.
You know now because you have just seen one of the lost boys emerge.
The holes in these seven great hollow trees are the ‘doors’ down to Peter’s home,
and he made seven because, despite his cleverness, he thought seven boys must need seven doors.

The boy who has emerged from his tree is Slightly,
who has perhaps been driven from the abode below by companions less musical than himself.
Quite possibly a genius Slightly has with him his home‐made whistle to which he capers entrancingly,
with no audience save a Never ostrich which is also musically inclined.
Unable to imitate Slightly’s graces the bird falls so low as to burlesque them
and is driven from the entertainment.
Other lost boys climb up the trunks or drop from branches,
and now we see the six of them, all in the skins of animals they think they have shot,
and so round and furry in them that if they fall they roll.
Tootles is not the least brave though the most unfortunate of this gallant band.
He has been in fewer adventures than any of them
because the big things constantly happen while he has stepped round the corner;
he will go off, for instance, in some quiet hour to gather firewood,
and then when he returns the others will be sweeping up the blood.
Instead of souring his nature this has sweetened it and he is the humblest of the band.
Nibs is more gay and debonair, Slightly more conceited.
Slightly thinks he remembers the days before he was lost, with their manners and customs.
Curly is a pickle, and so often has he had to deliver up his person when Peter said sternly,
‘Stand forth the one who did this thing,’ that now he stands forth whether he has done it or not.
The other two are First Twin and Second Twin,
who cannot be described because we should probably be describing the wrong one.
Hunkering on the ground or peeking out of their holes,
the six are not unlike village gossips gathered round the pump.
\end{Settings}

\begin{drama}

\tootlesspeaks
Has Peter come back yet, Slightly?

\slightlyspeaks[\delivery{with a solemnity that he thinks suits the occasion}]
No, Tootles, no.

\direction{They are like dogs waiting for the master to tell them that the day has begun.}

\curlyspeaks[\delivery{as if Peter might be listening}]
I do wish he would come back.

\tootlesspeaks
I am always afraid of the pirates when Peter is not here to protect us.

\slightlyspeaks
I am not afraid of pirates.
Nothing frightens me.
But I do wish Peter would come back and tell us whether he has heard anything more about Cinderella.

\secondtwinspeaks[\delivery{with diffidence}]
Slightly, I dreamt last night that the prince found Cinderella.

\firsttwinspeaks[\delivery{who is intellectually the superior of the two}]
Twin, I think you should not have dreamt that, for I didn’t,
and Peter may say we oughtn’t to dream differently, being twins, you know.

\tootlesspeaks
I am awfully anxious about Cinderella.
You see, not knowing anything about my own mother I am fond of thinking that she was rather like Cinderella.

\direction{This is received with derision.}

\nibsspeaks
All I remember about my mother is that she often said to father, ‘Oh how I wish I had a cheque book of my own.’
I don’t know what a cheque book is, but I should just love to give my mother one.

\slightlyspeaks[\delivery{as usual}]
My mother was fonder of me than your mothers were of you.
\delivery{Uproar.}
Oh yes, she was.
Peter had to make up names for you, but my mother had wrote my name on the pinafore I was lost in.
‘Slightly Soiled’; that’s my name.

\direction{%
(They fall upon him pugnaciously;
not that they are really worrying about their mothers, who are now as important to them as a piece of string,
but because any excuse is good enough for a shindy.
Not for long is he belaboured, for a sound is heard that sends them scurrying down their holes; in a second of time the scene is bereft of human life.
What they have heard from near‐by is a verse of the dreadful song with which on the Never Land the prates stealthily trumpet their approach—

\begin{verse}
	Yo ho, yo ho, the pirate life,\\
	The flag of skull and bones,\\
	A merry hour, a hempen rope,\\
	And hey for Davy Jones!
\end{verse}

The pirates appear upon the frozen river dragging a raft,
on which reclines among cushions that dark and fearful man, \character{Captain Jas Hook}.
A more villainous‐looking brotherhood of men never hung in a row on Execution dock.
Here, his great arms bare, pieces of eight in his ears as ornaments, is the handsome \cecco, who cut his name on the back of the governor of the prison at Gao.
Heavier in the pull is the gigantic black who has had many names since the first one terrified dusky children on the banks of the \emph{Guidjo‐mo}.
\character{Bill Jukes} comes next, every inch of him tattooed, the same \jukes who got six dozen on the \emph{Walrus} from \character{Flint}.
Following these are \cookson, said to be \character{Black Murphy}’s brother
\parenth{but this was never proved};
and \character{Gentleman Starkey}, once an usher in a school;
and \character{Skylights} \parenth{Morgan’s Skylights};
and \noodler, whose hands are fixed on backwards;
and the spectacled boatswain, \smee, the only Nonconformist in \hook’s crew;
and other ruffians long known and feared on the Spanish main.

Cruelest jewel in that dark setting is \hook himself, cadaverous and blackavised,
his hair dressed in long curls which look like black candles about to melt,
his eyes blue as the forget‐me‐not and of a profound insensibility,
save when he claws, at which time a red spot appears in them.
He has an iron hook instead of a right hand, and it is with this he claws.
He is never more sinister than when he is most polite,
and the elegance of his diction, the distinction of his demeanour,
show him one of a different class from his crew, a solitary among uncultured companions.
This courtliness impresses even his victims on the high seas,
who note that he always says ‘Sorry’ when prodding them along the flank.
A man of indomitable courage,
the only thing at which he flinches is the sight of his own blood, which is thick and of an unusual colour.
At his public school they said of him that he ‘bled yellow.’
In dress he apes the dandiacal associated with Charles II.,
having heard it said in an earlier period of his career that he bore a strange resemblance to the ill‐fated Stuarts.
A holder of his own contrivance is in his mouth enabling him to smoke two cigars at once.
Those, however, who have seen him in the flesh, which is an inadequate term for his earthly tenement,
agree that the grimmest part of him is his iron claw.

They continue their distasteful singing as they disembark—

\begin{verse}
	Avast, belay, yo ho, heave to,\\
	A‐pirating we go,\\
	And if we’re parted by a shot\\
	We’re sure to meet below!
\end{verse}

\nibs, the only one of the boys who has not sought safety in his tree, is seen for a moment near the lagoon,
and \starkey’s pistol is at once upraised.
The captain twists his hook in him.)%
}

\starkeyspeaks[\delivery{abject}]
Captain, let go!

\hookspeaks
Put back that pistol, first.

\starkeyspeaks
’Twas one of those boys you hate; I could have shot him dead.

\hookspeaks
Ay, and the sound would have brought Tiger Lily’s redskins on us.
Do you want to lose your scalp?

\smeespeaks[\delivery{wriggling his cutlass pleasantly}]
That is true.
Shall I after him, Captain, and tickle him with Johnny Corkscrew?
Johnny is a silent fellow.

\hookspeaks
Not now.
He is only one, and I want to mischief all the seven.
Scatter and look for them.
\delivery{The boatswain whistles his instructions, and the men disperse on their frightful errand.
With none to hear save \smee, \hook becomes confidential.}
Most of all I want their captain, Peter Pan.
’Twas he cut off my arm.
I have waited long to shake his hand with this.
\delivery{Luxuriating.}
Oh, I'll tear him!

\smeespeaks[\delivery{always ready for a chat}]
Yet I have oft heard you say your hook was worth a score of hands, for combing the hair and other homely uses.

\hookspeaks
If I was a mother I would pray to have my children born with this instead of that
\delivery{his left arm creeps nervously behind him.
He has a galling remembrance}¤
Smee, Pan flung my arm to a crocodile that happened to be passing by.

\smeespeaks
I have often noticed your strange dread of crocodiles.

\hookspeaks[\delivery{pettishly}]
Not of crocodiles but of that one crocodile.
\delivery{He lays bare a lacerated heart.}
The brute liked my arm so much, Smee, that he has followed me ever since, from sea to sea, and from land to land, licking his lips for the rest of me.

\smeespeaks[\delivery{looking for the bright side}]
In a way it is a sort of compliment.

\hookspeaks[\delivery{with dignity}]
I want no such compliments; I want Peter Pan, who first gave the brute his taste for me.
Smee, that crocodile would have had me before now, but by a lucky chance he swallowed a clock, and it goes tick, tick, tick, tick inside him; and so before he can reach me I hear the tick and bolt.
\delivery{He emits a hollow rumble.}
Once I heard it strike six within him.

\smeespeaks[\delivery{sombrely}]
Some day the clock will run down, and then he'll get you.

\hookspeaks[\delivery{a broken man}]
Ay, that is the fear that haunts me.
\delivery{He rises.}
Smee, this seat is hot; odds, bobs, hammer and tongs, I am burning.

\direction{He has been sitting, he thinks, on one of the island mushrooms, which are of enormous size.
But this is a hand‐painted one placed here in times of danger to conceal a chimney.
They remove it, and tell‐tale smoke issues; also, alas, the sound of children's voices.}

\smeespeaks
A chimney!

\hookspeaks[\delivery{avidly}]
Listen!
Smee, 'tis plain they live here, beneath the ground.
\delivery{He replaces the mushroom.
His brain works tortuously.}

\smeespeaks[\delivery{hopefully}]
Unrip your plan, Captain.

\hookspeaks
To return to the boat and cook a large rich cake of jolly thickness with sugar on it, green sugar.
There can be but one room below, for there is but one chimney.
The silly moles had not the sense to see that they did not need a door apiece.
We must leave the cake on the shore of the mermaids' lagoon.
These boys are always swimming about there, trying to catch the mermaids.
They will find the cake and gobble it up, because, having no mother, they don't know how dangerous 'tis to eat rich damp cake.
They will die!

\smeespeaks[\delivery{fascinated}]
It is the wickedest, prettiest policy ever I heard of.

\hookspeaks[\delivery{meaning well}]
Shake hands on 't.

\smeespeaks
No, Captain, no.

\direction{He has to link with the hook, but he does not join in the song.}

\hookspeaks
\begin{verse}
	Yo ho, yo ho, when I say ‘paw,'\\
	By fear they're overtook,\\
	Naught's left upon your bones when you\\
	Have shaken hands with Hook!
\end{verse}

\direction{Frightened by a tug at his hand, \smee is joining in the chorus when another sound stills them both.
It is a tick, tick as of a clock, whose significance \hook is, naturally, the first to recognise,
‘The crocodile!\@' he cries, and totters from the scene.
\smee follows.
A huge crocodile, of one thought compact, passes across, ticking, and oozes after them.
The wood is now so silent that you may be sure it is full of redskins.
\tigerlily comes first.
She is the belle of the Piccaninny tribe, whose braves would all have her to wife, but she wards them off with a hatchet.
She puts her ear to the ground and listens, then beckons,
and \character{Great Big Little Panther} and the tribe are around her, carpeting the ground.
Far away some one treads on a dry leaf.'}

\tigerlilyspeaks
Pirates!
\delivery{They do not draw their knives—the knives slip into their hands.}
Have um scalps?
What you say?

\pantherspeaks
Scalp um, oho, velly quick.

\speaker{The Braves}[\delivery{in corroboration}]
Ugh, ugh, wah.

\direction{A fire is lit and they dance round and over it till they seem part of the leaping flames.
\tigerlily invokes Manitou; the pipe of peace is broken;
and they crawl off like a long snake that has not fed for many moons.
\tootles peers after the tail and summons the other boys, who issue from their holes.}

\tootlesspeaks
They are gone.

\slightlyspeaks[\delivery{almost losing confidence in himself}]
I do wish Peter was here.

\firsttwinspeaks
H'sh!
What is that?
\delivery{He is gazing at the lagoon and shrinks back.}
It is wolves, and they are chasing Nibs!

\direction{The baying wolves are upon them quicker than any boy can scuttle down his tree.}

\nibsspeaks[\delivery{falling among his comrades}]
Save me, save me!

\tootlesspeaks
What should we do?

\secondtwinspeaks
What would Peter do?

\slightlyspeaks
Peter would look at them through his legs; let us do what Peter would do.

\direction{The boys advance backwards, looking between their legs at the snarling red‐eyed enemy, who trot away foiled.}

\firsttwinspeaks[\delivery{swaggering}]
We have saved you, Nibs.
Did you see the pirates?

\nibsspeaks[\delivery{sitting up, and agreeably aware that the centre of interest is now to pass to him}]
No, but I saw a wonderfuller thing, Twin.
\delivery{All mouths open for the information to be dropped into them.}
High over the lagoon I saw the loveliest great white bird.
It is flying this way.
\delivery{They search the firmament.}

\tootlesspeaks
What kind of a bird, do you think?

\nibsspeaks[\delivery{awed}]
I don't know; but it looked so weary, and as it flies it moans ‘Poor Wendy.'

\slightlyspeaks[\delivery{instantly}]
I remember now there are birds called Wendies.

\firsttwinspeaks[\delivery{who has flown to a high branch}]
See, it comes, the Wendy!
\delivery{They all see it now.}
How white it is!
\delivery{A dot of light is pursuing the bird malignantly.}

\tootlesspeaks
That is Tinker Bell.
Tink is trying to hurt the Wendy.
\delivery{He makes a cup of his hands and calls}¤
Hullo, Tink!
\delivery{A response comes down in the fairy language.}
She says Peter wants us to shoot the Wendy.

\nibsspeaks
Let us do what Peter wishes.

\slightlyspeaks
Ay, shoot it; quick, bows and arrows.

\tootlesspeaks[\delivery{first with his bow}]
Out of the way, Tink; I'll shoot it.
\delivery{His bolt goes home, and \wendy, who has been fluttering among the tree‐tops in her white nightgown, falls straight to earth.
No one could be more proud than \tootles.}
I have shot the Wendy; Peter will be so pleased.
\delivery{From some tree on which \tink is roosting comes the tinkle we can now translate, ‘You silly ass.'
\tootles falters.}
Why do you say that?
\delivery{The others feel that he may have blundered, and draw away from \tootles.}

\slightlyspeaks[\delivery{examining the fallen one more minutely}]
This is no bird; I think it must be a lady.

\nibsspeaks[\delivery{who would have preferred it to be a bird}]
And Tootles has killed her.

\curlyspeaks
Now I see, Peter was bringing her to us.
\delivery{They wonder for what object.}

\secondtwinspeaks
To take care of us?
\delivery{Undoubtedly for some diverting purpose.}

\speaker{Omnes}[\delivery{though every one of them had wanted to have a shot at her}]
Oh, Tootles!

\tootlesspeaks[\delivery{gulping}]
I did it.
When ladies used to come to me in dreams I said ‘Pretty mother,' but when she really came I shot her!
\delivery{He perceives the necessity of a solitary life for him.}
Friends, good‐bye.

\speaker{Several}[\delivery{not very enthusiastic}]
Don't go.

\tootlesspeaks
I must; I am so afraid of Peter.

\direction{He has gone but a step toward oblivion when he is stopped by a crowing as of some victorious cock.}

\speaker{Omnes}
Peter!

\direction{They make a paling of themselves in front of \wendy as \peter skims round the tree‐tops and reaches earth.}

\peterspeaks
Greeting, boys!
\delivery{Their silence chafes him.}
I am back; why do you not cheer?
Great news, boys, I have brought at last a mother for us all.

\slightlyspeaks[\delivery{vaguely}]
Ay, ay.

\peterspeaks
She flew this way; have you not seen her?

\secondtwinspeaks[\delivery{as \peter evidently thinks her important}]
Oh mournful day!

\tootlesspeaks[\delivery{making a break in the paling}]
Peter, I will show her to you.

\speaker{The Others}[\delivery{closing the gap}]
No, no.

\tootlesspeaks[\delivery{majestically}]
Stand back all, and let Peter see.

\direction{The paling dissolves, and \peter sees \wendy prone on the ground.}

\peterspeaks
Wendy, with an arrow in her heart!
\delivery{He plucks it out.}
Wendy is dead.
\delivery{He is not so much pained as puzzled.}

\curlyspeaks
I thought it was only flowers that die.

\peterspeaks
Perhaps she is frightened at being dead?
\delivery{None of them can say as to that.}
Whose arrow?
\delivery{Not one of them looks at \tootles.})

\tootlesspeaks
Mine, Peter.

\peterspeaks[\delivery{raising it as a dagger}]
Oh dastard hand!

\tootlesspeaks[\delivery{kneeling and baring his breast}]
Strike, Peter; strike true.

\peterspeaks[\delivery{undergoing a singular experience}]
I cannot strike; there is something stays my hand.

\direction{In fact \wendy's arm has risen.}

\nibsspeaks
’Tis she, the Wendy lady.
See, her arm.
\delivery{To help a friend}¤
I think she said ‘Poor Tootles.'

\peterspeaks[\delivery{investigating}]
She lives!

\slightlyspeaks[\delivery{authoritatively}]
The Wendy lady lives.
\delivery{The delightful feeling that they have been cleverer than they thought comes over them
and they applaud themselves.}

\peterspeaks[\delivery{holding up a button that is attached to her chain}]
See, the arrow struck against this.
It is a kiss I gave her; it has saved her life.

\slightlyspeaks
I remember kisses; let me see it.
\delivery{He takes it in his hand.}
Ay, that is a kiss.

\peterspeaks
Wendy, get better quickly and I'll take you to see the mermaids.
She is awfully anxious to see a mermaid.

\direction{\character{Tinker Bell}, who may have been off visiting her relations,
returns to the wood
and, under the impression that \wendy has been got rid of, is whistling as gaily as a canary.
She is not wholly heartless, but is so small that she has only room for one feeling at a time.}

\curlyspeaks
Listen to Tink rejoicing because she thinks the Wendy is dead!
\delivery{Regardless of spoiling another's pleasure}¤
Tink, the Wendy lives.

\direction{\tink gives expression to fury.}

\secondtwinspeaks[\delivery{tell‐tale}]
It was she who said that you wanted us to shoot the Wendy.

\peterspeaks
She said that?
Then listen, Tink, I am your friend no more.
\delivery{There is a note of acerbity in \tink's reply; it may mean ‘Who wants you?'}¤
Begone from me forever.
\delivery{Now it is a very wet tinkle.}

\curlyspeaks
She is crying.

\tootlesspeaks
She says she is your fairy.

\peterspeaks[\delivery{who knows they are not worth worrying about}]
Oh well, not for ever, but for a whole week.

\direction{\tink goes off sulking,
no doubt with the intention of giving all her friends an entirely false impression of \wendy's appearance.}

%\speakercontinues
Now what shall we do with Wendy?

\curlyspeaks
Let us carry her down into the house.

\slightlyspeaks
Ay, that is what one does with ladies.

\peterspeaks
No, you must not touch her; it wouldn't be sufficiently respectful.

\slightlyspeaks
That is what I was thinking.

\tootlesspeaks
But if she lies there she will die.

\slightlyspeaks
Ay, she will die.
It is a pity, but there is no way out.

\peterspeaks
Yes, there is.
Let us build a house around her!
\delivery{Cheers again, meaning that no difficulty baffles \peter.}
Leave all to me.
Bring the best of what we have.
Gut our house.
Be sharp.
\delivery{They race down their trees.}

\direction{While \peter is engrossed in measuring \wendy so that the house may fit her,
\john and \michael, who have probably landed on the island with a bump, wander forward,
so draggled and tired that if you were to ask \michael whether he is awake or asleep he would probably answer
‘I haven't tried yet.'}

\michaelspeaks[\delivery{bewildered}]
John, John, wake up.
Where is Nana, John?

\johnspeaks[\delivery{with the help of one eye but not always the same eye}]
It is true, we did fly!
\delivery{Thankfully}¤
And here is Peter.
Peter, is this the place?

\direction{\peter, alas, has already forgotten them, as soon maybe he will forget \wendy.
The first thing she should do now that she is here is to sew a handkerchief for him, and knot it as a jog to his memory.}

\peterspeaks[\delivery{curtly}]
Yes.

\michaelspeaks
Where is Wendy?
\delivery{\peter points.}

\johnspeaks[\delivery{who still wears his hat}]
She is asleep.

\michaelspeaks
John, let us wake her and get her to make supper for us.

\direction{Some of the boys emerge, and he pinches one.}

%\speakercontinues
John, look at them!

\peterspeaks[\delivery{still house‐building}]
Curly, see that these boys help in the building of the house.

\johnspeaks
Build a house?

\curlyspeaks
For the Wendy.

\johnspeaks[\delivery{feeling that there must be some mistake here}]
For Wendy?
Why, she is only a girl.

\curlyspeaks
That is why we are her servants.

\johnspeaks[\delivery{dazed}]
Are you Wendy's servants?

\peterspeaks
Yes, and you also.
Away with them.
\delivery{In another moment they are woodsmen hacking at trees, with \curly as overseer.}
Slightly, fetch a doctor.
\delivery{\slightly reels and goes.
He returns professionally in \john's hat.}
Please, sir, are you a doctor?

\slightlyspeaks[\delivery{trembling in his desire to give satisfaction}]
Yes, my little man.

\peterspeaks
Please, sir, a lady lies very ill.

\slightlyspeaks[\delivery{taking care not to fall over her}]
Tut, tut, where does she lie?

\peterspeaks
In yonder glade.
\delivery{It is a variation of a game they play.}

\slightlyspeaks
I will put a glass thing in her mouth.
\delivery{He inserts an imaginary thermometer in \wendy's mouth and gives it a moment to record its verdict.
He shakes it and then consults it.}

\peterspeaks[\delivery{anxiously}]
How is she?

\slightlyspeaks
Tut, tut, this has cured her.

\peterspeaks[\delivery{leaping joyously}]
I am glad.

\slightlyspeaks
I will call again in the evening.
Give her beef tea out of a cup with a spout to it, tut, tut.

\direction{The boys are running up with odd articles of furniture.}

\peterspeaks[\delivery{with an already fading recollection of the Darling nursery}]
These are not good enough for Wendy.
How I wish I knew the kind of house she would prefer!

\firsttwinspeaks
Peter, she is moving in her sleep.

\tootlesspeaks[\delivery{opening \wendy's mouth and gazing down into the depths}]
Lovely!

\peterspeaks
Oh, Wendy, if you could sing the kind of house you would like to have.

\direction{It is as if she had heard him.}

\wendyspeaks[\delivery{without opening her eyes}]
\begin{verse}
	I wish I had a woodland house,\\
	The littlest ever seen,\\
	With funny little red walls\\
	And roof of mossy green.
\end{verse}

\direction{In the time she sings this and two other verses,
such is the urgency of \peter's silent orders that they have knocked down trees,
laid a foundation and put up the walls and roof,
so that she is now hidden from view.
‘Windows' cries \peter, and \curly rushes them in,
‘Roses' and \tootles arrives breathless with a festoon for the door.
Thus springs into existence the most delicious little house for beginners.}

\firsttwinspeaks
I think it is finished.

\peterspeaks
There is no knocker on the door.
\delivery{\tootles hangs up the sole of his shoe.}
There is no chimney, we must have a chimney.
\delivery{They await his deliberations anxiously.}

\johnspeaks[\delivery{unwisely critical}]
It certainly does need a chimney.

\delivery{He is again wearing his hat, which \peter seizes, knocks the top off it and places on the roof.
In the friendliest way smoke begins to come out of the hat.}

\peterspeaks[\delivery{with his hand on the knocker}]
All look your best; the first impression is awfully important.
\delivery{he knocks, and after a dreadful moment of suspense, in which they cannot help wondering if any one is inside, the door opens and who should come out but \wendy!
She has evidently been tidying a little.
She is quite surprised to find that she has nine children.}

\wendyspeaks[\delivery{genteelly}]
Where am I?

\slightlyspeaks
Wendy lady, for you we built this house.

\nibsspeaks[and \tootles]
Oh, say you are pleased.

\wendyspeaks[\delivery{stroking the pretty thing}]
Lovely, darling house!

\firsttwinspeaks
And we are your children.

\wendyspeaks[\delivery{affecting surprise}]
Oh?

\speaker{Omnes}[\delivery{kneeling, with outstretched arms}]
Wendy lady, be our mother!
\delivery{Now that they know it is pretend they acclaim her greedily.}

\wendyspeaks[\delivery{not to make herself too cheap}]
Ought I?
Of course it is frightfully fascinating; but you see I am only a little girl; I have no real experience.

\speaker{Omnes}
That doesn't matter.
What we need is just a nice motherly person.

\wendyspeaks
Oh dear, I feel that is just exactly what I am.

\speaker{Omnes}
It is, it is, we saw it at once.

\wendyspeaks
Very well then, I will do my best.
\delivery{In their glee they go dancing obstreperously round the little house,
and she sees she must be firm with them as well as kind.}
Come inside at once, you naughty children, I am sure your feet are damp.
And before I put you to bed I have just time to finish the story of Cinderella.

\direction{They all troop into the enchanting house,
whose not least remarkable feature is that it holds them.
A vision of \liza passes, not perhaps because she has any right to be there;
but she has so few pleasures and is so young that we just let her have a peep at the little house.
By and by \peter comes out and marches up and down with drawn sword,
for the pirates can be heard carousing faraway on the lagoon, and the wolves are on the prowl.
The little house, its walls so red and its roof so mossy, looks very cosy and safe,
with a bright light showing through the blind, the chimney smoking beautifully, and \peter on guard.
On our last sight of him it is so dark
that we just guess he is the little figure who has fallen asleep by the door.
Dots of light come and go.
They are inquisitive fairies having a look at the house.
Any other child in their way they would mischief,
but they just tweak \peter's nose and pass on.
Fairies, you see, can touch him.}

\end{drama}

\endinput

% !TEX program = pdflatex
% !TEX encoding = UTF-8
% !TEX spellcheck = en_GB
% !TEX root = peter-pan.tex

\Act{The Mermaids’ Lagoon}

\endinput

<p><i>It is the end of a long playful day on the lagoon. The sun's rays have persuaded him to give them another five minutes, for one more race over the waters before he gathers them up and lets in the moon. There are many mermaids here, going plop-plop, and one might attempt to count the tails did they not flash and disappear so quickly. At times a lovely girl leaps in the air seeking to get rid of her excess of scales, which fall in a silver shower as she shakes them off. From the coral grottoes beneath the lagoon, where are the mermaids' bedchambers, comes fitful music.</i></p>

<p><i>One of the most bewitching of these blue-eyed creatures is lying lazily on Marooners' Rock, combing her long tresses and noting effects in a transparent shell. Peter and his band are in the water unseen behind the rock, whither they have tracked her as if she were a trout, and at a signal ten pairs of arms come whack upon the mermaid to enclose her. Alas, this is only what was meant to happen, for she hears the signal (which is the crow of a cock) and slips through their arms into the water. It has been such a near thing that there are scales on some of their hands. They climb on to the rock crestfallen.</i></p>

<p>WENDY \direct{preserving her scales as carefully as if they were rare postage stamps}¤
I did so want to catch a mermaid.</p>

<p>PETER \direct{getting rid of his}¤
It is awfully difficult to catch a mermaid.</p>

\direct{The mermaids at times find it just as difficult to catch him, though he sometimes joins them in their one game,which consists in lazily blowing their bubbles into the air and seeing who can catch them. The number of bubbles</i> PETER <i>has flown away with! When the weather grows</i> <i>cold mermaids migrate 'to the other side of the world, and he once went with a great shoal of them half the way.}

<p>They are such cruel creatures, Wendy, that they try to pull boys and girls like you into the water and drown them.</p>

<p>WENDY \direct{too guarded by this time to ask what he means 'precisely by 'like you,' though she is very desirous of knowing}¤
How hateful!</p>

\begin{stagedir}
(She is slightly different in appearance now, rather rounder, while</i> JOHN <i>and</i> MICHAEL <i>are not quite so round. The reason is that when new lost children arrive at his underground home</i> PETER <i>finds new trees for them to go up and down by, and instead of fitting the tree to them he makes them fit the tree. Sometimes it can be done by adding or removing garments, but if you are bumpy, or the tree is an odd shape, he has things done to you with a roller, and after that you fit.</i><br> 

<p><i>The other boys are now playing King of the Castle, throwing each other into the water, taking headers and so on; but these two continue to talk.)
\end{stagedir}

<p>PETER. Wendy, this is a fearfully important rock. It is called Marooners' Rock. Sailors are marooned, you know, when their captain leaves them on a rock and sails away.</p>

<p>WENDY. Leaves them on this little rock to drown?</p>

<p>PETER \direct{lightly}¤
Oh, they don't live long. Their hands are tied, so that they can't swim. When the tide is full this rock is covered with water, and then the sailor drowns.</p>

\begin{stagedir}
(WENDY <i>is uneasy as she surveys the rock, which is the only one in the lagoon and no larger than a table. Since she last looked around a threatening change has come over the scene. The sun has gone, but the moon has not come. What has come is a cold shiver across the waters which has sent all the wiser mermaids to their coral recesses. They know that evil is creeping over the lagoon. Of the boys</i> PETER <i>is of course the first to scent it, and he has leapt to his feet before the words strike the rock&mdash;
\end{stagedir}

<blockquote> 'And if we 're parted by a shot<br> We 're sure to meet below.'</blockquote>

\begin{stagedir}
The games on the rock and around it end so abruptly that several divers are checked in the air. There they hang waiting for the word of command from</i> PETER.<i>When they get it they strike the water simultaneously, and the rock is at once as bare as if suddenly they had been blown off it. Thus the pirates find it deserted when their dinghy strikes the rock and is nearly stove in by the concussion.)
\end{stagedir}

<p>SMEE. Luff, you spalpeen, luff!
\direct{They are</i> SMEE <i>and</i> STARKEY, <i>with</i> TIGER LILY, <i>their captive, bound hand and foot.}
What we have got to do is to hoist the redskin on to the rock and leave her there to drown.</p>

\direct{To one of her race this is an end darker than death by fire or torture, for it is written in the laws of the Piccaninnies that there is no path through water to the happy hunting ground. Yet her face is impassive; she is the daughter of a chief and must die as a chief's daughter; it is enough.}

<p>STARKEY \direct{chagrined because she does not mewl}¤
No mewling. This is your reward for prowling round the ship with a knife in your mouth.</p>

<p>TIGER LILLY \direct{stoically}¤
Enough said.</p>

<p>SMEE \direct{who would have preferred a farewell palaver}¤
So that's it! On to the rock with her, mate.</p>

<p>STARKEY \direct{experiencing for perhaps the last time the stirrings of a man}¤
Not so rough, Smee; roughish, but not so rough.</p>

<p>SMEE \direct{dragging her on to the rock}¤
It is the captain's orders.</p>

\direct{A stave has in some past time been driven into the rock, probably to mark the burial place of hidden treasure, and to this they moor the dinghy.}

WENDY \direct{in the water}¤
Poor Tiger Lily! 

<p>STARKEY. What was that?
\direct{The children bob.}</p>

<p>PETER \direct{who can imitate the captain's voice so perfectly that even the author has a dizzy feeling that at times he was really</i> HOOK}¤)
Ahoy there, you lubbers!</p>

<p>STARKEY. It is the captain; he must be swimming out to us.</p>

<p>SMEE \direct{calling}¤
We have put the redskin on the rock,Captain.</p>

<p>PETER. Set her free.</p>

<p>SMEE. But, Captain&mdash;&mdash;</p>

<p>PETER. Cut her bonds, or I 'll plunge my hook in you.</p>

<p>SMEE. This is queer:</p>

<p>STARKEY \direct{unmanned}¤
Let us follow the captain's orders.</p>

\direct{They undo the thongs and</i> TIGER LILY <i>slides between their legs into the lagoon, forgetting in her haste to utter her war-cry, but</i> PETER <i>utters it for her, so naturally that even the lost boys are deceived. It is at this moment that the voice of the true</i> HOOK <i>is heard.}

<p>HOOK. Boat ahoy!</p>

<p>SMEE \direct{relieved}¤
It is the captain.</p>

\direct{HOOK <i>is swimming, and they help him to scale the rock. He is in gloomy mood.}

<p>STARKEY. Captain, is all well?</p>

<p>SMEE. He sighs.</p>

<p>STARKEY. He sighs again.</p>

<p>SMEE \direct{counting}¤
And yet a third time he sighs.
\direct{With foreboding}¤
What's up, Captain?</p>

<p>HOOK \direct{who has perhaps found the large rich damp cake untouched}¤
The game is up. Those boys have found a mother!</p>

<p>STARKEY. Oh evil day!</p>

<p>SMEE. What is a mother?</p>

<p>WENDY \direct{horrified}¤
He doesn't know!</p>

<p>HOOK \direct{sharply}¤
What was that?</p>

\direct{PETER <i>makes the splash of a mermaid's tail.}

<p>STARKEY. One of them mermaids.</p>

<p>HOOK. Dost not know, Smee? A mother is&mdash;&mdash;
\direct{He finds it more difficult to explain than he had expected, and looks about him for an illustration. He finds one in a great bird which drifts past in a nest as large as the roomiest basin}¤
There is a lesson in mothers for you! The nest must have fallen intothe water, but would the bird desert her eggs?
\direct{PETER, <i>who is now more or less off his head, makes the sound of a bird answering in the negative. The nest is borne out of sight.}

<p>STARKEY. Maybe she is hanging about here to protect Peter?</p>

\direct{HOOK'S <i>face clouds still further and</i> PETER <i>just manages not to call out that he needs no protection.}

<p>SMEE \direct{not usually a man of ideas}¤
Captain, could we not kidnap these boys' mother and make her our mother?</p>

<p>HOOK. Obesity and bunions, 'tis a princely scheme. We will seize the children, make them walk the plank, and Wendy shall be our mother!</p>

<p>WENDY. Never!
\direct{Another splash from</i> PETER.}</p>

<p>HOOK. What say you, bullies?</p>

<p>SMEE. There is my hand on 't.</p>

<p>STARKEY. And mine.</p>

<p>HOOK. And there is my hook. Swear.
\direct{All swear.}
But I had forgot; where is the redskin?</p>

<p>SMEE \direct{shaken}¤
That is all right, Captain; we let her go.</p>

<p>HOOK \direct{terrible}¤
Let her go?</p>

<p>SMEE. 'Twas your own orders, Captain.</p>

<p>STARKEY \direct{whimpering}¤
You called over the water to us to let her go.</p>

<p>HOOK. Brimstone and gall, what cozening is here?
\direct{Disturbed by their faithful faces}¤
Lads, I gave no such order.</p>

<p>SMEE 'Tis passing queer.</p>

<p>HOOK \direct{addressing the immensities}¤
Spirit that haunts thisdark lagoon to-night, dost hear me?</p>

<p>PETER \direct{in the same voice}¤
Odds, bobs, hammer and tongs, I hear you.</p>

<p>HOOK \direct{gripping the stave for support}¤
Who are you, stranger, speak.</p>

<p>PETER \direct{who is only too ready to speak}¤
I am Jas Hook, Captain of the <i>Jolly Roger.</i></p>

<p>HOOK \direct{now white to the gills}¤
No, no, you are not.</p>

<p>PETER. Brimstone and gall, say that again and I 'll cast anchor in you.</p>

<p>HOOK. If you are Hook, come tell me, who am I?</p>

<p>PETER. A codfish, only a codfish.</p>

<p>HOOK \direct{aghast}¤
A codfish?</p>

<p>SMEE \direct{drawing back from him}¤
Have we been captained all this time by a codfish?</p>

<p>STARKEY. It's lowering to our pride.</p>

<p>HOOK \direct{feeling that his ego is slipping from him}¤
Don't desert me, bullies.</p>

<p>PETER \direct{top-heavy}¤
Paw, fish, paw!</p>

\direct{There is a touch of the feminine in</i> HOOK, <i>as in all the greatest prates, and it prompts him to try the guessing game.}

<p>HOOK. Have you another name?</p>

<p>PETER \direct{falling to the lure}¤
Ay, ay.</p>

<p>HOOK \direct{thirstily}¤
Vegetable?</p>

<p>PETER. No.</p>

<p>HOOK. Mineral?</p>

<p>PETER. No.</p>

<p>HOOK. Animal?</p>

<p>PETER \direct{after a hurried consultation with</i> TOOTLES}¤
Yes.</p>

<p>HOOK. Man?</p>

<p>PETER \direct{with scorn}¤
No.</p>

<p>HOOK. Boy?</p>

<p>PETER, Yes.</p>

<p>HOOK. Ordinary boy?</p>

<p>PETER. No!</p>

<p>HOOK. Wonderful boy?</p>

<p>PETER \direct{to</i> WENDY'S <i>distress}¤
Yes!</p>

<p>HOOK. Are you in England?</p>

<p>PETER. No.</p>

<p>HOOK. Are you here?</p>

<p>PETER. Yes.</p>

<p>HOOK \direct{beaten, though he feels he has very nearly got it}¤
Smee, you ask him some questions.</p>

<p>SMEE \direct{rummaging his brains}¤
I can't think of a thing,</p>

<p>PETER. Can't guess, can't guess!
\direct{Foundering in his cockiness}¤
Do you give it up?</p>

<p>HOOK \direct{eagerly}¤
Yes.</p>

<p>PETER. All of you?</p>

<p>SMEE and STARKEY. Yes.</p>

<p>PETER \direct{crowing}¤
Well, then, I am Peter Pan!</p>

\direct{Now they have him.}

<p>HOOK. Pan! Into the water, Smee. Starkey, mind the boat. Take him dead or alive!</p>

<p>PETER \direct{who still has all his baby teeth}¤
Boys, lam into the pirates!</p>

\direct{For <i>a moment the only two we can see are in the dinghy, where</i> JOHN <i>throws himself on</i> STARKEY. STARKEY <i>wriggles into the lagoon and</i> JOHN <i>leaps so quickly after him that he reaches it first. The impression left on</i> STARKEY <i>is that he is being attacked by the</i> TWINS. <i>The water becomes stained. The dinghy drifts away. Here and there a head shows in the water, and once it is the head of the crocodile. In the growing gloom some strike at their friends,</i> SLIGHTLY <i>getting</i> TOOTLES <i>in the fourth rib while he himself is pinked by</i> CURLY. <i>It looks as if the boys were getting the worse of it, which is perhaps just as well at this point, because</i> PETER, <i>who will be the determining factor in the end, has a perplexing way of changing sides if he is winning too easily.</i> HOOK'S <i>iron claw makes a circle of black water round him from which opponents flee like fishes. There is only one prepared to enter that dreadful circle. His name is</i> PAN. <i>Strangely, it is not in the water that they meet.</i> HOOK <i>has risen tothe rock to breathe, and at the same moment</i> PETER <i>scales it on the opposite side. The rock is now wet and as slippery as a ball, and they have to crawl rather than climb. Suddenly they are face to face.</i> PETER <i>gnashes his pretty teeth with joy, and is gathering himself for the spring when he sees he is higher up the rock than his foe. Courteously he waits;</i> HOOK <i>sees his intention, and taking advantage of it claws twice.</i> PETER <i>is untouched, but unfairness is what he never can get used to, and in his bewilderment he rolls off the rock. The crocodile, whose tick has been drowned in the strife, rears its jaws, and</i> HOOK, <i>who has almost stepped into them, is pursued by it to land. All is quiet on the lagoon now, not a sound save little waves nibbling at the rock, which is smaller than when we last looked at it. Two boys appear with the dinghy, and the others despite their wounds climb into it. They send the cry 'Peter&mdash;Wendy' across the waters, but no answer comes.}

<p>NIBS. They must be swimming home.</p>

<p>JOHN. Or flying.</p>

<p>FIRST TWIN. Yes, that is it. Let us be off and call to them as we go.</p>

\begin{stagedir}
(The dinghy disappears with its load, whose hearts would sink it if they knew of the peril of</i> WENDY <i>and her captain. From near and far away come the cries 'Peter&mdash;Wendy' till we no longer hear them.

Two small figures are now on the rock, but they have fainted. A mermaid who has dared to come back in the stillness stretches up her arms and is slowly pulling</i>WENDY <i>into the water to drown her.</i> WENDY <i>starts up just in time.)
\end{stagedir}

<p>WENDY. Peter!
\direct{He rouses himself and looks around him.}
Where are we, Peter?</p>

<p>PETER. We are on the rock, but it is getting smaller. Soon the water will be over it. Listen!</p>

\direct{They can hear the wash of the relentless little waves.}

<p>WENDY. We must go.</p>

<p>PETER. Yes.</p>

<p>WENDY. Shall we swim or fly?</p>

<p>PETER. Wendy, do you think you could swim or fly to the island without me?</p>

<p>WENDY. You know I couldn't, Peter, I am just a beginner.</p>

<p>PETER. Hook wounded me twice.
\direct{He believes it; he is so good at pretend that he feels the pain, his arms hang limp.}
I can neither swim nor fly.</p>

<p>WENDY. Do you mean we shall both be drowned?</p>

<p>PETER. Look how the water is rising!</p>

\direct{They cover their faces with their hands. Something touches</i> WENDY <i>as lightly as a kiss.}

<p>PETER \direct{with little interest}¤
It must be the tail of the kite we made for Michael; you remember it tore itself outof his hands and floated away.
\direct{He looks up and sees the kite sailing overhead.}
The kite! Why shouldn't it carry you?
\direct{He grips the tail and pulls, and the kite responds.}</p>

<p>WENDY. Both of us!</p>

<p>PETER. It can't lift two, Michael and Curly tried.</p>

\direct{She knows very well that if it can lift her it can lift him also, for she has been told by the boys as a deadly secret that one of the queer things about him is that he is no weight at all. But it is a forbidden subject.}

<p>WENDY. I won't go without you. Let us draw lots which is to stay behind.</p>

<p>PETER. And you a lady, never!
\direct{The tail is in her hands, and the kite is tugging hard. She holds out her mouth to</i> PETER, <i>but he knows they cannot do that.}
Ready, Wendy!</p>

\direct{The kite draws her out of sight across the lagoon.</i><br> <i> The waters are lapping over the rock now, and</i> PETER <i>knows that it will soon be submerged. Pale rays of light mingle with the moving clouds, and from the coral grottoes is to be heard a sound, at once the most musical and the most melancholy in the Never Land, the mermaids calling to the moon to rise.</i> PETER <i>is afraid at last, and a tremor runs through him, like a shudder passing over the lagoon; but on the lagoon one shudder follows another till there are hundreds of them, and he feels just the one.}

<p>PETER \direct{with a drum beating in his breast as if he were a real boy at last}¤
To die will be an awfully big adventure.</p>

\direct{The blind rises again, and the lagoon is now suffused with moonlight. He is on the rock still, but the water is over his feet. The nest is borne nearer, and the bird, after cooing a message to him, leaves it and wings her way upwards.</i> PETER, <i>who knows the bird language, slips into the nest, first removing the two eggs and placing them in</i> STARKEY'S <i>hat, which has been left on the stave.
The hat drifts away from the rock, but he uses the stave as a mast. The wind is driving him toward the open sea. He takes off his shirt, which he had forgotten to remove while bathing, and unfurls it as a sail. His vessel tacks, and he passes from sight, naked and victorious. The bird returns and sits on the hat.}

% !TEX program = pdflatex
% !TEX encoding = UTF-8
% !TEX spellcheck = en_GB
% !TEX root = peter-pan.tex

\Act{The Home under the Ground}

\begin{stagedir}
We see simultaneously the home under the ground, with the children in it
and the wood above ground with the redskins on it.
Below, the children are gobbling their evening meal;
above, the redskins are squatting in their blankets near the little house guarding the children from the pirates.
The only way of communicating between these two parties is by means of the hollow trees.

The home has an earthen floor, which is handy for digging in if you want to go fishing;
and owing to there being so many entrances there is not much wall space.
The table at which the lost ones are sitting is a board on top of a live tree trunk,
which has been cut flat but has such growing pains that the board rises as they eat,
and they have sometimes to pause in their meals to cut a bit more off the trunk.
Their seats are pumpkins or the large gay mushrooms of which we have seen an imitation one concealing the chimney.
There is an enormous fireplace which is in almost any part of the room where you care to light it,
and across this Wendy has stretched strings, made of fibre, from which she hangs her washing.
There are also various tomfool things in the room of no use whatever.

Michael’s basket bed is nailed high up on the wall as if to protect him from the cat,
but there is no indication at present of where the others sleep.
At the back between two of the tree trunks is a grindstone,
and near it is a lovely hole, the size of a band‐box,
with a gay curtain drawn across so that you cannot see what is inside.
This is Tink’s withdrawing‐room and bed‐chamber, and it is just as well that you cannot see inside,
for it is so exquisite in its decoration and in the personal apparel spread out on the bed
that you could scarcely resist making off with something.
Tink is within at present, as one can guess from a glow showing through the chinks.
It is her own glow, for though she has a chandelier for the look of the thing, of course she lights her residence herself.
She is probably wasting valuable time just now wondering whether to put on the smoky blue or the apple‐blossom.

All the boys except Peter are here, and Wendy has the head of the table,
smiling complacently at their captivating ways,
but doing her best at the same time to see that they keep the rules
about hands‐off‐the‐table, no‐two‐to‐speak‐at‐once, and so on.
She is wearing romantic woodland garments, sewn by herself, with red berries in her hair
which go charmingly with her complexion, as she knows;
indeed she searched for red berries the morning after she reached the island.
The boys are in picturesque attire of her contrivance,
and if these don’t always fit well the fault is not hers but the wearers,
for they constantly put on each other’s things when they put on anything at all.
Michael is in his cradle on the wall.
First Twin is apart on a high stool and wears a dunce’s cap, another invention of Wendy’s,
but not wholly successful because everybody wants to be dunce.

It is a pretend meal this evening, with nothing whatever on the table, not a mug, nor a crust, nor a spoon.
They often have these suppers and like them on occasions
as well as the other kind, which consist chiefly of bread‐fruit, tappa rolls, yams, mammee apples and banana splash,
washed down with calabashes of poe‐poe.
The pretend meals are not Wendy’s idea;
indeed she was rather startled to find, on arriving, that Peter knew of no other kind,
and she is not absolutely certain even now that he does eat the other kind,
though no one appears to do it more heartily.
He insists that the pretend meals should be partaken of with gusto,
and we see his band doing their best to obey orders.
\end{stagedir}

\begin{drama}

\wendyspeaks \direct{her fingers to her ears, for their chatter and clatter are deafening}¤
Silence!
Is your mug empty, Slightly?

\slightlyspeaks \direct{who would not say this if he had a mug}¤
Not quite empty, thank you.

\nibsspeaks
Mummy, he has not even begun to drink his poe‐poe.

\slightlyspeaks \direct{seizing his chance, for this is tale‐bearing}¤
I complain of Nibs!

\direct{\john holds up his hand.}

\wendyspeaks
Well, John?

\johnspeaks
May I sit in Peter’s chair as he is not here?

\wendyspeaks
In your father’s chair?
Certainly not.

\johnspeaks
He is not really our father.
He did not even know how to be a father till I showed him.

\direct{This is insubordination.}

\secondtwinspeaks
I complain of John!

\direct{The gentle \tootles raises his hand.}

\tootlesspeaks \direct{who has the poorest opinion of himself}¤
I don’t suppose Michael would let me be baby?

\michaelspeaks
No, I won’t.

\tootlesspeaks
May I be dunce?

\firsttwinspeaks \direct{from his perch}¤
No.
It’s awfully difficult to be dunce.

\tootlesspeaks
As I can’t be anything important would any of you like to see me do a trick?

\speaker{OMNES}
No.

\tootlesspeaks \direct{subsiding}¤
I hadn’t really any hope.

\direct{The tale‐telling breaks out again.}

\nibsspeaks
Slightly is coughing on the table.

\curlyspeaks
The twins began with tappa rolls.

\slightlyspeaks
I complain of Nibs!

\nibsspeaks
I complain of Slightly!

\wendyspeaks
Oh dear, I am sure I sometimes think that spinsters are to be envied.

\michaelspeaks
Wendy, I am too big for a cradle.

\wendyspeaks
You are the littlest, and a cradle is such a nice homely thing to have about a house.
You others can clear away now.
\direct{She sits down on a pumpkin near the fire to her usual evening occupation, darning.}
Every heel with a hole in it!

\begin{stagedir}
(The boys clear away with dispatch,
washing dishes they don’t have in a non‐existent sink and stowing them in a cupboard that isn’t there.
Instead of sawing the table‐leg to‐night they crush it into the ground like a concertina,
and are now ready for play, in which they indulge hilariously.

A movement of the Indians draws our attention to the scene above.
Hitherto, with the exception of \panther, who sits on guard on top of the little house,
they have been hunkering in their blankets, mute but picturesque;
now all rise and prostrate themselves before the majestic figure of \peter,
who approaches through the forest carrying a gun and game bag.
It is not exactly a gun.
He often wanders away alone with this weapon,
and when he comes back you are never absolutely certain whether he has had an adventure or not.
He may have forgotten it so completely that he says nothing about it;
and then when you go out you find the body.
On the other hand he may say a great deal about it, and yet you never find the body.
Sometimes he comes home with his face scratched,
and tells \wendy, as a thing of no importance,
that he got these marks from the little people for cheeking them at a fairy wedding,
and she listens politely, but she is never quite sure, you know;
indeed the only one who is sure about anything on the island is \peter.)
\end{stagedir}

\peterspeaks
The Great White Father is glad to see the Piccaninny braves protecting his wigwam from the pirates.

\tigerlilyspeaks
The Great White Father save me from pirates.
Me his velly nice friend now;
no let pirates hurt him.

\speaker{BRAVES}
Ugh, ugh, wah!

\tigerlilyspeaks
Tiger Lily has spoken.

\pantherspeaks
Loola, loola!
Great Big Little Panther has spoken.

\peterspeaks
It is well.
The Great White Father has spoken.

\direct{This has a note of finality about it, with the implied, ‘And now shut up’
which is never far from the courteous receptions of well‐meaning inferiors by born leaders of men.
He descends his tree, not unheard by \wendy.}

\wendyspeaks
Children, I hear your father’s step.
He likes you to meet him at the door.
\direct{\peter scatters pretend nuts among them and watches sharply to see that they crunch with relish.}
Peter, you just spoil them, you know!

\johnspeaks \direct{who would be incredulous if he dare}¤
Any sport, Peter?

\peterspeaks
Two tigers and a pirate.

\johnspeaks \direct{boldly}¤
Where are their heads?

\peterspeaks \direct{contracting his little brows.}
In the bag.

\johnspeaks \direct{No, he doesn’t say it.
He backs away.}

\wendyspeaks \direct{peeping into the bag}¤
They are beauties.
\direct{She has learned her lesson.}

\firsttwinspeaks
Mummy, we all want to dance.

\wendyspeaks
The mother of such an armful dance!

\slightlyspeaks
As it is Saturday night?

\direct{They have long lost count of the days,
but always if they want to do anything special they say this is Saturday night, and then they do it.}

\wendyspeaks
Of course it is Saturday night, Peter?
\direct{He shrugs an indifferent assent.}
On with your nighties first.

\direct{They disappear into various recesses,
and \peter and \wendy with her darning are left by the fire to dodder parentally.
She emphasises it by humming a verse of ‘John Anderson my Jo,’ which has not the desired effect on \peter.
She is too loving to be ignorant that he is not loving enough,
and she hesitates like one who knows the answer to her question.}

\speakercontinues
What is wrong, Peter?

\peterspeaks \direct{scared}¤
It is only pretend, isn’t it, that I am their father?

\wendyspeaks \direct{drooling}¤
Oh yes.

\direct{His sigh of relief is without consideration for her feelings.}

\speakercontinues
But they are ours, Peter, yours and mine.

\peterspeaks \direct{determined to get at facts, the only things that puzzle him}¤
But not really?

\wendyspeaks
Not if you don’t wish it.

\peterspeaks
I don’t.

\wendyspeaks \direct{knowing she ought not to probe but driven to it by something within}¤
What are your exact feelings for me, Peter?

\peterspeaks \direct{in the class‐room}¤
Those of a devoted son, Wendy.

\wendyspeaks \direct{turning away}¤
I thought so.

\peterspeaks
You are so puzzling.
Tiger Lily is just the same; there is something or other she wants to be to me,
but she says it is not my mother.

\wendyspeaks \direct{with spirit}¤
No, indeed it isn’t.

\peterspeaks
Then what is it?

\wendyspeaks
It isn’t for a lady to tell.

\direct{The curtain of the fairy chamber opens slightly,
and \tink, who has doubtless been eavesdropping, tinkles a laugh of scorn.}

\peterspeaks \direct{badgered}¤
I suppose she means that she wants to be my mother.

\direct{\tink’s comment is ‘You silly ass.’}

\wendyspeaks \direct{who has picked up some of the fairy words}¤
I almost agree with her!

\begin{stagedir}
(The arrival of the boys in their nightgowns turns \wendy’s mind to practical matters,
for the children have to be arranged in line and passed or not passed for cleanliness.
\slightly is the worst.
At last we see how they sleep,
for in a babel the great bed which stands on end by day against the wall is unloosed from custody
and lowered to the floor.
Though large, it is a tight fit for so many boys,
and \wendy has made a rule that there is to be no turning round until one gives the signal, when all turn at once.

\firsttwin is the best dancer and performs mightily on the bed and in it and out of it and over it
to an accompaniment of pillow fights by the less agile;
and then there is a rush at \wendy.)
\end{stagedir}

\nibsspeaks
Now the story you promised to tell us as soon as we were in bed!

\wendyspeaks \direct{severely}¤
As far as I can see you are not in bed yet.

\direct{They scramble into the bed, and the effect is as of a boxful of sardines.}

\wendyspeaks \direct{drawing up her stool}¤
Well, there was once a gentleman———

\curlyspeaks
I wish he had been a lady.

\nibsspeaks
I wish he had been a white rat.

\wendyspeaks
Quiet!
There was a lady also.
The gentleman’s name was Mr.\@ Darling and the lady’s name was Mrs.\@ Darling———

\johnspeaks
I knew them!

\michaelspeaks \direct{who has been allowed to join the circle}¤
I think I knew them.

\wendyspeaks
They were married, you know; and what do you think they had?

\nibsspeaks
White rats?

\wendyspeaks
No, they had three descendants.
White rats are descendants also.
Almost everything is a descendant.
Now these three children had a faithful nurse called Nana.

\michaelspeaks \direct{alas}¤
What a funny name!

\wendyspeaks
But Mr.\@ Darling—\direct{faltering} or was it Mrs.\@ Darling?—was angry with her
and chained her up in the yard;
so all the children flew away.
They flew away to the Never Land, where the lost boys are.

\curlyspeaks
I just thought they did;
I don’t know how it is, but I just thought they did.

\tootlesspeaks
Oh, Wendy, was one of the lost boys called Tootles?

\wendyspeaks
Yes, he was.

\tootlesspeaks \direct{dazzled}¤
Am I in a story?
Nibs, I am in a story!

\peterspeaks \direct{who is by the fire making *Pan’s* pipes with his knife,
and is determined that \wendy shall have fair play, however beastly a story he may think it}¤
A little less noise there.

\wendyspeaks \direct{melting over the beauty of her present performance, but without any real qualms}¤
Now I want you to consider the feelings of the unhappy parents with all their children flown away.
Think, oh think, of the empty beds.
\direct{The heartless ones think of them with glee.}

\firsttwinspeaks \direct{cheerfully}¤
It’s awfully sad.

\wendyspeaks
But our heroine knew that her mother would always leave the window open for her progeny to fly back by;
so they stayed away for years and had a lovely time.

\direct{\peter is interested at last.}

\firsttwinspeaks
Did they ever go back?

\wendyspeaks \direct{comfortably}¤
Let us now take a peep into the future.
Years have rolled by, and who is this elegant lady of uncertain age alighting at London station?

\direct{The tension is unbearable.}

\nibsspeaks
Oh, Wendy, who is she?

\wendyspeaks \direct{swelling}¤
Can it be—yes—no—yes, it is the fair Wendy!

\tootlesspeaks
I am glad.

\wendyspeaks
Who are the two noble portly figures accompanying her?
Can they be John and Michael?
They are.
\direct{Pride of \michael.}
‘See, dear brothers,’ says Wendy, pointing upward, ‘there is the window standing open.’
So up they flew to their loving parents, and pen cannot inscribe the happy scene over which we draw a veil.
\direct{Her triumph is spoilt by a groan from \peter and she hurries to him.}
Peter, what is it?
\direct{Thinking he is ill, and looking lower than his chest.}
Where is it?

\peterspeaks
It isn’t that kind of pain.
Wendy, you are wrong about mothers.
I thought like you about the window, so I stayed away for moons and moons,
and then I flew back, but the window was barred,
for my mother had forgotten all about me and there was another little boy sleeping in my bed.

\direct{This is a general damper.}

\johnspeaks
Wendy, let us go back!

\wendyspeaks
Are you sure mothers are like that?

\peterspeaks
Yes.

\wendyspeaks
John, Michael!
\direct{She clasps them to her.}

\firsttwinspeaks \direct{alarmed}¤
You are not to leave us, Wendy?

\wendyspeaks
I must.

\nibsspeaks
Not to‐night?

\wendyspeaks
At once.
Perhaps mother is in half‐mourning by this time!
Peter, will you make the necessary arrangements?

\direct{She asks it in the steely tones women adopt when they are prepared secretly for opposition.}

\peterspeaks \direct{coolly}¤
If you wish it.

\direct{He ascends his tree to give the redskins their instructions.
The lost boys gather threateningly round \wendy.}

\curlyspeaks
We won’t let you go!

\wendyspeaks \direct{with one of those inspirations women have, in an emergency,
to make use of some male who need otherwise have no hope}¤
Tootles, I appeal to you.

\tootlesspeaks \direct{leaping to his death if necessary}¤
I am just Tootles and nobody minds me, but the first who does not behave to Wendy I will blood him severely.
\direct{\peter returns.}

\peterspeaks \direct{with awful serenity}¤
Wendy, I told the braves to guide you through the wood as flying tires you so.
Then Tinker Bell will take you across the sea.
\direct{A shrill tinkle from the boudoir probably means ‘and drop her into it.’}

\nibsspeaks \direct{fingering the curtain which he is not allowed to open}¤
Tink, you are to get up and take Wendy on a journey.
\direct{Star‐eyed}¤
She says she won’t!

\peterspeaks \direct{taking a step toward that chamber}¤
If you don’t get up, Tink, and dress at once——¤
She is getting up!

\wendyspeaks \direct{quivering now that the time to depart has come}¤
Dear ones, if you will all come with me I feel almost sure I can get my father and mother to adopt you.

\direct{There is joy at this, not that they want parents, but novelty is their religion.}

\nibsspeaks
But won’t they think us rather a handful?

\wendyspeaks \direct{a swift reckoner}¤
Oh no, it will only mean having a few beds in the drawing‐room; they can be hidden behind screens on first Thursdays.

\direct{Everything depends on \peter.}

\speaker{OMNES}
Peter, may we go?

\peterspeaks \direct{carelessly through the pipes to which he is giving a finishing touch}¤
All right.

\direct{They scurry off to dress for the adventure.}

\wendyspeaks \direct{insinuatingly}¤
Get your clothes, Peter.

\peterspeaks \direct{skipping about and playing fairy music on his pipes, the only music he knows}¤
I am not going with you, Wendy.

\wendyspeaks
Yes, Peter!

\peterspeaks
No.

\direct{The lost ones run back gaily, each carrying a stick with a bundle on the end of it.}

\wendyspeaks
Peter isn’t coming!

\direct{All the faces go blank.}

\johnspeaks \direct{even \john}¤
Peter not coming!

\tootlesspeaks \direct{overthrown}¤
Why, Peter?

\peterspeaks \direct{his pipes more riotous than ever}¤
I just want always to be a little boy and to have fun.

\direct{There is a general fear that they are perhaps making the mistake of their lives.}

\speakercontinues
Now then, no fuss, no blubbering.
\direct{With dreadful cynicism}¤
I hope you will like your mothers!
Are you ready, Tink?
Then lead the way.

\begin{stagedir}
(\tink darts up any tree, but she is the only one.
The air above is suddenly rent with shrieks and the clash of steel.
Though they cannot see, the boys know that \hook and his crew are upon the Indians.
Mouths open and remain open, all in mute appeal to \peter.
He is the only boy on his feet now, a sword in his hand, the same he slew *Barbicue* with;
and in his eye is the lust of battle.

We can watch the carnage that is invisible to the children.
\hook has basely broken the two laws of Indian warfare,
which are that the redskins should attack first, and that it should be at dawn.
They have known the pirate whereabouts since, early in the night, one of \smee’s fingers crackled.
The brushwood has closed behind their scouts as silently as the sand on the mole;
for hours they have imitated the lonely call of the coyote;
no stratagem has been overlooked, but alas, they have trusted to the pale‐face’s honour to await an attack at dawn,
when his courage is known to be at the lowest ebb.
\hook falls upon them pell‐mell,
and one cannot withhold a reluctant admiration for the wit that conceived so subtle a scheme
and the fell genius with which it is carried out.
If the braves would rise quickly they might still have time to scalp,
but this they are forbidden to do by the traditions of their race,
for it is written that they must never express surprise in the presence of the pale‐face.
For a brief space they remain recumbent, not a muscle moving, as if the foe were here by invitation.
Thus perish the flower of the Piccaninnies, though not unavenged,
for with \name{LEAN WOLF} fall \name{ALF MASON} and \name{CANARY ROBB},
while other pirates to bite dust are \name{BLACK GILMOUR} and \name{ALAN HERB},
that same \name{HERB} who is still remembered at Manaos
for playing skittles with the mate of the \emph{Switch} for each other’s heads.
\name{CHAY TURLEY}, who laughed with the wrong side of his mouth \parenth{having no other},
is tomahawked by \panther,
who eventually cuts a way through the shambles with \tigerlily and a remnant of the tribe.

This onslaught passes and is gone like a fierce wind.
The victors wipe their cutlasses, and squint, ferret‐eyed, at their leader.
He remains, as ever, aloof in spirit and in substance.
He signs to them to descend the trees, for he is convinced that \pan is down there,
and though he has smoked the bees it is the honey he wants.
There is something in \peter that at all times goads this extraordinary man to frenzy;
it is the boy’s cockiness, which disturbs \hook like an insect.
If you have seen a lion in a cage futilely pursuing a sparrow you will know what is meant.
The pirates try to do their captain’s bidding, but the apertures prove to be not wide enough for them;
he cannot even ram them down with a pole.
He steals to the mouth of a tree and listens.)
\end{stagedir}

\peterspeaks \direct{prematurely}¤
All is over!

\wendyspeaks
But who has won?

\peterspeaks
Hst!
If the Indians have won they will beat the tom‐tom; it is always their signal of victory.

\direct{\hook licks his lips at this and signs to \smee, who is sitting on it, to hold up the tom‐tom.
He beats upon it with his claw, and listens for results.}

\tootlesspeaks
The tom‐tom!

\peterspeaks \direct{sheathing his sword}¤
An Indian victory!

\direct{The cheers from below are music to the black hearts above.}

\speakercontinues
You are quite safe now, Wendy.
Boys, good‐bye.
\direct{He resumes his pipes.}

\wendyspeaks
Peter, you will remember about changing your flannels, won’t you?*'*

\peterspeaks
Oh, all right!

\wendyspeaks
And this is your medicine.

\direct{She puts something into a shell and leaves it on a ledge between two of the trees.
It is only water, but she measures it out in drops.}

\peterspeaks
I won’t forget.

\wendyspeaks
Peter, what are you to me?

\peterspeaks \direct{through the pipes}¤
Your son, Wendy.

\wendyspeaks
Oh, good‐bye!

\direct{The travellers start upon their journey, little witting that \hook has issued his silent orders:
a man to the mouth of each tree, and a row of men between the trees and the little house.
As the children squeeze up they are plucked from their trees,
trussed, thrown like bales of cotton from one pirate to another, and so piled up in the little house.
The only one treated differently is \wendy, whom \hook escorts to the house on his arm with hateful politeness.
He signs to his dogs to be gone, and they depart through the wood,
carrying the little house with its strange merchandise and singing their ribald song.
The chimney of the little house emits a jet of smoke fitfully, as if not sure what it ought to do just now.

\hook and \peter are now, as it were, alone on the island.
Below, \peter is on the bed, asleep, no weapon near him;
above, \hook, armed to the teeth, is searching noiselessly for some tree down which the nastiness of him can descend.
Don’t be too much alarmed by this; it is precisely the situation \peter would have chosen;
indeed if the whole thing were pretend——.
One of his arms droops over the edge of the bed, a leg is arched,
and the mouth is not so tightly closed that we cannot see the little pearls.
He is dreaming, and in his dreams he is always in pursuit of a boy who was never here, nor anywhere:
the only boy who could beat him.

\hook finds the tree.
It is the one set apart for \slightly who being addicted when hot to the drinking of water has swelled in consequence
and surreptitiously scooped his tree for easier descent and egress.
Down this the pirate wriggles a passage.
In the aperture below his face emerges and goes green as he glares at the sleeping child.
Does no feeling of compassion disturb his sombre breast?
The man is not wholly evil: he has a \emph{Thesaurus} in his cabin, and is no mean performer on the flute.
What really warps him is a presentiment that he is about to fail.
This is not unconnected with a beatific smile on the face of the sleeper,
whom he cannot reach owing to being stuck at the foot of the tree.
He, however, sees the medicine shell within easy reach,
and to \wendy’s draught he adds from a bottle five drops of poison
distilled when he was weeping from the red in his eye.
The expression on \peter’s face merely implies that something heavenly is going on.
\hook worms his way upwards, and winding his cloak around him,
as if to conceal his person from the night of which he is the blackest part,
he stalks moodily toward the lagoon.

A dot of light flashes past him and darts down the nearest tree, looking for \peter, only for \peter,
quite indifferent about the others when she finds him safe.}

\peterspeaks \direct{stirring}¤
Who is that?
\direct{\tink has to tell her tale, in one long ungrammatical sentence.}
The redskins were defeated?
Wendy and the boys captured by the pirates!
I’ll rescue her, I’ll rescue her!
\direct{He leap first at his dagger, and then at his grindstone, to sharpen it.
\tink alights near the shell, and rings out a warning cry.}
Oh, that is just my medicine.
Poisoned?
Who could have poisoned it?
I promised Wendy to take it, and I will as soon as I have sharpened my dagger.
\direct{\tink, who sees its red colour and remembers the red in the *grate’s* eye,
nobly swallows the draught as \peter’s hand is reaching for it.}
Why, Tink, you have drunk my medicine!
\direct{She flutters strangely about the room, answering him now in a very thin tinkle.}
It was poisoned and you drank it to save my life!
Tink, dear Tink, are you dying?
\direct{He has never called her dear \tink before, and for a moment she is gay;
she alights on his shoulder, gives his chin a loving bite, whispers ‘You silly ass’ and falls on her tiny bed.
The boudoir, which is lit by her, flickers ominously.
He is on his knees by the opening.**}

\speakercontinues**
Her light is growing faint, and if it goes out, that means she is dead!
Her voice is so low I can scarcely tell what she is saying.
She says—she says she thinks she could get well again if children believed in fairies!
\direct{He rises and throws out his arms he knows not to whom, perhaps to the boys and girls of whom he is not one.}
Do you believe in fairies?
Say quick that you believe!
If you believe, clap your hands!
\direct{Many clap, some don’t, a few hiss.
Then perhaps there is a rush of Nanas to the nurseries to see what on earth is happening.
But \tink is saved.}
Oh, thank you, thank you, thank you!
And now to rescue Wendy!

\direct{\tink is already as merry and impudent as a *grig,* with not a thought for those who have saved her.
\peter ascends his tree as if he were shot up it.
What he is feeling is ‘\hook or me this time!’
He is frightfully happy.
He soon hits the trail, for the smoke from the house has lingered here and there to guide him.
He takes wing.}

\end{drama}

\endinput

% !TEX program = pdflatex
% !TEX encoding = UTF-8
% !TEX spellcheck = en_GB
% !TEX root = peter-pan.tex

\act

\Scene{The Pirate Ship}

\endinput

\begin{stagedir}
The stage directions for the opening of this scene are as follows:&mdash;

1 Circuit Amber checked to 80.\@ Battens, all Amber checked,
3 ship's lanterns alight.
Arcs: prompt perch 1.
Open dark Amber flooding back, O.P. perch open dark Amber flooding upper deck.
Arc on tall steps at back of cabin to flood back cloth.
Open dark Amber. Warning for slide. Plank ready. Call Hook.

In the strange light thus described we see what is happening on the deck of the \emph{Jolly Roger}, which is flying the skull and crossbones and lies low in the water. There is no need to call Hook, for he is here already, and indeed there is not a pirate aboard who would dare to call him. Most of them are at present carousing in the bowels of the vessel, but on the poop Mullins is visible, in the only great-coat on the ship, raking with his glass the monstrous rocks within which the lagoon is cooped. Such a look-out is supererogatory, for the pirate craft floats immune in the horror of her name.

From Hook's cabin at the back Starkey appears and leans over the bulwark, silently surveying the sullen waters. He is bare-headed and is perhaps thinking with bitterness of his hat,which he sometimes sees still drifting past him with the Never bird sitting on it. The black pirate is asleep on deck, yet even in his dreams rolling mechanically out of the way when Hook draws near. The only sound to be heard is made by Smee at his sewing-machine, which lends a touch of domesticity to the night.

Hook is now leaning against the mast, now prowling the deck, the double cigar in his mouth. With Peter surely at last removed from his path we, who know how vain a tabernacle is man, would not be surprised to find him bellied out by the winds of his success, but it is not so; he is still uneasy, looking long and meaninglessly at familiar objects, such as the ship's bell or the Long Tom, like one who may shortly be a stranger to them. It is as if Pan's terrible oath 'Hook or me this time!' had already boarded the ship.
\end{stagedir}

\begin{drama}

<p>HOOK \direct{communing with his ego}¤
How still the night is; nothing sounds alive. Now is the hour when children in their homes are a-bed; their lips bright-browned with the good-night chocolate, and their tongues drowsily searching for belated crumbs housed insecurely on their shining cheeks. Compare with them the children on this boat about to walk the plank. Split my infinitives, but 'tis my hour of triumph!
\direct{Clinging to this fair prospect he dances a few jubilant steps, but they fall below his usual form.}
And yet some disky spirit compels me now to make my dying speech, lest when dying there may be no time for it. All mortals envy me, yet better perhaps for Hook to have had less ambition! O fame, fame, thou glittering bauble, what if the very&amp;mdsah;&amp;&mdash;¤
\direct{SMEE, <i>engrossed in his labours at the sewing-machine, tears a piece of calico with a rending sound which makes the Solitary think for a moment that the untoward has happened to his garments.}
No little children love me. I am told they play at Peter Pan, and that the strongest always chooses to be Peter. They would rather be a Twin than Hook; they force the baby to be Hook. The baby! that is where the canker gnaws.
\direct{He contemplates his industrious boatswain.}
'Tis said they find Smee lovable. But an hour agone I found him letting the youngest of them try on his spectacles. Pathetic Smee, the Nonconformist pirate, a happy smile upon his face because he thinks they fear him! How can I break it to him that they think him lovable? No, bi-carbonate of Soda, no, not even&mdash;&mdash;¤
\direct{Another rending of the calico disturbs him, and he has a private consultation with</i> STARKEY, <i>who turns him round and evidently assures him that all is well. The peroration of his speech is nevertheless for ever lost, as eight bells strikes and his crew pour forth in bacchanalian orgy. From, the poop he watches their dance till it frets him beyond bearing.}
Quiet,you dogs, or I'll cast anchor in you!
\direct{He descends to a barrel on which there are playing-cards, and his crew stand waiting, as ever, like whipped curs.}
Are all the prisoners chained, sothat they can't fly away?</p>

<p>JUKES. Ay, ay, Captain.</p>

<p>HOOK. Then hoist them up.</p>

<p>STARKEY \direct{raising the door of the hold}¤
Tumble up, you ungentlemanly lubbers.</p>

\direct{The terrified boys are prodded up and tossed about the deck.</i> HOOK <i>seems to have forgotten them; he is sitting by the barrel with his cards.}

<p>HOOK \direct{suddenly}¤
So! Now then, you bullies, six of you walk the plank to-night, but I have room for two cabin-boys.
Which of you is it to be?
\direct{He returns to his cards.}</p>

<p>TOOTLES \direct{hoping to soothe him by putting the blame on the only person, vaguely remembered, who is always willing to act as a buffer}¤
You see, sir, I don't think my mother would like me to be a pirate. Would your mother like you to be a pirate, Slightly?</p>

<p>SLIGHTLY \direct{implying that otherwise it would be a pleasure to him to oblige}¤
I don't think so. Twin, would your mother like&mdash;&mdash;</p>

<p>HOOK. Stow this gab.
\direct{To</i> JOHN}¤
You boy, you look as if you had a little pluck in you. Didst never want to be a pirate, my hearty?</p>

<p>JOHN \direct{dazzled by being singled out}¤
When I was at school I&mdash;&mdash;what do you think, Michael?</p>

<p>MICHAEL \direct{stepping into prominence}¤
What would you call me if I joined?</p>

<p>HOOK. Blackbeard Joe.</p>

<p>MICHAEL. John, what do you think?</p>

<p>JOHN. Stop, should we still be respectful subjects of KingGeorge?</p>

<p>HOOK. You would have to swear 'Down with KingGeorge.'</p>

<p>JOHN \direct{grandly}¤
Then I refuse!</p>

<p>MICHAEL. And I refuse.</p>

<p>HOOK. That seals your doom. Bring up their mother.</p>

\direct{WENDY <i>is driven up from the hold and thrown to him. She sees at the first glance that the deck has not been scrubbed for years.}

So, my beauty, you are to see your children walk the plank. 

<p>WENDY \direct{with noble calmness}¤
Are they to die?</p>

<p>HOOK. They are. Silence all, for a mother's last words to her children.</p>

<p>WENDY. These are my last words. Dear boys, I feel that I have a message to you from your real mothers, and it is this, 'We hope our sons will die like English gentlemen.'</p>

\direct{The boys go on fire.}

<p>TOOTLES. I am going to do what my mother hopes. What are you to do, Twin?</p>

<p>FIRST TWIN. What my mother hopes. John, what are&mdash;&mdash;</p>

<p>HOOK. Tie her up! Get the plank ready.</p>

\begin{stagedir}
(WENDY <i>is roped to the mast; but no one regards her, for all eyes are fixed upon the plank now protruding from the poop over the ship's side. A great change, however, occurs in the time</i> HOOK <i>takes to raise his claw and point to this deadly engine. No one is now looking at the plank: for the tick, tick of the crocodile is heard. Yet it is not to bear on the crocodile that all eyes slew round, it is that they may bear on</i> HOOK. <i>Otherwise prisoners and captors are equally inert, like actors in some play who have found themselves 'on' in a scene in which they are not personally concerned. Even the iron claw hangs inactive, as if aware that the crocodile is not coming for it. Affection for their captain, now cowering from view, is not what has given</i> HOOK <i>his dominance over the crew, but as the menacing sound draws nearer they close their eyes respectfully.

<p><i>There is no crocodile. It is</i> PETER, <i>who has been circling the pirate ship, ticking as he flies far more superbly than any clock. He drops into the water and climbs aboard, warning the captives with upraised finger (but still ticking) not for the moment to give audible expression to their natural admiration. Only one pirate sees him,</i> WHIBBLES <i>of the eye patch, who comes up from below.</i> JOHN <i>claps a hand on</i> WHIBBLES's <i>mouth to stifle the groan; four boys hold him to prevent the thud;</i> PETER <i>delivers the blow, and the carrion is thrown overboard. 'One!' says</i> SLIGHTLY, <i>beginning to count.

<p>STARKEY <i>is the first pirate to open his eyes. The ship seems to him to be precisely as when he closed them. He cannot interpret the sparkle that has come into the faces of the captives, who are cleverly pretending to be as afraid as ever. He little knows that the door of the dark cabin has just closed on one more boy. Indeed it is for</i> HOOK<i>alone he looks, and he is a little surprised to see him.)
\end{stagedir}

<p>STARKEY \direct{hoarsely}¤
It is gone, Captain! There is not a sound.</p>

\direct{The tenement that is</i> HOOK <i>heaves tumultuously and he is himself again.}

<p>HOOK \direct{now convinced that some fair spirit watches over him}¤
Then here is to Johnny Plank&mdash;&amp;&mdash;</p>

<p> Avast, belay, the English brig<br> We took and quickly sank,<br> And for a warning to the crew<br> We made them walk the plank!</p>

\direct{As he sings he capers detestably along an imaginary plank and his copy-cats do likewise, joining in the chorus.}

<p> Yo ho, yo ho, the frisky cat,<br> You walks along it so,<br> Till it goes down and you goes down<br> To tooral looral lo!</p>

\direct{The brave children try to stem this monstrous torrent by breaking into the National Anthem.}

<p>STARKEY \direct{paling}¤
I don't like it, messmates!</p>

<p>HOOK. Stow that, Starkey. Do you boys want a touch of the cat before you walk the plank?
\direct{He is more pitiless than ever now that he believes he has a charmed life.}
Fetch the cat, Jukes; it is in the cabin.</p>

<p>JUKES. Ay, ay, sir.
\direct{It is one of his commonest remarks, and is only recorded now because he never makes another. The stage direction 'Exit</i> JUKES' <i>has in this case a special significance. But only the children know that some one is awaiting this unfortunate in the cabin, and</i> HOOK <i>tramples them down as he resumes his ditty:}</p>

<p> Yo ho, yo ho, the scratching cat<br> Its tails are nine you know,<br> And when they're writ upon your back,<br> You 're fit to&mdash;&mdash;</p>

\direct{The last words will ever remain a matter of conjecture, for from the dark cabin comes a curdling screech which wails through the ship and dies away. It is followed by a sound, almost more eerie in the circumstances, that can only be likened to the crowing of a cock.}

<p>HOOK. What was that?</p>

<p>SLIGHTLY \direct{solemnly}¤
Two!</p>

\direct{CECCO <i>swings into the cabin, and in a moment returns, livid.}

<p>HOOK \direct{with an effort}¤
What is the matter with Bill Jukes, you dog?</p>

<p>CECCO. The matter with him is he is dead&mdash;&mdash;stabbed.</p>

<p>PIRATES. Bill Jukes dead!</p>

<p>CECCO. The cabin is as black as a pit, but there is something terrible in there: the thing you heard a-crowing.</p>

<p>HOOK \direct{slowly}¤
Cecco, go back and fetch me out that doodle-doo.</p>

<p>CECCO \direct{unstrung}¤
No, Captain, no.
\direct{He supplicates on his knees, but his master advances on him implacably.}</p>

<p>HOOK \direct{in his most syrupy voice}¤
Did you say you would go, Cecco?</p>

\direct{CECCO <i>goes. All listen. There is one screech, one crow.}

<p>SLIGHTLY \direct{as if he were a bell tolling}¤
Three!</p>

<p>HOOK. 'Sdeath and oddsfish, who is to bring me out that doodle-doo?</p>

\direct{No one steps forward.}

<p>STARKEY \direct{injudiciously}¤
Wait till Cecco comes out.</p>

\direct{The black looks of some others encourage him.}

<p>HOOK. I think I heard you volunteer, Starkey.</p>

<p>STARKEY \direct{emphatically}¤
No, by thunder!</p>

<p>HOOK \direct{in that syrupy voice which might be more engaging when accompanied by his flute}¤
My hook thinks you did. Iwonder if it would not be advisable, Starkey, to humour the hook?</p>

<p>STARKEY. I'll swing before I go in there.</p>

<p>HOOK \direct{gleaming}¤
Is it mutiny?
Starkey is ringleader.
Shake hands, Starkey.</p>

\direct{STARKEY <i>recoils from the claw. It follows him till he leaps overboard.}

<p>Did any other gentleman say mutiny?</p>

\direct{They indicate that they did not even know the late</i> STARKEY.}

<p>SLIGHTLY. Four!</p>

<p>HOOK. I will bring out that doodle-doo myself.</p>

\direct{He raises a blunderbuss but casts it from him with a menacing gesture which means that he has more faith in the claw. With a lighted lantern in his hand he enters the cabin. Not a sound is to be heard now on the ship, unless it be</i> SLIGHTLY <i>wetting his lips to say 'Five.'</i> HOOK <i>staggers out.}

<p>HOOK \direct{unsteadily}¤
Something blew out the light.</p>

<p>MULLINS \direct{with dark meaning}¤
Some&mdash;thing?</p>

<p>NOODLER. What of Cecco?</p>

<p>HOOK. He is as dead as Jukes.</p>

\direct{They are superstitious like all sailors, and</i> MULLINS <i>has planted a dire conception in their minds.}

<p>COOKSON. They do say as the surest sign a ship's accurst is when there is one aboard more than can be accounted for.</p>

<p>NOODLER. I've heard he allus boards the pirate craft at last.
\direct{With dreadful significance}¤
Has he a tail, Captain?</p>

<p>MULLINS. They say that when he comes it is in the likeness of the wickedest man aboard.</p>

<p>COOKSON \direct{clinching it}¤
Has he a hook, Captain?</p>

\direct{Knives and pistols come to hand, and there is a general cry 'The ship is doomed!' But it is not his dogs that can frighten</i> JAS HOOK. <i>Hearing something like a cheer from the boys he wheels round, and his face brings them to their knees.}

<p>HOOK. So you like it, do you! By Caius and Balbus, bullies, here is a notion: open the cabin door and drive them in. Let them fight the doodle-doo for their lives. If they kill him we are so much the better; if he kills them we are none the worse.</p>

\begin{stagedir}
(This masterly stroke restores their confidence; and the boys, affecting fear, are driven into the cabin. Desperadoes though the pirates are, some of them have been boys themselves, and all turn their backs to the cabin and listen, with arms outstretched to it as if to ward off the horrors that are being enacted there.

Relieved by Peter of their manacles, and armed with such weapons as they can lay their hands on, the boys steal out softly as snowflakes, and under their captain's hushed order find hiding-places on the poop. He releases</i>WENDY; <i>and now it would be easy for them all to fly away, but it is to be</i> HOOK <i>or him this time. He signs to her to join the others, and with awful grimness folding her cloak around him, the hood over his head, he takes her place by the mast, and crows.)
\end{stagedir}

<p>MULLINS. The doodle-doo has killed them all!</p>

<p>SEVERAL. The ship 's bewitched.</p>

\direct{They are snapping at</i> HOOK <i>again.}

<p>HOOK. I 've thought it out, lads; there is a Jonah aboard.</p>

<p>SEVERAL \direct{advancing upon him}¤
Ay, a man with a hook.</p>

\direct{If he were to withdraw one step their knives would be in him, but he does not flinch.}

<p>HOOK \direct{temporising}¤
No, lads, no, it is the girl. Never was luck on a pirate ship wi' a woman aboard. We'll right the ship when she has gone.'</p>

<p>MULLINS \direct{lowering his cutlass}¤
It's worth trying.</p>

<p>HOOK. Throw the girl overboard.</p>

<p>MULLINS \direct{jeering}¤
There is none can save you now, missy.</p>

<p>PETER. There is one.</p>

<p>MULLINS. Who is that?</p>

<p>PETER \direct{casting off the cloak}¤
Peter Pan, the avenger!</p>

\direct{He continues standing there to let the effect sink in.}

<p>HOOK \direct{throwing out a suggestion}¤
Cleave him to the brisket.</p>

\direct{But he has a sinking that this boy has no brisket*?*}

<p>NOODLER. The ship 's accurst!</p>

<p>PETER. Down, boys, and at them!</p>

\direct{The boys leap from their concealment and the clash of arms resounds through the vessel. Man to man the pirates are the stronger, but they are unnerved by the sud</i><i>denness of the onslaught and they scatter, thus enabling their opponents to hunt in couples and choose their quarry. Some are hurled into the lagoon; others are dragged from dark recesses. There is no boy whose weapon is not reeking save</i> SLIGHTLY, <i>who runs about with a lantern, counting, ever counting.}

<p>WENDY \direct{meeting</i> MICHAEL <i>in a moment's lull}¤
Oh,Michael, stay with me, protect me!</p>

<p>MICHAEL \direct{reeling}¤
Wendy, I've killed a pirate!</p>

<p>WENDY. It's awful, awful.</p>

<p>MICHAEL. No, it isn't, I like it, I like it.</p>

\direct{He casts himself into the group of boys who are encircling</i> HOOK. <i>Again and again they close upon him and again and again he hews a clear space.}

<p>HOOK. Back, back, you mice. It's Hook; do you like him?
\direct{He lifts up</i> MICHAEL <i>with his claw and uses him as a buckler. A terrible voice breaks in.}</p>

<p>PETER. Put up your swords, boys. This man is mine.</p>

\direct{HOOK <i>shakes</i> MICHAEL <i>off his claw as if he were a drop of water, and these two antagonists face each other for their final bout. They measure swords at arms' length ,make a sweeping motion with them, and bringing the points to the deck rest their hands upon the hilts.}

<p>HOOK \direct{with curling lip}¤
So, Pan, this is all your doing!</p>

<p>PETER. Ay, Jas Hook, it is all my doing.</p>

<p>HOOK. Proud and insolent youth, prepare to meet thy doom.</p>

<p>PETER. Dark and sinister man, have at thee.</p>

\begin{stagedir}
(Some say that he had to ask</i> TOOTLES <i>whether the word was sinister or canister.

HOOK <i>or</i> PETER <i>this time! They fall to without another word.</i> PETER <i>is a rare swordsman, and parries with dazzling rapidity, sometimes before the other can make his stroke.</i> HOOK, <i>if not quite so nimble in wrist play, has the advantage of a yard or two in reach, but though they close he cannot give the quietus with his claw, which seems to find nothing to tear at. He does not, especially in the most heated moments, quite see</i> PETER, <i>who to his eyes, now blurred or opened clearly for the first time, is less like a boy than a mote of dust dancing in the sun. By some impalpable stroke</i> HOOK's <i>sword is whipped from his grasp, and when he stoops to raise it a little foot is on its blade. There is no deep gash on</i> HOOK, <i>but he is suffering torment as from innumerable jags.)
\end{stagedir}

<p>BOYS \direct{exulting}¤
Now, Peter, now!</p>

\direct{PETER <i>raises the sword by its blade, and with an inclination of the head that is perhaps slightly overdone, presents the hilt to his enemy.}

<p>HOOK. 'Tis some fiend fighting me! Pan, who and what art thou?</p>

\direct{The children listen eagerly for the answer, none quite so eagerly as</i> WENDY.}

<p>PETER \direct{at a venture}¤
I'm youth, I'm joy, I'm a little bird that has broken out of the egg.</p>

<p>HOOK. To 't again!</p>

\direct{He has now a damp feeling that this boy is the weapon which is to strike him from the lists of man; but the grandeur of his mind still holds and, true to the traditions of his flag, he fights on like a human flail.</i> PETER <i>flutters round and through and over these gyrations as if the wind of them blew him out of the danger zone ,and again and again he darts in and jags.}

<p>HOOK \direct{stung to madness}¤
I'll fire the powder magazine.
\direct{He disappears they know not where.}</p>

<p>CHILDREN. Peter, save us!</p>

\direct{PETER, <i>alas, goes the wrong way and</i> HOOK <i>returns.}

<p>HOOK \direct{sitting on the hold with gloomy satisfaction}¤
In two minutes the ship will be blown to pieces.</p>

\direct{They cast themselves before him in entreaty.}

<p>CHILDREN. Mercy, mercy!</p>

<p>HOOK. Back, you pewling spawn. I'll show you now the road to dusty death. A holocaust of children, there is something grand in the idea!</p>

\begin{stagedir}
(PETER <i>appears with the smoking bomb in his hand, and tosses it overboard.</i> HOOK <i>has not really had much hope, and he rushes at his other persecutors with his head down like some exasperated bull in the ring; but with bantering cries they easily elude him by flying among the rigging.

<i>Where is</i> PETER? <i>The incredible boy has apparently forgotten the recent doings, and is sitting on a barrel playing upon his pipes. This may surprise others but does not surprise</i> HOOK. <i>Lifting a blunderbuss he strikes forlornly not at the boy but at the barrel, which is hurled across the deck.</i> PETER <i>remains sitting in the air still playing upon his pipes. At this sight the great heart of</i> HOOK <i>breaks. That not wholly unheroic figure climbs the bulwarks murmuring</i> 'Floreat Etona,' <i>and prostrates himself into the water, where the crocodile is waitingfor him open-mouthed.</i> HOOK <i>knows the purpose of this yawning cavity, but after what he has gone through he enters it like one greeting a friend.</i></p>

<p><i>The curtain rises to show</i> PETER <i>a very Napoleon on his ship. It must not rise again lest we see him on the poop in</i> HOOK's <i>hat and cigars, and with a small iron claw.)
\end{stagedir}

% !TEX program = pdflatex
% !TEX encoding = UTF-8
% !TEX spellcheck = en_GB
% !TEX root = peter-pan.tex

\Scene{The Nursery and the Tree-Tops}

\begin{stagedir}
In a sort of way he understands what she means by ‘Yes, I know,’
but in most sorts of ways he doesn’t.
It has something to do with the riddle of his being.
If he could get the hang of the thing his cry might become
‘To live would be an awfully big adventure!\@’
but he can never quite get the hang of it,
and so no one is as gay as he.
With rapturous face he produces his pipes,
and the Never birds and the fairies gather closer,
till the roof of the little house is so thick with his admirers
that some of them fall down the chimney.
He plays on and on till we wake up.)
\end{stagedir}


\backmatter

% !TEX program = pdflatex
% !TEX encoding = UTF-8
% !TEX spellcheck = en_GB
% !TEX root = peter-pan.tex

\chapter{About Project Gutenberg}

% !TEX program = pdflatex
% !TEX encoding = UTF-8
% !TEX spellcheck = en_GB
% !TEX root = peter-pan.tex

\chapter{About the Text}

\begin{adjustwidth}{0.5in}{0.5in}
\begin{center}
\setlength{\parskip}{\baselineskip}

This book was typeset with the \LaTeX\ typesetting system created by Leslie Lamport,
using the \textsc{memoir} class originally by Peter Wilson.

\emph{\theplaytitle} was set using
the pre-release version 2.0 of
Massimiliano Dominici’s \textsc{dramatist} package,
and special thanks are due him for his help.

The body text is set in KP-Light by Christophe Caignaert,
from the \textsc{Johannes Kepler} project.
\end{center}
\end{adjustwidth}

\endinput


\end{document}
