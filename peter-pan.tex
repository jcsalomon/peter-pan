% !TEX program = pdflatex
% !TEX encoding = UTF-8
% !TEX spellcheck = en_GB

\documentclass[draft]{memoir}

\usepackage[utf8]{inputenc}
\usepackage[T1]{fontenc}
\usepackage[onlyrm,oldstylenums,largesmallcaps,noamsmath,notextcomp]{kpfonts}

\usepackage[paper=cream]{createspace}

\usepackage{dramatist}

\title{Peter Pan}
\author{J. M. Barrie}
\date{}

% See ‹http://tex.stackexchange.com/questions/59390/spacefactor-unicode-curly-quotes/2966›
\AtBeginDocument{
  \sfcode\csname\encodingdefault\string\textquotedblright\endcsname=0
}

\newcommand{\theend}{\fancybreak{\scshape{The End}}}

\begin{document}
\frontmatter

\maketitle
\tableofcontents*

\mainmatter

\book{Peter and Wendy}

\chapter{Peter Breaks Through}

All children, except one, grow up.
They soon know that they will grow up,
and the way Wendy knew was this.
One day when she was two years old she was playing in a garden,
and she plucked another flower and ran with it to her mother.
I suppose she must have looked rather delightful,
for Mrs.\@ Darling put her hand to her heart and cried,
“Oh, why can't you remain like this for ever!”
This was all that passed between them on the subject,
but henceforth Wendy knew that she must grow up.
You always know after you are two.
Two is the beginning of the end.

\chapter{The Shadow}

\chapter{Come Away, Come Away!}

\chapter{The Flight}

\chapter{The Island Come True}

\chapter{The Little House}

\chapter{The Home under the Ground}

\chapter{The Mermaids’ Lagoon}

\chapter{The Never Bird}

\chapter{The Happy Home}

\chapter{Wendy’s Story}

\chapter{The Children are Carried Off}

\chapter{Do You Believe in Fairies?}

\chapter{The Pirate Ship}

\chapter{“Hook or Me this Time”}

\chapter{The Return Home}

\chapter{When Wendy Grew Up}

As you look at Wendy,
you may see her hair becoming white,
and her figure little again,
for all this happened long ago.
Jane is now a common grown-up,
with a daughter called Margaret;
and every spring cleaning time,
except when he forgets,
Peter comes for Margaret and takes her to the Neverland,
where she tells him stories about himself,
to which he listens eagerly.
When Margaret grows up she will have a daughter,
who is to be Peter’s mother in turn;
and thus it will go on,
so long as children are gay and innocent and heartless.

\theend

\book[The Boy Who Would Not Grow Up]{Peter Pan\\\emph{or}\\The Boy Who Would Not Grow Up}

Produced at the Duke of York’s Theatre on December 27, 1904.

The play ran for 145 performances.

\chapter[To the Five]{To the Five\\A Dedication}

Some disquieting confessions must be made in printing at last the play of \emph{Peter Pan};
among them this, that I have no recollection of having written it.
Of that, however, anon.
What I want to do first is to give Peter to the Five without whom he never would have existed.
I hope, my dear sirs,
that in memory of what we have been to each other
you will accept this dedication with your friend’s love.
The play of Peter is streaky with you still,
though none may see this save ourselves.
A score of Acts had to be left out, and you were in them all.
We first brought Peter down, didn’t we, with a blunt-headed arrow in Kensington Gardens?
I seem to remember that we believed we had killed him,
though he was only winded,
and that after a spasm of exultation in our prowess
the more soft hearted among us wept and all of us thought of the police.
There was not one of you who would not have sworn as an eye-witness to this occurrence;
no doubt I was abetting,
but you used to provide corroboration that was never given to you by me.
As for myself,
I suppose I always knew that I made Peter by rubbing the five of you violently together,
as savages with two sticks produce a flame.
That is all he is, the spark I got from you.

\Act{The Nursery}

\begin{stagedir}
The night nursery of the Darling family,
which is the scene of our opening Act,
is at the top of a rather depressed street in Bloomsbury.
We have a right to place it where we will,
and the reason Bloomsbury is chosen is that Mr.\@ Roget once lived there.
So did we in days when his \emph{Thesaurus} was our only companion in London;
and we whom he has helped to wend our way through life
have always wanted to pay him a little compliment.
The Darlings therefore lived in Bloomsbury.
\end{stagedir}

\Act{The Never Land}

\Act{The Mermaids’ Lagoon}

\Act{The Home under the Ground}

\act

\Scene{The Pirate Ship}

\Scene{The Nursery and the Tree-Tops}

\begin{stagedir}
In a sort of way he understands what she means by ‘Yes, I know,’
but in most sorts of ways he doesn’t.
It has something to do with the riddle of his being.
If he could get the hang of the thing his cry might become
‘To live would be an awfully big adventure!\@’
but he can never quite get the hang of it,
and so no one is as gay as he.
With rapturous face he produces his pipes,
and the Never birds and the fairies gather closer,
till the roof of the little house is so thick with his admirers
that some of them fall down the chimney.
He plays on and on till we wake up.)
\end{stagedir}

\theend

\backmatter
\end{document}
