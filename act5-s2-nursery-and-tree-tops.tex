% !TEX program = pdflatex
% !TEX encoding = UTF-8
% !TEX spellcheck = en_GB
% !TEX root = peter-pan.tex

\Scene{The Nursery and the Tree-Tops}

\begin{stagedir}
The old nursery appears again with everything just as it was at the beginning of the play,
except that the kennel has gone and that the window is standing open.
So Peter was wrong about mothers;
indeed there is no subject on which he is so likely to be wrong.

Mrs.\@ Darling is asleep on a chair near the window, her eyes tired with searching the heavens.
Nana is stretched out listless on the floor.
She is the cynical one,
and though custom has made her hang the children’s night things on the fire-guard for an airing,
she surveys them not hopefully but with some self-contempt.
\end{stagedir}

\begin{drama}

\mrsdarlingspeaks \direct{starting up as if we had whispered to her that her brats are coming back}¤
Wendy, John, Michael!
\direct{\nana lifts a sympathetic paw to the poor soul’s lap.}
I see you have put their night things out again, Nana!
It touches my heart to watch you do that night after night.
But they will never come back.

\direct{In trouble the difference of station can be completely ignored,
and it is not strange to see these two using the same handkerchief.
Enter \liza, who in the gentleness with which the house has been run of late
is perhaps a little more masterful than of yore.}

\lizaspeaks \direct{feeling herself degraded by the announcement}¤
Nana’s dinner is served.

\direct{\nana, who quite understands what are \liza’s feelings,
departs for the dining-room with *our* exasperating leisureliness,
instead of running, as we would all do if we followed our instincts.}

\lizaspeaks
To think I have a master as have changed places with his dog!

\mrsdarlingspeaks \direct{gently}¤
Out of remorse, Liza.

\lizaspeaks \direct{surely exaggerating}¤
I am a married woman myself.
I don’t think it’s respectable to go to his office in a kennel,
with the street boys running alongside cheering.
\direct{Even this does not rouse her mistress, which may have been the honourable intention.}
There, that is the cab fetching him back!
\direct{Amid interested cheers from the street the kennel is conveyed to its old place by a cabby and friend,
and \mrdarling scrambles out of it in his office clothes.}

\mrdarlingspeaks \direct{giving her his hat loftily}¤
If you will be so good, Liza.
\direct{The cheering is resumed.}
It is very gratifying!

\lizaspeaks \direct{contemptuous}¤
Lot of little boys.

\mrdarlingspeaks \direct{with the new sweetness of one who has sworn never to lose his temper again}¤
There were several adults to-day.

\direct{She goes off scornfully with the hat and the two men, but he has not a word of reproach for her.
It ought to melt us when we see how humbly grateful he is for a kiss from his wife,
so much more than he feels he deserves.
One may think he is wrong to exchange into the kennel,
but sorrow has taught him that he is the kind of man who whatever he does contritely he must do to excess;
otherwise he soon abandons doing it.}

\mrsdarlingspeaks \direct{who has known this for quite a long time}¤
What sort of a day have you had, George?

\direct{He is sitting on the floor by the kennel.}

\mrdarlingspeaks
There were never less than a hundred running round the cab cheering,
and when we passed the Stock Exchange the members came out and waved.

\direct{He is exultant but uncertain of himself, and with a word she could dispirit him utterly.}

\mrsdarlingspeaks \direct{bravely}¤
I am so proud, George.

\mrdarlingspeaks \direct{commendation from the dearest quarter ever going to his head}¤
I have been put on a picture postcard, dear.

\mrsdarlingspeaks \direct{nobly}¤
Never!

\mrdarlingspeaks \direct{thoughtlessly}¤
Ah, Mary, we should not be such celebrities if the children hadn’t flown away.

\mrsdarlingspeaks \direct{startled}¤
George, you are sure you are not enjoying it?,

\mrdarlingspeaks \direct{anxiously}¤
Enjoying it!
See my punishment: living in a kennel.

\mrsdarlingspeaks
Forgive me, dear one.

\mrdarlingspeaks
It is I who need forgiveness, always I, never you.
And now I feel drowsy.
\direct{He retires into the kennel.}
Won’t you play me to sleep on the nursery piano?
And shut that window, Mary dearest; I feel a draught.

\mrsdarlingspeaks
Oh, George, never ask me to do that.
The window must always be left open, for them, always, always.

\direct{She goes into the day nursery, from which we presently hear her playing the sad song of Margaret.
She little knows that her last remark has been overheard by a boy crouching at the window.
He steals into the room accompanied by a ball of light.}

\peterspeaks
Tink, where are you?
Quick, close the window.
\direct{It closes.}
Bar it.
\direct{The bar slams down.}
Now when Wendy comes she will think her mother has barred her out, and she will have to come back to me!
\direct{\tinkerbell sulks.}
Now, Tink, you and I must go out by the door.
\direct{Doors, however, are confusing things to those who are used to windows, and he is puzzled when he finds that this one does not open on to the firmament.
He tries the other, and sees the piano player.}
It is Wendy’s mother!
\direct{\tink pops on to his shoulder and they peep together.}
She is a pretty lady, but not so pretty as my mother.
\direct{This is a pure guess.}
She is making the box say ‘Come home, Wendy.’
You will never see Wendy again, lady, for the window is barred!
\direct{He flutters about the room joyously like a bird, but has to return to that door.}
She has laid her head down on the box.
There are two wet things sitting on her eyes.
As soon as they go away another two come and sit on her eyes.
\direct{She is heard moaning ‘Wendy, Wendy, Wendy.’}
She wants me to unbar the window.
I won’t!
She is awfully fond of Wendy.
I am fond of her too.
We can’t both have her, lady!
\direct{A funny feeling comes over him.}
Come on, Tink; we don’t want any silly mothers.

\begin{stagedir}
(He opens the window and they fly out.

It is thus that the truants find entrance easy when they alight on the sill,
\john to his credit having the tired \michael on his shoulders.
They have nothing else to their credit;
no compunction for what they have done, not the tiniest fear that any just person may be awaiting them with a stick.
The youngest is in a daze,
but the two others are shining virtuously like holy people who are about to give two other people a treat.)
\end{stagedir}

\michaelspeaks \direct{looking about him}¤
I think I have been here before.

\johnspeaks
It’s your home, you stupid.

\wendyspeaks
There is your old bed, Michael.

\michaelspeaks
I had nearly forgotten.

\johnspeaks
I say, the kennel!

\wendyspeaks
Perhaps Nana is in it.

\johnspeaks \direct{peering}¤
There is a man asleep in it.

\wendyspeaks \direct{remembering him by the bald patch}¤
It’s father!

\johnspeaks
So it is!

\michaelspeaks
Let me see father.
\direct{Disappointed}¤
He is not as big as the pirate I killed.

\johnspeaks \direct{perplexed}¤
Wendy, surely father didn’t use to sleep in the kennel?

\wendyspeaks \direct{with misgivings}¤
Perhaps we don’t remember the old life as well as we thought we did.

\johnspeaks \direct{chilled}¤
It is very careless of mother not to be here when we come back.

\direct{The piano is heard again.}

\wendyspeaks
H’sh!
\direct{She goes to the door and peeps.}
That is her playing!
\direct{They all have a peep.}

\michaelspeaks
Who is that lady?

\johnspeaks
H’sh!
It’s mother.

\michaelspeaks
Then are you not really our mother, Wendy?

\wendyspeaks \direct{with conviction}¤
Oh dear, it is quite time to be back!

\johnspeaks
Let us creep in and put our hands over her eyes.

\wendyspeaks \direct{more considerate}¤
No, let us break it to her gently.

\direct{She slips between the sheets of her bed;
and the others, seeing the idea at once, get into their beds.
Then when the music stops they cover their heads.
There are now three distinct bumps in the beds.
\mrsdarling sees the bumps as soon as she comes in, but she does not believe she sees them.}

\mrsdarlingspeaks
I see them in their beds so often in my dreams that I seem still to see them when I am awake!
I’ll not look again.
\direct{She sits down and turns away her face from the bump,
though of course they are still reflected in her mind.}
So often their silver voices call me, my little children whom I’ll see no more.

\direct{Silver voices is a good one, especially about \john; but the heads pop up.}

\wendyspeaks \direct{perhaps rather silvery}¤
Mother!

\mrsdarlingspeaks \direct{without moving}¤
That is Wendy.

\johnspeaks \direct{quite gruff}¤
Mother!

\mrsdarlingspeaks
Now it is John.

\michaelspeaks \direct{no better than a squeak}¤
Mother!

\mrsdarlingspeaks
Now Michael.
And when they call I stretch out my arms to them, but they never come, they never come!

\begin{stagedir}
(This time, however, they come, and there is joy once more in the Darling household.
The little boy who is crouching at the window sees the joke of the bumps in the beds,
but cannot understand what all the rest of the fuss is about.

The scene changes from the inside of the house to the outside,
and we see \mrdarling romping in at the door, with the lost boys hanging gaily to his coat-tails.
Some may conclude that \wendy has told them to wait outside until she explains the situation to her mother,
who has then sent \mrdarling down to tell them that they are adopted.
Of course they could have flown in by the window like a covey of birds,
but they think it better fun to enter by a door.
There is a moment’s trouble about \slightly, who somehow gets shut out.
Fortunately \liza finds him.)
\end{stagedir}

\lizaspeaks
What is the matter, boy?

\slightlyspeaks
They have all got a mother except me.

\lizaspeaks \direct{starting back}¤
Is your name Slightly?

\slightlyspeaks
Yes’m.

\lizaspeaks
Then I am your mother.

\slightlyspeaks
How do you know?

\lizaspeaks \direct{the good-natured creature}¤
I feel it in my bones.

\direct{They go into the house and there is none hazier now than \slightly,
unless it be \nana as she passes with the importance of a nurse who will never have another day off.
\wendy looks out at the nursery window and sees a friend below,
who is hovering in the air knocking off tall hats with his feet.
The wearers don’t see him.
They are too old.
You can’t see \peter if you are old.
They think he is a draught at the corner.}

\wendyspeaks
Peter!

\peterspeaks \direct{looking up casually}¤
Hullo, Wendy.

\direct{She flies down to him, to the horror of her mother, who rushes to the window.}

\wendyspeaks \direct{making a last attempt}¤
You don’t feel you would like to say anything to my parents, Peter, about a very sweet subject?

\peterspeaks
No, Wendy.

\wendyspeaks
About me, Peter?

\peterspeaks
No.
\direct{He gets out his pipes, which she knows is a very bad sign.
She appeals with her arms to \mrsdarling,
who is probably thinking that these children will all need to be tied to their beds at night.}

\mrsdarlingspeaks \direct{from the window}¤
Peter, where are you?
Let me adopt you too.

\direct{She is the loveliest age for a woman, but too old to see \peter clearly.}

\peterspeaks
Would you send me to school?

\mrsdarlingspeaks \direct{obligingly}¤
Yes.

\peterspeaks
And then to an office?

\mrsdarlingspeaks
I suppose so.

\peterspeaks
Soon I should be a man?

\mrsdarlingspeaks
Very soon.

\peterspeaks \direct{passionately}¤
I don’t want to go to school and learn solemn things.
No one is going to catch me, lady, and make me a man.
I want always to be a little boy and to have fun.

\direct{So perhaps he thinks, but it is only his greatest pretend.}

\mrsdarlingspeaks \direct{shivering every time \wendy pursues him in the air}¤
Where are you to live, Peter?

\peterspeaks
In the house we built for Wendy.
The fairies are to put it high up among the tree-tops where they sleep at night.

\wendyspeaks \direct{rapturously}¤
To think of it!

\mrsdarlingspeaks
I thought all the fairies were dead.

\wendyspeaks \direct{almost reprovingly}¤
No indeed!
Their mothers drop the babies into the Never birds’ nests, all mixed up with the eggs,
and the mauve fairies are boys and the white ones are girls,
and there are some colours who don’t know what they are.
The row the children and the birds make at bath time is positively deafening.

\peterspeaks
I throw things at them.

\wendyspeaks
You will be rather lonely in the evenings, Peter.

\peterspeaks
I shall have Tink.

\wendyspeaks \direct{flying up to the window}¤
Mother, may I go?

\mrsdarlingspeaks \direct{gripping her for ever}¤
Certainly not.
I have got you home again, and I mean to keep you.

\wendyspeaks
But he does so need a mother.

\mrsdarlingspeaks
So do you, my love.

\peterspeaks
Oh, all right.

\mrsdarlingspeaks \direct{magnanimously}¤
But, Peter, I shall let her go to you once a year for a week to do your spring cleaning.

\direct{\wendy revels in this,
but \peter, who has no notion what a spring cleaning is, waves a rather careless thanks.}

\mrsdarlingspeaks
Say good-night, Wendy.

\wendyspeaks
I couldn’t go down just for a minute?

\mrsdarlingspeaks
No.

\wendyspeaks
Good-night, Peter!

\peterspeaks
Good-night, Wendy!

\wendyspeaks
Peter, you won’t forget me, will you, before spring-cleaning time comes?

\direct{There is no answer, for he is already soaring high.
For a moment after he is gone we still hear the pipes.
\mrsdarling closes and bars the window.}

\plainbreak{1}

\begin{stagedir}
We are dreaming now of the Never Land a year later.
It is bed-time on the island, and the blind goes up to the whispers of the lovely Never music.
The blue haze that makes the wood below magical by day comes up to the tree-tops to sleep,
and through it we see numberless nests all lit up, fairies and birds quarrelling for possession,
others flying around just for the fun of the thing
and perhaps making bets about where the little house will appear to-night.
It always comes and snuggles on some tree-top, but you can never be sure which;
here it is again, you see John’s hat first as up comes the house
so softly that it knocks some gossips off their perch.
When it has settled comfortably it lights up, and out come Peter and Wendy.

Wendy looks a little older, but Peter is just the same.
She is cloaked for a journey, and a sad confession must be made about her;
she flies so badly now that she has to use a broomstick.
\end{stagedir}

\wendyspeaks \direct{who knows better this time than to be demonstrative at partings}¤
Well, good-bye, Peter; and remember not to bite your nails.

\peterspeaks
Good-bye, Wendy.

\wendyspeaks
I’ll tell mother all about the spring cleaning and the house.

\peterspeaks \direct{who sometimes forgets that she has been here before}¤
You do like the house?

\wendyspeaks
Of course it is small.
But most people of our size wouldn’t have a house at all.
\direct{She should not have mentioned size, for he has already expressed displeasure at her growth.
Another thing, one he has scarcely noticed, though it disturbs her,
is that she does not see him quite so clearly now as she used to do.}
When you come for me next year, Peter—you will come, won’t you?

\peterspeaks
Yes.
\direct{Gloating}¤
To hear stories about me!

\wendyspeaks
It is so queer that the stories you like best should be the ones about yourself.

\peterspeaks \direct{touchy}¤
Well, then?

\wendyspeaks
Fancy your forgetting the lost boys, and even Captain Hook!

\peterspeaks
Well, then?

\wendyspeaks
I haven’t seen Tink this time.

\peterspeaks
Who?

\wendyspeaks
Oh dear!
I suppose it is because you have so many adventures.

\peterspeaks \direct{relieved}¤
’Course it is.

\wendyspeaks
If another little girl—if one younger than I am—¤
\direct{She can’t go on.}
Oh, Peter, how I wish I could take you up and squdge you!
\direct{He draws back.}
Yes, I know.
\direct{She gets astride her broomstick.}
Home!
\direct{It carries her from him over the tree-tops.

In a sort of way he understands what she means by ‘Yes, I know,’
but in most sorts of ways he doesn’t.
It has something to do with the riddle of his being.
If he could get the hang of the thing his cry might become
‘To live would be an awfully big adventure!\@’
but he can never quite get the hang of it,
and so no one is as gay as he.
With rapturous face he produces his pipes,
and the Never birds and the fairies gather closer,
till the roof of the little house is so thick with his admirers
that some of them fall down the chimney.
He plays on and on till we wake up.)}

\end{drama}

\endinput
