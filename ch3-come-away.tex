% !TEX program = pdflatex
% !TEX encoding = UTF-8
% !TEX spellcheck = en_GB
% !TEX root = peter-pan.tex

\chapter{Come Away, Come Away!}

For a moment after Mr.\@ and Mrs.\@ Darling left the house
the night‐lights by the beds of the three children continued to burn clearly.
They were awfully nice little night‐lights,
and one cannot help wishing that they could have kept awake to see Peter;
but Wendy’s light blinked and gave such a yawn that the other two yawned also,
and before they could close their mouths all the three went out.

There was another light in the room now, a thousand times brighter than the night‐lights,
and in the time we have taken to say this, it had been in all the drawers in the nursery,
looking for Peter’s shadow, rummaged the wardrobe and turned every pocket inside out.
It was not really a light;
it made this light by flashing about so quickly,
but when it came to rest for a second you saw it was a fairy, no longer than your hand, but still growing.
It was a girl called Tinker Bell exquisitely gowned in a skeleton leaf, cut low and square,
through which her figure could be seen to the best advantage.
She was slightly inclined to \emph{embonpoint}.

A moment after the fairy’s entrance the window was blown open by the breathing of the little stars,
and Peter dropped in.
He had carried Tinker Bell part of the way, and his hand was still messy with the fairy dust.

“Tinker Bell,” he called softly, after making sure that the children were asleep,
“Tink, where are you?”
She was in a jug for the moment, and liking it extremely;
she had never been in a jug before.

“Oh, do come out of that jug, and tell me, do you know where they put my shadow?”

The loveliest tinkle as of golden bells answered him.
It is the fairy language.
You ordinary children can never hear it,
but if you were to hear it you would know that you had heard it once before.

Tink said that the shadow was in the big box.
She meant the chest of drawers, and Peter jumped at the drawers,
scattering their contents to the floor with both hands, as kings toss ha’pence to the crowd.
In a moment he had recovered his shadow,
and in his delight he forgot that he had shut Tinker Bell up in the drawer.

If he thought at all, but I don’t believe he ever thought,
it was that he and his shadow, when brought near each other, would join like drops of water,
and when they did not he was appalled.
He tried to stick it on with soap from the bathroom, but that also failed.
A shudder passed through Peter, and he sat on the floor and cried.

His sobs woke Wendy, and she sat up in bed.
She was not alarmed to see a stranger crying on the nursery floor;
she was only pleasantly interested.

“Boy,” she said courteously, “why are you crying?”

Peter could be exceeding polite also,
having learned the grand manner at fairy ceremonies,
and he rose and bowed to her beautifully.
She was much pleased, and bowed beautifully to him from the bed.

“What’s your name?\@” he asked.

“Wendy Moira Angela Darling,” she replied with some satisfaction.
“What is your name?”

“Peter Pan.”

She was already sure that he must be Peter,
but it did seem a comparatively short name.

“Is that all?”

“Yes,” he said rather sharply.
He felt for the first time that it was a shortish name.

“I’m so sorry,” said Wendy Moira Angela.

“It doesn’t matter,” Peter gulped.

She asked where he lived.

“Second to the right,” said Peter,
“and then straight on till morning.”

“What a funny address!”

Peter had a sinking.
For the first time he felt that perhaps it was a funny address.

“No, it isn’t,” he said.

“I mean,” Wendy said nicely, remembering that she was hostess,
“is that what they put on the letters?”

He wished she had not mentioned letters.

“Don’t get any letters,” he said contemptuously.

“But your mother gets letters?”

“Don’t have a mother,” he said.
Not only had he no mother, but he had not the slightest desire to have one.
He thought them very over‐rated persons.
Wendy, however, felt at once that she was in the presence of a tragedy.

“O Peter, no wonder you were crying,” she said, and got out of bed and ran to him.

“I wasn’t crying about mothers,” he said rather indignantly.
“I was crying because I can’t get my shadow to stick on.
Besides, I wasn’t crying.”

“It has come off?”

“Yes.”

Then Wendy saw the shadow on the floor, looking so draggled,
and she was frightfully sorry for Peter.
“How awful!\@” she said,
but she could not help smiling when she saw that he had been trying to stick it on with soap.
How exactly like a boy!

Fortunately she knew at once what to do.
“It must be sewn on,” she said, just a little patronisingly.

“What’s sewn?\@” he asked.

“You’re dreadfully ignorant.”

“No, I’m not.”

But she was exulting in his ignorance.
“I shall sew it on for you, my little man,” she said,
though he was tall as herself,
and she got out her housewife, and sewed the shadow on to Peter’s foot.

“I daresay it will hurt a little,” she warned him.

“Oh, I shan’t cry,” said Peter,
who was already of the opinion that he had never cried in his life.
And he clenched his teeth and did not cry,
and soon his shadow was behaving properly, though still a little creased.

“Perhaps I should have ironed it,” Wendy said thoughtfully,
but Peter, boylike, was indifferent to appearances,
and he was now jumping about in the wildest glee.
Alas, he had already forgotten that he owed his bliss to Wendy.
He thought he had attached the shadow himself.
“How clever I am!\@” he crowed rapturously, “oh, the cleverness of me!”

It is humiliating to have to confess that this conceit of Peter was one of his most fascinating qualities.
To put it with brutal frankness, there never was a cockier boy.

But for the moment Wendy was shocked.
“You conceit,” she exclaimed, with frightful sarcasm;
“of course I did nothing!”

“You did a little,” Peter said carelessly, and continued to dance.

“A little!\@” she replied with \emph{hauteur};
“if I am no use I can at least withdraw,”
and she sprang in the most dignified way into bed and covered her face with the blankets.

To induce her to look up he pretended to be going away,
and when this failed he sat on the end of the bed and tapped her gently with his foot.
“Wendy,” he said, “don’t withdraw.
I can’t help crowing, Wendy, when I’m pleased with myself.”
Still she would not look up, though she was listening eagerly.
“Wendy,” he continued, in a voice that no woman has ever yet been able to resist,
“Wendy, one girl is more use than twenty boys.”

Now Wendy was every inch a woman,
though there were not very many inches,
and she peeped out of the bed‐clothes.

“Do you really think so, Peter?”

“Yes, I do.”

“I think it’s perfectly sweet of you,” she declared, “and I’ll get up again,”
and she sat with him on the side of the bed.
She also said she would give him a kiss if he liked,
but Peter did not know what she meant, and he held out his hand expectantly.

“Surely you know what a kiss is?\@” she asked, aghast.

“I shall know when you give it to me,” he replied stiffly,
and not to hurt his feeling she gave him a thimble.

“Now,” said he, “shall I give you a kiss?\@”
and she replied with a slight primness, “If you please.”
She made herself rather cheap by inclining her face toward him,
but he merely dropped an acorn button into her hand,
so she slowly returned her face to where it had been before,
and said nicely that she would wear his kiss on the chain around her neck.
It was lucky that she did put it on that chain, for it was afterwards to save her life.

When people in our set are introduced, it is customary for them to ask each other’s age,
and so Wendy, who always liked to do the correct thing, asked Peter how old he was.
It was not really a happy question to ask him;
it was like an examination paper that asks grammar,
when what you want to be asked is Kings of England.

“I don’t know,” he replied uneasily, “but I am quite young.”
He really knew nothing about it, he had merely suspicions,
but he said at a venture, “Wendy, I ran away the day I was born.”

Wendy was quite surprised, but interested;
and she indicated in the charming drawing‐room manner, by a touch on her night‐gown,
that he could sit nearer her.

“It was because I heard father and mother,” he explained in a low voice,
“talking about what I was to be when I became a man.”
He was extraordinarily agitated now.
“I don’t want ever to be a man,” he said with passion.
“I want always to be a little boy and to have fun.
So I ran away to Kensington Gardens and lived a long long time among the fairies.”

She gave him a look of the most intense admiration,
and he thought it was because he had run away,
but it was really because he knew fairies.
Wendy had lived such a home life that to know fairies struck her as quite delightful.
She poured out questions about them, to his surprise,
for they were rather a nuisance to him, getting in his way and so on,
and indeed he sometimes had to give them a hiding.
Still, he liked them on the whole, and he told her about the beginning of fairies.

“You see, Wendy, when the first baby laughed for the first time, its laugh broke into a thousand pieces,
and they all went skipping about, and that was the beginning of fairies.”

Tedious talk this, but being a stay‐at‐home she liked it.

“And so,” he went on good‐naturedly, “there ought to be one fairy for every boy and girl.”

“Ought to be?
Isn’t there?”

“No.
You see children know such a lot now, they soon don’t believe in fairies,
and every time a child says, ‘I don’t believe in fairies,’ there is a fairy somewhere that falls down dead.”

Really, he thought they had now talked enough about fairies,
and it struck him that Tinker Bell was keeping very quiet.
“I can’t think where she has gone to,” he said, rising, and he called Tink by name.
Wendy’s heart went flutter with a sudden thrill.

“Peter,” she cried, clutching him,
“you don’t mean to tell me that there is a fairy in this room!”

“She was here just now,” he said a little impatiently.
“You don’t hear her, do you?\@” and they both listened.

“The only sound I hear,” said Wendy, “is like a tinkle of bells.”

“Well, that’s Tink, that’s the fairy language.
I think I hear her too.”

The sound came from the chest of drawers, and Peter made a merry face.
No one could ever look quite so merry as Peter, and the loveliest of gurgles was his laugh.
He had his first laugh still.

“Wendy,” he whispered gleefully, “I do believe I shut her up in the drawer!”

He let poor Tink out of the drawer, and she flew about the nursery screaming with fury.
“You shouldn’t say such things,” Peter retorted.
“Of course I’m very sorry, but how could I know you were in the drawer?”

Wendy was not listening to him.
“O Peter,” she cried, “if she would only stand still and let me see her!”

“They hardly ever stand still,” he said,
but for one moment Wendy saw the romantic figure come to rest on the cuckoo clock.
“O the lovely!\@” she cried, though Tink’s face was still distorted with passion.

“Tink,” said Peter amiably, “this lady says she wishes you were her fairy.”

Tinker Bell answered insolently.

“What does she say, Peter?”

He had to translate.
“She is not very polite.
She says you are a great ugly girl, and that she is my fairy.”

He tried to argue with Tink.
“You know you can’t be my fairy, Tink, because I am an gentleman and you are a lady.”

To this Tink replied in these words, “You silly ass,” and disappeared into the bathroom.
“She is quite a common fairy,” Peter explained apologetically,
“she is called Tinker Bell because she mends the pots and kettles.”

They were together in the armchair by this time,
and Wendy plied him with more questions.

“If you don’t live in Kensington Gardens now—”

“Sometimes I do still.”

“But where do you live mostly now?”

“With the lost boys.”

“Who are they?”

“They are the children who fall out of their perambulators when the nurse is looking the other way.
If they are not claimed in seven days they are sent far away to the Neverland to defray expenses.
I’m captain.”

“What fun it must be!”

“Yes,” said cunning Peter, “but we are rather lonely.
You see we have no female companionship.”

“Are none of the others girls?”

“Oh, no;
girls, you know, are much too clever to fall out of their prams.”

This flattered Wendy immensely.
“I think,” she said, “it is perfectly lovely the way you talk about girls;
John there just despises us.”

For reply Peter rose and kicked John out of bed, blankets and all; one kick.
This seemed to Wendy rather forward for a first meeting,
and she told him with spirit that he was not captain in her house.
However, John continued to sleep so placidly on the floor that she allowed him to remain there.
“And I know you meant to be kind,” she said, relenting, “so you may give me a kiss.”

For the moment she had forgotten his ignorance about kisses.
“I thought you would want it back,” he said a little bitterly,
and offered to return her the thimble.

“Oh dear,” said the nice Wendy,
“I don’t mean a kiss, I mean a thimble.”

“What’s that?”

“It’s like this.”
She kissed him.

“Funny!\@” said Peter gravely.
“Now shall I give you a thimble?”

“If you wish to,” said Wendy, keeping her head erect this time.

Peter thimbled her, and almost immediately she screeched.
“What is it, Wendy?”

“It was exactly as if someone were pulling my hair.”

“That must have been Tink.
I never knew her so naughty before.”

And indeed Tink was darting about again, using offensive language.

“She says she will do that to you, Wendy, every time I give you a thimble.”

“But why?”

“Why, Tink?”

Again Tink replied, “You silly ass.”
Peter could not understand why, but Wendy understood,
and she was just slightly disappointed when he admitted that he came to the nursery window
not to see her but to listen to stories.

“You see, I don’t know any stories.
None of the lost boys knows any stories.”

“How perfectly awful,” Wendy said.

“Do you know,” Peter asked “why swallows build in the eaves of houses?
It is to listen to the stories.
O Wendy, your mother was telling you such a lovely story.”

“Which story was it?”

“About the prince who couldn’t find the lady who wore the glass slipper.”

“Peter,” said Wendy excitedly, “that was Cinderella,
and he found her, and they lived happily ever after.”

Peter was so glad that he rose from the floor, where they had been sitting, and hurried to the window.

“Where are you going?\@” she cried with misgiving.

“To tell the other boys.”

“Don’t go Peter,” she entreated,
“I know such lots of stories.”

Those were her precise words, so there can be no denying that it was she who first tempted him.

He came back, and there was a greedy look in his eyes now which ought to have alarmed her, but did not.

“Oh, the stories I could tell to the boys!\@” she cried,
and then Peter gripped her and began to draw her toward the window.

“Let me go!\@” she ordered him.

“Wendy, do come with me and tell the other boys.”

Of course she was very pleased to be asked,
but she said, “Oh dear, I can’t.
Think of mummy!
Besides, I can’t fly.”

“I’ll teach you.”

“Oh, how lovely to fly.”

“I’ll teach you how to jump on the wind’s back, and then away we go.”

“Oo!\@” she exclaimed rapturously.

“Wendy, Wendy, when you are sleeping in your silly bed
you might be flying about with me saying funny things to the stars.”

“Oo!”

“And, Wendy, there are mermaids.”

“Mermaids!
With tails?”

“Such long tails.”

“Oh,” cried Wendy, “to see a mermaid!”

He had become frightfully cunning.
“Wendy,” he said, “how we should all respect you.”

She was wriggling her body in distress.
It was quite as if she were trying to remain on the nursery floor.

But he had no pity for her.

“Wendy,” he said, the sly one, “you could tuck us in at night.”

“Oo!”

“None of us has ever been tucked in at night.”

“Oo,” and her arms went out to him.

“And you could darn our clothes, and make pockets for us.
None of us has any pockets.”

How could she resist.
“Of course it’s awfully fascinating!\@” she cried.
“Peter, would you teach John and Michael to fly too?”

“If you like,” he said indifferently, and she ran to John and Michael and shook them.
“Wake up,” she cried, “Peter Pan has come and he is to teach us to fly.”

John rubbed his eyes.
“Then I shall get up,” he said.
Of course he was on the floor already.
“Hallo,” he said, “I am up!”

Michael was up by this time also, looking as sharp as a knife with six blades and a saw,
but Peter suddenly signed silence.
Their faces assumed the awful craftiness of children listening for sounds from the grown‐up world.
All was as still as salt.
Then everything was right.
No, stop!
Everything was wrong.
Nana, who had been barking distressfully all the evening, was quiet now.
It was her silence they had heard.

“Out with the light!
Hide!
Quick!\@” cried John, taking command for the only time throughout the whole adventure.
And thus when Liza entered, holding Nana, the nursery seemed quite its old self, very dark,
and you would have sworn you heard its three wicked inmates breathing angelically as they slept.
They were really doing it artfully from behind the window curtains.

Liza was in a bad temper, for she was mixing the Christmas puddings in the kitchen,
and had been drawn from them, with a raisin still on her cheek, by Nana’s absurd suspicions.
She thought the best way of getting a little quiet was to take Nana to the nursery for a moment,
but in custody of course.

“There, you suspicious brute,” she said, not sorry that Nana was in disgrace.
“They are perfectly safe, aren’t they?
Every one of the little angels sound asleep in bed.
Listen to their gentle breathing.”

Here Michael, encouraged by his success, breathed so loudly that they were nearly detected.
Nana knew that kind of breathing, and she tried to drag herself out of Liza’s clutches.

But Liza was dense.
“No more of it, Nana,” she said sternly, pulling her out of the room.
“I warn you if you bark again I shall go straight for master and missus and bring them home from the party,
and then, oh, won’t master whip you, just.”

She tied the unhappy dog up again, but do you think Nana ceased to bark?
Bring master and missus home from the party!
Why, that was just what she wanted.
Do you think she cared whether she was whipped so long as her charges were safe?
Unfortunately Liza returned to her puddings,
and Nana, seeing that no help would come from her, strained and strained at the chain until at last she broke it.
In another moment she had burst into the dining‐room of 27 and flung up her paws to heaven,
her most expressive way of making a communication.
Mr.\@ and Mrs.\@ Darling knew at once that something terrible was happening in their nursery,
and without a good‐bye to their hostess they rushed into the street.

But it was now ten minutes since three scoundrels had been breathing behind the curtains,
and Peter Pan can do a great deal in ten minutes.

We now return to the nursery.

“It’s all right,” John announced, emerging from his hiding‐place.
“I say, Peter, can you really fly?”

Instead of troubling to answer him Peter flew around the room, taking the mantelpiece on the way.

“How topping!\@” said John and Michael.

“How sweet!\@” cried Wendy.

“Yes, I’m sweet, oh, I am sweet!\@” said Peter, forgetting his manners again.

It looked delightfully easy, and they tried it first from the floor and then from the beds,
but they always went down instead of up.

“I say, how do you do it?\@” asked John, rubbing his knee.
He was quite a practical boy.

“You just think lovely wonderful thoughts,” Peter explained, “and they lift you up in the air.”

He showed them again.

“You’re so nippy at it,” John said, “couldn’t you do it very slowly once?”

Peter did it both slowly and quickly.
“I’ve got it now, Wendy!\@” cried John, but soon he found he had not.
Not one of them could fly an inch,
though even Michael was in words of two syllables, and Peter did not know A from Z\@.

Of course Peter had been trifling with them,
for no one can fly unless the fairy dust has been blown on him.
Fortunately, as we have mentioned, one of his hands was messy with it,
and he blew some on each of them, with the most superb results.

“Now just wiggle your shoulders this way,” he said, “and let go.”

They were all on their beds, and gallant Michael let go first.
He did not quite mean to let go, but he did it,
and immediately he was borne across the room.

“I flewed!\@” he screamed while still in mid‐air.

John let go and met Wendy near the bathroom.

“Oh, lovely!”

“Oh, ripping!”

“Look at me!”

“Look at me!”

“Look at me!”

They were not nearly so elegant as Peter, they could not help kicking a little,
but their heads were bobbing against the ceiling,
and there is almost nothing so delicious as that.
Peter gave Wendy a hand at first, but had to desist, Tink was so indignant.

Up and down they went, and round and round.
Heavenly was Wendy’s word.

“I say,” cried John, “why shouldn’t we all go out?”

Of course it was to this that Peter had been luring them.

Michael was ready:
he wanted to see how long it took him to do a billion miles.
But Wendy hesitated.

“Mermaids!\@” said Peter again.

“Oo!”

“And there are pirates.”

“Pirates,” cried John, seizing his Sunday hat, “let us go at once.”

It was just at this moment that Mr.\@ and Mrs.\@ Darling hurried with Nana out of 27.
They ran into the middle of the street to look up at the nursery window;
and, yes, it was still shut,
but the room was ablaze with light,
and most heart‐gripping sight of all,
they could see in shadow on the curtain three little figures in night attire circling round and round,
not on the floor but in the air.

Not three figures, four!

In a tremble they opened the street door.
Mr.\@ Darling would have rushed upstairs, but Mrs.\@ Darling signed him to go softly.
She even tried to make her heart go softly.

Will they reach the nursery in time?
If so, how delightful for them, and we shall all breathe a sigh of relief, but there will be no story.
On the other hand, if they are not in time, I solemnly promise that it will all come right in the end.

They would have reached the nursery in time had it not been that the little stars were watching them.
Once again the stars blew the window open, and that smallest star of all called out:

“\emph{Cave}, Peter!”

Then Peter knew that there was not a moment to lose.
“Come,” he cried imperiously, and soared out at once into the night,
followed by John and Michael and Wendy.

Mr.\@ and Mrs.\@ Darling and Nana rushed into the nursery too late.
The birds were flown.

\endinput
