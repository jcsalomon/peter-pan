% !TEX program = pdflatex
% !TEX encoding = UTF-8
% !TEX spellcheck = en_GB
% !TEX root = peter-pan.tex

\chapter{Do You Believe in Fairies?}

\endinput


Chapter 13 DO YOU BELIEVE IN FAIRIES?


The more quickly this horror is disposed of the better. The first to
emerge from his tree was Curly. He rose out of it into the arms of Cecco,
who flung him to Smee, who flung him to Starkey, who flung him to Bill
Jukes, who flung him to Noodler, and so he was tossed from one to another
till he fell at the feet of the black pirate. All the boys were plucked
from their trees in this ruthless manner; and several of them were in the
air at a time, like bales of goods flung from hand to hand.


A different treatment was accorded to Wendy, who came last. With ironical
politeness Hook raised his hat to her, and, offering her his arm, escorted
her to the spot where the others were being gagged. He did it with such an
air, he was so frightfully \emph{distingué}, that she
was too fascinated to cry out. She was only a little girl.


Perhaps it is tell-tale to divulge that for a moment Hook entranced her,
and we tell on her only because her slip led to strange results. Had she
haughtily unhanded him (and we should have loved to write it of her), she
would have been hurled through the air like the others, and then Hook
would probably not have been present at the tying of the children; and had
he not been at the tying he would not have discovered Slightly's secret,
and without the secret he could not presently have made his foul attempt
on Peter's life.


They were tied to prevent their flying away, doubled up with their knees
close to their ears; and for the trussing of them the black pirate had cut
a rope into nine equal pieces. All went well until Slightly's turn came,
when he was found to be like those irritating parcels that use up all the
string in going round and leave no tags with which to tie a knot.
The pirates kicked him in their rage, just as you kick the parcel (though
in fairness you should kick the string); and strange to say it was Hook
who told them to belay their violence. His lip was curled with malicious
triumph. While his dogs were merely sweating because every time they tried
to pack the unhappy lad tight in one part he bulged out in another, Hook's
master mind had gone far beneath Slightly's surface, probing not for
effects but for causes; and his exultation showed that he had found them.
Slightly, white to the gills, knew that Hook had surprised
his secret, which was this, that no boy so blown out could use a tree
wherein an average man need stick. Poor Slightly, most wretched of all the
children now, for he was in a panic about Peter, bitterly regretted what
he had done. Madly addicted to the drinking of water when he was hot, he
had swelled in consequence to his present girth, and instead of reducing
himself to fit his tree he had, unknown to the others, whittled his tree
to make it fit him.


Sufficient of this Hook guessed to persuade him that Peter at last lay at
his mercy, but no word of the dark design that now formed in the
subterranean caverns of his mind crossed his lips; he merely signed that
the captives were to be conveyed to the ship, and that he would be alone.


How to convey them? Hunched up in their ropes they might indeed be rolled
down hill like barrels, but most of the way lay through a morass. Again
Hook's genius surmounted difficulties. He indicated that the little house
must be used as a conveyance. The children were flung into it, four stout
pirates raised it on their shoulders, the others fell in behind, and
singing the hateful pirate chorus the strange procession set off through
the wood. I don't know whether any of the children were crying; if so, the
singing drowned the sound; but as the little house disappeared in the
forest, a brave though tiny jet of smoke issued from its chimney as if
defying Hook.


Hook saw it, and it did Peter a bad service. It dried up any trickle of
pity for him that may have remained in the pirate's infuriated breast.


The first thing he did on finding himself alone in the fast falling night
was to tiptoe to Slightly's tree, and make sure that it provided him with
a passage. Then for long he remained brooding; his hat of ill omen on the
sward, so that any gentle breeze which had arisen might play refreshingly
through his hair. Dark as were his thoughts his blue eyes were as soft as
the periwinkle. Intently he listened for any sound from the nether world,
but all was as silent below as above; the house under the ground seemed to
be but one more empty tenement in the void. Was that boy asleep, or did he
stand waiting at the foot of Slightly's tree, with his dagger in his hand?


There was no way of knowing, save by going down. Hook let his cloak slip
softly to the ground, and then biting his lips till a lewd blood stood on
them, he stepped into the tree. He was a brave man, but for a moment he
had to stop there and wipe his brow, which was dripping like a candle.
Then, silently, he let himself go into the unknown.


He arrived unmolested at the foot of the shaft, and stood still again,
biting at his breath, which had almost left him. As his eyes became
accustomed to the dim light various objects in the home under the trees
took shape; but the only one on which his greedy gaze rested, long sought
for and found at last, was the great bed. On the bed lay Peter fast
asleep.


Unaware of the tragedy being enacted above, Peter had continued, for a
little time after the children left, to play gaily on his pipes: no doubt
rather a forlorn attempt to prove to himself that he did not care. Then he
decided not to take his medicine, so as to grieve Wendy. Then he lay down
on the bed outside the coverlet, to vex her still more; for she had always
tucked them inside it, because you never know that you may not grow chilly
at the turn of the night. Then he nearly cried; but it struck him how
indignant she would be if he laughed instead; so he laughed a haughty
laugh and fell asleep in the middle of it.


Sometimes, though not often, he had dreams, and they were more painful
than the dreams of other boys. For hours he could not be separated from
these dreams, though he wailed piteously in them. They had to do, I think,
with the riddle of his existence. At such times it had been Wendy's custom
to take him out of bed and sit with him on her lap, soothing him in dear
ways of her own invention, and when he grew calmer to put him back to bed
before he quite woke up, so that he should not know of the indignity to
which she had subjected him. But on this occasion he had fallen at once
into a dreamless sleep. One arm dropped over the edge of the bed, one leg
was arched, and the unfinished part of his laugh was stranded on his
mouth, which was open, showing the little pearls.


Thus defenceless Hook found him. He stood silent at the foot of the tree
looking across the chamber at his enemy. Did no feeling of compassion
disturb his sombre breast? The man was not wholly evil; he loved flowers
(I have been told) and sweet music (he was himself no mean performer on
the harpsichord); and, let it be frankly admitted, the idyllic nature of
the scene stirred him profoundly. Mastered by his better self he would
have returned reluctantly up the tree, but for one thing.


What stayed him was Peter's impertinent appearance as he slept. The open
mouth, the drooping arm, the arched knee: they were such a personification
of cockiness as, taken together, will never again, one may hope, be
presented to eyes so sensitive to their offensiveness. They steeled Hook's
heart. If his rage had broken him into a hundred pieces every one of them
would have disregarded the incident, and leapt at the sleeper.


Though a light from the one lamp shone dimly on the bed, Hook stood in
darkness himself, and at the first stealthy step forward he discovered an
obstacle, the door of Slightly's tree. It did not entirely fill the
aperture, and he had been looking over it. Feeling for the catch, he found
to his fury that it was low down, beyond his reach. To his disordered
brain it seemed then that the irritating quality in Peter's face and
figure visibly increased, and he rattled the door and flung himself
against it. Was his enemy to escape him after all?


But what was that? The red in his eye had caught sight of Peter's medicine
standing on a ledge within easy reach. He fathomed what it was
straightaway, and immediately knew that the sleeper was in his power.


Lest he should be taken alive, Hook always carried about his person a
dreadful drug, blended by himself of all the death-dealing rings that had
come into his possession. These he had boiled down into a yellow liquid
quite unknown to science, which was probably the most virulent poison in
existence.


Five drops of this he now added to Peter's cup. His hand shook, but it was
in exultation rather than in shame. As he did it he avoided glancing at
the sleeper, but not lest pity should unnerve him; merely to avoid
spilling. Then one long gloating look he cast upon his victim, and
turning, wormed his way with difficulty up the tree. As he emerged at the
top he looked the very spirit of evil breaking from its hole. Donning his
hat at its most rakish angle, he wound his cloak around him, holding one
end in front as if to conceal his person from the night, of which it was
the blackest part, and muttering strangely to himself, stole away through
the trees.


Peter slept on. The light guttered and went out, leaving
the tenement in darkness; but still he slept. It must have been not less
than ten o'clock by the crocodile, when he suddenly sat up in his bed,
wakened by he knew not what. It was a soft cautious tapping on the door of
his tree.


Soft and cautious, but in that stillness it was sinister. Peter felt for
his dagger till his hand gripped it. Then he spoke.


"Who is that?"


For long there was no answer: then again the knock.


"Who are you?"


No answer.


He was thrilled, and he loved being thrilled. In two strides he reached
the door. Unlike Slightly's door, it filled the aperture, so
that he could not see beyond it, nor could the one knocking see him.


"I won't open unless you speak," Peter cried.


Then at last the visitor spoke, in a lovely bell-like voice.


"Let me in, Peter."


It was Tink, and quickly he unbarred to her. She flew in excitedly, her
face flushed and her dress stained with mud.


"What is it?"


"Oh, you could never guess!" she cried, and offered him three guesses.
"Out with it!" he shouted, and in one ungrammatical sentence, as long as
the ribbons that conjurers pull from their mouths, she told of
the capture of Wendy and the boys.


Peter's heart bobbed up and down as he listened. Wendy bound, and on the
pirate ship; she who loved everything to be just so!


"I'll rescue her!" he cried, leaping at his weapons. As he leapt he
thought of something he could do to please her. He could take his
medicine.


His hand closed on the fatal draught.


"No!" shrieked Tinker Bell, who had heard Hook mutter about his deed as he
sped through the forest.


"Why not?"


"It is poisoned."


"Poisoned? Who could have poisoned it?"


"Hook."


"Don't be silly. How could Hook have got down here?"


Alas, Tinker Bell could not explain this, for even she did not know the
dark secret of Slightly's tree. Nevertheless Hook's words had left no room
for doubt. The cup was poisoned.


"Besides," said Peter, quite believing himself "I never fell asleep."


He raised the cup. No time for words now; time for deeds; and with one of
her lightning movements Tink got between his lips and the draught, and
drained it to the dregs.


"Why, Tink, how dare you drink my medicine?"


But she did not answer. Already she was reeling in the air.


"What is the matter with you?" cried Peter, suddenly afraid.


"It was poisoned, Peter," she told him softly; "and now I am going to be
dead."


"O Tink, did you drink it to save me?"


"Yes."


"But why, Tink?"


Her wings would scarcely carry her now, but in reply she alighted on his
shoulder and gave his nose a loving bite. She whispered in his ear "You
silly ass," and then, tottering to her chamber, lay down on the bed.


His head almost filled the fourth wall of her little room as he knelt near
her in distress. Every moment her light was growing fainter; and he knew
that if it went out she would be no more. She liked his tears so much that
she put out her beautiful finger and let them run over it.


Her voice was so low that at first he could not make out what she said.
Then he made it out. She was saying that she thought she could get well
again if children believed in fairies.


Peter flung out his arms. There were no children there, and it was night
time; but he addressed all who might be dreaming of the Neverland, and who
were therefore nearer to him than you think: boys and girls in their
nighties, and naked papooses in their baskets hung from trees.


"Do you believe?" he cried.


Tink sat up in bed almost briskly to listen to her fate.


She fancied she heard answers in the affirmative, and then again she
wasn't sure.


"What do you think?" she asked Peter.


"If you believe," he shouted to them, "clap your hands; don't let Tink
die."


Many clapped.


Some didn't.


A few beasts hissed.


The clapping stopped suddenly; as if countless mothers had rushed to their
nurseries to see what on earth was happening; but already Tink was saved.
First her voice grew strong, then she popped out of bed, then she was
flashing through the room more merry and impudent than ever. She never
thought of thanking those who believed, but she would have like to get at
the ones who had hissed.


"And now to rescue Wendy!"


The moon was riding in a cloudy heaven when Peter rose from his tree,
begirt with weapons and wearing little else, to set out upon his
perilous quest. It was not such a night as he would have chosen. He had
hoped to fly, keeping not far from the ground so that nothing unwonted
should escape his eyes; but in that fitful light to have flown low would
have meant trailing his shadow through the trees, thus disturbing birds
and acquainting a watchful foe that he was astir.


He regretted now that he had given the birds of the island such strange
names that they are very wild and difficult of approach.


There was no other course but to press forward in redskin fashion, at
which happily he was an adept. But in what direction, for he
could not be sure that the children had been taken to the ship? A light
fall of snow had obliterated all footmarks; and a deathly silence pervaded
the island, as if for a space Nature stood still in horror of the recent
carnage. He had taught the children something of the forest lore that he
had himself learned from Tiger Lily and Tinker Bell, and knew that in
their dire hour they were not likely to forget it. Slightly, if he had an
opportunity, would blaze the trees, for instance, Curly
would drop seeds, and Wendy would leave her handkerchief at some important
place. The morning was needed to search for such guidance, and he could
not wait. The upper world had called him, but would give no help.


The crocodile passed him, but not another living thing, not a sound, not a
movement; and yet he knew well that sudden death might be at the next
tree, or stalking him from behind.


He swore this terrible oath: "Hook or me this time."


Now he crawled forward like a snake, and again erect, he darted across a
space on which the moonlight played, one finger on his lip and his dagger
at the ready. He was frightfully happy.

